\documentclass[12pt, letterpaper]{article}
\usepackage{url, graphicx, epstopdf, amsmath}

% page layout
\setlength{\topmargin}{-0.25in}
\setlength{\textheight}{9.5in}
\setlength{\headheight}{0in}
\setlength{\headsep}{0in}
\setlength{\parindent}{1.1\baselineskip}

\newcommand{\examheader}[1]{
\noindent
Name:\rule[-1ex]{0.40\textwidth}{0.1pt}
NetID:\rule[-1ex]{0.13\textwidth}{0.1pt}
Lab time:\rule[-1ex]{0.13\textwidth}{0.1pt}

\section*{\textsl{Einstein's Universe} {#1}}
\setcounter{problem}{1}}


% problem formatting
\newcommand{\problemname}{Problem}
\newcounter{problem}\setcounter{problem}{1}
\newenvironment{problem}{%
  \addvspace{\baselineskip}\noindent\textbf{Problem~\theproblem:}\refstepcounter{problem}
}{%
  \par\addvspace{\baselineskip}
}

% words
\newcommand{\foreign}[1]{\textsl{#1}}
\newcommand{\vs}{\foreign{vs}}

% math
\newcommand{\dd}{\mathrm{d}}
\newcommand{\e}{\mathrm{e}}

% primary or base units
\newcommand{\rad}{\mathrm{rad}}
\newcommand{\kg}{\mathrm{kg}}
\newcommand{\m}{\mathrm{m}}
\newcommand{\s}{\mathrm{s}}

% secondary units
\renewcommand{\deg}{\mathrm{deg}}
\newcommand{\g}{\mathrm{g}}
\newcommand{\km}{\mathrm{km}}
\newcommand{\cm}{\mathrm{cm}}
\newcommand{\mm}{\mathrm{mm}}
\newcommand{\mum}{\mathrm{\mu m}}
\newcommand{\ft}{\mathrm{ft}}
\newcommand{\mi}{\mathrm{mi}}
\newcommand{\AU}{\mathrm{AU}}
\newcommand{\Mpc}{\mathrm{Mpc}}
\newcommand{\ns}{\mathrm{ns}}
\newcommand{\h}{\mathrm{h}}
\newcommand{\yr}{\mathrm{yr}}
\newcommand{\N}{\mathrm{N}}
\newcommand{\J}{\mathrm{J}}
\newcommand{\eV}{\mathrm{eV}}
\newcommand{\W}{\mathrm{W}}
\newcommand{\Pa}{\mathrm{Pa}}

% derived units
\newcommand{\mps}{\m\,\s^{-1}}
\newcommand{\mph}{\mi\,\h^{-1}}
\newcommand{\mpss}{\m\,\s^{-2}}

% random stuff
\sloppy\sloppypar\raggedbottom\frenchspacing


\begin{document}

\section*{\textsl{Einstein's Universe} Problem Set 1}

This problem set is not to be handed in for credit. But it is due
before \textbf{Thursday September 19}, when some of these problems
will appear in part, near-verbatim, on Term Exam 1.

\begin{problem}
How many cubic millimeters are there in a cubic centimeter? How many
cubic centimeters are there in a liter? How many liters are there in a
cubic meter?
\end{problem}

\begin{problem}
How many molecules of air are there in one cubic millimeter?
Approximately! You might want to look up the following things: How
many molecules are there in a mole? What is the volume of a mole at
STP? What is STP? Express your answer in scientific notation, and only
give your answer to two digits of accuracy..
\end{problem}

\begin{problem}
How many molecules of water are there in one cubic millimeter?
Approximately! Water is $\mathrm{H}_2\mathrm{O}$; use that to figure
out the mass of one mole of water. Then use the density of water to
figure out what fraction of a mole is in one cubic millimeter. Again,
use scientific notation and only give two digits of accuracy.
\end{problem}

\begin{problem}
What is the speed of light in parsecs per year? Give your answer to
only two digits of accuracy.
\end{problem}

\begin{problem}
What is the energy (in joules $\J$) of a photon with a wavelength of
$\lambda = 0.5\,\mum$? You will have to look up the relationship
between energy $E$ and wavelength $\lambda$; it involves the speed of
light $c$ and Planck's constant $h$. How many photons per unit time
(number per second) does a $2\,\W$ LED emit if you assume that all
that power goes into $0.5\,\mum$ photons? You might have to look up
the definition of a watt $\W$. Use scientific notation to express your
answer and don't forget your units.
\end{problem}

\begin{problem}
What are the base units of the expression $2\,n\,m\,v^2$, where $n$
has units of number per volume, $m$ has units of mass, and $v$ has
units of speed? The base units are $\kg$ (kilograms), $\m$ (meters),
and $\s$ (seconds). Compare these to the base units of a force and a
pressure.
\end{problem}

\end{document}
