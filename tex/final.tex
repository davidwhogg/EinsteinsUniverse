\documentclass[12pt, letterpaper]{article}
\usepackage{url, graphicx, epstopdf, amsmath}

% page layout
\setlength{\topmargin}{-0.25in}
\setlength{\textheight}{9.5in}
\setlength{\headheight}{0in}
\setlength{\headsep}{0in}
\setlength{\parindent}{1.1\baselineskip}

\newcommand{\examheader}[1]{
\noindent
Name:\rule[-1ex]{0.40\textwidth}{0.1pt}
NetID:\rule[-1ex]{0.13\textwidth}{0.1pt}
Lab time:\rule[-1ex]{0.13\textwidth}{0.1pt}

\section*{\textsl{Einstein's Universe} {#1}}
\setcounter{problem}{1}}


% problem formatting
\newcommand{\problemname}{Problem}
\newcounter{problem}\setcounter{problem}{1}
\newenvironment{problem}{%
  \addvspace{\baselineskip}\noindent\textbf{Problem~\theproblem:}\refstepcounter{problem}
}{%
  \par\addvspace{\baselineskip}
}

% words
\newcommand{\foreign}[1]{\textsl{#1}}
\newcommand{\vs}{\foreign{vs}}

% math
\newcommand{\dd}{\mathrm{d}}
\newcommand{\e}{\mathrm{e}}

% primary or base units
\newcommand{\rad}{\mathrm{rad}}
\newcommand{\kg}{\mathrm{kg}}
\newcommand{\m}{\mathrm{m}}
\newcommand{\s}{\mathrm{s}}

% secondary units
\renewcommand{\deg}{\mathrm{deg}}
\newcommand{\g}{\mathrm{g}}
\newcommand{\km}{\mathrm{km}}
\newcommand{\cm}{\mathrm{cm}}
\newcommand{\mm}{\mathrm{mm}}
\newcommand{\mum}{\mathrm{\mu m}}
\newcommand{\ft}{\mathrm{ft}}
\newcommand{\mi}{\mathrm{mi}}
\newcommand{\AU}{\mathrm{AU}}
\newcommand{\Mpc}{\mathrm{Mpc}}
\newcommand{\ns}{\mathrm{ns}}
\newcommand{\h}{\mathrm{h}}
\newcommand{\yr}{\mathrm{yr}}
\newcommand{\N}{\mathrm{N}}
\newcommand{\J}{\mathrm{J}}
\newcommand{\eV}{\mathrm{eV}}
\newcommand{\W}{\mathrm{W}}
\newcommand{\Pa}{\mathrm{Pa}}

% derived units
\newcommand{\mps}{\m\,\s^{-1}}
\newcommand{\mph}{\mi\,\h^{-1}}
\newcommand{\mpss}{\m\,\s^{-2}}

% random stuff
\sloppy\sloppypar\raggedbottom\frenchspacing

\pagestyle{plain}

\begin{document}

\noindent
Name: \rule[-1ex]{0.60\textwidth}{0.1pt}
NetID: \rule[-1ex]{0.20\textwidth}{0.1pt}

\section*{\textsl{Einstein's Universe} Final Exam}
\setcounter{problem}{1}

\begin{problem}
(From Problem Set 1)
How many molecules of air are there in a cubic centimeter?
You will have to scale up your answer on the problem set, which
was for a cubic millimeter.
\end{problem}

\vfill ~

\begin{problem}
(From the reading)
Classical mechanics, or Newtonian mechanics, is only valid in certain
circumstances. When do the laws of classical mechanics, like $F =
m\,a$ for example, become wrong or break down? There are many answers
to this problem; I will take anything correct.
\end{problem}

\vfill ~

\begin{problem}
(From Problem Set 1)
A Joule is a unit of energy. What are the SI base units for 1\,J? That is,
express 1\,J in terms of kg, s, and m.
\end{problem}

\vfill ~

\begin{problem}
(From Problem Set 1)
What is the approximate thickness of a stack of 10,000 100-dollar bills?
That is, a million dollars!
No need to be precise, and use any units you like.
\end{problem}

\vfill ~

\clearpage

\begin{problem}
(From the Kinematics Lab)
Here is a data table of times, positions, and velocities in SI units:\\
\rule{1.0in}{0pt}\begin{tabular}{c|c|c}
time $t$ ($\s$) & position $x$ ($\m$) & velocity $v$ ($\m\,\s^{-1}$) \\
\hline
1.1 & 1.15 & 1.0 \\
2.3 & 1.51 & 1.6 \\
3.7 & 2.00 & 2.3 \\
\hline
\end{tabular}\\
What is the average acceleration in the time interval from $2.3\,\s$ to $3.7\,\s$?
\end{problem}

\vfill ~

\begin{problem}
(From Problem Set 2)
Give a combination of a mass $M$, a length $L$, and a tension $T$ that will have
units of \emph{time}.
\end{problem}

\vfill ~

\begin{problem}
(From the reading)
What, roughly, is the principle of relativity? State it in 30 words or less!
\end{problem}

\vfill ~

\begin{problem}
(From Problem Set 2)
If the work function (binding energy) for an electron in some metal is 2\,eV, can
a photon of wavelength $0.5\,\mu\m$ unbind an electron from that metal?
\end{problem}

\vfill ~

\cleardoublepage

\begin{problem}
(From Lecture on 2019-09-26)
This wave on a string is moving to the left. The string is moving only
up and down. Draw arrows at points A, B, and C showing which way those
bits of string are moving.\\
\includegraphics{wavepulse.png}
\end{problem}

\vfill ~

\begin{problem}
(From the Newton's Second Law Lab)
A $100\,\g$ mass will have roughly what weight in $N$? Roughly!
\end{problem}

\vfill ~

\begin{problem}
(From the reading)
Muons live for a couple of milliseconds. Naively, they can't travel more than few hundred meters,
even traveling near the speed of light. And yet, they often are observed to travel many kilometers.
How is this possible?
\end{problem}

\vfill ~

\begin{problem}
(From Problem Set 3, problem 2)
If something travels at 0.99\,c, what is the corresponding Lorentz factor?
\end{problem}

\vfill ~

\clearpage

\begin{problem}
(From the Photoelectric Effect Lab)
Which photons have more energy? Ultraviolet or Green?
\end{problem}

\vfill ~

\begin{problem}
(From the reading)
Why did Einstein want to have astronomers precisely observe a Solar eclipse?
What was he hoping they would see?
\end{problem}

\vfill ~

\begin{problem}
(From Problem Set 4, problem 3)
Which produces more energy per unit mass of fuel: Nuclear fusion reactions or chemical reactions?
\end{problem}

\vfill ~

\begin{problem}
(From the reading)
If you are in a rocket that is moving in the $x$-direction with
respect to the Earth at $0.75\,c$ and, inside the rocket, you are
moving at $0.75\,c$ in the $x$ direction with respect to the rocket,
how fast are you moving with respect to the Earth? No need to
calculate. All I want to know is: Are you moving closer to $0.75\,c$,
$0.95\,c$, or $1.5\,c$?
\end{problem}

\vfill ~

\cleardoublepage

\begin{problem}
(From Lecture, 2019-10-31)
How does the vomit comet make everyone inside experience weightlessness?
What are the pilots doing to make that happen?
\end{problem}

\vfill ~

\begin{problem}
(From Problem Set 5)
How long does it take light to travel a distance corresponding to the altitude of the mountain K2?
\end{problem}

\vfill ~

\begin{problem}
(From Problem Set 5)
What is the Doppler factor corresponding to a velocity of $2\times 10^8\,\mps$?
\end{problem}

\vfill ~

\begin{problem}
(From Problem Set 5)
What is the insolation on Jupiter in watts per square meter?
\end{problem}

\vfill ~

\clearpage

\begin{problem}
(From Lecture)
What is the age of the Universe?
\end{problem}

\vfill ~

\begin{problem}
(From Problem Set 6)
The star S2 orbits the black hole at the center of our Galaxy.
How fast is this star moving, on average or roughly?
Don't forget your units, here or elsewhere on the exam!
\end{problem}

\vfill ~

\begin{problem}
(From Lecture)
What is the relationship between an acceleration $a$, a change in velocity $\Delta v$ and a small time interval $\Delta t$? Give an equation.
\end{problem}

\vfill ~

\begin{problem}
(From Problem Set 6)
Why did the black hole merger discovered by \textsl{LIGO} produce a final black hole that is less massive than the sum of the masses of the two black holes that merged to make it? Or, where did the mass go?
\end{problem}

\vfill ~

\clearpage

\begin{problem}
State one true thing about the world that you learned in this class,
which you expect to remember for many years.
\end{problem}

\vfill ~

\begin{problem}
Einstein brought two changes to physics: Relativity (special and general) and
quantization of energy and mass. Which do you think is more important, and why?
\end{problem}

\vfill ~

~

\vfill ~

~

\vfill ~

\end{document}

