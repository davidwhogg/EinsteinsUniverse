\documentclass[12pt, letterpaper]{article}
\usepackage{url, graphicx, epstopdf, amsmath}

% page layout
\setlength{\topmargin}{-0.25in}
\setlength{\textheight}{9.5in}
\setlength{\headheight}{0in}
\setlength{\headsep}{0in}
\setlength{\parindent}{1.1\baselineskip}

\newcommand{\examheader}[1]{
\noindent
Name:\rule[-1ex]{0.40\textwidth}{0.1pt}
NetID:\rule[-1ex]{0.13\textwidth}{0.1pt}
Lab time:\rule[-1ex]{0.13\textwidth}{0.1pt}

\section*{\textsl{Einstein's Universe} {#1}}
\setcounter{problem}{1}}


% problem formatting
\newcommand{\problemname}{Problem}
\newcounter{problem}\setcounter{problem}{1}
\newenvironment{problem}{%
  \addvspace{\baselineskip}\noindent\textbf{Problem~\theproblem:}\refstepcounter{problem}
}{%
  \par\addvspace{\baselineskip}
}

% words
\newcommand{\foreign}[1]{\textsl{#1}}
\newcommand{\vs}{\foreign{vs}}

% math
\newcommand{\dd}{\mathrm{d}}
\newcommand{\e}{\mathrm{e}}

% primary or base units
\newcommand{\rad}{\mathrm{rad}}
\newcommand{\kg}{\mathrm{kg}}
\newcommand{\m}{\mathrm{m}}
\newcommand{\s}{\mathrm{s}}

% secondary units
\renewcommand{\deg}{\mathrm{deg}}
\newcommand{\g}{\mathrm{g}}
\newcommand{\km}{\mathrm{km}}
\newcommand{\cm}{\mathrm{cm}}
\newcommand{\mm}{\mathrm{mm}}
\newcommand{\mum}{\mathrm{\mu m}}
\newcommand{\ft}{\mathrm{ft}}
\newcommand{\mi}{\mathrm{mi}}
\newcommand{\AU}{\mathrm{AU}}
\newcommand{\Mpc}{\mathrm{Mpc}}
\newcommand{\ns}{\mathrm{ns}}
\newcommand{\h}{\mathrm{h}}
\newcommand{\yr}{\mathrm{yr}}
\newcommand{\N}{\mathrm{N}}
\newcommand{\J}{\mathrm{J}}
\newcommand{\eV}{\mathrm{eV}}
\newcommand{\W}{\mathrm{W}}
\newcommand{\Pa}{\mathrm{Pa}}

% derived units
\newcommand{\mps}{\m\,\s^{-1}}
\newcommand{\mph}{\mi\,\h^{-1}}
\newcommand{\mpss}{\m\,\s^{-2}}

% random stuff
\sloppy\sloppypar\raggedbottom\frenchspacing

\pagestyle{plain}

\begin{document}

\noindent
Name: \rule[-1ex]{0.60\textwidth}{0.1pt}
NetID: \rule[-1ex]{0.20\textwidth}{0.1pt}

\section*{\textsl{Einstein's Universe} Final Exam}
\setcounter{problem}{1}

\begin{problem}
  (From Problem Set 1)
  What, approximately, is the cargo volume of an armored truck?
\end{problem}

\vfill ~

\begin{problem}
  (From Problem Set 1)
  What is the mass of one mole of water?
\end{problem}

\vfill ~

\begin{problem}
  (From Term Exam 1)
  Classical mechanics, or Newtonian mechanics, is only valid in certain
circumstances. When do the laws of classical mechanics, like $\vec{F} =
m\,\vec{a}$ for example, become wrong or break down? There are many answers
to this problem; I will take anything correct.
\end{problem}

\vfill ~

\begin{problem}
  (From Term Exam 1)
  Which of the following physical quantities are vectors?
\\
\textsl{(a)}~energy,
\textsl{(b)}~mass,
\textsl{(c)}~force,
\textsl{(d)}~momentum,
\textsl{(e)}~acceleration.
\end{problem}

\vfill ~

\clearpage

\begin{problem}
(From Problem Set 2)
\end{problem}

\vfill ~

\begin{problem}
(From Problem Set 2)
\end{problem}

\vfill ~

\begin{problem}
(From Term Exam 2)
\end{problem}

\vfill ~

\begin{problem}
(From Term Exam 2)
\end{problem}

\vfill ~

\clearpage

\begin{problem}
(From Problem Set 3)
\end{problem}

\vfill ~

\begin{problem}
(From Problem Set 3)
\end{problem}

\vfill ~

\begin{problem}
(From Term Exam 3)
\end{problem}

\vfill ~

\begin{problem}
(From Term Exam 3)
\end{problem}

\vfill ~

\clearpage

\begin{problem}
(From Problem Set 4)
\end{problem}

\vfill ~

\begin{problem}
(From Problem Set 4)
\end{problem}

\vfill ~

\begin{problem}
(From Term Exam 4)
\end{problem}

\vfill ~

\begin{problem}
(From Term Exam 4)
\end{problem}

\vfill ~

\clearpage

\begin{problem}
(From Problem Set 5)
\end{problem}

\vfill ~

\begin{problem}
(From Problem Set 5)
\end{problem}

\vfill ~

\begin{problem}
(From Term Exam 5)
\end{problem}

\vfill ~

\begin{problem}
(From Term Exam 5)
\end{problem}

\vfill ~

\clearpage

\begin{problem}
  (From Problem Set 6)
  What is the mass of the black hole at the center of the Milky Way?
\end{problem}

\vfill ~

\begin{problem}
  (From Problem Set 6)
  How much energy was radiated away in gravitational radiation in event GW150914?
  You can give your answer in energy units, or mass units (since, after all, there
  is mass--energy equivalence).
\end{problem}

\vfill ~

\begin{problem}
(From Term Exam 6)
\end{problem}

\vfill ~

\begin{problem}
(From Term Exam 6)
\end{problem}

\vfill ~

\clearpage

\begin{problem}
State one true (?) thing that you learned in this class,
which you expect to remember for many years.
\end{problem}

\vfill ~

\begin{problem}
Einstein brought two changes to physics: relativity (special and general) and
quantization of energy and mass. Which do you think is more important, and why?
\end{problem}

\vfill ~

~

\vfill ~

~

\vfill ~

\end{document}

