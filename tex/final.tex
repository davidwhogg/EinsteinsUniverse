\documentclass[12pt, letterpaper]{article}
\usepackage{url, graphicx, epstopdf, amsmath}

% page layout
\setlength{\topmargin}{-0.25in}
\setlength{\textheight}{9.5in}
\setlength{\headheight}{0in}
\setlength{\headsep}{0in}
\setlength{\parindent}{1.1\baselineskip}

\newcommand{\examheader}[1]{
\noindent
Name:\rule[-1ex]{0.40\textwidth}{0.1pt}
NetID:\rule[-1ex]{0.13\textwidth}{0.1pt}
Lab time:\rule[-1ex]{0.13\textwidth}{0.1pt}

\section*{\textsl{Einstein's Universe} {#1}}
\setcounter{problem}{1}}


% problem formatting
\newcommand{\problemname}{Problem}
\newcounter{problem}\setcounter{problem}{1}
\newenvironment{problem}{%
  \addvspace{\baselineskip}\noindent\textbf{Problem~\theproblem:}\refstepcounter{problem}
}{%
  \par\addvspace{\baselineskip}
}

% words
\newcommand{\foreign}[1]{\textsl{#1}}
\newcommand{\vs}{\foreign{vs}}

% math
\newcommand{\dd}{\mathrm{d}}
\newcommand{\e}{\mathrm{e}}

% primary or base units
\newcommand{\rad}{\mathrm{rad}}
\newcommand{\kg}{\mathrm{kg}}
\newcommand{\m}{\mathrm{m}}
\newcommand{\s}{\mathrm{s}}

% secondary units
\renewcommand{\deg}{\mathrm{deg}}
\newcommand{\g}{\mathrm{g}}
\newcommand{\km}{\mathrm{km}}
\newcommand{\cm}{\mathrm{cm}}
\newcommand{\mm}{\mathrm{mm}}
\newcommand{\mum}{\mathrm{\mu m}}
\newcommand{\ft}{\mathrm{ft}}
\newcommand{\mi}{\mathrm{mi}}
\newcommand{\AU}{\mathrm{AU}}
\newcommand{\Mpc}{\mathrm{Mpc}}
\newcommand{\ns}{\mathrm{ns}}
\newcommand{\h}{\mathrm{h}}
\newcommand{\yr}{\mathrm{yr}}
\newcommand{\N}{\mathrm{N}}
\newcommand{\J}{\mathrm{J}}
\newcommand{\eV}{\mathrm{eV}}
\newcommand{\W}{\mathrm{W}}
\newcommand{\Pa}{\mathrm{Pa}}

% derived units
\newcommand{\mps}{\m\,\s^{-1}}
\newcommand{\mph}{\mi\,\h^{-1}}
\newcommand{\mpss}{\m\,\s^{-2}}

% random stuff
\sloppy\sloppypar\raggedbottom\frenchspacing

\pagestyle{plain}

\begin{document}

\noindent
Name: \rule[-1ex]{0.60\textwidth}{0.1pt}
NetID: \rule[-1ex]{0.20\textwidth}{0.1pt}

\section*{\textsl{Einstein's Universe} Final Exam}
\setcounter{problem}{1}

\begin{problem}
  (From Problem Set 1)
  What, approximately, is the cargo volume of an armored truck?
\end{problem}

\vfill ~

\begin{problem}
  (From Problem Set 1)
  What is the mass of one mole of water?
\end{problem}

\vfill ~

\begin{problem}
  (From Term Exam 1)
  Classical mechanics, or Newtonian mechanics, is only valid in certain
circumstances. When do the laws of classical mechanics, like $\vec{F} =
m\,\vec{a}$ for example, become wrong or break down? There are many answers
to this problem; I will take anything correct.
\end{problem}

\vfill ~

\begin{problem}
  (From Term Exam 1)
  Which of the following physical quantities are vectors?
\\
\textsl{(a)}~energy,
\textsl{(b)}~mass,
\textsl{(c)}~force,
\textsl{(d)}~momentum,
\textsl{(e)}~acceleration.
\end{problem}

\vfill ~

\clearpage

\begin{problem}
  (From Problem Set 2)
  What is the mass $M$ you found for a piano string?
\end{problem}

\vfill ~

\begin{problem}
  (From Problem Set 2)
  What, approximately, is the volume of an ice molecule?
\end{problem}

\vfill ~

\begin{problem}
  (From Term Exam 2)
  This wave on a string is moving to the left. The string is moving only
up and down. Draw arrows at points A, B, and C showing which way those
bits of string are moving.\\
\includegraphics{wavepulse.png}
\end{problem}

\vfill ~

\begin{problem}
  (From Term Exam 2)
  A $50\,\g$ mass will have roughly what weight in $N$? Roughly!
\end{problem}

\vfill ~

\clearpage

\begin{problem}
  (From Problem Set 3)
  What is the Lorentz factor $\gamma$ for something moving relative
  to you at speed $v = 0.99\,c$?
\end{problem}

\vfill ~

\begin{problem}
  (From Problem Set 3)
  What is the distance from us to the Galactic Center?
\end{problem}

\vfill ~

\begin{problem}
  (From Term Exam 3)
  Reproduce here the triangular path of light you drew for the moving-light-clock problem (Problem Set 3, problem 3), and label the lengths
of the sides of the triangle.
\end{problem}

\vfill ~

\begin{problem}
  (From Term Exam 3)
  Muons live for a couple of milliseconds. Naively, therefore, they can't travel more than few hundred meters,
even traveling near the speed of light. And yet, they often are observed to travel many kilometers.
How is this possible?
\end{problem}

\vfill ~

\clearpage

\begin{problem}
  (From Problem Set 4)
  Which produces more energy per unit mass?
  Burning fossil fuels,
  nuclear fission with uranium fuel, or
  nuclear fusion with hydrogen fuel?
\end{problem}

\vfill ~

\begin{problem}
  (From Problem Set 4)
  An elevator in New York City is accelerating downwards
  at acceleration $g$ (the gravitational acceleration).
  What is the magnitude of the normal force on a box of mass $M$ inside this elevator?
\end{problem}

\vfill ~

\begin{problem}
  (From Term Exam 4)
  In normal space, the straight line (or geodesic) is the path of \emph{shortest
total distance} between two points. In spacetime, the geodesic is the path
of what?
\end{problem}

\vfill ~

\begin{problem}
  (From Term Exam 4)
  If you are in a rocket that is moving in the $x$-direction with
respect to the Earth at $0.75\,c$ and, inside the rocket, you are
moving at $0.75\,c$ in the $x$ direction with respect to the rocket,
how fast are you moving with respect to the Earth? No need to
calculate. All I want to know is: Are you moving closer to $0.75\,c$,
$0.95\,c$, or $1.5\,c$?
\end{problem}

\vfill ~

\clearpage

\begin{problem}
  (From Problem Set 5)
  What, roughly, is the insolation on the surface of Pluto?
\end{problem}

\vfill ~

\begin{problem}
  (From Problem Set 5)
  What is the recession speed (relative to us) of a galaxy that has a Doppler
  factor of 10?
\end{problem}

\vfill ~

\begin{problem}
  (From Term Exam 5)
  If a galaxy is moving away at 5 percent of the speed of light, the K
line (which is at a rest-frame or natural wavelength of around $\lambda =
3900$\,\AA), will be shifted to the red. What, roughly, will be the change
$\Delta\lambda$ in the wavelength of the line?
\end{problem}

\vfill ~

\begin{problem}
  (From Term Exam 5)
  What was the policy introduced at the University in Berlin in 1933 that caused Einstein's job to be terminated?
\end{problem}

\vfill ~

\clearpage

\begin{problem}
  (From Problem Set 6)
  What is the mass of the black hole at the center of the Milky Way?
\end{problem}

\vfill ~

\begin{problem}
  (From Problem Set 6)
  How much energy was radiated away in gravitational radiation in event GW150914?
  You can give your answer in energy units, or mass units (since, after all, there
  is mass--energy equivalence).
\end{problem}

\vfill ~

\begin{problem}
  (From Term Exam 6)
  HOGG
\end{problem}

\vfill ~

\begin{problem}
  (From Term Exam 6)
  HOGG
\end{problem}

\vfill ~

\clearpage

\begin{problem}
State one true (?) thing that you learned in this class,
which you expect to remember for many years.
\end{problem}

\vfill ~

\begin{problem}
Einstein brought two changes to physics: relativity (special and general) and
quantization of energy and mass. Which do you think is more important, and why?
\end{problem}

\vfill ~

~

\vfill ~

~

\vfill ~

\end{document}

