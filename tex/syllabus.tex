\documentclass[12pt]{article}
\usepackage{fancyhdr}
\usepackage{afterpage}

\addtolength{\textheight}{2.10in}
\addtolength{\topmargin}{-0.90in}
\setlength{\headsep}{\baselineskip}
\setlength{\footskip}{\baselineskip}

\renewcommand{\paragraph}[1]{\par\smallskip\noindent\textbf{#1}}

\pagestyle{fancyplain}
\fancyhf{}
\fancyfoot[R]{\sffamily{\thepage}}
\renewcommand{\headrule}{}
\renewcommand{\footrule}{}
\renewcommand{\sectionmark}{}

\begin{document}\sloppy\sloppypar\raggedbottom\frenchspacing

\section*{Einstein's Universe}
\afterpage{\fancyhead[R]{\fancyplain{}\sffamily{Prof David W. Hogg}}}
\afterpage{\fancyhead[L]{\fancyplain{}\sffamily{\textsl{Einstein's Universe}, 2019 Fall}}}
NYU CORE-UA 0204 --- This syllabus is for the 2019 Fall semester.

\bigskip
\noindent
\begin{tabular}{llll}
                 &                             & \textsl{office} & \textsl{office hours} \\
\hline
\textsl{lecture} & \textbf{Prof David W. Hogg} & 726 Broadway & Tu 09:00--11:00 \\
                 & david.hogg@nyu.edu          & rm 1050      & Th 09:00--11:00 \\
\hline
\textsl{admin}   & Aisling Dunne               & Silver       \\
                 & ??                          & rm ??        \\
\hline
\end{tabular}

\subsection*{Aims}

This course is intended to introduce you to interesting and important
concepts in contemporary physics.
It is also designed to provide you with useful quantitative reasoning
and conceptual reasoning skills that can be useful in many contexts.
In addition, you will learn a little bit about the history of science,
and the development of physical ideas.

This course is organized around breakthroughs that were made or
facilitated by the work of Albert Einstein.
As we will learn, Einstein did not do everything that is attributed to
him!
But he was at the center of some of the very important discoveries and
revolutions in physics, so he provides an excellent focus for a course
like this.

\subsection*{Reading, Lectures, and Problems}

There will be assigned reading every week.

\subsection*{Labs}

\subsection*{Exams}

There will be 6 small, low-stakes Term Examinations during the term
and one Final Exam. The Exams happen in the last 30 minutes of class
time, on dates given in the schedule below. The Exams will take place
in the lecture room. The scope of each Exam will be made clear in
lecture, but in brief, the Term Exams will concentrate on the material
in the previous few weeks.

\textbf{Why so many Exams?} Research shows that students learn a full
semester of material better when they are presented with frequent,
low-stakes tests. The Exams are low-stakes in that each is only worth
a small fraction of your final grade, and when they are combined for
grading, we will drop the lowest two scores among them. Each Exam
gives you an opportunity to recall the material of the course, and
also learn where you need to do more work.

The Exams will be open notes. Any written or printed documents are
permitted in the Exam room. On the other hand, electronic devices that
can connect to a mobile-phone network or internet are
forbidden. Furthermore, you do not need a calculator, so no electronic
devices will be permitted at all.

\textbf{Missed Exams will be graded zero unless there is a properly
  documented medical excuse or a religious conflict}. If there is a
properly documented medical excuse or an announced-in-advance (to Prof
Hogg) religious conflict for a missed Term Exam, it will be pro-rated
out of the total score. \textbf{No special arrangements will be made
  and no excuses will be granted for travel conflicts}, no matter
what. If you have a non-medical emergency or non-religious conflict
that prevents you from making an Exam, you will have to speak with a
Dean of your College, not with the staff of this course.

(If you are re-taking part of this course, or have any unusual
circumstance, you must contact Prof Hogg at the beginning of the
semester, before the first Exam, to notify him of this situation.)

If you arrive late for any Exam, you will not be given extra time. If
you fail to obey any of the instructions given to you by course staff
before, during, or after any Exam, your Exam may be graded zero or you
may be subject to academic honesty proceedings.

\subsection*{Grading}

Grades will be based on a total score generated with these percentages:

\smallskip\noindent
\begin{tabular}{lr}
\hline
laboratories                  &  30 \\
best four Term Exams combined &  40 \\
Final Exam                    &  30 \\
\hline
total                         & 100 \\
\hline
\end{tabular}

\smallskip\noindent
Note that the combination of all Term Exams will involve dropping
(ignoring) your lowest two Exam scores.

Grades will be assigned in one-to-one correspondence with the total
score according to the following percentage ranges:

\smallskip\noindent
\begin{tabular}{lcccccccc}
\hline
total score greater than: & 85 & 80   & 75   & 66 & 61   & 56   & 45 & 30 \\
final grade at least:     & A  & A$-$ & B$+$ & B  & B$-$ & C$+$ & C  & D  \\
\hline
\end{tabular}

\medskip
\textbf{Why these absolute grading policies?} Relative grading
policies (in which, say, x percent of the class gets an A) make the
student-student interactions essentially competitive. Prof Hogg wants
the students in this course to interact cooperatively, not
competitively. If you work with your fellow students, and everyone
does better on the Problem Sets and Exams, everyone's grade prospects
improve.

\subsection*{Help}

All of the teaching staff on this course have office hours (listed
above), and if you can't make their office hours, you should feel free
to contact them about the material of the course by email. You get
much more out of office hours or even email contact if you have a
specific question ready in advance.

Importantly, your best learning resource is your fellow students. Form
a study group (ideally with students of comparable ability) and work
together on the lecture material, on relevant reading, and on the
labs. Choose a regular time and meet. Multiple lines of research show
that students who make use of peer support learn better and perform
better on the Exams. They also have more fun.

\subsection*{Rules and Comments}

\paragraph{respect for diversity:}
It is Prof Hogg's intent that students from all backgrounds and
perspectives be served well by this course, that students' learning
needs be addressed, and that the diversity that students bring to this
class be viewed as a resource, strength and benefit. Your suggestions
are encouraged and appreciated. Please let Prof Hogg or the other
staff of the class know if any issues arise or how we can improve our
approaches, for you personally or for other students or for students
from minoritized groups more broadly.

\paragraph{audio recordings:}
While you are not forbidden from making audio recordings during class,
you must not post, publish, or share them with others, not even in
small sound bites. This is because the classroom setting is a private
setting in which everyone should feel free to speak plainly and
without regrets. Failure to obey this rule will be considered an act
of academic dishonesty.

\paragraph{disabilities:}
If you have an arrangement with the Center for Students with
Disabilities, you must present the relevant forms to Prof Hogg one
week in advance of each of the Exams.

\paragraph{academic honesty:}
The lightest consequence for academic dishonesty that Prof Hogg
considers consistent with University and Departmental policy is a
reduced grade (by one full letter) in the course and a recommended
disciplinary action by the College. Academic dishonesty includes (in
addition to the usual kinds of cheating) misrepresenting matters of
material importance to the instructors.

\paragraph{staying current:}
It is every student's individual responsibility to stay up-to-date
with the syllabus and any emails sent by the staff regarding the
course. Having a broken email address, having an overfull inbox, or
not being registered properly in Albert or NYU Classes will not be
accepted as excuses for missing things or not knowing about events or
assignments.
Related to that, 
you must have read and understood the content of this syllabus. If
there are things here you don't understand, you must ask Prof Hogg
about them.

\paragraph{feedback:}
Please ask questions during lectures and recitations. If there is
something you don't understand, many other students are having the
same trouble, guaranteed. If there is some aspect of the pace,
content, or structure of the course you don't like, or any other
feedback you would like to give, please let Prof Hogg know as soon as
possible. If you wait until course evaluation forms are handed out at
the end of the semester, you will have benefited next year's class at
the expense of your own!

\paragraph{legalese:}
We apologize for the legal tone of this section of the syllabus. The
subject of physics is great fun; operating a sizeable course can be
exasperating. All of the staff of this course will do everything we
can to make this course interesting and enjoyable for
everyone. Physics isn't just fun for Prof Hogg; it is his profession
and his calling.

\subsection*{Calendar}
\noindent
\begin{tabular}{lll}
\textbf{week} & \textbf{subjects} & \textbf{assignments} \\
\hline
foo & bar & whatever \\
\hline
\end{tabular}

\end{document}
