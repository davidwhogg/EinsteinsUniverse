\documentclass[12pt]{article}
\usepackage{fancyhdr}
\usepackage{afterpage}
\usepackage{url}

\addtolength{\textheight}{2.05in}
\addtolength{\topmargin}{-0.90in}
\setlength{\headsep}{\baselineskip}
\setlength{\footskip}{1.5\baselineskip}

\renewcommand{\paragraph}[1]{\par\smallskip\noindent\textbf{#1}}

\pagestyle{fancyplain}
\fancyhf{}
\fancyfoot[R]{\sffamily{\thepage}}
\renewcommand{\headrule}{}
\renewcommand{\footrule}{}
\renewcommand{\sectionmark}{}

\begin{document}\sloppy\sloppypar\raggedbottom\frenchspacing

\section*{Einstein's Universe}
\afterpage{\fancyhead[R]{\fancyplain{}\sffamily{Prof David W. Hogg}}}
\afterpage{\fancyhead[L]{\fancyplain{}\sffamily{\textsl{Einstein's Universe}, 2019 Fall}}}
NYU CORE-UA 0204 --- This syllabus is for the 2019 Fall semester.

\bigskip
\noindent
\textbf{David W. Hogg}, Professor of Physics and Data Science \\
david.hogg@nyu.edu \hfill \url{http://cosmo.nyu.edu/hogg/} \\
726 Broadway, room 1050 \hfill \textsl{office hours:} Tu, Th 9--11

\bigskip
\noindent
Albert Einstein was involved in four important breakthroughs in
physics.
Two of these are special relativity and general relativity, which
changed how we think about space and time, and brought an
understanding of the equivalence of matter and energy.
The other two breakthroughs are the photoelectric effect and brownian
motion, which helped to establish that both matter and energy are quantized,
not continuous.
Einstein's contributions were truly theoretical, in that his work
unified or explained phenomena that were well established by the
experimental physicists of his time, but which were very hard to
understand in the classical, common-sense physical picture that
prevailed.
Einstein did not participate much in the tremendous scientific
revolutions that followed from his work, but we will learn about
them in this class:
We will learn about how time and space are different for different
observers, and how they are affected by gravity.
We will learn about how we know that the world is made up of atoms,
and how matter and energy are quantized and obey the strange rules of
quantum mechanics.
We will learn about how the theory of gravity directly predicts both
the existence of black holes and the expansion of the Universe, and
how both of these things are now extremely well established.

\subsection*{Aims}

This course is intended to:
\begin{itemize}
\item
introduce you to interesting and important concepts in contemporary
physics, and
\item
provide you with some quantitative reasoning and conceptual
reasoning skills that can be useful in many contexts, and
\item
teach you a little bit about the history of science,
and the development of physical ideas.
\end{itemize}

\subsection*{Reading, Lectures, and Problems}

There are two Lectures every week, and there
will be assigned readings every week.
The readings will come from the book \textit{Einstein} by Isaacson,
the book \textit{Relativity Demystified} by Wolfson,
and various articles on Wikipedia.
Read these carefully, but also critically.
You will be expected (on the Exams) to be familiar with the content of
the readings and the Lectures.

In addition, there will be problem sets each week, which you are
expected to do.
These problem sets will not be handed in, nor will they be graded.
But some of the problems, or parts of the problems, will appear
(verbatim) on the Exams.
Since the Exams will be open notes, it behooves you not just to do
these problems, but to bring your solutions with you on Exam days.

You are encouraged to work together with your fellow students on the
reading and on the problem sets.
You will do better in the class (and have more fun) if you work in a
team.
But since you will be examined individually, make sure you personally
understand the material, even if you work it out in a group.

At no point will we give out solutions to the problems. \textbf{Why no
  solutions?} There is research that shows that students learn more if
they do not have a reference solution to a problem they are
facing. This has various consequences. One is that if you are doing
problems with friends, and one has a solution, do not rely on this
solution as a reference solution. For one, it might be wrong. And for
two, if you treat it as correct, you won't learn as much. Another
consequence is that if you don't understand how to do a problem, or if
you are not sure of your solution, you should bring it to your Lab
Instructor or Prof Hogg for discussion.

\subsection*{Labs}

There is an experimental Lab session each week. These are a required
part of the course and make up a significant part of your total grade.
In addition, material from the Labs will appear on the Term Exams.
Your Lab Instructor is a physicst and teacher, who can help you.

Each Lab Instructor will hold a weekly office hour where you can
discuss readings, Lectures, problem sets, and Labs.
Office locations and office hour time and day will be announced by
your Lab Instructor during the first Lab session.

You must be
registered for one Lab section. You will have to submit a Lab Report
for each experiment performed. The Lab Report has to include answers
to all questions and any data you may have collected. The Lab Report
will be due in Lab one week after the experiment has been performed.

Your Lab grade will be based on the Lab experiment and Report
for each of the sessions.  At the end of the semester, your lowest Lab
grade will be dropped. This can be applied to an unexcused absence.

\textbf{Missed Labs will be graded zero unless there is a properly
  documented medical excuse or a religious conflict}. You must notify
your Lab instructor in advance in writing if you will miss a Lab for
religious reasons. All other absences will be considered unexcused and
will result in a Lab grade of zero. \textbf{You cannot make up a Lab by
attending a Lab session for which you are not registered.}
In accordance with NYU CAS Core policy, if you have more than two
unexcused Lab absences during the semester, you will not be given a
passing grade in the class.

If you arrive 10 minutes late (or later) for the Lab session you will
lose the participation credit for that Lab session.

Late Lab Reports will be penalized five points for each day late
(excluding weekend days). If you wish to submit a late Lab Report you
must do so by making an arrangement with your Lab Instructor.

\subsection*{Exams}

There will be 6 small, low-stakes Term Examinations during the term
and one Final Exam. The Exams happen in the last 30 minutes of class
time, on dates given in the schedule below. The Exams will take place
in the Lecture room. The scope of each Exam will be made clear in
Lecture, but in brief, the Term Exams will concentrate on the material
in the previous few weeks.

\textbf{Why so many Exams?} Research shows that students learn a full
semester of material better when they are presented with frequent,
low-stakes tests. The Exams are low-stakes in that each is only worth
a small fraction of your final grade, and when they are combined for
grading, we will drop the lowest two scores among them. Each Exam
gives you an opportunity to recall the material of the course, and
also learn where you need to do more work.

The Exams will be open notes. Any written or printed documents are
permitted in the Exam room. On the other hand, electronic devices that
can connect to a mobile-phone network or internet are
forbidden. Furthermore, you do not need a calculator, so no electronic
devices will be permitted at all.

\textbf{Missed Exams will be graded zero unless there is a properly
  documented medical excuse or a religious conflict}. If there is a
properly documented medical excuse or an announced-in-advance (to Prof
Hogg) religious conflict for a missed Term Exam, it will be pro-rated
out of the total score. \textbf{No special arrangements will be made
  and no excuses will be granted for travel conflicts}, no matter
what. If you have a non-medical emergency or non-religious conflict
that prevents you from making an Exam, you will have to speak with your
advisor or a Dean of the College, who can ask us to make an exception.

If you arrive late for any Exam, you will not be given extra time. If
you fail to obey any of the instructions given to you by course staff
before, during, or after any Exam, your Exam may be graded zero or you
may be subject to academic honesty proceedings.

\subsection*{Grading}

Grades will be based on a total score generated with these percentages:

\bigskip\noindent
\begin{tabular}{lr}
\hline
Labs (including participation) &  30 \\
best five Term Exams combined  &  40 \\
Final Exam                     &  30 \\
\hline
total                          & 100 \\
\hline
\end{tabular}

\bigskip\noindent
Note that the combination of all Term Exams will involve dropping
(ignoring) your lowest Exam score.

Provided that you have no more than two unexcused Lab absences (see
above), grades will be assigned in one-to-one correspondence with the
total score according to the following percentage ranges:

\bigskip\noindent
\begin{tabular}{lcccccccc}
\hline
total score greater than: & 85 & 80   & 75   & 66 & 61   & 56   & 45 & 30 \\
final grade at least:     & A  & A$-$ & B$+$ & B  & B$-$ & C$+$ & C  & D  \\
\hline
\end{tabular}

\textbf{Why these absolute grading policies?} Relative grading
policies (in which, say, x percent of the class gets an A) make the
student-student interactions essentially competitive. Prof Hogg wants
the students in this course to interact cooperatively, not
competitively. If you work with your fellow students, and everyone
does better on the Exams, everyone's grade improves.

\subsection*{Help}

All of the staff involved in this course have office hours.
If you can't make their office hours, you should feel free
to contact them about the material of the course by email. You get
much more out of office hours or even email contact if you have a
specific question ready in advance.

Importantly, your best learning resource is your fellow students. Form
a study group (ideally with students of comparable ability) and work
together on the Lecture material, on relevant reading, and on the
Labs. Choose a regular time and meet. Multiple lines of research show
that students who make use of peer support learn better and perform
better on the Exams. They also have more fun.

\subsection*{Rules and Comments}

\paragraph{inclusivity:}
It is Prof Hogg's intent that students from all backgrounds and
perspectives be served well by this course, that students' learning
needs be addressed, and that the diversity that students bring to this
class be viewed as a resource, strength and benefit. Your suggestions
are encouraged and appreciated. Please let Prof Hogg or the other
staff of the class know if any issues arise or how we can improve our
approaches, for you personally or for other students or for students
from minoritized groups more broadly.

\paragraph{audio recordings:}
While you are not forbidden from making audio recordings during class,
you must not post, publish, or share them with others, not even in
small sound bites. This is because the classroom setting is a private
setting in which everyone should feel free to speak plainly and
without regrets. Failure to obey this rule will be considered an act
of academic dishonesty.

\paragraph{disabilities:}
If you have an arrangement with the Center for Students with
Disabilities, you must present the relevant forms to Prof Hogg one
week in advance of each of the Exams.

\paragraph{academic honesty:}
The lightest consequence for academic dishonesty that Prof Hogg
considers consistent with University and Departmental policy is a
reduced grade (by one full letter) in the course and a recommended
disciplinary action by the College. Academic dishonesty includes (in
addition to the usual kinds of cheating) misrepresenting matters of
material importance to the instructors.

\paragraph{staying current:}
It is every student's individual responsibility to stay up-to-date
with the syllabus and any emails sent by the staff regarding the
course. Having a broken email address, having an overfull inbox, or
not being registered properly in Albert or NYU Classes will not be
accepted as excuses for missing things or not knowing about events or
assignments.
Related to that, 
you must have read and understood the content of this syllabus. If
there are things here you don't understand, you must ask Prof Hogg
about them.

\paragraph{feedback:}
Please ask questions during Lectures and Labs. If there is
something you don't understand, many other students are having the
same trouble, guaranteed. If there is some aspect of the pace,
content, or structure of the course you don't like, or any other
feedback you would like to give, please let Prof Hogg know as soon as
possible. If you wait until course evaluation forms are handed out at
the end of the semester, you will have benefited next year's class at
the expense of your own!

\paragraph{legalese:}
We apologize for the legal tone of this section of the syllabus. The
subject of physics is great fun; operating a sizeable course can be
challenging. All of the staff of this course will do everything we can
to make this course interesting and enjoyable for everyone. Physics
isn't just fun for Prof Hogg; it is his profession and his calling.

\subsection*{Calendar}\raggedright
\begin{description}
\item[week of Sep 02:] In Lectures this week we will discuss the role
  Einstein played in physics, and how it has been distorted over
  time. We will discuss the changes in physics that took place between
  1870 and 1930 and Einstein's contributions thereto.

  \textsl{There will be no Labs this week.}

  Your reading assignment is
  Isaacson,~\textit{Einstein}, Ch.~1.

\item[week of Sep 09:] Lectures: Momentum, force, collisions,
  Newtonian physics and $F=m\,a$.

  Lab: Math Review

  Reading: Isaacson,~Chs.~2--3;
  Wikipedia:~\textit{Classical~mechanics} (but read around the
  calculus; you don't need to understand the calculus)..

\item[week of Sep 16:] Lectures: How do you measure the masses of
  atoms? Brownian motion. The kinetic theory of gases.

  \textbf{Term Exam 1} in class on Thursday Sep 19.

  Lab: Kinematics

  Reading: Isaacson,~Chs.~3--4;
  Wikipedia:~\textit{Kinetic~theory~of~gases} (but ignore the parts on
  transport and viscosity).

\item[week of Sep 23:] Lectures: Electromagnetism. How waves
  work. Sound waves and light waves. Superpositions of waves.

  Lab: Interference and Diffraction of Light.

  Reading:
  Isaacson,~Chs.~5--6;
  Wikipedia:~\textit{Wave} (ignore the calculus);
  Wikipedia:~\textit{Diffraction};
  Wikipedia:~\textit{Wave~interference}.

\item[week of Sep 30:] Lectures: The photoelectric effect.
  Atom--photon interactions. Quantum mechanics, entanglement.

  \textbf{Term Exam 2} in class on Thursday Sep 19.

  Lab: Measuring the Speed of Sound.

  Reading:
  Wolfson,~\textit{Simply~Einstein:~Relativity~Demystified}, Chs.~1--4;
  Wikipedia:~\textit{Photoelectric~Effect};
  Wikipedia:~\textit{Quantum~mechanics} (but ignore all the mathematical parts).

\item[week of Oct 07:] Lectures: The principle of relativity. The
  importance of the speed of light. The (failed) experiments to
  determine absolute motion. The crisis of theoretical physics in
  1899-ish.

  Lab: The Photoelectric Effect.

  Reading:
  Isaacson,~Ch.~7--8.;
  Wolfson,~Chs.~5--9.

\item[week of Oct 14:] Lectures: Time dilation.

  \textsl{No Labs this week.}

  Reading:
  Isaacson,~Chs.~9--10;
  Wolfson,~Chs.~10--11.

\item[week of Oct 21:] Lectures: Length contraction and the paradoxes
  of special relativity.

  \textbf{Term Exam 3} in class on Tuesday Oct 22.

  Lab: The Michelson Interferometer.

  Reading: Isaacson,~Chs.~11--12;
           Wolfson,~Chs.~12--14.

\item[week of Oct 28:] Lectures: The equivalence of mass and energy,
  or $E = m\,c^2$. The equivalence of gravitational mass and
  gravitational charge.

  Lab: Special Relativity.

  Reading: Isaacson,~Chs.~13--14;
  Wikipedia:~\textit{Mass-energy~equivalence}.

\item[week of Nov 04:] Lectures: General relativity. The curvature of
  spacetime (or the acceleration of space).

  \textbf{Term Exam 4} in class on Tuesday Nov 05.

  Lab: The Principle of Equivalence.

  Reading: Isaacson,~Chs.~15--16;
  Wikipedia:~\textit{General~Relativity}.

\item[week of Nov 11:] Lectures: How do we know that the Universe
  is expanding? How do we measure the expansion precisely?

  Lab: Observing Cosmological Redshift.

  Reading: Isaacson,~Chs.~17--18;
  Wikipedia:~\textit{Expansion~of~the~universe}.

\item[week of Nov 18:] Lectures: Gravitational orbits. The Solar
  System. Kepler's Laws. Escape velocity.

  \textbf{Term Exam 5} in class on Tuesday Nov 19.

  Lab: Hubble's Law and the Expanding Universe.

  Reading: Isaacson,~Chs.~19--20;
           Wolfson,~Chs.~15--16.

\item[week of Nov 25:] Lectures: Black holes. Tidal forces and tidal
  disruption.

  \textsl{No Labs this week.}

  Reading: Isaacson,~Chs.~20--21;
  Wikipedia:~\textit{Black Hole}.

\item[week of Dec 02:] Lectures: The black hole at the center of the
  Milky Way and the black hole at the center of M87.

  Lab: Final-exam review 1

  Reading: Isaacson,~Chs.~22--23;
  Wikipedia:~\textit{Supermassive~black~hole};
  Wikipedia:~\textit{Event~Horizon~Telescope}.

\item[week of Dec 09:] Lectures: Gravitational radiation and its
  discovery and use for astrophysics and cosmology. LIGO project and
  discoveries.

  \textbf{Term Exam 6} in class on Thursday Dec 12.

  Lab: Final-exam review 2

  Reading: Isaacson,~Chs.~24--25;
  Wikipedia:~\textit{LIGO}.

\item[Final Exam:] Tuesday Dec 17 at 10:00 in the Lecture room.
\end{description}
\end{document}
