\documentclass[12pt, letterpaper]{article}
\include{eu}

\begin{document}

\section*{\textsl{Einstein's Universe} Problem Set 4}

This problem set is not to be handed in for credit. But it is due
before \textbf{Tuesday November 9}, when some of these problems
will appear in part, near-verbatim, on Term Exam 4.

\begin{problem}
The energy $E$ of a particle of mass $m$ moving
at speed $v$ is given by:
\begin{equation}\label{eq:energy}
E = \gamma\,m\,c^2
\end{equation}
\begin{equation}
\gamma = \frac{1}{\sqrt{1 - v^2/c^2}}
\end{equation}
This means that if something is at rest, its energy is $m\,c^2$.
We can define a \emph{kinetic energy} to be the difference
between the total energy of an object given by
equation~(\ref{eq:energy}) and the rest energy $m\,c^2$.

What would be the kinetic energy of a baseball moving at 80~percent of the
speed of light? Give your answer in joules but also in tons of TNT
equivalent. For context, the worst nuclear weapons are measured in
megatons.

Do you think we could ever accelerate a baseball to this speed?
\end{problem}

%% \begin{problem}
%% As we said in lecture, if you heat something up, it becomes more
%% massive! Imagine you took 1\,kg of water and heated it up by 50~deg~C.
%% By how much would its mass increase because of this heating?
%% You might have to look up the heat capacity of water.
%% \end{problem}

\begin{problem}
  HOGG: FOR NEXT YEAR: SAY MORE ABOUT UNITS.

  HOGG: FOR NEXT YEAR: TONNES is TONNES OF CO2 not TONNES OF C.

A typical American has a carbon footprint of 40-ish tonnes per year.
If somehow, magically, all of that fossil-fuel consumption could be
converted to nuclear fission energy, how much nuclear reactor fuel
would you need to use?

\textsl{(a)}~To perform this calculation, look up the energy you
get for every carbon triple bond you use up (use that for the carbon
mass-to-energy conversion, assming you release one carbon atom for
every triple bond broken). Also look up the fission energy
per uranium atom, and the fusion energy you would get if you could fuse
hydrogen to iron.

Note that the energies will have slightly different units: The chemical
energy will be joules or eV per bond, the fission energy will be joules
or MeV per atom, and the fusion energy will be joules or MeV per nucleon.

\textsl{(b)}~Now, using what you looked up, compute how much energy
corresponds to 40~tonnes of carbon, assuming that you started with one
carbon triple bond per carbon atom (a very optimistic assumption!).
Give your answer in joules. You will have to use the fact that a mole of
carbon atoms is 12~g.

\textsl{(c)}~Now compute how much uranium you would need to fission to
get the same amount of energy. You will have to use the molar mass of
Uranium here.
That is, if an American converts 40 tonnes of carbon fossil-fuel usage
per year over to uranium fission reactor nuclear energy, how much
uranium would that person use per year? Give your answer in kg.

\textsl{(d)}~Now consider nuclear \emph{fusion}: 
How much fusion fuel would each American need each year? Give your answer
in kg. And note that this could be derived directly from water, in principle!

I'm only looking for rough answers here.
\end{problem}

\begin{problem}
A box of mass $M$ sits on the floor of an elevator at rest. Gravity
(which has a strength set by the local acceleration due to gravity
$g$) pulls the box and the floor pushes the box. The \emph{net} or
total force on the box is zero. What is the force on the box from the
floor? Give both the magnitude and the direction.

Now imagine that the elevator is accelerating upwards at acceleration $a$.
Now the two forces on the box don't balance! Because, after all, the box
is accelerating upwards. What is the force on the box from the floor in this
case?

Now imagine that the elevator is accelerating downwards at acceleration $a$.
Same question.

Now imagine that the elevator is accelerating downwards at acceleration $a = g$.
Same question. Look up the ``vomit comet'' and tell me what this problem has to
do with that airplane.
\end{problem}

\end{document}
