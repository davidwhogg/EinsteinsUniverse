\documentclass[12pt, letterpaper]{article}
\include{eu}

\begin{document}

\section*{\textsl{Einstein's Universe} Problem Set 4}

This problem set is not to be handed in for credit. But it is due
before \textbf{Tuesday November 05}, when some of these problems
will appear in part, near-verbatim, on Term Exam 4.

\begin{problem}
The energy $E$ of a particle of mass $m$ moving
at speed $v$ is given by:
\begin{equation}\label{eq:energy}
E^2 = m^2\,c^4 + p^2\,c^2
\end{equation}
\begin{equation}
p = \gamma\,m\,v
\end{equation}
\begin{equation}
\gamma = \frac{1}{\sqrt{1 - v^2/c^2}}
\end{equation}
This means that if something is at rest, it's energy is $m\,c^2$.
We can define a \emph{kinetic energy} to be the difference
between the total energy of an object given by
equation~(\ref{eq:energy}) and the rest energy $m\,c^2$.

What would be the kinetic energy of a baseball moving at half the
speed of light? Give your answer in Joules but also in tons of TNT
equivalent. For context, the worst nuclear weapons are measured in
megatons.

Do you think we could ever accelerate a baseball to half the speed of
light?
\end{problem}

\end{document}
