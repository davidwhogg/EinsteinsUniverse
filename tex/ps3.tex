\documentclass[12pt, letterpaper]{article}
\usepackage{url, graphicx, epstopdf, amsmath}

% page layout
\setlength{\topmargin}{-0.25in}
\setlength{\textheight}{9.5in}
\setlength{\headheight}{0in}
\setlength{\headsep}{0in}
\setlength{\parindent}{1.1\baselineskip}

\newcommand{\examheader}[1]{
\noindent
Name:\rule[-1ex]{0.40\textwidth}{0.1pt}
NetID:\rule[-1ex]{0.13\textwidth}{0.1pt}
Lab time:\rule[-1ex]{0.13\textwidth}{0.1pt}

\section*{\textsl{Einstein's Universe} {#1}}
\setcounter{problem}{1}}


% problem formatting
\newcommand{\problemname}{Problem}
\newcounter{problem}\setcounter{problem}{1}
\newenvironment{problem}{%
  \addvspace{\baselineskip}\noindent\textbf{Problem~\theproblem:}\refstepcounter{problem}
}{%
  \par\addvspace{\baselineskip}
}

% words
\newcommand{\foreign}[1]{\textsl{#1}}
\newcommand{\vs}{\foreign{vs}}

% math
\newcommand{\dd}{\mathrm{d}}
\newcommand{\e}{\mathrm{e}}

% primary or base units
\newcommand{\rad}{\mathrm{rad}}
\newcommand{\kg}{\mathrm{kg}}
\newcommand{\m}{\mathrm{m}}
\newcommand{\s}{\mathrm{s}}

% secondary units
\renewcommand{\deg}{\mathrm{deg}}
\newcommand{\g}{\mathrm{g}}
\newcommand{\km}{\mathrm{km}}
\newcommand{\cm}{\mathrm{cm}}
\newcommand{\mm}{\mathrm{mm}}
\newcommand{\mum}{\mathrm{\mu m}}
\newcommand{\ft}{\mathrm{ft}}
\newcommand{\mi}{\mathrm{mi}}
\newcommand{\AU}{\mathrm{AU}}
\newcommand{\Mpc}{\mathrm{Mpc}}
\newcommand{\ns}{\mathrm{ns}}
\newcommand{\h}{\mathrm{h}}
\newcommand{\yr}{\mathrm{yr}}
\newcommand{\N}{\mathrm{N}}
\newcommand{\J}{\mathrm{J}}
\newcommand{\eV}{\mathrm{eV}}
\newcommand{\W}{\mathrm{W}}
\newcommand{\Pa}{\mathrm{Pa}}

% derived units
\newcommand{\mps}{\m\,\s^{-1}}
\newcommand{\mph}{\mi\,\h^{-1}}
\newcommand{\mpss}{\m\,\s^{-2}}

% random stuff
\sloppy\sloppypar\raggedbottom\frenchspacing


\begin{document}

\section*{\textsl{Einstein's Universe} Problem Set 3}

This problem set is not to be handed in for credit. But it is due
before \textbf{Tuesday October 26}, when some of these problems
will appear in part, near-verbatim, on Term Exam 3.

\begin{problem}
HOGG: CHANGE THIS TO A HIGHER SPEED NEXT YEAR.
Look up the distance to the nearby star Alpha Centauri. And look up the
distance to the center of the Milky Way. Convert (or get) both of those
distances in light years.

Now imagine traveling to both of those places in a spaceship that can
travel at half the speed of light. To answer these questions, you
might want to look up ``length contraction''. You might also want to
look at the round-trip stellar travel example in Wolfson, Ch.~10.

In the frame of reference of that spaceship, traveling to Alpha
Centauri, how far away is Alpha Centauri? In the frame of reference of
another spaceship, traveling to the Galactic Center, how far away is
the Galactic Center?

How long does it take to get to those two locations? Give two answers:
One for the reference frame of the Earth (essentially at rest in the
Galaxy), and one for the reference frame of the astronauts on the
spaceship.
\end{problem}

\begin{problem}
HOGG DROP THE JET PLANE EXAMPLE FOR NEXT YEAR
Look up the ``twin paradox'' and answer the following questions:

Imagine one of your friends is a flight attendant and spends five years
flying around in jet planes. By how much will your friend's age be different
than it would have been if the friend had spent that time hanging out
at home. Older or younger?

Now imagine the same but for a friend who is on the International
Space Station for five years?

(In detail this question is weird because flights and space stations
go around the Earth, not out and back. But just do the calculation as
if your friend just went out and back. The answer you get is close to
the true answer.)

Now imagine the same but for a friend who spends five years on a journey
to a distant star, traveling at $0.2\,c$.

Now the same but $0.99\,c$. Here you might want to say how much has your
friend aged, instead of the \emph{correction} to the age.
\end{problem}

\begin{problem}
A light-clock like we used in the Lectures has a one-way distance of
one meter (so a round-trip distance of 2 meters) for the light, in its
rest frame. Now imagine that there is a frame (let's call it the Lab
Frame) in which this light clock is moving at 0.95 the speed of
light. Draw a diagram (to scale!) of the path of the light in the Lab
Frame. Assume, like in Lecture, that the light clock is aligned
perpendicular to the direction of motion in the Lab Frame.

How long does the round-trip light trip (from LED to mirror and back
to detector) take in the Lab Frame? It covers two sides of a
triangle. Label all three sides of that triangle with their lengths in
units of meters.
\end{problem}

\end{document}
