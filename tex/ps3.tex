\documentclass[12pt, letterpaper]{article}
\include{eu}

\begin{document}

\section*{\textsl{Einstein's Universe} Problem Set 3}

This problem set is not to be handed in for credit. But it is due
before \textbf{Tuesday October 22}, when some of these problems
will appear in part, near-verbatim, on Term Exam 3.

\begin{problem}
Look up the ``twin paradox'' and answer the following questions:

Imagine one of your friends is a flight attendant and spends five years
flying around in jet planes. By how much will your friend's age be different
than it would have been if the friend had spent that time hanging out
at home. Older or younger?

Now imagine the same but for a friend who is on the International
Space Station for five years?

(In detail this question is weird because flights and space stations
go around the Earth, not out and back. But just do the calculation as
if your friend just went out and back. The answer you get is close to
the true answer.)

Now imagine the same but for a friend who spends five years on a journey
to a distant star, traveling at $0.2\,c$.

Now the same but $0.99\,c$. Here you might want to say how much has your
friend aged, instead of the \emph{correction} to the age.
\end{problem}

\begin{problem}
We said (I hope) that
\begin{equation}\label{eq:energy}
E^2 = m^2\,c^4 + p^2\,c^2
\end{equation}
and
\begin{equation}
p = \gamma\,m\,v
\end{equation}
\begin{equation}
\gamma = \frac{1}{\sqrt{1 - v^2/c^2}}
\end{equation}
This means that if something is at rest, it's energy is $m\,c^2$. That
means that we can define a \emph{kinetic energy} to be the difference
between the total energy of an object given by
equation~(\ref{eq:energy}) and the rest energy $m\,c^2$.

What would be the kinetic energy of a baseball moving at half the
speed of light? Give your answer in Joules but also in tons of TNT
equivalent. For context, the worst nuclear weapons are measured in
megatons.

Do you think we could ever accelerate a baseball to half the speed of
light?
\end{problem}

\begin{problem}
Look up the distance to the nearby star Alpha Centauri. And look up the
distance to the center of the Milky Way. Convert (or get) both of those
distances in light years.

Now imagine traveling to both of those places in a spaceship that can
travel at half the speed of light. How long does it take to get to
those stars? Give two answers: One for the reference frame of the
Earth (essentially at rest in the Galaxy), and one for the reference
frame of the astronauts on the spaceship.
\end{problem}

\begin{problem}
World
\end{problem}

\begin{problem}
Yo
\end{problem}

\end{document}
