\documentclass[12pt, letterpaper]{article}
\usepackage{url, graphicx, epstopdf, amsmath}

% page layout
\setlength{\topmargin}{-0.25in}
\setlength{\textheight}{9.5in}
\setlength{\headheight}{0in}
\setlength{\headsep}{0in}
\setlength{\parindent}{1.1\baselineskip}

\newcommand{\examheader}[1]{
\noindent
Name:\rule[-1ex]{0.40\textwidth}{0.1pt}
NetID:\rule[-1ex]{0.13\textwidth}{0.1pt}
Lab time:\rule[-1ex]{0.13\textwidth}{0.1pt}

\section*{\textsl{Einstein's Universe} {#1}}
\setcounter{problem}{1}}


% problem formatting
\newcommand{\problemname}{Problem}
\newcounter{problem}\setcounter{problem}{1}
\newenvironment{problem}{%
  \addvspace{\baselineskip}\noindent\textbf{Problem~\theproblem:}\refstepcounter{problem}
}{%
  \par\addvspace{\baselineskip}
}

% words
\newcommand{\foreign}[1]{\textsl{#1}}
\newcommand{\vs}{\foreign{vs}}

% math
\newcommand{\dd}{\mathrm{d}}
\newcommand{\e}{\mathrm{e}}

% primary or base units
\newcommand{\rad}{\mathrm{rad}}
\newcommand{\kg}{\mathrm{kg}}
\newcommand{\m}{\mathrm{m}}
\newcommand{\s}{\mathrm{s}}

% secondary units
\renewcommand{\deg}{\mathrm{deg}}
\newcommand{\g}{\mathrm{g}}
\newcommand{\km}{\mathrm{km}}
\newcommand{\cm}{\mathrm{cm}}
\newcommand{\mm}{\mathrm{mm}}
\newcommand{\mum}{\mathrm{\mu m}}
\newcommand{\ft}{\mathrm{ft}}
\newcommand{\mi}{\mathrm{mi}}
\newcommand{\AU}{\mathrm{AU}}
\newcommand{\Mpc}{\mathrm{Mpc}}
\newcommand{\ns}{\mathrm{ns}}
\newcommand{\h}{\mathrm{h}}
\newcommand{\yr}{\mathrm{yr}}
\newcommand{\N}{\mathrm{N}}
\newcommand{\J}{\mathrm{J}}
\newcommand{\eV}{\mathrm{eV}}
\newcommand{\W}{\mathrm{W}}
\newcommand{\Pa}{\mathrm{Pa}}

% derived units
\newcommand{\mps}{\m\,\s^{-1}}
\newcommand{\mph}{\mi\,\h^{-1}}
\newcommand{\mpss}{\m\,\s^{-2}}

% random stuff
\sloppy\sloppypar\raggedbottom\frenchspacing


\begin{document}

\section*{\textsl{Einstein's Universe} Problem Set 5}

This problem set is not to be handed in for credit. But it is due
before \textbf{Thursday November 21}, when some of these problems
will appear in part, near-verbatim, on Term Exam 5.

\begin{problem}
Hogg said in lecture that the exact Doppler factor
\begin{equation}
\sqrt{\frac{1 + {v/c}}{1 - {v/c}}}
\end{equation}
for small $v/c$ can be approximated by
\begin{equation}
1 + \frac{v}{c}
\end{equation}
Check this by computing both the exact factor and the approximate factor for
the speeds $2\,\mps$, $2\times10^4\,\mps$, and $2\times 10^8\,\mps$.
What do you conclude?
\end{problem}

\begin{problem}
There is a gravitational redshift between a clock at sea level and a clock at the
top of the (very tall) mountain K2.
Imagine that you have two identical clocks, one at sea level and one at the top of K2.
If they are both started on 2020 January 1,
and both run until 2020 December 31, by how much (roughly) will their times differ
at the end of the year? Give your answer in seconds.
Which one will have run more quickly, and which more slowly?
\end{problem}

\begin{problem}
The expansion rate of the Universe is usually given as something like $70\,\km\,\s^{-1}\,\Mpc^{-1}$.
If you convert this quantity to SI base units of kg, m, s, what is this number and in what units?
What do you think this means for the age of the Universe?
Give an estimate for the age of the Universe, in $s$ and in billions of years.
\end{problem}

%% \begin{problem}
%% The Earth spins once per day. Look up the circumference of the Earth and use it to estimate
%% (very roughly) the speed at which we are moving because of this spin.
%% Now look up how fast the Earth is moving around the Sun in its orbit.
%% Now look up how fast the Sun is moving in its orbit around the Galaxy.
%% Now look up how fast the Galaxy is moving with respect to the cosmic microwave background.
%% Convert each of these speeds into a Doppler factor.
%% \end{problem}

\begin{problem}
The insolation---the Solar energy delivered to the surface of the
Earth---is about 1 kW per square meter. This is the energy you feel
when you are on the beach on a sunny day. What would be the insolation
at Jupiter in those same units?  What about at Pluto?
\end{problem}

\begin{problem}
Because the Universe is expanding, more distant objects are receding from us faster.
The most distant galaxies and quasars we have ever observed have Doppler factors
around 10 (they are redshifted by a factor of 10)!
What recession speed $v$ does this correspond to?
Give your answer both in SI units ($\mps$) and in units of the speed of light $c$.
\end{problem}

\end{document}
