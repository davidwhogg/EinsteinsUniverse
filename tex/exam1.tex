
\documentclass[12pt, letterpaper]{article}
\usepackage{url, graphicx, epstopdf, amsmath}

% page layout
\setlength{\topmargin}{-0.25in}
\setlength{\textheight}{9.5in}
\setlength{\headheight}{0in}
\setlength{\headsep}{0in}
\setlength{\parindent}{1.1\baselineskip}

\newcommand{\examheader}[1]{
\noindent
Name:\rule[-1ex]{0.40\textwidth}{0.1pt}
NetID:\rule[-1ex]{0.13\textwidth}{0.1pt}
Lab time:\rule[-1ex]{0.13\textwidth}{0.1pt}

\section*{\textsl{Einstein's Universe} {#1}}
\setcounter{problem}{1}}


% problem formatting
\newcommand{\problemname}{Problem}
\newcounter{problem}\setcounter{problem}{1}
\newenvironment{problem}{%
  \addvspace{\baselineskip}\noindent\textbf{Problem~\theproblem:}\refstepcounter{problem}
}{%
  \par\addvspace{\baselineskip}
}

% words
\newcommand{\foreign}[1]{\textsl{#1}}
\newcommand{\vs}{\foreign{vs}}

% math
\newcommand{\dd}{\mathrm{d}}
\newcommand{\e}{\mathrm{e}}

% primary or base units
\newcommand{\rad}{\mathrm{rad}}
\newcommand{\kg}{\mathrm{kg}}
\newcommand{\m}{\mathrm{m}}
\newcommand{\s}{\mathrm{s}}

% secondary units
\renewcommand{\deg}{\mathrm{deg}}
\newcommand{\g}{\mathrm{g}}
\newcommand{\km}{\mathrm{km}}
\newcommand{\cm}{\mathrm{cm}}
\newcommand{\mm}{\mathrm{mm}}
\newcommand{\mum}{\mathrm{\mu m}}
\newcommand{\ft}{\mathrm{ft}}
\newcommand{\mi}{\mathrm{mi}}
\newcommand{\AU}{\mathrm{AU}}
\newcommand{\Mpc}{\mathrm{Mpc}}
\newcommand{\ns}{\mathrm{ns}}
\newcommand{\h}{\mathrm{h}}
\newcommand{\yr}{\mathrm{yr}}
\newcommand{\N}{\mathrm{N}}
\newcommand{\J}{\mathrm{J}}
\newcommand{\eV}{\mathrm{eV}}
\newcommand{\W}{\mathrm{W}}
\newcommand{\Pa}{\mathrm{Pa}}

% derived units
\newcommand{\mps}{\m\,\s^{-1}}
\newcommand{\mph}{\mi\,\h^{-1}}
\newcommand{\mpss}{\m\,\s^{-2}}

% random stuff
\sloppy\sloppypar\raggedbottom\frenchspacing

\pagestyle{empty}

\begin{document}



\noindent
Name: \rule[-1ex]{0.60\textwidth}{0.1pt}
NetID: \rule[-1ex]{0.20\textwidth}{0.1pt}

\section*{\textsl{Einstein's Universe} Term Exam 1}
\setcounter{problem}{1}


\begin{problem} (From the reading)
What musical instrument did Einstein most enjoy playing?
\end{problem}


\vfill ~

\begin{problem} (From the reading)
Classical mechanics, or Newtonian mechanics, is only valid in certain
circumstances. When do the laws of classical mechanics, like $F =
m\,a$ for example, become wrong or break down? There are many answers
to this problem; I will take anything correct.
\end{problem}


\vfill ~

\begin{problem} (From Problem Set 1)
What is the relationship between the energy $E$ and wavelength
$\lambda$ of a photon? Give a formula that involves energy $E$,
Planck's Constant $h$, the speed of light $c$, and wavelength
$\lambda$ (or whatever you need).
\end{problem}

\vfill ~

\begin{problem} (From Problem Set 1)
What is the approximate thickness of a stack of 1000 20-dollar bills?
No need to be precise, and use any units you like.
\end{problem}


\vfill ~


\clearpage


\begin{problem} (From Lecture on 2019-09-17)
If you are traveling at 60 miles per hour, how long does
it take you to go 300 miles?
\end{problem}


\vfill ~

\begin{problem} (From Lecture on 2019-09-05)
The molar weight of water is $18\,\g$. How many molecules would there
be, therefore, in $18\,\g$ of water? You don't need a calculator for
this.
\end{problem}


\vfill ~

\begin{problem} (From the Kinematics Lab)
Here is a data table of times, positions, and velocities in SI units:\\
\rule{1.0in}{0pt}\begin{tabular}{c|c|c}
time $t$ ($\s$) & position $x$ ($\m$) & velocity $v$ ($\m\,\s^{-1}$) \\
\hline
1 & 1.15 & 1.3 \\
2 & 2.60 & 1.6 \\
3 & 4.35 & 1.9 \\
\hline
\end{tabular}\\
What is the average acceleration in the time interval from $2\,\s$ to $3\,\s$?
\end{problem}


\vfill ~

\begin{problem} (From the Math Review Lab)
What is this number? Give your answer in scientific notation.
$$
\frac{(7\times10^{-34})\times(3\times10^8)}{5\times10^{-7}}
$$
You don't need a calculator to solve this problem (\textit{hint: $3/5=0.6$}).
\end{problem}


\vfill ~


\cleardoublepage



\noindent
Name: \rule[-1ex]{0.60\textwidth}{0.1pt}
NetID: \rule[-1ex]{0.20\textwidth}{0.1pt}

\section*{\textsl{Einstein's Universe} Term Exam 1}
\setcounter{problem}{1}


\begin{problem} (From Problem Set 1)
What is the relationship between the energy $E$ and wavelength
$\lambda$ of a photon? Give a formula that involves energy $E$,
Planck's Constant $h$, the speed of light $c$, and wavelength
$\lambda$ (or whatever you need).
\end{problem}

\vfill ~

\begin{problem} (From Lecture on 2019-09-05)
The molar weight of water is $18\,\g$. How many molecules would there
be, therefore, in $18\,\g$ of water? You don't need a calculator for
this.
\end{problem}


\vfill ~

\begin{problem} (From the reading)
What musical instrument did Einstein most enjoy playing?
\end{problem}


\vfill ~

\begin{problem} (From Problem Set 1)
What is the approximate thickness of a stack of 1000 20-dollar bills?
No need to be precise, and use any units you like.
\end{problem}


\vfill ~


\clearpage


\begin{problem} (From Lecture on 2019-09-17)
If you are traveling at 60 miles per hour, how long does
it take you to go 300 miles?
\end{problem}


\vfill ~

\begin{problem} (From the Math Review Lab)
What is this number? Give your answer in scientific notation.
$$
\frac{(7\times10^{-34})\times(3\times10^8)}{5\times10^{-7}}
$$
You don't need a calculator to solve this problem (\textit{hint: $3/5=0.6$}).
\end{problem}


\vfill ~

\begin{problem} (From the Kinematics Lab)
Here is a data table of times, positions, and velocities in SI units:\\
\rule{1.0in}{0pt}\begin{tabular}{c|c|c}
time $t$ ($\s$) & position $x$ ($\m$) & velocity $v$ ($\m\,\s^{-1}$) \\
\hline
1 & 1.15 & 1.3 \\
2 & 2.60 & 1.6 \\
3 & 4.35 & 1.9 \\
\hline
\end{tabular}\\
What is the average acceleration in the time interval from $2\,\s$ to $3\,\s$?
\end{problem}


\vfill ~

\begin{problem} (From the reading)
Classical mechanics, or Newtonian mechanics, is only valid in certain
circumstances. When do the laws of classical mechanics, like $F =
m\,a$ for example, become wrong or break down? There are many answers
to this problem; I will take anything correct.
\end{problem}


\vfill ~


\cleardoublepage



\noindent
Name: \rule[-1ex]{0.60\textwidth}{0.1pt}
NetID: \rule[-1ex]{0.20\textwidth}{0.1pt}

\section*{\textsl{Einstein's Universe} Term Exam 1}
\setcounter{problem}{1}


\begin{problem} (From the reading)
What musical instrument did Einstein most enjoy playing?
\end{problem}


\vfill ~

\begin{problem} (From the Math Review Lab)
What is this number? Give your answer in scientific notation.
$$
\frac{(7\times10^{-34})\times(3\times10^8)}{5\times10^{-7}}
$$
You don't need a calculator to solve this problem (\textit{hint: $3/5=0.6$}).
\end{problem}


\vfill ~

\begin{problem} (From Lecture on 2019-09-05)
The molar weight of water is $18\,\g$. How many molecules would there
be, therefore, in $18\,\g$ of water? You don't need a calculator for
this.
\end{problem}


\vfill ~

\begin{problem} (From the reading)
Classical mechanics, or Newtonian mechanics, is only valid in certain
circumstances. When do the laws of classical mechanics, like $F =
m\,a$ for example, become wrong or break down? There are many answers
to this problem; I will take anything correct.
\end{problem}


\vfill ~


\clearpage


\begin{problem} (From Problem Set 1)
What is the relationship between the energy $E$ and wavelength
$\lambda$ of a photon? Give a formula that involves energy $E$,
Planck's Constant $h$, the speed of light $c$, and wavelength
$\lambda$ (or whatever you need).
\end{problem}

\vfill ~

\begin{problem} (From the Kinematics Lab)
Here is a data table of times, positions, and velocities in SI units:\\
\rule{1.0in}{0pt}\begin{tabular}{c|c|c}
time $t$ ($\s$) & position $x$ ($\m$) & velocity $v$ ($\m\,\s^{-1}$) \\
\hline
1 & 1.15 & 1.3 \\
2 & 2.60 & 1.6 \\
3 & 4.35 & 1.9 \\
\hline
\end{tabular}\\
What is the average acceleration in the time interval from $2\,\s$ to $3\,\s$?
\end{problem}


\vfill ~

\begin{problem} (From Problem Set 1)
What is the approximate thickness of a stack of 1000 20-dollar bills?
No need to be precise, and use any units you like.
\end{problem}


\vfill ~

\begin{problem} (From Lecture on 2019-09-17)
If you are traveling at 60 miles per hour, how long does
it take you to go 300 miles?
\end{problem}


\vfill ~


\cleardoublepage



\noindent
Name: \rule[-1ex]{0.60\textwidth}{0.1pt}
NetID: \rule[-1ex]{0.20\textwidth}{0.1pt}

\section*{\textsl{Einstein's Universe} Term Exam 1}
\setcounter{problem}{1}


\begin{problem} (From the reading)
What musical instrument did Einstein most enjoy playing?
\end{problem}


\vfill ~

\begin{problem} (From Problem Set 1)
What is the approximate thickness of a stack of 1000 20-dollar bills?
No need to be precise, and use any units you like.
\end{problem}


\vfill ~

\begin{problem} (From the reading)
Classical mechanics, or Newtonian mechanics, is only valid in certain
circumstances. When do the laws of classical mechanics, like $F =
m\,a$ for example, become wrong or break down? There are many answers
to this problem; I will take anything correct.
\end{problem}


\vfill ~

\begin{problem} (From Lecture on 2019-09-05)
The molar weight of water is $18\,\g$. How many molecules would there
be, therefore, in $18\,\g$ of water? You don't need a calculator for
this.
\end{problem}


\vfill ~


\clearpage


\begin{problem} (From the Kinematics Lab)
Here is a data table of times, positions, and velocities in SI units:\\
\rule{1.0in}{0pt}\begin{tabular}{c|c|c}
time $t$ ($\s$) & position $x$ ($\m$) & velocity $v$ ($\m\,\s^{-1}$) \\
\hline
1 & 1.15 & 1.3 \\
2 & 2.60 & 1.6 \\
3 & 4.35 & 1.9 \\
\hline
\end{tabular}\\
What is the average acceleration in the time interval from $2\,\s$ to $3\,\s$?
\end{problem}


\vfill ~

\begin{problem} (From Problem Set 1)
What is the relationship between the energy $E$ and wavelength
$\lambda$ of a photon? Give a formula that involves energy $E$,
Planck's Constant $h$, the speed of light $c$, and wavelength
$\lambda$ (or whatever you need).
\end{problem}

\vfill ~

\begin{problem} (From the Math Review Lab)
What is this number? Give your answer in scientific notation.
$$
\frac{(7\times10^{-34})\times(3\times10^8)}{5\times10^{-7}}
$$
You don't need a calculator to solve this problem (\textit{hint: $3/5=0.6$}).
\end{problem}


\vfill ~

\begin{problem} (From Lecture on 2019-09-17)
If you are traveling at 60 miles per hour, how long does
it take you to go 300 miles?
\end{problem}


\vfill ~


\cleardoublepage



\noindent
Name: \rule[-1ex]{0.60\textwidth}{0.1pt}
NetID: \rule[-1ex]{0.20\textwidth}{0.1pt}

\section*{\textsl{Einstein's Universe} Term Exam 1}
\setcounter{problem}{1}


\begin{problem} (From Problem Set 1)
What is the approximate thickness of a stack of 1000 20-dollar bills?
No need to be precise, and use any units you like.
\end{problem}


\vfill ~

\begin{problem} (From the Kinematics Lab)
Here is a data table of times, positions, and velocities in SI units:\\
\rule{1.0in}{0pt}\begin{tabular}{c|c|c}
time $t$ ($\s$) & position $x$ ($\m$) & velocity $v$ ($\m\,\s^{-1}$) \\
\hline
1 & 1.15 & 1.3 \\
2 & 2.60 & 1.6 \\
3 & 4.35 & 1.9 \\
\hline
\end{tabular}\\
What is the average acceleration in the time interval from $2\,\s$ to $3\,\s$?
\end{problem}


\vfill ~

\begin{problem} (From the reading)
What musical instrument did Einstein most enjoy playing?
\end{problem}


\vfill ~

\begin{problem} (From Lecture on 2019-09-05)
The molar weight of water is $18\,\g$. How many molecules would there
be, therefore, in $18\,\g$ of water? You don't need a calculator for
this.
\end{problem}


\vfill ~


\clearpage


\begin{problem} (From the reading)
Classical mechanics, or Newtonian mechanics, is only valid in certain
circumstances. When do the laws of classical mechanics, like $F =
m\,a$ for example, become wrong or break down? There are many answers
to this problem; I will take anything correct.
\end{problem}


\vfill ~

\begin{problem} (From Problem Set 1)
What is the relationship between the energy $E$ and wavelength
$\lambda$ of a photon? Give a formula that involves energy $E$,
Planck's Constant $h$, the speed of light $c$, and wavelength
$\lambda$ (or whatever you need).
\end{problem}

\vfill ~

\begin{problem} (From the Math Review Lab)
What is this number? Give your answer in scientific notation.
$$
\frac{(7\times10^{-34})\times(3\times10^8)}{5\times10^{-7}}
$$
You don't need a calculator to solve this problem (\textit{hint: $3/5=0.6$}).
\end{problem}


\vfill ~

\begin{problem} (From Lecture on 2019-09-17)
If you are traveling at 60 miles per hour, how long does
it take you to go 300 miles?
\end{problem}


\vfill ~


\cleardoublepage



\noindent
Name: \rule[-1ex]{0.60\textwidth}{0.1pt}
NetID: \rule[-1ex]{0.20\textwidth}{0.1pt}

\section*{\textsl{Einstein's Universe} Term Exam 1}
\setcounter{problem}{1}


\begin{problem} (From Problem Set 1)
What is the approximate thickness of a stack of 1000 20-dollar bills?
No need to be precise, and use any units you like.
\end{problem}


\vfill ~

\begin{problem} (From Lecture on 2019-09-17)
If you are traveling at 60 miles per hour, how long does
it take you to go 300 miles?
\end{problem}


\vfill ~

\begin{problem} (From the reading)
What musical instrument did Einstein most enjoy playing?
\end{problem}


\vfill ~

\begin{problem} (From the Math Review Lab)
What is this number? Give your answer in scientific notation.
$$
\frac{(7\times10^{-34})\times(3\times10^8)}{5\times10^{-7}}
$$
You don't need a calculator to solve this problem (\textit{hint: $3/5=0.6$}).
\end{problem}


\vfill ~


\clearpage


\begin{problem} (From the Kinematics Lab)
Here is a data table of times, positions, and velocities in SI units:\\
\rule{1.0in}{0pt}\begin{tabular}{c|c|c}
time $t$ ($\s$) & position $x$ ($\m$) & velocity $v$ ($\m\,\s^{-1}$) \\
\hline
1 & 1.15 & 1.3 \\
2 & 2.60 & 1.6 \\
3 & 4.35 & 1.9 \\
\hline
\end{tabular}\\
What is the average acceleration in the time interval from $2\,\s$ to $3\,\s$?
\end{problem}


\vfill ~

\begin{problem} (From Problem Set 1)
What is the relationship between the energy $E$ and wavelength
$\lambda$ of a photon? Give a formula that involves energy $E$,
Planck's Constant $h$, the speed of light $c$, and wavelength
$\lambda$ (or whatever you need).
\end{problem}

\vfill ~

\begin{problem} (From Lecture on 2019-09-05)
The molar weight of water is $18\,\g$. How many molecules would there
be, therefore, in $18\,\g$ of water? You don't need a calculator for
this.
\end{problem}


\vfill ~

\begin{problem} (From the reading)
Classical mechanics, or Newtonian mechanics, is only valid in certain
circumstances. When do the laws of classical mechanics, like $F =
m\,a$ for example, become wrong or break down? There are many answers
to this problem; I will take anything correct.
\end{problem}


\vfill ~


\cleardoublepage



\noindent
Name: \rule[-1ex]{0.60\textwidth}{0.1pt}
NetID: \rule[-1ex]{0.20\textwidth}{0.1pt}

\section*{\textsl{Einstein's Universe} Term Exam 1}
\setcounter{problem}{1}


\begin{problem} (From the reading)
What musical instrument did Einstein most enjoy playing?
\end{problem}


\vfill ~

\begin{problem} (From the Math Review Lab)
What is this number? Give your answer in scientific notation.
$$
\frac{(7\times10^{-34})\times(3\times10^8)}{5\times10^{-7}}
$$
You don't need a calculator to solve this problem (\textit{hint: $3/5=0.6$}).
\end{problem}


\vfill ~

\begin{problem} (From the Kinematics Lab)
Here is a data table of times, positions, and velocities in SI units:\\
\rule{1.0in}{0pt}\begin{tabular}{c|c|c}
time $t$ ($\s$) & position $x$ ($\m$) & velocity $v$ ($\m\,\s^{-1}$) \\
\hline
1 & 1.15 & 1.3 \\
2 & 2.60 & 1.6 \\
3 & 4.35 & 1.9 \\
\hline
\end{tabular}\\
What is the average acceleration in the time interval from $2\,\s$ to $3\,\s$?
\end{problem}


\vfill ~

\begin{problem} (From Lecture on 2019-09-17)
If you are traveling at 60 miles per hour, how long does
it take you to go 300 miles?
\end{problem}


\vfill ~


\clearpage


\begin{problem} (From Lecture on 2019-09-05)
The molar weight of water is $18\,\g$. How many molecules would there
be, therefore, in $18\,\g$ of water? You don't need a calculator for
this.
\end{problem}


\vfill ~

\begin{problem} (From Problem Set 1)
What is the relationship between the energy $E$ and wavelength
$\lambda$ of a photon? Give a formula that involves energy $E$,
Planck's Constant $h$, the speed of light $c$, and wavelength
$\lambda$ (or whatever you need).
\end{problem}

\vfill ~

\begin{problem} (From the reading)
Classical mechanics, or Newtonian mechanics, is only valid in certain
circumstances. When do the laws of classical mechanics, like $F =
m\,a$ for example, become wrong or break down? There are many answers
to this problem; I will take anything correct.
\end{problem}


\vfill ~

\begin{problem} (From Problem Set 1)
What is the approximate thickness of a stack of 1000 20-dollar bills?
No need to be precise, and use any units you like.
\end{problem}


\vfill ~


\cleardoublepage



\noindent
Name: \rule[-1ex]{0.60\textwidth}{0.1pt}
NetID: \rule[-1ex]{0.20\textwidth}{0.1pt}

\section*{\textsl{Einstein's Universe} Term Exam 1}
\setcounter{problem}{1}


\begin{problem} (From the reading)
Classical mechanics, or Newtonian mechanics, is only valid in certain
circumstances. When do the laws of classical mechanics, like $F =
m\,a$ for example, become wrong or break down? There are many answers
to this problem; I will take anything correct.
\end{problem}


\vfill ~

\begin{problem} (From Lecture on 2019-09-17)
If you are traveling at 60 miles per hour, how long does
it take you to go 300 miles?
\end{problem}


\vfill ~

\begin{problem} (From Problem Set 1)
What is the relationship between the energy $E$ and wavelength
$\lambda$ of a photon? Give a formula that involves energy $E$,
Planck's Constant $h$, the speed of light $c$, and wavelength
$\lambda$ (or whatever you need).
\end{problem}

\vfill ~

\begin{problem} (From Lecture on 2019-09-05)
The molar weight of water is $18\,\g$. How many molecules would there
be, therefore, in $18\,\g$ of water? You don't need a calculator for
this.
\end{problem}


\vfill ~


\clearpage


\begin{problem} (From the reading)
What musical instrument did Einstein most enjoy playing?
\end{problem}


\vfill ~

\begin{problem} (From the Math Review Lab)
What is this number? Give your answer in scientific notation.
$$
\frac{(7\times10^{-34})\times(3\times10^8)}{5\times10^{-7}}
$$
You don't need a calculator to solve this problem (\textit{hint: $3/5=0.6$}).
\end{problem}


\vfill ~

\begin{problem} (From the Kinematics Lab)
Here is a data table of times, positions, and velocities in SI units:\\
\rule{1.0in}{0pt}\begin{tabular}{c|c|c}
time $t$ ($\s$) & position $x$ ($\m$) & velocity $v$ ($\m\,\s^{-1}$) \\
\hline
1 & 1.15 & 1.3 \\
2 & 2.60 & 1.6 \\
3 & 4.35 & 1.9 \\
\hline
\end{tabular}\\
What is the average acceleration in the time interval from $2\,\s$ to $3\,\s$?
\end{problem}


\vfill ~

\begin{problem} (From Problem Set 1)
What is the approximate thickness of a stack of 1000 20-dollar bills?
No need to be precise, and use any units you like.
\end{problem}


\vfill ~


\cleardoublepage



\noindent
Name: \rule[-1ex]{0.60\textwidth}{0.1pt}
NetID: \rule[-1ex]{0.20\textwidth}{0.1pt}

\section*{\textsl{Einstein's Universe} Term Exam 1}
\setcounter{problem}{1}


\begin{problem} (From Problem Set 1)
What is the approximate thickness of a stack of 1000 20-dollar bills?
No need to be precise, and use any units you like.
\end{problem}


\vfill ~

\begin{problem} (From Lecture on 2019-09-17)
If you are traveling at 60 miles per hour, how long does
it take you to go 300 miles?
\end{problem}


\vfill ~

\begin{problem} (From Lecture on 2019-09-05)
The molar weight of water is $18\,\g$. How many molecules would there
be, therefore, in $18\,\g$ of water? You don't need a calculator for
this.
\end{problem}


\vfill ~

\begin{problem} (From the reading)
What musical instrument did Einstein most enjoy playing?
\end{problem}


\vfill ~


\clearpage


\begin{problem} (From the reading)
Classical mechanics, or Newtonian mechanics, is only valid in certain
circumstances. When do the laws of classical mechanics, like $F =
m\,a$ for example, become wrong or break down? There are many answers
to this problem; I will take anything correct.
\end{problem}


\vfill ~

\begin{problem} (From the Math Review Lab)
What is this number? Give your answer in scientific notation.
$$
\frac{(7\times10^{-34})\times(3\times10^8)}{5\times10^{-7}}
$$
You don't need a calculator to solve this problem (\textit{hint: $3/5=0.6$}).
\end{problem}


\vfill ~

\begin{problem} (From Problem Set 1)
What is the relationship between the energy $E$ and wavelength
$\lambda$ of a photon? Give a formula that involves energy $E$,
Planck's Constant $h$, the speed of light $c$, and wavelength
$\lambda$ (or whatever you need).
\end{problem}

\vfill ~

\begin{problem} (From the Kinematics Lab)
Here is a data table of times, positions, and velocities in SI units:\\
\rule{1.0in}{0pt}\begin{tabular}{c|c|c}
time $t$ ($\s$) & position $x$ ($\m$) & velocity $v$ ($\m\,\s^{-1}$) \\
\hline
1 & 1.15 & 1.3 \\
2 & 2.60 & 1.6 \\
3 & 4.35 & 1.9 \\
\hline
\end{tabular}\\
What is the average acceleration in the time interval from $2\,\s$ to $3\,\s$?
\end{problem}


\vfill ~


\cleardoublepage



\noindent
Name: \rule[-1ex]{0.60\textwidth}{0.1pt}
NetID: \rule[-1ex]{0.20\textwidth}{0.1pt}

\section*{\textsl{Einstein's Universe} Term Exam 1}
\setcounter{problem}{1}


\begin{problem} (From the reading)
Classical mechanics, or Newtonian mechanics, is only valid in certain
circumstances. When do the laws of classical mechanics, like $F =
m\,a$ for example, become wrong or break down? There are many answers
to this problem; I will take anything correct.
\end{problem}


\vfill ~

\begin{problem} (From Problem Set 1)
What is the relationship between the energy $E$ and wavelength
$\lambda$ of a photon? Give a formula that involves energy $E$,
Planck's Constant $h$, the speed of light $c$, and wavelength
$\lambda$ (or whatever you need).
\end{problem}

\vfill ~

\begin{problem} (From Lecture on 2019-09-17)
If you are traveling at 60 miles per hour, how long does
it take you to go 300 miles?
\end{problem}


\vfill ~

\begin{problem} (From the reading)
What musical instrument did Einstein most enjoy playing?
\end{problem}


\vfill ~


\clearpage


\begin{problem} (From the Kinematics Lab)
Here is a data table of times, positions, and velocities in SI units:\\
\rule{1.0in}{0pt}\begin{tabular}{c|c|c}
time $t$ ($\s$) & position $x$ ($\m$) & velocity $v$ ($\m\,\s^{-1}$) \\
\hline
1 & 1.15 & 1.3 \\
2 & 2.60 & 1.6 \\
3 & 4.35 & 1.9 \\
\hline
\end{tabular}\\
What is the average acceleration in the time interval from $2\,\s$ to $3\,\s$?
\end{problem}


\vfill ~

\begin{problem} (From Problem Set 1)
What is the approximate thickness of a stack of 1000 20-dollar bills?
No need to be precise, and use any units you like.
\end{problem}


\vfill ~

\begin{problem} (From Lecture on 2019-09-05)
The molar weight of water is $18\,\g$. How many molecules would there
be, therefore, in $18\,\g$ of water? You don't need a calculator for
this.
\end{problem}


\vfill ~

\begin{problem} (From the Math Review Lab)
What is this number? Give your answer in scientific notation.
$$
\frac{(7\times10^{-34})\times(3\times10^8)}{5\times10^{-7}}
$$
You don't need a calculator to solve this problem (\textit{hint: $3/5=0.6$}).
\end{problem}


\vfill ~


\cleardoublepage



\noindent
Name: \rule[-1ex]{0.60\textwidth}{0.1pt}
NetID: \rule[-1ex]{0.20\textwidth}{0.1pt}

\section*{\textsl{Einstein's Universe} Term Exam 1}
\setcounter{problem}{1}


\begin{problem} (From the reading)
What musical instrument did Einstein most enjoy playing?
\end{problem}


\vfill ~

\begin{problem} (From Lecture on 2019-09-17)
If you are traveling at 60 miles per hour, how long does
it take you to go 300 miles?
\end{problem}


\vfill ~

\begin{problem} (From Problem Set 1)
What is the relationship between the energy $E$ and wavelength
$\lambda$ of a photon? Give a formula that involves energy $E$,
Planck's Constant $h$, the speed of light $c$, and wavelength
$\lambda$ (or whatever you need).
\end{problem}

\vfill ~

\begin{problem} (From Problem Set 1)
What is the approximate thickness of a stack of 1000 20-dollar bills?
No need to be precise, and use any units you like.
\end{problem}


\vfill ~


\clearpage


\begin{problem} (From the Math Review Lab)
What is this number? Give your answer in scientific notation.
$$
\frac{(7\times10^{-34})\times(3\times10^8)}{5\times10^{-7}}
$$
You don't need a calculator to solve this problem (\textit{hint: $3/5=0.6$}).
\end{problem}


\vfill ~

\begin{problem} (From the Kinematics Lab)
Here is a data table of times, positions, and velocities in SI units:\\
\rule{1.0in}{0pt}\begin{tabular}{c|c|c}
time $t$ ($\s$) & position $x$ ($\m$) & velocity $v$ ($\m\,\s^{-1}$) \\
\hline
1 & 1.15 & 1.3 \\
2 & 2.60 & 1.6 \\
3 & 4.35 & 1.9 \\
\hline
\end{tabular}\\
What is the average acceleration in the time interval from $2\,\s$ to $3\,\s$?
\end{problem}


\vfill ~

\begin{problem} (From Lecture on 2019-09-05)
The molar weight of water is $18\,\g$. How many molecules would there
be, therefore, in $18\,\g$ of water? You don't need a calculator for
this.
\end{problem}


\vfill ~

\begin{problem} (From the reading)
Classical mechanics, or Newtonian mechanics, is only valid in certain
circumstances. When do the laws of classical mechanics, like $F =
m\,a$ for example, become wrong or break down? There are many answers
to this problem; I will take anything correct.
\end{problem}


\vfill ~


\cleardoublepage



\noindent
Name: \rule[-1ex]{0.60\textwidth}{0.1pt}
NetID: \rule[-1ex]{0.20\textwidth}{0.1pt}

\section*{\textsl{Einstein's Universe} Term Exam 1}
\setcounter{problem}{1}


\begin{problem} (From Lecture on 2019-09-17)
If you are traveling at 60 miles per hour, how long does
it take you to go 300 miles?
\end{problem}


\vfill ~

\begin{problem} (From Lecture on 2019-09-05)
The molar weight of water is $18\,\g$. How many molecules would there
be, therefore, in $18\,\g$ of water? You don't need a calculator for
this.
\end{problem}


\vfill ~

\begin{problem} (From the reading)
Classical mechanics, or Newtonian mechanics, is only valid in certain
circumstances. When do the laws of classical mechanics, like $F =
m\,a$ for example, become wrong or break down? There are many answers
to this problem; I will take anything correct.
\end{problem}


\vfill ~

\begin{problem} (From the reading)
What musical instrument did Einstein most enjoy playing?
\end{problem}


\vfill ~


\clearpage


\begin{problem} (From Problem Set 1)
What is the relationship between the energy $E$ and wavelength
$\lambda$ of a photon? Give a formula that involves energy $E$,
Planck's Constant $h$, the speed of light $c$, and wavelength
$\lambda$ (or whatever you need).
\end{problem}

\vfill ~

\begin{problem} (From the Kinematics Lab)
Here is a data table of times, positions, and velocities in SI units:\\
\rule{1.0in}{0pt}\begin{tabular}{c|c|c}
time $t$ ($\s$) & position $x$ ($\m$) & velocity $v$ ($\m\,\s^{-1}$) \\
\hline
1 & 1.15 & 1.3 \\
2 & 2.60 & 1.6 \\
3 & 4.35 & 1.9 \\
\hline
\end{tabular}\\
What is the average acceleration in the time interval from $2\,\s$ to $3\,\s$?
\end{problem}


\vfill ~

\begin{problem} (From Problem Set 1)
What is the approximate thickness of a stack of 1000 20-dollar bills?
No need to be precise, and use any units you like.
\end{problem}


\vfill ~

\begin{problem} (From the Math Review Lab)
What is this number? Give your answer in scientific notation.
$$
\frac{(7\times10^{-34})\times(3\times10^8)}{5\times10^{-7}}
$$
You don't need a calculator to solve this problem (\textit{hint: $3/5=0.6$}).
\end{problem}


\vfill ~


\cleardoublepage



\noindent
Name: \rule[-1ex]{0.60\textwidth}{0.1pt}
NetID: \rule[-1ex]{0.20\textwidth}{0.1pt}

\section*{\textsl{Einstein's Universe} Term Exam 1}
\setcounter{problem}{1}


\begin{problem} (From the Kinematics Lab)
Here is a data table of times, positions, and velocities in SI units:\\
\rule{1.0in}{0pt}\begin{tabular}{c|c|c}
time $t$ ($\s$) & position $x$ ($\m$) & velocity $v$ ($\m\,\s^{-1}$) \\
\hline
1 & 1.15 & 1.3 \\
2 & 2.60 & 1.6 \\
3 & 4.35 & 1.9 \\
\hline
\end{tabular}\\
What is the average acceleration in the time interval from $2\,\s$ to $3\,\s$?
\end{problem}


\vfill ~

\begin{problem} (From Lecture on 2019-09-05)
The molar weight of water is $18\,\g$. How many molecules would there
be, therefore, in $18\,\g$ of water? You don't need a calculator for
this.
\end{problem}


\vfill ~

\begin{problem} (From Problem Set 1)
What is the approximate thickness of a stack of 1000 20-dollar bills?
No need to be precise, and use any units you like.
\end{problem}


\vfill ~

\begin{problem} (From the reading)
Classical mechanics, or Newtonian mechanics, is only valid in certain
circumstances. When do the laws of classical mechanics, like $F =
m\,a$ for example, become wrong or break down? There are many answers
to this problem; I will take anything correct.
\end{problem}


\vfill ~


\clearpage


\begin{problem} (From the Math Review Lab)
What is this number? Give your answer in scientific notation.
$$
\frac{(7\times10^{-34})\times(3\times10^8)}{5\times10^{-7}}
$$
You don't need a calculator to solve this problem (\textit{hint: $3/5=0.6$}).
\end{problem}


\vfill ~

\begin{problem} (From the reading)
What musical instrument did Einstein most enjoy playing?
\end{problem}


\vfill ~

\begin{problem} (From Lecture on 2019-09-17)
If you are traveling at 60 miles per hour, how long does
it take you to go 300 miles?
\end{problem}


\vfill ~

\begin{problem} (From Problem Set 1)
What is the relationship between the energy $E$ and wavelength
$\lambda$ of a photon? Give a formula that involves energy $E$,
Planck's Constant $h$, the speed of light $c$, and wavelength
$\lambda$ (or whatever you need).
\end{problem}

\vfill ~


\cleardoublepage



\noindent
Name: \rule[-1ex]{0.60\textwidth}{0.1pt}
NetID: \rule[-1ex]{0.20\textwidth}{0.1pt}

\section*{\textsl{Einstein's Universe} Term Exam 1}
\setcounter{problem}{1}


\begin{problem} (From the reading)
Classical mechanics, or Newtonian mechanics, is only valid in certain
circumstances. When do the laws of classical mechanics, like $F =
m\,a$ for example, become wrong or break down? There are many answers
to this problem; I will take anything correct.
\end{problem}


\vfill ~

\begin{problem} (From Problem Set 1)
What is the approximate thickness of a stack of 1000 20-dollar bills?
No need to be precise, and use any units you like.
\end{problem}


\vfill ~

\begin{problem} (From the Math Review Lab)
What is this number? Give your answer in scientific notation.
$$
\frac{(7\times10^{-34})\times(3\times10^8)}{5\times10^{-7}}
$$
You don't need a calculator to solve this problem (\textit{hint: $3/5=0.6$}).
\end{problem}


\vfill ~

\begin{problem} (From Lecture on 2019-09-05)
The molar weight of water is $18\,\g$. How many molecules would there
be, therefore, in $18\,\g$ of water? You don't need a calculator for
this.
\end{problem}


\vfill ~


\clearpage


\begin{problem} (From Lecture on 2019-09-17)
If you are traveling at 60 miles per hour, how long does
it take you to go 300 miles?
\end{problem}


\vfill ~

\begin{problem} (From the reading)
What musical instrument did Einstein most enjoy playing?
\end{problem}


\vfill ~

\begin{problem} (From Problem Set 1)
What is the relationship between the energy $E$ and wavelength
$\lambda$ of a photon? Give a formula that involves energy $E$,
Planck's Constant $h$, the speed of light $c$, and wavelength
$\lambda$ (or whatever you need).
\end{problem}

\vfill ~

\begin{problem} (From the Kinematics Lab)
Here is a data table of times, positions, and velocities in SI units:\\
\rule{1.0in}{0pt}\begin{tabular}{c|c|c}
time $t$ ($\s$) & position $x$ ($\m$) & velocity $v$ ($\m\,\s^{-1}$) \\
\hline
1 & 1.15 & 1.3 \\
2 & 2.60 & 1.6 \\
3 & 4.35 & 1.9 \\
\hline
\end{tabular}\\
What is the average acceleration in the time interval from $2\,\s$ to $3\,\s$?
\end{problem}


\vfill ~


\cleardoublepage



\noindent
Name: \rule[-1ex]{0.60\textwidth}{0.1pt}
NetID: \rule[-1ex]{0.20\textwidth}{0.1pt}

\section*{\textsl{Einstein's Universe} Term Exam 1}
\setcounter{problem}{1}


\begin{problem} (From the reading)
What musical instrument did Einstein most enjoy playing?
\end{problem}


\vfill ~

\begin{problem} (From the Kinematics Lab)
Here is a data table of times, positions, and velocities in SI units:\\
\rule{1.0in}{0pt}\begin{tabular}{c|c|c}
time $t$ ($\s$) & position $x$ ($\m$) & velocity $v$ ($\m\,\s^{-1}$) \\
\hline
1 & 1.15 & 1.3 \\
2 & 2.60 & 1.6 \\
3 & 4.35 & 1.9 \\
\hline
\end{tabular}\\
What is the average acceleration in the time interval from $2\,\s$ to $3\,\s$?
\end{problem}


\vfill ~

\begin{problem} (From Lecture on 2019-09-05)
The molar weight of water is $18\,\g$. How many molecules would there
be, therefore, in $18\,\g$ of water? You don't need a calculator for
this.
\end{problem}


\vfill ~

\begin{problem} (From Problem Set 1)
What is the relationship between the energy $E$ and wavelength
$\lambda$ of a photon? Give a formula that involves energy $E$,
Planck's Constant $h$, the speed of light $c$, and wavelength
$\lambda$ (or whatever you need).
\end{problem}

\vfill ~


\clearpage


\begin{problem} (From the Math Review Lab)
What is this number? Give your answer in scientific notation.
$$
\frac{(7\times10^{-34})\times(3\times10^8)}{5\times10^{-7}}
$$
You don't need a calculator to solve this problem (\textit{hint: $3/5=0.6$}).
\end{problem}


\vfill ~

\begin{problem} (From Problem Set 1)
What is the approximate thickness of a stack of 1000 20-dollar bills?
No need to be precise, and use any units you like.
\end{problem}


\vfill ~

\begin{problem} (From the reading)
Classical mechanics, or Newtonian mechanics, is only valid in certain
circumstances. When do the laws of classical mechanics, like $F =
m\,a$ for example, become wrong or break down? There are many answers
to this problem; I will take anything correct.
\end{problem}


\vfill ~

\begin{problem} (From Lecture on 2019-09-17)
If you are traveling at 60 miles per hour, how long does
it take you to go 300 miles?
\end{problem}


\vfill ~


\cleardoublepage



\noindent
Name: \rule[-1ex]{0.60\textwidth}{0.1pt}
NetID: \rule[-1ex]{0.20\textwidth}{0.1pt}

\section*{\textsl{Einstein's Universe} Term Exam 1}
\setcounter{problem}{1}


\begin{problem} (From the Math Review Lab)
What is this number? Give your answer in scientific notation.
$$
\frac{(7\times10^{-34})\times(3\times10^8)}{5\times10^{-7}}
$$
You don't need a calculator to solve this problem (\textit{hint: $3/5=0.6$}).
\end{problem}


\vfill ~

\begin{problem} (From Lecture on 2019-09-05)
The molar weight of water is $18\,\g$. How many molecules would there
be, therefore, in $18\,\g$ of water? You don't need a calculator for
this.
\end{problem}


\vfill ~

\begin{problem} (From Problem Set 1)
What is the relationship between the energy $E$ and wavelength
$\lambda$ of a photon? Give a formula that involves energy $E$,
Planck's Constant $h$, the speed of light $c$, and wavelength
$\lambda$ (or whatever you need).
\end{problem}

\vfill ~

\begin{problem} (From Problem Set 1)
What is the approximate thickness of a stack of 1000 20-dollar bills?
No need to be precise, and use any units you like.
\end{problem}


\vfill ~


\clearpage


\begin{problem} (From the reading)
What musical instrument did Einstein most enjoy playing?
\end{problem}


\vfill ~

\begin{problem} (From the reading)
Classical mechanics, or Newtonian mechanics, is only valid in certain
circumstances. When do the laws of classical mechanics, like $F =
m\,a$ for example, become wrong or break down? There are many answers
to this problem; I will take anything correct.
\end{problem}


\vfill ~

\begin{problem} (From Lecture on 2019-09-17)
If you are traveling at 60 miles per hour, how long does
it take you to go 300 miles?
\end{problem}


\vfill ~

\begin{problem} (From the Kinematics Lab)
Here is a data table of times, positions, and velocities in SI units:\\
\rule{1.0in}{0pt}\begin{tabular}{c|c|c}
time $t$ ($\s$) & position $x$ ($\m$) & velocity $v$ ($\m\,\s^{-1}$) \\
\hline
1 & 1.15 & 1.3 \\
2 & 2.60 & 1.6 \\
3 & 4.35 & 1.9 \\
\hline
\end{tabular}\\
What is the average acceleration in the time interval from $2\,\s$ to $3\,\s$?
\end{problem}


\vfill ~


\cleardoublepage



\noindent
Name: \rule[-1ex]{0.60\textwidth}{0.1pt}
NetID: \rule[-1ex]{0.20\textwidth}{0.1pt}

\section*{\textsl{Einstein's Universe} Term Exam 1}
\setcounter{problem}{1}


\begin{problem} (From Lecture on 2019-09-05)
The molar weight of water is $18\,\g$. How many molecules would there
be, therefore, in $18\,\g$ of water? You don't need a calculator for
this.
\end{problem}


\vfill ~

\begin{problem} (From the reading)
What musical instrument did Einstein most enjoy playing?
\end{problem}


\vfill ~

\begin{problem} (From Problem Set 1)
What is the approximate thickness of a stack of 1000 20-dollar bills?
No need to be precise, and use any units you like.
\end{problem}


\vfill ~

\begin{problem} (From Problem Set 1)
What is the relationship between the energy $E$ and wavelength
$\lambda$ of a photon? Give a formula that involves energy $E$,
Planck's Constant $h$, the speed of light $c$, and wavelength
$\lambda$ (or whatever you need).
\end{problem}

\vfill ~


\clearpage


\begin{problem} (From the Kinematics Lab)
Here is a data table of times, positions, and velocities in SI units:\\
\rule{1.0in}{0pt}\begin{tabular}{c|c|c}
time $t$ ($\s$) & position $x$ ($\m$) & velocity $v$ ($\m\,\s^{-1}$) \\
\hline
1 & 1.15 & 1.3 \\
2 & 2.60 & 1.6 \\
3 & 4.35 & 1.9 \\
\hline
\end{tabular}\\
What is the average acceleration in the time interval from $2\,\s$ to $3\,\s$?
\end{problem}


\vfill ~

\begin{problem} (From the Math Review Lab)
What is this number? Give your answer in scientific notation.
$$
\frac{(7\times10^{-34})\times(3\times10^8)}{5\times10^{-7}}
$$
You don't need a calculator to solve this problem (\textit{hint: $3/5=0.6$}).
\end{problem}


\vfill ~

\begin{problem} (From the reading)
Classical mechanics, or Newtonian mechanics, is only valid in certain
circumstances. When do the laws of classical mechanics, like $F =
m\,a$ for example, become wrong or break down? There are many answers
to this problem; I will take anything correct.
\end{problem}


\vfill ~

\begin{problem} (From Lecture on 2019-09-17)
If you are traveling at 60 miles per hour, how long does
it take you to go 300 miles?
\end{problem}


\vfill ~


\cleardoublepage



\noindent
Name: \rule[-1ex]{0.60\textwidth}{0.1pt}
NetID: \rule[-1ex]{0.20\textwidth}{0.1pt}

\section*{\textsl{Einstein's Universe} Term Exam 1}
\setcounter{problem}{1}


\begin{problem} (From the reading)
Classical mechanics, or Newtonian mechanics, is only valid in certain
circumstances. When do the laws of classical mechanics, like $F =
m\,a$ for example, become wrong or break down? There are many answers
to this problem; I will take anything correct.
\end{problem}


\vfill ~

\begin{problem} (From Lecture on 2019-09-17)
If you are traveling at 60 miles per hour, how long does
it take you to go 300 miles?
\end{problem}


\vfill ~

\begin{problem} (From the Math Review Lab)
What is this number? Give your answer in scientific notation.
$$
\frac{(7\times10^{-34})\times(3\times10^8)}{5\times10^{-7}}
$$
You don't need a calculator to solve this problem (\textit{hint: $3/5=0.6$}).
\end{problem}


\vfill ~

\begin{problem} (From Lecture on 2019-09-05)
The molar weight of water is $18\,\g$. How many molecules would there
be, therefore, in $18\,\g$ of water? You don't need a calculator for
this.
\end{problem}


\vfill ~


\clearpage


\begin{problem} (From the reading)
What musical instrument did Einstein most enjoy playing?
\end{problem}


\vfill ~

\begin{problem} (From Problem Set 1)
What is the approximate thickness of a stack of 1000 20-dollar bills?
No need to be precise, and use any units you like.
\end{problem}


\vfill ~

\begin{problem} (From Problem Set 1)
What is the relationship between the energy $E$ and wavelength
$\lambda$ of a photon? Give a formula that involves energy $E$,
Planck's Constant $h$, the speed of light $c$, and wavelength
$\lambda$ (or whatever you need).
\end{problem}

\vfill ~

\begin{problem} (From the Kinematics Lab)
Here is a data table of times, positions, and velocities in SI units:\\
\rule{1.0in}{0pt}\begin{tabular}{c|c|c}
time $t$ ($\s$) & position $x$ ($\m$) & velocity $v$ ($\m\,\s^{-1}$) \\
\hline
1 & 1.15 & 1.3 \\
2 & 2.60 & 1.6 \\
3 & 4.35 & 1.9 \\
\hline
\end{tabular}\\
What is the average acceleration in the time interval from $2\,\s$ to $3\,\s$?
\end{problem}


\vfill ~


\cleardoublepage



\noindent
Name: \rule[-1ex]{0.60\textwidth}{0.1pt}
NetID: \rule[-1ex]{0.20\textwidth}{0.1pt}

\section*{\textsl{Einstein's Universe} Term Exam 1}
\setcounter{problem}{1}


\begin{problem} (From Lecture on 2019-09-05)
The molar weight of water is $18\,\g$. How many molecules would there
be, therefore, in $18\,\g$ of water? You don't need a calculator for
this.
\end{problem}


\vfill ~

\begin{problem} (From the Math Review Lab)
What is this number? Give your answer in scientific notation.
$$
\frac{(7\times10^{-34})\times(3\times10^8)}{5\times10^{-7}}
$$
You don't need a calculator to solve this problem (\textit{hint: $3/5=0.6$}).
\end{problem}


\vfill ~

\begin{problem} (From Problem Set 1)
What is the approximate thickness of a stack of 1000 20-dollar bills?
No need to be precise, and use any units you like.
\end{problem}


\vfill ~

\begin{problem} (From Lecture on 2019-09-17)
If you are traveling at 60 miles per hour, how long does
it take you to go 300 miles?
\end{problem}


\vfill ~


\clearpage


\begin{problem} (From Problem Set 1)
What is the relationship between the energy $E$ and wavelength
$\lambda$ of a photon? Give a formula that involves energy $E$,
Planck's Constant $h$, the speed of light $c$, and wavelength
$\lambda$ (or whatever you need).
\end{problem}

\vfill ~

\begin{problem} (From the Kinematics Lab)
Here is a data table of times, positions, and velocities in SI units:\\
\rule{1.0in}{0pt}\begin{tabular}{c|c|c}
time $t$ ($\s$) & position $x$ ($\m$) & velocity $v$ ($\m\,\s^{-1}$) \\
\hline
1 & 1.15 & 1.3 \\
2 & 2.60 & 1.6 \\
3 & 4.35 & 1.9 \\
\hline
\end{tabular}\\
What is the average acceleration in the time interval from $2\,\s$ to $3\,\s$?
\end{problem}


\vfill ~

\begin{problem} (From the reading)
What musical instrument did Einstein most enjoy playing?
\end{problem}


\vfill ~

\begin{problem} (From the reading)
Classical mechanics, or Newtonian mechanics, is only valid in certain
circumstances. When do the laws of classical mechanics, like $F =
m\,a$ for example, become wrong or break down? There are many answers
to this problem; I will take anything correct.
\end{problem}


\vfill ~


\cleardoublepage



\noindent
Name: \rule[-1ex]{0.60\textwidth}{0.1pt}
NetID: \rule[-1ex]{0.20\textwidth}{0.1pt}

\section*{\textsl{Einstein's Universe} Term Exam 1}
\setcounter{problem}{1}


\begin{problem} (From the reading)
What musical instrument did Einstein most enjoy playing?
\end{problem}


\vfill ~

\begin{problem} (From the Math Review Lab)
What is this number? Give your answer in scientific notation.
$$
\frac{(7\times10^{-34})\times(3\times10^8)}{5\times10^{-7}}
$$
You don't need a calculator to solve this problem (\textit{hint: $3/5=0.6$}).
\end{problem}


\vfill ~

\begin{problem} (From the reading)
Classical mechanics, or Newtonian mechanics, is only valid in certain
circumstances. When do the laws of classical mechanics, like $F =
m\,a$ for example, become wrong or break down? There are many answers
to this problem; I will take anything correct.
\end{problem}


\vfill ~

\begin{problem} (From Lecture on 2019-09-05)
The molar weight of water is $18\,\g$. How many molecules would there
be, therefore, in $18\,\g$ of water? You don't need a calculator for
this.
\end{problem}


\vfill ~


\clearpage


\begin{problem} (From Problem Set 1)
What is the relationship between the energy $E$ and wavelength
$\lambda$ of a photon? Give a formula that involves energy $E$,
Planck's Constant $h$, the speed of light $c$, and wavelength
$\lambda$ (or whatever you need).
\end{problem}

\vfill ~

\begin{problem} (From Problem Set 1)
What is the approximate thickness of a stack of 1000 20-dollar bills?
No need to be precise, and use any units you like.
\end{problem}


\vfill ~

\begin{problem} (From the Kinematics Lab)
Here is a data table of times, positions, and velocities in SI units:\\
\rule{1.0in}{0pt}\begin{tabular}{c|c|c}
time $t$ ($\s$) & position $x$ ($\m$) & velocity $v$ ($\m\,\s^{-1}$) \\
\hline
1 & 1.15 & 1.3 \\
2 & 2.60 & 1.6 \\
3 & 4.35 & 1.9 \\
\hline
\end{tabular}\\
What is the average acceleration in the time interval from $2\,\s$ to $3\,\s$?
\end{problem}


\vfill ~

\begin{problem} (From Lecture on 2019-09-17)
If you are traveling at 60 miles per hour, how long does
it take you to go 300 miles?
\end{problem}


\vfill ~


\cleardoublepage



\noindent
Name: \rule[-1ex]{0.60\textwidth}{0.1pt}
NetID: \rule[-1ex]{0.20\textwidth}{0.1pt}

\section*{\textsl{Einstein's Universe} Term Exam 1}
\setcounter{problem}{1}


\begin{problem} (From the Math Review Lab)
What is this number? Give your answer in scientific notation.
$$
\frac{(7\times10^{-34})\times(3\times10^8)}{5\times10^{-7}}
$$
You don't need a calculator to solve this problem (\textit{hint: $3/5=0.6$}).
\end{problem}


\vfill ~

\begin{problem} (From the reading)
Classical mechanics, or Newtonian mechanics, is only valid in certain
circumstances. When do the laws of classical mechanics, like $F =
m\,a$ for example, become wrong or break down? There are many answers
to this problem; I will take anything correct.
\end{problem}


\vfill ~

\begin{problem} (From the reading)
What musical instrument did Einstein most enjoy playing?
\end{problem}


\vfill ~

\begin{problem} (From Lecture on 2019-09-17)
If you are traveling at 60 miles per hour, how long does
it take you to go 300 miles?
\end{problem}


\vfill ~


\clearpage


\begin{problem} (From Problem Set 1)
What is the approximate thickness of a stack of 1000 20-dollar bills?
No need to be precise, and use any units you like.
\end{problem}


\vfill ~

\begin{problem} (From Problem Set 1)
What is the relationship between the energy $E$ and wavelength
$\lambda$ of a photon? Give a formula that involves energy $E$,
Planck's Constant $h$, the speed of light $c$, and wavelength
$\lambda$ (or whatever you need).
\end{problem}

\vfill ~

\begin{problem} (From the Kinematics Lab)
Here is a data table of times, positions, and velocities in SI units:\\
\rule{1.0in}{0pt}\begin{tabular}{c|c|c}
time $t$ ($\s$) & position $x$ ($\m$) & velocity $v$ ($\m\,\s^{-1}$) \\
\hline
1 & 1.15 & 1.3 \\
2 & 2.60 & 1.6 \\
3 & 4.35 & 1.9 \\
\hline
\end{tabular}\\
What is the average acceleration in the time interval from $2\,\s$ to $3\,\s$?
\end{problem}


\vfill ~

\begin{problem} (From Lecture on 2019-09-05)
The molar weight of water is $18\,\g$. How many molecules would there
be, therefore, in $18\,\g$ of water? You don't need a calculator for
this.
\end{problem}


\vfill ~


\cleardoublepage



\noindent
Name: \rule[-1ex]{0.60\textwidth}{0.1pt}
NetID: \rule[-1ex]{0.20\textwidth}{0.1pt}

\section*{\textsl{Einstein's Universe} Term Exam 1}
\setcounter{problem}{1}


\begin{problem} (From Problem Set 1)
What is the approximate thickness of a stack of 1000 20-dollar bills?
No need to be precise, and use any units you like.
\end{problem}


\vfill ~

\begin{problem} (From the reading)
What musical instrument did Einstein most enjoy playing?
\end{problem}


\vfill ~

\begin{problem} (From the Math Review Lab)
What is this number? Give your answer in scientific notation.
$$
\frac{(7\times10^{-34})\times(3\times10^8)}{5\times10^{-7}}
$$
You don't need a calculator to solve this problem (\textit{hint: $3/5=0.6$}).
\end{problem}


\vfill ~

\begin{problem} (From the reading)
Classical mechanics, or Newtonian mechanics, is only valid in certain
circumstances. When do the laws of classical mechanics, like $F =
m\,a$ for example, become wrong or break down? There are many answers
to this problem; I will take anything correct.
\end{problem}


\vfill ~


\clearpage


\begin{problem} (From the Kinematics Lab)
Here is a data table of times, positions, and velocities in SI units:\\
\rule{1.0in}{0pt}\begin{tabular}{c|c|c}
time $t$ ($\s$) & position $x$ ($\m$) & velocity $v$ ($\m\,\s^{-1}$) \\
\hline
1 & 1.15 & 1.3 \\
2 & 2.60 & 1.6 \\
3 & 4.35 & 1.9 \\
\hline
\end{tabular}\\
What is the average acceleration in the time interval from $2\,\s$ to $3\,\s$?
\end{problem}


\vfill ~

\begin{problem} (From Lecture on 2019-09-05)
The molar weight of water is $18\,\g$. How many molecules would there
be, therefore, in $18\,\g$ of water? You don't need a calculator for
this.
\end{problem}


\vfill ~

\begin{problem} (From Problem Set 1)
What is the relationship between the energy $E$ and wavelength
$\lambda$ of a photon? Give a formula that involves energy $E$,
Planck's Constant $h$, the speed of light $c$, and wavelength
$\lambda$ (or whatever you need).
\end{problem}

\vfill ~

\begin{problem} (From Lecture on 2019-09-17)
If you are traveling at 60 miles per hour, how long does
it take you to go 300 miles?
\end{problem}


\vfill ~


\cleardoublepage



\noindent
Name: \rule[-1ex]{0.60\textwidth}{0.1pt}
NetID: \rule[-1ex]{0.20\textwidth}{0.1pt}

\section*{\textsl{Einstein's Universe} Term Exam 1}
\setcounter{problem}{1}


\begin{problem} (From Lecture on 2019-09-17)
If you are traveling at 60 miles per hour, how long does
it take you to go 300 miles?
\end{problem}


\vfill ~

\begin{problem} (From the reading)
What musical instrument did Einstein most enjoy playing?
\end{problem}


\vfill ~

\begin{problem} (From Lecture on 2019-09-05)
The molar weight of water is $18\,\g$. How many molecules would there
be, therefore, in $18\,\g$ of water? You don't need a calculator for
this.
\end{problem}


\vfill ~

\begin{problem} (From Problem Set 1)
What is the relationship between the energy $E$ and wavelength
$\lambda$ of a photon? Give a formula that involves energy $E$,
Planck's Constant $h$, the speed of light $c$, and wavelength
$\lambda$ (or whatever you need).
\end{problem}

\vfill ~


\clearpage


\begin{problem} (From the Math Review Lab)
What is this number? Give your answer in scientific notation.
$$
\frac{(7\times10^{-34})\times(3\times10^8)}{5\times10^{-7}}
$$
You don't need a calculator to solve this problem (\textit{hint: $3/5=0.6$}).
\end{problem}


\vfill ~

\begin{problem} (From Problem Set 1)
What is the approximate thickness of a stack of 1000 20-dollar bills?
No need to be precise, and use any units you like.
\end{problem}


\vfill ~

\begin{problem} (From the Kinematics Lab)
Here is a data table of times, positions, and velocities in SI units:\\
\rule{1.0in}{0pt}\begin{tabular}{c|c|c}
time $t$ ($\s$) & position $x$ ($\m$) & velocity $v$ ($\m\,\s^{-1}$) \\
\hline
1 & 1.15 & 1.3 \\
2 & 2.60 & 1.6 \\
3 & 4.35 & 1.9 \\
\hline
\end{tabular}\\
What is the average acceleration in the time interval from $2\,\s$ to $3\,\s$?
\end{problem}


\vfill ~

\begin{problem} (From the reading)
Classical mechanics, or Newtonian mechanics, is only valid in certain
circumstances. When do the laws of classical mechanics, like $F =
m\,a$ for example, become wrong or break down? There are many answers
to this problem; I will take anything correct.
\end{problem}


\vfill ~


\cleardoublepage



\noindent
Name: \rule[-1ex]{0.60\textwidth}{0.1pt}
NetID: \rule[-1ex]{0.20\textwidth}{0.1pt}

\section*{\textsl{Einstein's Universe} Term Exam 1}
\setcounter{problem}{1}


\begin{problem} (From Problem Set 1)
What is the approximate thickness of a stack of 1000 20-dollar bills?
No need to be precise, and use any units you like.
\end{problem}


\vfill ~

\begin{problem} (From the Kinematics Lab)
Here is a data table of times, positions, and velocities in SI units:\\
\rule{1.0in}{0pt}\begin{tabular}{c|c|c}
time $t$ ($\s$) & position $x$ ($\m$) & velocity $v$ ($\m\,\s^{-1}$) \\
\hline
1 & 1.15 & 1.3 \\
2 & 2.60 & 1.6 \\
3 & 4.35 & 1.9 \\
\hline
\end{tabular}\\
What is the average acceleration in the time interval from $2\,\s$ to $3\,\s$?
\end{problem}


\vfill ~

\begin{problem} (From the Math Review Lab)
What is this number? Give your answer in scientific notation.
$$
\frac{(7\times10^{-34})\times(3\times10^8)}{5\times10^{-7}}
$$
You don't need a calculator to solve this problem (\textit{hint: $3/5=0.6$}).
\end{problem}


\vfill ~

\begin{problem} (From the reading)
What musical instrument did Einstein most enjoy playing?
\end{problem}


\vfill ~


\clearpage


\begin{problem} (From the reading)
Classical mechanics, or Newtonian mechanics, is only valid in certain
circumstances. When do the laws of classical mechanics, like $F =
m\,a$ for example, become wrong or break down? There are many answers
to this problem; I will take anything correct.
\end{problem}


\vfill ~

\begin{problem} (From Lecture on 2019-09-17)
If you are traveling at 60 miles per hour, how long does
it take you to go 300 miles?
\end{problem}


\vfill ~

\begin{problem} (From Problem Set 1)
What is the relationship between the energy $E$ and wavelength
$\lambda$ of a photon? Give a formula that involves energy $E$,
Planck's Constant $h$, the speed of light $c$, and wavelength
$\lambda$ (or whatever you need).
\end{problem}

\vfill ~

\begin{problem} (From Lecture on 2019-09-05)
The molar weight of water is $18\,\g$. How many molecules would there
be, therefore, in $18\,\g$ of water? You don't need a calculator for
this.
\end{problem}


\vfill ~


\cleardoublepage



\noindent
Name: \rule[-1ex]{0.60\textwidth}{0.1pt}
NetID: \rule[-1ex]{0.20\textwidth}{0.1pt}

\section*{\textsl{Einstein's Universe} Term Exam 1}
\setcounter{problem}{1}


\begin{problem} (From Problem Set 1)
What is the relationship between the energy $E$ and wavelength
$\lambda$ of a photon? Give a formula that involves energy $E$,
Planck's Constant $h$, the speed of light $c$, and wavelength
$\lambda$ (or whatever you need).
\end{problem}

\vfill ~

\begin{problem} (From the reading)
Classical mechanics, or Newtonian mechanics, is only valid in certain
circumstances. When do the laws of classical mechanics, like $F =
m\,a$ for example, become wrong or break down? There are many answers
to this problem; I will take anything correct.
\end{problem}


\vfill ~

\begin{problem} (From Lecture on 2019-09-17)
If you are traveling at 60 miles per hour, how long does
it take you to go 300 miles?
\end{problem}


\vfill ~

\begin{problem} (From the Kinematics Lab)
Here is a data table of times, positions, and velocities in SI units:\\
\rule{1.0in}{0pt}\begin{tabular}{c|c|c}
time $t$ ($\s$) & position $x$ ($\m$) & velocity $v$ ($\m\,\s^{-1}$) \\
\hline
1 & 1.15 & 1.3 \\
2 & 2.60 & 1.6 \\
3 & 4.35 & 1.9 \\
\hline
\end{tabular}\\
What is the average acceleration in the time interval from $2\,\s$ to $3\,\s$?
\end{problem}


\vfill ~


\clearpage


\begin{problem} (From Problem Set 1)
What is the approximate thickness of a stack of 1000 20-dollar bills?
No need to be precise, and use any units you like.
\end{problem}


\vfill ~

\begin{problem} (From the Math Review Lab)
What is this number? Give your answer in scientific notation.
$$
\frac{(7\times10^{-34})\times(3\times10^8)}{5\times10^{-7}}
$$
You don't need a calculator to solve this problem (\textit{hint: $3/5=0.6$}).
\end{problem}


\vfill ~

\begin{problem} (From the reading)
What musical instrument did Einstein most enjoy playing?
\end{problem}


\vfill ~

\begin{problem} (From Lecture on 2019-09-05)
The molar weight of water is $18\,\g$. How many molecules would there
be, therefore, in $18\,\g$ of water? You don't need a calculator for
this.
\end{problem}


\vfill ~


\cleardoublepage



\noindent
Name: \rule[-1ex]{0.60\textwidth}{0.1pt}
NetID: \rule[-1ex]{0.20\textwidth}{0.1pt}

\section*{\textsl{Einstein's Universe} Term Exam 1}
\setcounter{problem}{1}


\begin{problem} (From the reading)
What musical instrument did Einstein most enjoy playing?
\end{problem}


\vfill ~

\begin{problem} (From Problem Set 1)
What is the relationship between the energy $E$ and wavelength
$\lambda$ of a photon? Give a formula that involves energy $E$,
Planck's Constant $h$, the speed of light $c$, and wavelength
$\lambda$ (or whatever you need).
\end{problem}

\vfill ~

\begin{problem} (From the Kinematics Lab)
Here is a data table of times, positions, and velocities in SI units:\\
\rule{1.0in}{0pt}\begin{tabular}{c|c|c}
time $t$ ($\s$) & position $x$ ($\m$) & velocity $v$ ($\m\,\s^{-1}$) \\
\hline
1 & 1.15 & 1.3 \\
2 & 2.60 & 1.6 \\
3 & 4.35 & 1.9 \\
\hline
\end{tabular}\\
What is the average acceleration in the time interval from $2\,\s$ to $3\,\s$?
\end{problem}


\vfill ~

\begin{problem} (From Lecture on 2019-09-17)
If you are traveling at 60 miles per hour, how long does
it take you to go 300 miles?
\end{problem}


\vfill ~


\clearpage


\begin{problem} (From the Math Review Lab)
What is this number? Give your answer in scientific notation.
$$
\frac{(7\times10^{-34})\times(3\times10^8)}{5\times10^{-7}}
$$
You don't need a calculator to solve this problem (\textit{hint: $3/5=0.6$}).
\end{problem}


\vfill ~

\begin{problem} (From Problem Set 1)
What is the approximate thickness of a stack of 1000 20-dollar bills?
No need to be precise, and use any units you like.
\end{problem}


\vfill ~

\begin{problem} (From Lecture on 2019-09-05)
The molar weight of water is $18\,\g$. How many molecules would there
be, therefore, in $18\,\g$ of water? You don't need a calculator for
this.
\end{problem}


\vfill ~

\begin{problem} (From the reading)
Classical mechanics, or Newtonian mechanics, is only valid in certain
circumstances. When do the laws of classical mechanics, like $F =
m\,a$ for example, become wrong or break down? There are many answers
to this problem; I will take anything correct.
\end{problem}


\vfill ~


\cleardoublepage



\noindent
Name: \rule[-1ex]{0.60\textwidth}{0.1pt}
NetID: \rule[-1ex]{0.20\textwidth}{0.1pt}

\section*{\textsl{Einstein's Universe} Term Exam 1}
\setcounter{problem}{1}


\begin{problem} (From Lecture on 2019-09-17)
If you are traveling at 60 miles per hour, how long does
it take you to go 300 miles?
\end{problem}


\vfill ~

\begin{problem} (From the reading)
What musical instrument did Einstein most enjoy playing?
\end{problem}


\vfill ~

\begin{problem} (From the reading)
Classical mechanics, or Newtonian mechanics, is only valid in certain
circumstances. When do the laws of classical mechanics, like $F =
m\,a$ for example, become wrong or break down? There are many answers
to this problem; I will take anything correct.
\end{problem}


\vfill ~

\begin{problem} (From the Kinematics Lab)
Here is a data table of times, positions, and velocities in SI units:\\
\rule{1.0in}{0pt}\begin{tabular}{c|c|c}
time $t$ ($\s$) & position $x$ ($\m$) & velocity $v$ ($\m\,\s^{-1}$) \\
\hline
1 & 1.15 & 1.3 \\
2 & 2.60 & 1.6 \\
3 & 4.35 & 1.9 \\
\hline
\end{tabular}\\
What is the average acceleration in the time interval from $2\,\s$ to $3\,\s$?
\end{problem}


\vfill ~


\clearpage


\begin{problem} (From Problem Set 1)
What is the approximate thickness of a stack of 1000 20-dollar bills?
No need to be precise, and use any units you like.
\end{problem}


\vfill ~

\begin{problem} (From Lecture on 2019-09-05)
The molar weight of water is $18\,\g$. How many molecules would there
be, therefore, in $18\,\g$ of water? You don't need a calculator for
this.
\end{problem}


\vfill ~

\begin{problem} (From the Math Review Lab)
What is this number? Give your answer in scientific notation.
$$
\frac{(7\times10^{-34})\times(3\times10^8)}{5\times10^{-7}}
$$
You don't need a calculator to solve this problem (\textit{hint: $3/5=0.6$}).
\end{problem}


\vfill ~

\begin{problem} (From Problem Set 1)
What is the relationship between the energy $E$ and wavelength
$\lambda$ of a photon? Give a formula that involves energy $E$,
Planck's Constant $h$, the speed of light $c$, and wavelength
$\lambda$ (or whatever you need).
\end{problem}

\vfill ~


\cleardoublepage



\noindent
Name: \rule[-1ex]{0.60\textwidth}{0.1pt}
NetID: \rule[-1ex]{0.20\textwidth}{0.1pt}

\section*{\textsl{Einstein's Universe} Term Exam 1}
\setcounter{problem}{1}


\begin{problem} (From Lecture on 2019-09-17)
If you are traveling at 60 miles per hour, how long does
it take you to go 300 miles?
\end{problem}


\vfill ~

\begin{problem} (From the Math Review Lab)
What is this number? Give your answer in scientific notation.
$$
\frac{(7\times10^{-34})\times(3\times10^8)}{5\times10^{-7}}
$$
You don't need a calculator to solve this problem (\textit{hint: $3/5=0.6$}).
\end{problem}


\vfill ~

\begin{problem} (From Problem Set 1)
What is the approximate thickness of a stack of 1000 20-dollar bills?
No need to be precise, and use any units you like.
\end{problem}


\vfill ~

\begin{problem} (From Problem Set 1)
What is the relationship between the energy $E$ and wavelength
$\lambda$ of a photon? Give a formula that involves energy $E$,
Planck's Constant $h$, the speed of light $c$, and wavelength
$\lambda$ (or whatever you need).
\end{problem}

\vfill ~


\clearpage


\begin{problem} (From the Kinematics Lab)
Here is a data table of times, positions, and velocities in SI units:\\
\rule{1.0in}{0pt}\begin{tabular}{c|c|c}
time $t$ ($\s$) & position $x$ ($\m$) & velocity $v$ ($\m\,\s^{-1}$) \\
\hline
1 & 1.15 & 1.3 \\
2 & 2.60 & 1.6 \\
3 & 4.35 & 1.9 \\
\hline
\end{tabular}\\
What is the average acceleration in the time interval from $2\,\s$ to $3\,\s$?
\end{problem}


\vfill ~

\begin{problem} (From the reading)
What musical instrument did Einstein most enjoy playing?
\end{problem}


\vfill ~

\begin{problem} (From Lecture on 2019-09-05)
The molar weight of water is $18\,\g$. How many molecules would there
be, therefore, in $18\,\g$ of water? You don't need a calculator for
this.
\end{problem}


\vfill ~

\begin{problem} (From the reading)
Classical mechanics, or Newtonian mechanics, is only valid in certain
circumstances. When do the laws of classical mechanics, like $F =
m\,a$ for example, become wrong or break down? There are many answers
to this problem; I will take anything correct.
\end{problem}


\vfill ~


\cleardoublepage



\noindent
Name: \rule[-1ex]{0.60\textwidth}{0.1pt}
NetID: \rule[-1ex]{0.20\textwidth}{0.1pt}

\section*{\textsl{Einstein's Universe} Term Exam 1}
\setcounter{problem}{1}


\begin{problem} (From the Math Review Lab)
What is this number? Give your answer in scientific notation.
$$
\frac{(7\times10^{-34})\times(3\times10^8)}{5\times10^{-7}}
$$
You don't need a calculator to solve this problem (\textit{hint: $3/5=0.6$}).
\end{problem}


\vfill ~

\begin{problem} (From the reading)
Classical mechanics, or Newtonian mechanics, is only valid in certain
circumstances. When do the laws of classical mechanics, like $F =
m\,a$ for example, become wrong or break down? There are many answers
to this problem; I will take anything correct.
\end{problem}


\vfill ~

\begin{problem} (From Lecture on 2019-09-05)
The molar weight of water is $18\,\g$. How many molecules would there
be, therefore, in $18\,\g$ of water? You don't need a calculator for
this.
\end{problem}


\vfill ~

\begin{problem} (From Lecture on 2019-09-17)
If you are traveling at 60 miles per hour, how long does
it take you to go 300 miles?
\end{problem}


\vfill ~


\clearpage


\begin{problem} (From Problem Set 1)
What is the relationship between the energy $E$ and wavelength
$\lambda$ of a photon? Give a formula that involves energy $E$,
Planck's Constant $h$, the speed of light $c$, and wavelength
$\lambda$ (or whatever you need).
\end{problem}

\vfill ~

\begin{problem} (From the reading)
What musical instrument did Einstein most enjoy playing?
\end{problem}


\vfill ~

\begin{problem} (From the Kinematics Lab)
Here is a data table of times, positions, and velocities in SI units:\\
\rule{1.0in}{0pt}\begin{tabular}{c|c|c}
time $t$ ($\s$) & position $x$ ($\m$) & velocity $v$ ($\m\,\s^{-1}$) \\
\hline
1 & 1.15 & 1.3 \\
2 & 2.60 & 1.6 \\
3 & 4.35 & 1.9 \\
\hline
\end{tabular}\\
What is the average acceleration in the time interval from $2\,\s$ to $3\,\s$?
\end{problem}


\vfill ~

\begin{problem} (From Problem Set 1)
What is the approximate thickness of a stack of 1000 20-dollar bills?
No need to be precise, and use any units you like.
\end{problem}


\vfill ~


\cleardoublepage



\noindent
Name: \rule[-1ex]{0.60\textwidth}{0.1pt}
NetID: \rule[-1ex]{0.20\textwidth}{0.1pt}

\section*{\textsl{Einstein's Universe} Term Exam 1}
\setcounter{problem}{1}


\begin{problem} (From the reading)
What musical instrument did Einstein most enjoy playing?
\end{problem}


\vfill ~

\begin{problem} (From Problem Set 1)
What is the relationship between the energy $E$ and wavelength
$\lambda$ of a photon? Give a formula that involves energy $E$,
Planck's Constant $h$, the speed of light $c$, and wavelength
$\lambda$ (or whatever you need).
\end{problem}

\vfill ~

\begin{problem} (From the Math Review Lab)
What is this number? Give your answer in scientific notation.
$$
\frac{(7\times10^{-34})\times(3\times10^8)}{5\times10^{-7}}
$$
You don't need a calculator to solve this problem (\textit{hint: $3/5=0.6$}).
\end{problem}


\vfill ~

\begin{problem} (From Problem Set 1)
What is the approximate thickness of a stack of 1000 20-dollar bills?
No need to be precise, and use any units you like.
\end{problem}


\vfill ~


\clearpage


\begin{problem} (From Lecture on 2019-09-05)
The molar weight of water is $18\,\g$. How many molecules would there
be, therefore, in $18\,\g$ of water? You don't need a calculator for
this.
\end{problem}


\vfill ~

\begin{problem} (From Lecture on 2019-09-17)
If you are traveling at 60 miles per hour, how long does
it take you to go 300 miles?
\end{problem}


\vfill ~

\begin{problem} (From the Kinematics Lab)
Here is a data table of times, positions, and velocities in SI units:\\
\rule{1.0in}{0pt}\begin{tabular}{c|c|c}
time $t$ ($\s$) & position $x$ ($\m$) & velocity $v$ ($\m\,\s^{-1}$) \\
\hline
1 & 1.15 & 1.3 \\
2 & 2.60 & 1.6 \\
3 & 4.35 & 1.9 \\
\hline
\end{tabular}\\
What is the average acceleration in the time interval from $2\,\s$ to $3\,\s$?
\end{problem}


\vfill ~

\begin{problem} (From the reading)
Classical mechanics, or Newtonian mechanics, is only valid in certain
circumstances. When do the laws of classical mechanics, like $F =
m\,a$ for example, become wrong or break down? There are many answers
to this problem; I will take anything correct.
\end{problem}


\vfill ~


\cleardoublepage



\noindent
Name: \rule[-1ex]{0.60\textwidth}{0.1pt}
NetID: \rule[-1ex]{0.20\textwidth}{0.1pt}

\section*{\textsl{Einstein's Universe} Term Exam 1}
\setcounter{problem}{1}


\begin{problem} (From the Kinematics Lab)
Here is a data table of times, positions, and velocities in SI units:\\
\rule{1.0in}{0pt}\begin{tabular}{c|c|c}
time $t$ ($\s$) & position $x$ ($\m$) & velocity $v$ ($\m\,\s^{-1}$) \\
\hline
1 & 1.15 & 1.3 \\
2 & 2.60 & 1.6 \\
3 & 4.35 & 1.9 \\
\hline
\end{tabular}\\
What is the average acceleration in the time interval from $2\,\s$ to $3\,\s$?
\end{problem}


\vfill ~

\begin{problem} (From the Math Review Lab)
What is this number? Give your answer in scientific notation.
$$
\frac{(7\times10^{-34})\times(3\times10^8)}{5\times10^{-7}}
$$
You don't need a calculator to solve this problem (\textit{hint: $3/5=0.6$}).
\end{problem}


\vfill ~

\begin{problem} (From the reading)
What musical instrument did Einstein most enjoy playing?
\end{problem}


\vfill ~

\begin{problem} (From Lecture on 2019-09-17)
If you are traveling at 60 miles per hour, how long does
it take you to go 300 miles?
\end{problem}


\vfill ~


\clearpage


\begin{problem} (From Problem Set 1)
What is the relationship between the energy $E$ and wavelength
$\lambda$ of a photon? Give a formula that involves energy $E$,
Planck's Constant $h$, the speed of light $c$, and wavelength
$\lambda$ (or whatever you need).
\end{problem}

\vfill ~

\begin{problem} (From the reading)
Classical mechanics, or Newtonian mechanics, is only valid in certain
circumstances. When do the laws of classical mechanics, like $F =
m\,a$ for example, become wrong or break down? There are many answers
to this problem; I will take anything correct.
\end{problem}


\vfill ~

\begin{problem} (From Lecture on 2019-09-05)
The molar weight of water is $18\,\g$. How many molecules would there
be, therefore, in $18\,\g$ of water? You don't need a calculator for
this.
\end{problem}


\vfill ~

\begin{problem} (From Problem Set 1)
What is the approximate thickness of a stack of 1000 20-dollar bills?
No need to be precise, and use any units you like.
\end{problem}


\vfill ~


\cleardoublepage



\noindent
Name: \rule[-1ex]{0.60\textwidth}{0.1pt}
NetID: \rule[-1ex]{0.20\textwidth}{0.1pt}

\section*{\textsl{Einstein's Universe} Term Exam 1}
\setcounter{problem}{1}


\begin{problem} (From Problem Set 1)
What is the relationship between the energy $E$ and wavelength
$\lambda$ of a photon? Give a formula that involves energy $E$,
Planck's Constant $h$, the speed of light $c$, and wavelength
$\lambda$ (or whatever you need).
\end{problem}

\vfill ~

\begin{problem} (From Lecture on 2019-09-17)
If you are traveling at 60 miles per hour, how long does
it take you to go 300 miles?
\end{problem}


\vfill ~

\begin{problem} (From Lecture on 2019-09-05)
The molar weight of water is $18\,\g$. How many molecules would there
be, therefore, in $18\,\g$ of water? You don't need a calculator for
this.
\end{problem}


\vfill ~

\begin{problem} (From the reading)
What musical instrument did Einstein most enjoy playing?
\end{problem}


\vfill ~


\clearpage


\begin{problem} (From the Math Review Lab)
What is this number? Give your answer in scientific notation.
$$
\frac{(7\times10^{-34})\times(3\times10^8)}{5\times10^{-7}}
$$
You don't need a calculator to solve this problem (\textit{hint: $3/5=0.6$}).
\end{problem}


\vfill ~

\begin{problem} (From Problem Set 1)
What is the approximate thickness of a stack of 1000 20-dollar bills?
No need to be precise, and use any units you like.
\end{problem}


\vfill ~

\begin{problem} (From the reading)
Classical mechanics, or Newtonian mechanics, is only valid in certain
circumstances. When do the laws of classical mechanics, like $F =
m\,a$ for example, become wrong or break down? There are many answers
to this problem; I will take anything correct.
\end{problem}


\vfill ~

\begin{problem} (From the Kinematics Lab)
Here is a data table of times, positions, and velocities in SI units:\\
\rule{1.0in}{0pt}\begin{tabular}{c|c|c}
time $t$ ($\s$) & position $x$ ($\m$) & velocity $v$ ($\m\,\s^{-1}$) \\
\hline
1 & 1.15 & 1.3 \\
2 & 2.60 & 1.6 \\
3 & 4.35 & 1.9 \\
\hline
\end{tabular}\\
What is the average acceleration in the time interval from $2\,\s$ to $3\,\s$?
\end{problem}


\vfill ~


\cleardoublepage



\noindent
Name: \rule[-1ex]{0.60\textwidth}{0.1pt}
NetID: \rule[-1ex]{0.20\textwidth}{0.1pt}

\section*{\textsl{Einstein's Universe} Term Exam 1}
\setcounter{problem}{1}


\begin{problem} (From Lecture on 2019-09-17)
If you are traveling at 60 miles per hour, how long does
it take you to go 300 miles?
\end{problem}


\vfill ~

\begin{problem} (From Lecture on 2019-09-05)
The molar weight of water is $18\,\g$. How many molecules would there
be, therefore, in $18\,\g$ of water? You don't need a calculator for
this.
\end{problem}


\vfill ~

\begin{problem} (From Problem Set 1)
What is the relationship between the energy $E$ and wavelength
$\lambda$ of a photon? Give a formula that involves energy $E$,
Planck's Constant $h$, the speed of light $c$, and wavelength
$\lambda$ (or whatever you need).
\end{problem}

\vfill ~

\begin{problem} (From Problem Set 1)
What is the approximate thickness of a stack of 1000 20-dollar bills?
No need to be precise, and use any units you like.
\end{problem}


\vfill ~


\clearpage


\begin{problem} (From the Math Review Lab)
What is this number? Give your answer in scientific notation.
$$
\frac{(7\times10^{-34})\times(3\times10^8)}{5\times10^{-7}}
$$
You don't need a calculator to solve this problem (\textit{hint: $3/5=0.6$}).
\end{problem}


\vfill ~

\begin{problem} (From the reading)
What musical instrument did Einstein most enjoy playing?
\end{problem}


\vfill ~

\begin{problem} (From the Kinematics Lab)
Here is a data table of times, positions, and velocities in SI units:\\
\rule{1.0in}{0pt}\begin{tabular}{c|c|c}
time $t$ ($\s$) & position $x$ ($\m$) & velocity $v$ ($\m\,\s^{-1}$) \\
\hline
1 & 1.15 & 1.3 \\
2 & 2.60 & 1.6 \\
3 & 4.35 & 1.9 \\
\hline
\end{tabular}\\
What is the average acceleration in the time interval from $2\,\s$ to $3\,\s$?
\end{problem}


\vfill ~

\begin{problem} (From the reading)
Classical mechanics, or Newtonian mechanics, is only valid in certain
circumstances. When do the laws of classical mechanics, like $F =
m\,a$ for example, become wrong or break down? There are many answers
to this problem; I will take anything correct.
\end{problem}


\vfill ~


\cleardoublepage



\noindent
Name: \rule[-1ex]{0.60\textwidth}{0.1pt}
NetID: \rule[-1ex]{0.20\textwidth}{0.1pt}

\section*{\textsl{Einstein's Universe} Term Exam 1}
\setcounter{problem}{1}


\begin{problem} (From the reading)
What musical instrument did Einstein most enjoy playing?
\end{problem}


\vfill ~

\begin{problem} (From Lecture on 2019-09-17)
If you are traveling at 60 miles per hour, how long does
it take you to go 300 miles?
\end{problem}


\vfill ~

\begin{problem} (From Lecture on 2019-09-05)
The molar weight of water is $18\,\g$. How many molecules would there
be, therefore, in $18\,\g$ of water? You don't need a calculator for
this.
\end{problem}


\vfill ~

\begin{problem} (From the reading)
Classical mechanics, or Newtonian mechanics, is only valid in certain
circumstances. When do the laws of classical mechanics, like $F =
m\,a$ for example, become wrong or break down? There are many answers
to this problem; I will take anything correct.
\end{problem}


\vfill ~


\clearpage


\begin{problem} (From Problem Set 1)
What is the approximate thickness of a stack of 1000 20-dollar bills?
No need to be precise, and use any units you like.
\end{problem}


\vfill ~

\begin{problem} (From the Math Review Lab)
What is this number? Give your answer in scientific notation.
$$
\frac{(7\times10^{-34})\times(3\times10^8)}{5\times10^{-7}}
$$
You don't need a calculator to solve this problem (\textit{hint: $3/5=0.6$}).
\end{problem}


\vfill ~

\begin{problem} (From the Kinematics Lab)
Here is a data table of times, positions, and velocities in SI units:\\
\rule{1.0in}{0pt}\begin{tabular}{c|c|c}
time $t$ ($\s$) & position $x$ ($\m$) & velocity $v$ ($\m\,\s^{-1}$) \\
\hline
1 & 1.15 & 1.3 \\
2 & 2.60 & 1.6 \\
3 & 4.35 & 1.9 \\
\hline
\end{tabular}\\
What is the average acceleration in the time interval from $2\,\s$ to $3\,\s$?
\end{problem}


\vfill ~

\begin{problem} (From Problem Set 1)
What is the relationship between the energy $E$ and wavelength
$\lambda$ of a photon? Give a formula that involves energy $E$,
Planck's Constant $h$, the speed of light $c$, and wavelength
$\lambda$ (or whatever you need).
\end{problem}

\vfill ~


\cleardoublepage



\noindent
Name: \rule[-1ex]{0.60\textwidth}{0.1pt}
NetID: \rule[-1ex]{0.20\textwidth}{0.1pt}

\section*{\textsl{Einstein's Universe} Term Exam 1}
\setcounter{problem}{1}


\begin{problem} (From the reading)
Classical mechanics, or Newtonian mechanics, is only valid in certain
circumstances. When do the laws of classical mechanics, like $F =
m\,a$ for example, become wrong or break down? There are many answers
to this problem; I will take anything correct.
\end{problem}


\vfill ~

\begin{problem} (From Problem Set 1)
What is the approximate thickness of a stack of 1000 20-dollar bills?
No need to be precise, and use any units you like.
\end{problem}


\vfill ~

\begin{problem} (From Lecture on 2019-09-05)
The molar weight of water is $18\,\g$. How many molecules would there
be, therefore, in $18\,\g$ of water? You don't need a calculator for
this.
\end{problem}


\vfill ~

\begin{problem} (From the Math Review Lab)
What is this number? Give your answer in scientific notation.
$$
\frac{(7\times10^{-34})\times(3\times10^8)}{5\times10^{-7}}
$$
You don't need a calculator to solve this problem (\textit{hint: $3/5=0.6$}).
\end{problem}


\vfill ~


\clearpage


\begin{problem} (From Lecture on 2019-09-17)
If you are traveling at 60 miles per hour, how long does
it take you to go 300 miles?
\end{problem}


\vfill ~

\begin{problem} (From the reading)
What musical instrument did Einstein most enjoy playing?
\end{problem}


\vfill ~

\begin{problem} (From the Kinematics Lab)
Here is a data table of times, positions, and velocities in SI units:\\
\rule{1.0in}{0pt}\begin{tabular}{c|c|c}
time $t$ ($\s$) & position $x$ ($\m$) & velocity $v$ ($\m\,\s^{-1}$) \\
\hline
1 & 1.15 & 1.3 \\
2 & 2.60 & 1.6 \\
3 & 4.35 & 1.9 \\
\hline
\end{tabular}\\
What is the average acceleration in the time interval from $2\,\s$ to $3\,\s$?
\end{problem}


\vfill ~

\begin{problem} (From Problem Set 1)
What is the relationship between the energy $E$ and wavelength
$\lambda$ of a photon? Give a formula that involves energy $E$,
Planck's Constant $h$, the speed of light $c$, and wavelength
$\lambda$ (or whatever you need).
\end{problem}

\vfill ~


\cleardoublepage



\noindent
Name: \rule[-1ex]{0.60\textwidth}{0.1pt}
NetID: \rule[-1ex]{0.20\textwidth}{0.1pt}

\section*{\textsl{Einstein's Universe} Term Exam 1}
\setcounter{problem}{1}


\begin{problem} (From Problem Set 1)
What is the relationship between the energy $E$ and wavelength
$\lambda$ of a photon? Give a formula that involves energy $E$,
Planck's Constant $h$, the speed of light $c$, and wavelength
$\lambda$ (or whatever you need).
\end{problem}

\vfill ~

\begin{problem} (From the Kinematics Lab)
Here is a data table of times, positions, and velocities in SI units:\\
\rule{1.0in}{0pt}\begin{tabular}{c|c|c}
time $t$ ($\s$) & position $x$ ($\m$) & velocity $v$ ($\m\,\s^{-1}$) \\
\hline
1 & 1.15 & 1.3 \\
2 & 2.60 & 1.6 \\
3 & 4.35 & 1.9 \\
\hline
\end{tabular}\\
What is the average acceleration in the time interval from $2\,\s$ to $3\,\s$?
\end{problem}


\vfill ~

\begin{problem} (From Lecture on 2019-09-17)
If you are traveling at 60 miles per hour, how long does
it take you to go 300 miles?
\end{problem}


\vfill ~

\begin{problem} (From the reading)
What musical instrument did Einstein most enjoy playing?
\end{problem}


\vfill ~


\clearpage


\begin{problem} (From Problem Set 1)
What is the approximate thickness of a stack of 1000 20-dollar bills?
No need to be precise, and use any units you like.
\end{problem}


\vfill ~

\begin{problem} (From Lecture on 2019-09-05)
The molar weight of water is $18\,\g$. How many molecules would there
be, therefore, in $18\,\g$ of water? You don't need a calculator for
this.
\end{problem}


\vfill ~

\begin{problem} (From the Math Review Lab)
What is this number? Give your answer in scientific notation.
$$
\frac{(7\times10^{-34})\times(3\times10^8)}{5\times10^{-7}}
$$
You don't need a calculator to solve this problem (\textit{hint: $3/5=0.6$}).
\end{problem}


\vfill ~

\begin{problem} (From the reading)
Classical mechanics, or Newtonian mechanics, is only valid in certain
circumstances. When do the laws of classical mechanics, like $F =
m\,a$ for example, become wrong or break down? There are many answers
to this problem; I will take anything correct.
\end{problem}


\vfill ~


\cleardoublepage



\noindent
Name: \rule[-1ex]{0.60\textwidth}{0.1pt}
NetID: \rule[-1ex]{0.20\textwidth}{0.1pt}

\section*{\textsl{Einstein's Universe} Term Exam 1}
\setcounter{problem}{1}


\begin{problem} (From the Kinematics Lab)
Here is a data table of times, positions, and velocities in SI units:\\
\rule{1.0in}{0pt}\begin{tabular}{c|c|c}
time $t$ ($\s$) & position $x$ ($\m$) & velocity $v$ ($\m\,\s^{-1}$) \\
\hline
1 & 1.15 & 1.3 \\
2 & 2.60 & 1.6 \\
3 & 4.35 & 1.9 \\
\hline
\end{tabular}\\
What is the average acceleration in the time interval from $2\,\s$ to $3\,\s$?
\end{problem}


\vfill ~

\begin{problem} (From Problem Set 1)
What is the relationship between the energy $E$ and wavelength
$\lambda$ of a photon? Give a formula that involves energy $E$,
Planck's Constant $h$, the speed of light $c$, and wavelength
$\lambda$ (or whatever you need).
\end{problem}

\vfill ~

\begin{problem} (From the reading)
What musical instrument did Einstein most enjoy playing?
\end{problem}


\vfill ~

\begin{problem} (From the Math Review Lab)
What is this number? Give your answer in scientific notation.
$$
\frac{(7\times10^{-34})\times(3\times10^8)}{5\times10^{-7}}
$$
You don't need a calculator to solve this problem (\textit{hint: $3/5=0.6$}).
\end{problem}


\vfill ~


\clearpage


\begin{problem} (From Lecture on 2019-09-05)
The molar weight of water is $18\,\g$. How many molecules would there
be, therefore, in $18\,\g$ of water? You don't need a calculator for
this.
\end{problem}


\vfill ~

\begin{problem} (From the reading)
Classical mechanics, or Newtonian mechanics, is only valid in certain
circumstances. When do the laws of classical mechanics, like $F =
m\,a$ for example, become wrong or break down? There are many answers
to this problem; I will take anything correct.
\end{problem}


\vfill ~

\begin{problem} (From Lecture on 2019-09-17)
If you are traveling at 60 miles per hour, how long does
it take you to go 300 miles?
\end{problem}


\vfill ~

\begin{problem} (From Problem Set 1)
What is the approximate thickness of a stack of 1000 20-dollar bills?
No need to be precise, and use any units you like.
\end{problem}


\vfill ~


\cleardoublepage



\noindent
Name: \rule[-1ex]{0.60\textwidth}{0.1pt}
NetID: \rule[-1ex]{0.20\textwidth}{0.1pt}

\section*{\textsl{Einstein's Universe} Term Exam 1}
\setcounter{problem}{1}


\begin{problem} (From Problem Set 1)
What is the relationship between the energy $E$ and wavelength
$\lambda$ of a photon? Give a formula that involves energy $E$,
Planck's Constant $h$, the speed of light $c$, and wavelength
$\lambda$ (or whatever you need).
\end{problem}

\vfill ~

\begin{problem} (From the reading)
Classical mechanics, or Newtonian mechanics, is only valid in certain
circumstances. When do the laws of classical mechanics, like $F =
m\,a$ for example, become wrong or break down? There are many answers
to this problem; I will take anything correct.
\end{problem}


\vfill ~

\begin{problem} (From the reading)
What musical instrument did Einstein most enjoy playing?
\end{problem}


\vfill ~

\begin{problem} (From Lecture on 2019-09-17)
If you are traveling at 60 miles per hour, how long does
it take you to go 300 miles?
\end{problem}


\vfill ~


\clearpage


\begin{problem} (From Lecture on 2019-09-05)
The molar weight of water is $18\,\g$. How many molecules would there
be, therefore, in $18\,\g$ of water? You don't need a calculator for
this.
\end{problem}


\vfill ~

\begin{problem} (From the Math Review Lab)
What is this number? Give your answer in scientific notation.
$$
\frac{(7\times10^{-34})\times(3\times10^8)}{5\times10^{-7}}
$$
You don't need a calculator to solve this problem (\textit{hint: $3/5=0.6$}).
\end{problem}


\vfill ~

\begin{problem} (From the Kinematics Lab)
Here is a data table of times, positions, and velocities in SI units:\\
\rule{1.0in}{0pt}\begin{tabular}{c|c|c}
time $t$ ($\s$) & position $x$ ($\m$) & velocity $v$ ($\m\,\s^{-1}$) \\
\hline
1 & 1.15 & 1.3 \\
2 & 2.60 & 1.6 \\
3 & 4.35 & 1.9 \\
\hline
\end{tabular}\\
What is the average acceleration in the time interval from $2\,\s$ to $3\,\s$?
\end{problem}


\vfill ~

\begin{problem} (From Problem Set 1)
What is the approximate thickness of a stack of 1000 20-dollar bills?
No need to be precise, and use any units you like.
\end{problem}


\vfill ~


\cleardoublepage



\noindent
Name: \rule[-1ex]{0.60\textwidth}{0.1pt}
NetID: \rule[-1ex]{0.20\textwidth}{0.1pt}

\section*{\textsl{Einstein's Universe} Term Exam 1}
\setcounter{problem}{1}


\begin{problem} (From Lecture on 2019-09-05)
The molar weight of water is $18\,\g$. How many molecules would there
be, therefore, in $18\,\g$ of water? You don't need a calculator for
this.
\end{problem}


\vfill ~

\begin{problem} (From Lecture on 2019-09-17)
If you are traveling at 60 miles per hour, how long does
it take you to go 300 miles?
\end{problem}


\vfill ~

\begin{problem} (From Problem Set 1)
What is the approximate thickness of a stack of 1000 20-dollar bills?
No need to be precise, and use any units you like.
\end{problem}


\vfill ~

\begin{problem} (From Problem Set 1)
What is the relationship between the energy $E$ and wavelength
$\lambda$ of a photon? Give a formula that involves energy $E$,
Planck's Constant $h$, the speed of light $c$, and wavelength
$\lambda$ (or whatever you need).
\end{problem}

\vfill ~


\clearpage


\begin{problem} (From the reading)
What musical instrument did Einstein most enjoy playing?
\end{problem}


\vfill ~

\begin{problem} (From the Math Review Lab)
What is this number? Give your answer in scientific notation.
$$
\frac{(7\times10^{-34})\times(3\times10^8)}{5\times10^{-7}}
$$
You don't need a calculator to solve this problem (\textit{hint: $3/5=0.6$}).
\end{problem}


\vfill ~

\begin{problem} (From the Kinematics Lab)
Here is a data table of times, positions, and velocities in SI units:\\
\rule{1.0in}{0pt}\begin{tabular}{c|c|c}
time $t$ ($\s$) & position $x$ ($\m$) & velocity $v$ ($\m\,\s^{-1}$) \\
\hline
1 & 1.15 & 1.3 \\
2 & 2.60 & 1.6 \\
3 & 4.35 & 1.9 \\
\hline
\end{tabular}\\
What is the average acceleration in the time interval from $2\,\s$ to $3\,\s$?
\end{problem}


\vfill ~

\begin{problem} (From the reading)
Classical mechanics, or Newtonian mechanics, is only valid in certain
circumstances. When do the laws of classical mechanics, like $F =
m\,a$ for example, become wrong or break down? There are many answers
to this problem; I will take anything correct.
\end{problem}


\vfill ~


\cleardoublepage



\noindent
Name: \rule[-1ex]{0.60\textwidth}{0.1pt}
NetID: \rule[-1ex]{0.20\textwidth}{0.1pt}

\section*{\textsl{Einstein's Universe} Term Exam 1}
\setcounter{problem}{1}


\begin{problem} (From Problem Set 1)
What is the relationship between the energy $E$ and wavelength
$\lambda$ of a photon? Give a formula that involves energy $E$,
Planck's Constant $h$, the speed of light $c$, and wavelength
$\lambda$ (or whatever you need).
\end{problem}

\vfill ~

\begin{problem} (From the reading)
Classical mechanics, or Newtonian mechanics, is only valid in certain
circumstances. When do the laws of classical mechanics, like $F =
m\,a$ for example, become wrong or break down? There are many answers
to this problem; I will take anything correct.
\end{problem}


\vfill ~

\begin{problem} (From Problem Set 1)
What is the approximate thickness of a stack of 1000 20-dollar bills?
No need to be precise, and use any units you like.
\end{problem}


\vfill ~

\begin{problem} (From Lecture on 2019-09-05)
The molar weight of water is $18\,\g$. How many molecules would there
be, therefore, in $18\,\g$ of water? You don't need a calculator for
this.
\end{problem}


\vfill ~


\clearpage


\begin{problem} (From the Kinematics Lab)
Here is a data table of times, positions, and velocities in SI units:\\
\rule{1.0in}{0pt}\begin{tabular}{c|c|c}
time $t$ ($\s$) & position $x$ ($\m$) & velocity $v$ ($\m\,\s^{-1}$) \\
\hline
1 & 1.15 & 1.3 \\
2 & 2.60 & 1.6 \\
3 & 4.35 & 1.9 \\
\hline
\end{tabular}\\
What is the average acceleration in the time interval from $2\,\s$ to $3\,\s$?
\end{problem}


\vfill ~

\begin{problem} (From Lecture on 2019-09-17)
If you are traveling at 60 miles per hour, how long does
it take you to go 300 miles?
\end{problem}


\vfill ~

\begin{problem} (From the reading)
What musical instrument did Einstein most enjoy playing?
\end{problem}


\vfill ~

\begin{problem} (From the Math Review Lab)
What is this number? Give your answer in scientific notation.
$$
\frac{(7\times10^{-34})\times(3\times10^8)}{5\times10^{-7}}
$$
You don't need a calculator to solve this problem (\textit{hint: $3/5=0.6$}).
\end{problem}


\vfill ~


\cleardoublepage



\noindent
Name: \rule[-1ex]{0.60\textwidth}{0.1pt}
NetID: \rule[-1ex]{0.20\textwidth}{0.1pt}

\section*{\textsl{Einstein's Universe} Term Exam 1}
\setcounter{problem}{1}


\begin{problem} (From Problem Set 1)
What is the relationship between the energy $E$ and wavelength
$\lambda$ of a photon? Give a formula that involves energy $E$,
Planck's Constant $h$, the speed of light $c$, and wavelength
$\lambda$ (or whatever you need).
\end{problem}

\vfill ~

\begin{problem} (From Lecture on 2019-09-05)
The molar weight of water is $18\,\g$. How many molecules would there
be, therefore, in $18\,\g$ of water? You don't need a calculator for
this.
\end{problem}


\vfill ~

\begin{problem} (From Lecture on 2019-09-17)
If you are traveling at 60 miles per hour, how long does
it take you to go 300 miles?
\end{problem}


\vfill ~

\begin{problem} (From the reading)
Classical mechanics, or Newtonian mechanics, is only valid in certain
circumstances. When do the laws of classical mechanics, like $F =
m\,a$ for example, become wrong or break down? There are many answers
to this problem; I will take anything correct.
\end{problem}


\vfill ~


\clearpage


\begin{problem} (From the Kinematics Lab)
Here is a data table of times, positions, and velocities in SI units:\\
\rule{1.0in}{0pt}\begin{tabular}{c|c|c}
time $t$ ($\s$) & position $x$ ($\m$) & velocity $v$ ($\m\,\s^{-1}$) \\
\hline
1 & 1.15 & 1.3 \\
2 & 2.60 & 1.6 \\
3 & 4.35 & 1.9 \\
\hline
\end{tabular}\\
What is the average acceleration in the time interval from $2\,\s$ to $3\,\s$?
\end{problem}


\vfill ~

\begin{problem} (From the reading)
What musical instrument did Einstein most enjoy playing?
\end{problem}


\vfill ~

\begin{problem} (From Problem Set 1)
What is the approximate thickness of a stack of 1000 20-dollar bills?
No need to be precise, and use any units you like.
\end{problem}


\vfill ~

\begin{problem} (From the Math Review Lab)
What is this number? Give your answer in scientific notation.
$$
\frac{(7\times10^{-34})\times(3\times10^8)}{5\times10^{-7}}
$$
You don't need a calculator to solve this problem (\textit{hint: $3/5=0.6$}).
\end{problem}


\vfill ~


\cleardoublepage



\noindent
Name: \rule[-1ex]{0.60\textwidth}{0.1pt}
NetID: \rule[-1ex]{0.20\textwidth}{0.1pt}

\section*{\textsl{Einstein's Universe} Term Exam 1}
\setcounter{problem}{1}


\begin{problem} (From Problem Set 1)
What is the relationship between the energy $E$ and wavelength
$\lambda$ of a photon? Give a formula that involves energy $E$,
Planck's Constant $h$, the speed of light $c$, and wavelength
$\lambda$ (or whatever you need).
\end{problem}

\vfill ~

\begin{problem} (From the Kinematics Lab)
Here is a data table of times, positions, and velocities in SI units:\\
\rule{1.0in}{0pt}\begin{tabular}{c|c|c}
time $t$ ($\s$) & position $x$ ($\m$) & velocity $v$ ($\m\,\s^{-1}$) \\
\hline
1 & 1.15 & 1.3 \\
2 & 2.60 & 1.6 \\
3 & 4.35 & 1.9 \\
\hline
\end{tabular}\\
What is the average acceleration in the time interval from $2\,\s$ to $3\,\s$?
\end{problem}


\vfill ~

\begin{problem} (From the Math Review Lab)
What is this number? Give your answer in scientific notation.
$$
\frac{(7\times10^{-34})\times(3\times10^8)}{5\times10^{-7}}
$$
You don't need a calculator to solve this problem (\textit{hint: $3/5=0.6$}).
\end{problem}


\vfill ~

\begin{problem} (From Lecture on 2019-09-05)
The molar weight of water is $18\,\g$. How many molecules would there
be, therefore, in $18\,\g$ of water? You don't need a calculator for
this.
\end{problem}


\vfill ~


\clearpage


\begin{problem} (From the reading)
Classical mechanics, or Newtonian mechanics, is only valid in certain
circumstances. When do the laws of classical mechanics, like $F =
m\,a$ for example, become wrong or break down? There are many answers
to this problem; I will take anything correct.
\end{problem}


\vfill ~

\begin{problem} (From Lecture on 2019-09-17)
If you are traveling at 60 miles per hour, how long does
it take you to go 300 miles?
\end{problem}


\vfill ~

\begin{problem} (From the reading)
What musical instrument did Einstein most enjoy playing?
\end{problem}


\vfill ~

\begin{problem} (From Problem Set 1)
What is the approximate thickness of a stack of 1000 20-dollar bills?
No need to be precise, and use any units you like.
\end{problem}


\vfill ~


\cleardoublepage



\noindent
Name: \rule[-1ex]{0.60\textwidth}{0.1pt}
NetID: \rule[-1ex]{0.20\textwidth}{0.1pt}

\section*{\textsl{Einstein's Universe} Term Exam 1}
\setcounter{problem}{1}


\begin{problem} (From the reading)
What musical instrument did Einstein most enjoy playing?
\end{problem}


\vfill ~

\begin{problem} (From the reading)
Classical mechanics, or Newtonian mechanics, is only valid in certain
circumstances. When do the laws of classical mechanics, like $F =
m\,a$ for example, become wrong or break down? There are many answers
to this problem; I will take anything correct.
\end{problem}


\vfill ~

\begin{problem} (From the Kinematics Lab)
Here is a data table of times, positions, and velocities in SI units:\\
\rule{1.0in}{0pt}\begin{tabular}{c|c|c}
time $t$ ($\s$) & position $x$ ($\m$) & velocity $v$ ($\m\,\s^{-1}$) \\
\hline
1 & 1.15 & 1.3 \\
2 & 2.60 & 1.6 \\
3 & 4.35 & 1.9 \\
\hline
\end{tabular}\\
What is the average acceleration in the time interval from $2\,\s$ to $3\,\s$?
\end{problem}


\vfill ~

\begin{problem} (From Lecture on 2019-09-17)
If you are traveling at 60 miles per hour, how long does
it take you to go 300 miles?
\end{problem}


\vfill ~


\clearpage


\begin{problem} (From the Math Review Lab)
What is this number? Give your answer in scientific notation.
$$
\frac{(7\times10^{-34})\times(3\times10^8)}{5\times10^{-7}}
$$
You don't need a calculator to solve this problem (\textit{hint: $3/5=0.6$}).
\end{problem}


\vfill ~

\begin{problem} (From Problem Set 1)
What is the approximate thickness of a stack of 1000 20-dollar bills?
No need to be precise, and use any units you like.
\end{problem}


\vfill ~

\begin{problem} (From Lecture on 2019-09-05)
The molar weight of water is $18\,\g$. How many molecules would there
be, therefore, in $18\,\g$ of water? You don't need a calculator for
this.
\end{problem}


\vfill ~

\begin{problem} (From Problem Set 1)
What is the relationship between the energy $E$ and wavelength
$\lambda$ of a photon? Give a formula that involves energy $E$,
Planck's Constant $h$, the speed of light $c$, and wavelength
$\lambda$ (or whatever you need).
\end{problem}

\vfill ~


\cleardoublepage



\noindent
Name: \rule[-1ex]{0.60\textwidth}{0.1pt}
NetID: \rule[-1ex]{0.20\textwidth}{0.1pt}

\section*{\textsl{Einstein's Universe} Term Exam 1}
\setcounter{problem}{1}


\begin{problem} (From Problem Set 1)
What is the relationship between the energy $E$ and wavelength
$\lambda$ of a photon? Give a formula that involves energy $E$,
Planck's Constant $h$, the speed of light $c$, and wavelength
$\lambda$ (or whatever you need).
\end{problem}

\vfill ~

\begin{problem} (From the Math Review Lab)
What is this number? Give your answer in scientific notation.
$$
\frac{(7\times10^{-34})\times(3\times10^8)}{5\times10^{-7}}
$$
You don't need a calculator to solve this problem (\textit{hint: $3/5=0.6$}).
\end{problem}


\vfill ~

\begin{problem} (From the reading)
What musical instrument did Einstein most enjoy playing?
\end{problem}


\vfill ~

\begin{problem} (From the Kinematics Lab)
Here is a data table of times, positions, and velocities in SI units:\\
\rule{1.0in}{0pt}\begin{tabular}{c|c|c}
time $t$ ($\s$) & position $x$ ($\m$) & velocity $v$ ($\m\,\s^{-1}$) \\
\hline
1 & 1.15 & 1.3 \\
2 & 2.60 & 1.6 \\
3 & 4.35 & 1.9 \\
\hline
\end{tabular}\\
What is the average acceleration in the time interval from $2\,\s$ to $3\,\s$?
\end{problem}


\vfill ~


\clearpage


\begin{problem} (From Lecture on 2019-09-17)
If you are traveling at 60 miles per hour, how long does
it take you to go 300 miles?
\end{problem}


\vfill ~

\begin{problem} (From Lecture on 2019-09-05)
The molar weight of water is $18\,\g$. How many molecules would there
be, therefore, in $18\,\g$ of water? You don't need a calculator for
this.
\end{problem}


\vfill ~

\begin{problem} (From the reading)
Classical mechanics, or Newtonian mechanics, is only valid in certain
circumstances. When do the laws of classical mechanics, like $F =
m\,a$ for example, become wrong or break down? There are many answers
to this problem; I will take anything correct.
\end{problem}


\vfill ~

\begin{problem} (From Problem Set 1)
What is the approximate thickness of a stack of 1000 20-dollar bills?
No need to be precise, and use any units you like.
\end{problem}


\vfill ~


\cleardoublepage



\noindent
Name: \rule[-1ex]{0.60\textwidth}{0.1pt}
NetID: \rule[-1ex]{0.20\textwidth}{0.1pt}

\section*{\textsl{Einstein's Universe} Term Exam 1}
\setcounter{problem}{1}


\begin{problem} (From the Kinematics Lab)
Here is a data table of times, positions, and velocities in SI units:\\
\rule{1.0in}{0pt}\begin{tabular}{c|c|c}
time $t$ ($\s$) & position $x$ ($\m$) & velocity $v$ ($\m\,\s^{-1}$) \\
\hline
1 & 1.15 & 1.3 \\
2 & 2.60 & 1.6 \\
3 & 4.35 & 1.9 \\
\hline
\end{tabular}\\
What is the average acceleration in the time interval from $2\,\s$ to $3\,\s$?
\end{problem}


\vfill ~

\begin{problem} (From Lecture on 2019-09-05)
The molar weight of water is $18\,\g$. How many molecules would there
be, therefore, in $18\,\g$ of water? You don't need a calculator for
this.
\end{problem}


\vfill ~

\begin{problem} (From the reading)
Classical mechanics, or Newtonian mechanics, is only valid in certain
circumstances. When do the laws of classical mechanics, like $F =
m\,a$ for example, become wrong or break down? There are many answers
to this problem; I will take anything correct.
\end{problem}


\vfill ~

\begin{problem} (From Lecture on 2019-09-17)
If you are traveling at 60 miles per hour, how long does
it take you to go 300 miles?
\end{problem}


\vfill ~


\clearpage


\begin{problem} (From Problem Set 1)
What is the approximate thickness of a stack of 1000 20-dollar bills?
No need to be precise, and use any units you like.
\end{problem}


\vfill ~

\begin{problem} (From the reading)
What musical instrument did Einstein most enjoy playing?
\end{problem}


\vfill ~

\begin{problem} (From the Math Review Lab)
What is this number? Give your answer in scientific notation.
$$
\frac{(7\times10^{-34})\times(3\times10^8)}{5\times10^{-7}}
$$
You don't need a calculator to solve this problem (\textit{hint: $3/5=0.6$}).
\end{problem}


\vfill ~

\begin{problem} (From Problem Set 1)
What is the relationship between the energy $E$ and wavelength
$\lambda$ of a photon? Give a formula that involves energy $E$,
Planck's Constant $h$, the speed of light $c$, and wavelength
$\lambda$ (or whatever you need).
\end{problem}

\vfill ~


\cleardoublepage



\noindent
Name: \rule[-1ex]{0.60\textwidth}{0.1pt}
NetID: \rule[-1ex]{0.20\textwidth}{0.1pt}

\section*{\textsl{Einstein's Universe} Term Exam 1}
\setcounter{problem}{1}


\begin{problem} (From Lecture on 2019-09-05)
The molar weight of water is $18\,\g$. How many molecules would there
be, therefore, in $18\,\g$ of water? You don't need a calculator for
this.
\end{problem}


\vfill ~

\begin{problem} (From the Math Review Lab)
What is this number? Give your answer in scientific notation.
$$
\frac{(7\times10^{-34})\times(3\times10^8)}{5\times10^{-7}}
$$
You don't need a calculator to solve this problem (\textit{hint: $3/5=0.6$}).
\end{problem}


\vfill ~

\begin{problem} (From the reading)
What musical instrument did Einstein most enjoy playing?
\end{problem}


\vfill ~

\begin{problem} (From Problem Set 1)
What is the relationship between the energy $E$ and wavelength
$\lambda$ of a photon? Give a formula that involves energy $E$,
Planck's Constant $h$, the speed of light $c$, and wavelength
$\lambda$ (or whatever you need).
\end{problem}

\vfill ~


\clearpage


\begin{problem} (From Problem Set 1)
What is the approximate thickness of a stack of 1000 20-dollar bills?
No need to be precise, and use any units you like.
\end{problem}


\vfill ~

\begin{problem} (From the Kinematics Lab)
Here is a data table of times, positions, and velocities in SI units:\\
\rule{1.0in}{0pt}\begin{tabular}{c|c|c}
time $t$ ($\s$) & position $x$ ($\m$) & velocity $v$ ($\m\,\s^{-1}$) \\
\hline
1 & 1.15 & 1.3 \\
2 & 2.60 & 1.6 \\
3 & 4.35 & 1.9 \\
\hline
\end{tabular}\\
What is the average acceleration in the time interval from $2\,\s$ to $3\,\s$?
\end{problem}


\vfill ~

\begin{problem} (From the reading)
Classical mechanics, or Newtonian mechanics, is only valid in certain
circumstances. When do the laws of classical mechanics, like $F =
m\,a$ for example, become wrong or break down? There are many answers
to this problem; I will take anything correct.
\end{problem}


\vfill ~

\begin{problem} (From Lecture on 2019-09-17)
If you are traveling at 60 miles per hour, how long does
it take you to go 300 miles?
\end{problem}


\vfill ~


\cleardoublepage



\noindent
Name: \rule[-1ex]{0.60\textwidth}{0.1pt}
NetID: \rule[-1ex]{0.20\textwidth}{0.1pt}

\section*{\textsl{Einstein's Universe} Term Exam 1}
\setcounter{problem}{1}


\begin{problem} (From the Kinematics Lab)
Here is a data table of times, positions, and velocities in SI units:\\
\rule{1.0in}{0pt}\begin{tabular}{c|c|c}
time $t$ ($\s$) & position $x$ ($\m$) & velocity $v$ ($\m\,\s^{-1}$) \\
\hline
1 & 1.15 & 1.3 \\
2 & 2.60 & 1.6 \\
3 & 4.35 & 1.9 \\
\hline
\end{tabular}\\
What is the average acceleration in the time interval from $2\,\s$ to $3\,\s$?
\end{problem}


\vfill ~

\begin{problem} (From the reading)
What musical instrument did Einstein most enjoy playing?
\end{problem}


\vfill ~

\begin{problem} (From Problem Set 1)
What is the approximate thickness of a stack of 1000 20-dollar bills?
No need to be precise, and use any units you like.
\end{problem}


\vfill ~

\begin{problem} (From Problem Set 1)
What is the relationship between the energy $E$ and wavelength
$\lambda$ of a photon? Give a formula that involves energy $E$,
Planck's Constant $h$, the speed of light $c$, and wavelength
$\lambda$ (or whatever you need).
\end{problem}

\vfill ~


\clearpage


\begin{problem} (From Lecture on 2019-09-17)
If you are traveling at 60 miles per hour, how long does
it take you to go 300 miles?
\end{problem}


\vfill ~

\begin{problem} (From the reading)
Classical mechanics, or Newtonian mechanics, is only valid in certain
circumstances. When do the laws of classical mechanics, like $F =
m\,a$ for example, become wrong or break down? There are many answers
to this problem; I will take anything correct.
\end{problem}


\vfill ~

\begin{problem} (From the Math Review Lab)
What is this number? Give your answer in scientific notation.
$$
\frac{(7\times10^{-34})\times(3\times10^8)}{5\times10^{-7}}
$$
You don't need a calculator to solve this problem (\textit{hint: $3/5=0.6$}).
\end{problem}


\vfill ~

\begin{problem} (From Lecture on 2019-09-05)
The molar weight of water is $18\,\g$. How many molecules would there
be, therefore, in $18\,\g$ of water? You don't need a calculator for
this.
\end{problem}


\vfill ~


\cleardoublepage



\noindent
Name: \rule[-1ex]{0.60\textwidth}{0.1pt}
NetID: \rule[-1ex]{0.20\textwidth}{0.1pt}

\section*{\textsl{Einstein's Universe} Term Exam 1}
\setcounter{problem}{1}


\begin{problem} (From Problem Set 1)
What is the relationship between the energy $E$ and wavelength
$\lambda$ of a photon? Give a formula that involves energy $E$,
Planck's Constant $h$, the speed of light $c$, and wavelength
$\lambda$ (or whatever you need).
\end{problem}

\vfill ~

\begin{problem} (From the Math Review Lab)
What is this number? Give your answer in scientific notation.
$$
\frac{(7\times10^{-34})\times(3\times10^8)}{5\times10^{-7}}
$$
You don't need a calculator to solve this problem (\textit{hint: $3/5=0.6$}).
\end{problem}


\vfill ~

\begin{problem} (From the reading)
What musical instrument did Einstein most enjoy playing?
\end{problem}


\vfill ~

\begin{problem} (From Problem Set 1)
What is the approximate thickness of a stack of 1000 20-dollar bills?
No need to be precise, and use any units you like.
\end{problem}


\vfill ~


\clearpage


\begin{problem} (From Lecture on 2019-09-05)
The molar weight of water is $18\,\g$. How many molecules would there
be, therefore, in $18\,\g$ of water? You don't need a calculator for
this.
\end{problem}


\vfill ~

\begin{problem} (From the reading)
Classical mechanics, or Newtonian mechanics, is only valid in certain
circumstances. When do the laws of classical mechanics, like $F =
m\,a$ for example, become wrong or break down? There are many answers
to this problem; I will take anything correct.
\end{problem}


\vfill ~

\begin{problem} (From Lecture on 2019-09-17)
If you are traveling at 60 miles per hour, how long does
it take you to go 300 miles?
\end{problem}


\vfill ~

\begin{problem} (From the Kinematics Lab)
Here is a data table of times, positions, and velocities in SI units:\\
\rule{1.0in}{0pt}\begin{tabular}{c|c|c}
time $t$ ($\s$) & position $x$ ($\m$) & velocity $v$ ($\m\,\s^{-1}$) \\
\hline
1 & 1.15 & 1.3 \\
2 & 2.60 & 1.6 \\
3 & 4.35 & 1.9 \\
\hline
\end{tabular}\\
What is the average acceleration in the time interval from $2\,\s$ to $3\,\s$?
\end{problem}


\vfill ~


\cleardoublepage



\noindent
Name: \rule[-1ex]{0.60\textwidth}{0.1pt}
NetID: \rule[-1ex]{0.20\textwidth}{0.1pt}

\section*{\textsl{Einstein's Universe} Term Exam 1}
\setcounter{problem}{1}


\begin{problem} (From the Kinematics Lab)
Here is a data table of times, positions, and velocities in SI units:\\
\rule{1.0in}{0pt}\begin{tabular}{c|c|c}
time $t$ ($\s$) & position $x$ ($\m$) & velocity $v$ ($\m\,\s^{-1}$) \\
\hline
1 & 1.15 & 1.3 \\
2 & 2.60 & 1.6 \\
3 & 4.35 & 1.9 \\
\hline
\end{tabular}\\
What is the average acceleration in the time interval from $2\,\s$ to $3\,\s$?
\end{problem}


\vfill ~

\begin{problem} (From Lecture on 2019-09-17)
If you are traveling at 60 miles per hour, how long does
it take you to go 300 miles?
\end{problem}


\vfill ~

\begin{problem} (From the reading)
What musical instrument did Einstein most enjoy playing?
\end{problem}


\vfill ~

\begin{problem} (From the Math Review Lab)
What is this number? Give your answer in scientific notation.
$$
\frac{(7\times10^{-34})\times(3\times10^8)}{5\times10^{-7}}
$$
You don't need a calculator to solve this problem (\textit{hint: $3/5=0.6$}).
\end{problem}


\vfill ~


\clearpage


\begin{problem} (From Problem Set 1)
What is the approximate thickness of a stack of 1000 20-dollar bills?
No need to be precise, and use any units you like.
\end{problem}


\vfill ~

\begin{problem} (From Lecture on 2019-09-05)
The molar weight of water is $18\,\g$. How many molecules would there
be, therefore, in $18\,\g$ of water? You don't need a calculator for
this.
\end{problem}


\vfill ~

\begin{problem} (From the reading)
Classical mechanics, or Newtonian mechanics, is only valid in certain
circumstances. When do the laws of classical mechanics, like $F =
m\,a$ for example, become wrong or break down? There are many answers
to this problem; I will take anything correct.
\end{problem}


\vfill ~

\begin{problem} (From Problem Set 1)
What is the relationship between the energy $E$ and wavelength
$\lambda$ of a photon? Give a formula that involves energy $E$,
Planck's Constant $h$, the speed of light $c$, and wavelength
$\lambda$ (or whatever you need).
\end{problem}

\vfill ~


\cleardoublepage



\noindent
Name: \rule[-1ex]{0.60\textwidth}{0.1pt}
NetID: \rule[-1ex]{0.20\textwidth}{0.1pt}

\section*{\textsl{Einstein's Universe} Term Exam 1}
\setcounter{problem}{1}


\begin{problem} (From the reading)
What musical instrument did Einstein most enjoy playing?
\end{problem}


\vfill ~

\begin{problem} (From Lecture on 2019-09-17)
If you are traveling at 60 miles per hour, how long does
it take you to go 300 miles?
\end{problem}


\vfill ~

\begin{problem} (From the Math Review Lab)
What is this number? Give your answer in scientific notation.
$$
\frac{(7\times10^{-34})\times(3\times10^8)}{5\times10^{-7}}
$$
You don't need a calculator to solve this problem (\textit{hint: $3/5=0.6$}).
\end{problem}


\vfill ~

\begin{problem} (From the reading)
Classical mechanics, or Newtonian mechanics, is only valid in certain
circumstances. When do the laws of classical mechanics, like $F =
m\,a$ for example, become wrong or break down? There are many answers
to this problem; I will take anything correct.
\end{problem}


\vfill ~


\clearpage


\begin{problem} (From Lecture on 2019-09-05)
The molar weight of water is $18\,\g$. How many molecules would there
be, therefore, in $18\,\g$ of water? You don't need a calculator for
this.
\end{problem}


\vfill ~

\begin{problem} (From Problem Set 1)
What is the approximate thickness of a stack of 1000 20-dollar bills?
No need to be precise, and use any units you like.
\end{problem}


\vfill ~

\begin{problem} (From Problem Set 1)
What is the relationship between the energy $E$ and wavelength
$\lambda$ of a photon? Give a formula that involves energy $E$,
Planck's Constant $h$, the speed of light $c$, and wavelength
$\lambda$ (or whatever you need).
\end{problem}

\vfill ~

\begin{problem} (From the Kinematics Lab)
Here is a data table of times, positions, and velocities in SI units:\\
\rule{1.0in}{0pt}\begin{tabular}{c|c|c}
time $t$ ($\s$) & position $x$ ($\m$) & velocity $v$ ($\m\,\s^{-1}$) \\
\hline
1 & 1.15 & 1.3 \\
2 & 2.60 & 1.6 \\
3 & 4.35 & 1.9 \\
\hline
\end{tabular}\\
What is the average acceleration in the time interval from $2\,\s$ to $3\,\s$?
\end{problem}


\vfill ~


\cleardoublepage



\noindent
Name: \rule[-1ex]{0.60\textwidth}{0.1pt}
NetID: \rule[-1ex]{0.20\textwidth}{0.1pt}

\section*{\textsl{Einstein's Universe} Term Exam 1}
\setcounter{problem}{1}


\begin{problem} (From Lecture on 2019-09-17)
If you are traveling at 60 miles per hour, how long does
it take you to go 300 miles?
\end{problem}


\vfill ~

\begin{problem} (From the reading)
Classical mechanics, or Newtonian mechanics, is only valid in certain
circumstances. When do the laws of classical mechanics, like $F =
m\,a$ for example, become wrong or break down? There are many answers
to this problem; I will take anything correct.
\end{problem}


\vfill ~

\begin{problem} (From Problem Set 1)
What is the relationship between the energy $E$ and wavelength
$\lambda$ of a photon? Give a formula that involves energy $E$,
Planck's Constant $h$, the speed of light $c$, and wavelength
$\lambda$ (or whatever you need).
\end{problem}

\vfill ~

\begin{problem} (From the Kinematics Lab)
Here is a data table of times, positions, and velocities in SI units:\\
\rule{1.0in}{0pt}\begin{tabular}{c|c|c}
time $t$ ($\s$) & position $x$ ($\m$) & velocity $v$ ($\m\,\s^{-1}$) \\
\hline
1 & 1.15 & 1.3 \\
2 & 2.60 & 1.6 \\
3 & 4.35 & 1.9 \\
\hline
\end{tabular}\\
What is the average acceleration in the time interval from $2\,\s$ to $3\,\s$?
\end{problem}


\vfill ~


\clearpage


\begin{problem} (From Lecture on 2019-09-05)
The molar weight of water is $18\,\g$. How many molecules would there
be, therefore, in $18\,\g$ of water? You don't need a calculator for
this.
\end{problem}


\vfill ~

\begin{problem} (From Problem Set 1)
What is the approximate thickness of a stack of 1000 20-dollar bills?
No need to be precise, and use any units you like.
\end{problem}


\vfill ~

\begin{problem} (From the Math Review Lab)
What is this number? Give your answer in scientific notation.
$$
\frac{(7\times10^{-34})\times(3\times10^8)}{5\times10^{-7}}
$$
You don't need a calculator to solve this problem (\textit{hint: $3/5=0.6$}).
\end{problem}


\vfill ~

\begin{problem} (From the reading)
What musical instrument did Einstein most enjoy playing?
\end{problem}


\vfill ~


\cleardoublepage



\noindent
Name: \rule[-1ex]{0.60\textwidth}{0.1pt}
NetID: \rule[-1ex]{0.20\textwidth}{0.1pt}

\section*{\textsl{Einstein's Universe} Term Exam 1}
\setcounter{problem}{1}


\begin{problem} (From Lecture on 2019-09-05)
The molar weight of water is $18\,\g$. How many molecules would there
be, therefore, in $18\,\g$ of water? You don't need a calculator for
this.
\end{problem}


\vfill ~

\begin{problem} (From the Math Review Lab)
What is this number? Give your answer in scientific notation.
$$
\frac{(7\times10^{-34})\times(3\times10^8)}{5\times10^{-7}}
$$
You don't need a calculator to solve this problem (\textit{hint: $3/5=0.6$}).
\end{problem}


\vfill ~

\begin{problem} (From Problem Set 1)
What is the relationship between the energy $E$ and wavelength
$\lambda$ of a photon? Give a formula that involves energy $E$,
Planck's Constant $h$, the speed of light $c$, and wavelength
$\lambda$ (or whatever you need).
\end{problem}

\vfill ~

\begin{problem} (From Problem Set 1)
What is the approximate thickness of a stack of 1000 20-dollar bills?
No need to be precise, and use any units you like.
\end{problem}


\vfill ~


\clearpage


\begin{problem} (From the reading)
Classical mechanics, or Newtonian mechanics, is only valid in certain
circumstances. When do the laws of classical mechanics, like $F =
m\,a$ for example, become wrong or break down? There are many answers
to this problem; I will take anything correct.
\end{problem}


\vfill ~

\begin{problem} (From Lecture on 2019-09-17)
If you are traveling at 60 miles per hour, how long does
it take you to go 300 miles?
\end{problem}


\vfill ~

\begin{problem} (From the reading)
What musical instrument did Einstein most enjoy playing?
\end{problem}


\vfill ~

\begin{problem} (From the Kinematics Lab)
Here is a data table of times, positions, and velocities in SI units:\\
\rule{1.0in}{0pt}\begin{tabular}{c|c|c}
time $t$ ($\s$) & position $x$ ($\m$) & velocity $v$ ($\m\,\s^{-1}$) \\
\hline
1 & 1.15 & 1.3 \\
2 & 2.60 & 1.6 \\
3 & 4.35 & 1.9 \\
\hline
\end{tabular}\\
What is the average acceleration in the time interval from $2\,\s$ to $3\,\s$?
\end{problem}


\vfill ~


\cleardoublepage



\noindent
Name: \rule[-1ex]{0.60\textwidth}{0.1pt}
NetID: \rule[-1ex]{0.20\textwidth}{0.1pt}

\section*{\textsl{Einstein's Universe} Term Exam 1}
\setcounter{problem}{1}


\begin{problem} (From Problem Set 1)
What is the approximate thickness of a stack of 1000 20-dollar bills?
No need to be precise, and use any units you like.
\end{problem}


\vfill ~

\begin{problem} (From Problem Set 1)
What is the relationship between the energy $E$ and wavelength
$\lambda$ of a photon? Give a formula that involves energy $E$,
Planck's Constant $h$, the speed of light $c$, and wavelength
$\lambda$ (or whatever you need).
\end{problem}

\vfill ~

\begin{problem} (From the reading)
Classical mechanics, or Newtonian mechanics, is only valid in certain
circumstances. When do the laws of classical mechanics, like $F =
m\,a$ for example, become wrong or break down? There are many answers
to this problem; I will take anything correct.
\end{problem}


\vfill ~

\begin{problem} (From the Math Review Lab)
What is this number? Give your answer in scientific notation.
$$
\frac{(7\times10^{-34})\times(3\times10^8)}{5\times10^{-7}}
$$
You don't need a calculator to solve this problem (\textit{hint: $3/5=0.6$}).
\end{problem}


\vfill ~


\clearpage


\begin{problem} (From Lecture on 2019-09-05)
The molar weight of water is $18\,\g$. How many molecules would there
be, therefore, in $18\,\g$ of water? You don't need a calculator for
this.
\end{problem}


\vfill ~

\begin{problem} (From the reading)
What musical instrument did Einstein most enjoy playing?
\end{problem}


\vfill ~

\begin{problem} (From Lecture on 2019-09-17)
If you are traveling at 60 miles per hour, how long does
it take you to go 300 miles?
\end{problem}


\vfill ~

\begin{problem} (From the Kinematics Lab)
Here is a data table of times, positions, and velocities in SI units:\\
\rule{1.0in}{0pt}\begin{tabular}{c|c|c}
time $t$ ($\s$) & position $x$ ($\m$) & velocity $v$ ($\m\,\s^{-1}$) \\
\hline
1 & 1.15 & 1.3 \\
2 & 2.60 & 1.6 \\
3 & 4.35 & 1.9 \\
\hline
\end{tabular}\\
What is the average acceleration in the time interval from $2\,\s$ to $3\,\s$?
\end{problem}


\vfill ~


\cleardoublepage



\noindent
Name: \rule[-1ex]{0.60\textwidth}{0.1pt}
NetID: \rule[-1ex]{0.20\textwidth}{0.1pt}

\section*{\textsl{Einstein's Universe} Term Exam 1}
\setcounter{problem}{1}


\begin{problem} (From Problem Set 1)
What is the approximate thickness of a stack of 1000 20-dollar bills?
No need to be precise, and use any units you like.
\end{problem}


\vfill ~

\begin{problem} (From the reading)
What musical instrument did Einstein most enjoy playing?
\end{problem}


\vfill ~

\begin{problem} (From the Math Review Lab)
What is this number? Give your answer in scientific notation.
$$
\frac{(7\times10^{-34})\times(3\times10^8)}{5\times10^{-7}}
$$
You don't need a calculator to solve this problem (\textit{hint: $3/5=0.6$}).
\end{problem}


\vfill ~

\begin{problem} (From the reading)
Classical mechanics, or Newtonian mechanics, is only valid in certain
circumstances. When do the laws of classical mechanics, like $F =
m\,a$ for example, become wrong or break down? There are many answers
to this problem; I will take anything correct.
\end{problem}


\vfill ~


\clearpage


\begin{problem} (From Problem Set 1)
What is the relationship between the energy $E$ and wavelength
$\lambda$ of a photon? Give a formula that involves energy $E$,
Planck's Constant $h$, the speed of light $c$, and wavelength
$\lambda$ (or whatever you need).
\end{problem}

\vfill ~

\begin{problem} (From Lecture on 2019-09-17)
If you are traveling at 60 miles per hour, how long does
it take you to go 300 miles?
\end{problem}


\vfill ~

\begin{problem} (From Lecture on 2019-09-05)
The molar weight of water is $18\,\g$. How many molecules would there
be, therefore, in $18\,\g$ of water? You don't need a calculator for
this.
\end{problem}


\vfill ~

\begin{problem} (From the Kinematics Lab)
Here is a data table of times, positions, and velocities in SI units:\\
\rule{1.0in}{0pt}\begin{tabular}{c|c|c}
time $t$ ($\s$) & position $x$ ($\m$) & velocity $v$ ($\m\,\s^{-1}$) \\
\hline
1 & 1.15 & 1.3 \\
2 & 2.60 & 1.6 \\
3 & 4.35 & 1.9 \\
\hline
\end{tabular}\\
What is the average acceleration in the time interval from $2\,\s$ to $3\,\s$?
\end{problem}


\vfill ~


\cleardoublepage



\noindent
Name: \rule[-1ex]{0.60\textwidth}{0.1pt}
NetID: \rule[-1ex]{0.20\textwidth}{0.1pt}

\section*{\textsl{Einstein's Universe} Term Exam 1}
\setcounter{problem}{1}


\begin{problem} (From the Math Review Lab)
What is this number? Give your answer in scientific notation.
$$
\frac{(7\times10^{-34})\times(3\times10^8)}{5\times10^{-7}}
$$
You don't need a calculator to solve this problem (\textit{hint: $3/5=0.6$}).
\end{problem}


\vfill ~

\begin{problem} (From Problem Set 1)
What is the approximate thickness of a stack of 1000 20-dollar bills?
No need to be precise, and use any units you like.
\end{problem}


\vfill ~

\begin{problem} (From Lecture on 2019-09-05)
The molar weight of water is $18\,\g$. How many molecules would there
be, therefore, in $18\,\g$ of water? You don't need a calculator for
this.
\end{problem}


\vfill ~

\begin{problem} (From the Kinematics Lab)
Here is a data table of times, positions, and velocities in SI units:\\
\rule{1.0in}{0pt}\begin{tabular}{c|c|c}
time $t$ ($\s$) & position $x$ ($\m$) & velocity $v$ ($\m\,\s^{-1}$) \\
\hline
1 & 1.15 & 1.3 \\
2 & 2.60 & 1.6 \\
3 & 4.35 & 1.9 \\
\hline
\end{tabular}\\
What is the average acceleration in the time interval from $2\,\s$ to $3\,\s$?
\end{problem}


\vfill ~


\clearpage


\begin{problem} (From the reading)
Classical mechanics, or Newtonian mechanics, is only valid in certain
circumstances. When do the laws of classical mechanics, like $F =
m\,a$ for example, become wrong or break down? There are many answers
to this problem; I will take anything correct.
\end{problem}


\vfill ~

\begin{problem} (From Lecture on 2019-09-17)
If you are traveling at 60 miles per hour, how long does
it take you to go 300 miles?
\end{problem}


\vfill ~

\begin{problem} (From Problem Set 1)
What is the relationship between the energy $E$ and wavelength
$\lambda$ of a photon? Give a formula that involves energy $E$,
Planck's Constant $h$, the speed of light $c$, and wavelength
$\lambda$ (or whatever you need).
\end{problem}

\vfill ~

\begin{problem} (From the reading)
What musical instrument did Einstein most enjoy playing?
\end{problem}


\vfill ~


\cleardoublepage



\noindent
Name: \rule[-1ex]{0.60\textwidth}{0.1pt}
NetID: \rule[-1ex]{0.20\textwidth}{0.1pt}

\section*{\textsl{Einstein's Universe} Term Exam 1}
\setcounter{problem}{1}


\begin{problem} (From the Kinematics Lab)
Here is a data table of times, positions, and velocities in SI units:\\
\rule{1.0in}{0pt}\begin{tabular}{c|c|c}
time $t$ ($\s$) & position $x$ ($\m$) & velocity $v$ ($\m\,\s^{-1}$) \\
\hline
1 & 1.15 & 1.3 \\
2 & 2.60 & 1.6 \\
3 & 4.35 & 1.9 \\
\hline
\end{tabular}\\
What is the average acceleration in the time interval from $2\,\s$ to $3\,\s$?
\end{problem}


\vfill ~

\begin{problem} (From Problem Set 1)
What is the approximate thickness of a stack of 1000 20-dollar bills?
No need to be precise, and use any units you like.
\end{problem}


\vfill ~

\begin{problem} (From Lecture on 2019-09-05)
The molar weight of water is $18\,\g$. How many molecules would there
be, therefore, in $18\,\g$ of water? You don't need a calculator for
this.
\end{problem}


\vfill ~

\begin{problem} (From the reading)
What musical instrument did Einstein most enjoy playing?
\end{problem}


\vfill ~


\clearpage


\begin{problem} (From Lecture on 2019-09-17)
If you are traveling at 60 miles per hour, how long does
it take you to go 300 miles?
\end{problem}


\vfill ~

\begin{problem} (From the Math Review Lab)
What is this number? Give your answer in scientific notation.
$$
\frac{(7\times10^{-34})\times(3\times10^8)}{5\times10^{-7}}
$$
You don't need a calculator to solve this problem (\textit{hint: $3/5=0.6$}).
\end{problem}


\vfill ~

\begin{problem} (From Problem Set 1)
What is the relationship between the energy $E$ and wavelength
$\lambda$ of a photon? Give a formula that involves energy $E$,
Planck's Constant $h$, the speed of light $c$, and wavelength
$\lambda$ (or whatever you need).
\end{problem}

\vfill ~

\begin{problem} (From the reading)
Classical mechanics, or Newtonian mechanics, is only valid in certain
circumstances. When do the laws of classical mechanics, like $F =
m\,a$ for example, become wrong or break down? There are many answers
to this problem; I will take anything correct.
\end{problem}


\vfill ~


\cleardoublepage



\noindent
Name: \rule[-1ex]{0.60\textwidth}{0.1pt}
NetID: \rule[-1ex]{0.20\textwidth}{0.1pt}

\section*{\textsl{Einstein's Universe} Term Exam 1}
\setcounter{problem}{1}


\begin{problem} (From the reading)
What musical instrument did Einstein most enjoy playing?
\end{problem}


\vfill ~

\begin{problem} (From the Kinematics Lab)
Here is a data table of times, positions, and velocities in SI units:\\
\rule{1.0in}{0pt}\begin{tabular}{c|c|c}
time $t$ ($\s$) & position $x$ ($\m$) & velocity $v$ ($\m\,\s^{-1}$) \\
\hline
1 & 1.15 & 1.3 \\
2 & 2.60 & 1.6 \\
3 & 4.35 & 1.9 \\
\hline
\end{tabular}\\
What is the average acceleration in the time interval from $2\,\s$ to $3\,\s$?
\end{problem}


\vfill ~

\begin{problem} (From Problem Set 1)
What is the approximate thickness of a stack of 1000 20-dollar bills?
No need to be precise, and use any units you like.
\end{problem}


\vfill ~

\begin{problem} (From the Math Review Lab)
What is this number? Give your answer in scientific notation.
$$
\frac{(7\times10^{-34})\times(3\times10^8)}{5\times10^{-7}}
$$
You don't need a calculator to solve this problem (\textit{hint: $3/5=0.6$}).
\end{problem}


\vfill ~


\clearpage


\begin{problem} (From Lecture on 2019-09-05)
The molar weight of water is $18\,\g$. How many molecules would there
be, therefore, in $18\,\g$ of water? You don't need a calculator for
this.
\end{problem}


\vfill ~

\begin{problem} (From Problem Set 1)
What is the relationship between the energy $E$ and wavelength
$\lambda$ of a photon? Give a formula that involves energy $E$,
Planck's Constant $h$, the speed of light $c$, and wavelength
$\lambda$ (or whatever you need).
\end{problem}

\vfill ~

\begin{problem} (From the reading)
Classical mechanics, or Newtonian mechanics, is only valid in certain
circumstances. When do the laws of classical mechanics, like $F =
m\,a$ for example, become wrong or break down? There are many answers
to this problem; I will take anything correct.
\end{problem}


\vfill ~

\begin{problem} (From Lecture on 2019-09-17)
If you are traveling at 60 miles per hour, how long does
it take you to go 300 miles?
\end{problem}


\vfill ~


\cleardoublepage



\noindent
Name: \rule[-1ex]{0.60\textwidth}{0.1pt}
NetID: \rule[-1ex]{0.20\textwidth}{0.1pt}

\section*{\textsl{Einstein's Universe} Term Exam 1}
\setcounter{problem}{1}


\begin{problem} (From Lecture on 2019-09-17)
If you are traveling at 60 miles per hour, how long does
it take you to go 300 miles?
\end{problem}


\vfill ~

\begin{problem} (From the reading)
Classical mechanics, or Newtonian mechanics, is only valid in certain
circumstances. When do the laws of classical mechanics, like $F =
m\,a$ for example, become wrong or break down? There are many answers
to this problem; I will take anything correct.
\end{problem}


\vfill ~

\begin{problem} (From Lecture on 2019-09-05)
The molar weight of water is $18\,\g$. How many molecules would there
be, therefore, in $18\,\g$ of water? You don't need a calculator for
this.
\end{problem}


\vfill ~

\begin{problem} (From the reading)
What musical instrument did Einstein most enjoy playing?
\end{problem}


\vfill ~


\clearpage


\begin{problem} (From the Math Review Lab)
What is this number? Give your answer in scientific notation.
$$
\frac{(7\times10^{-34})\times(3\times10^8)}{5\times10^{-7}}
$$
You don't need a calculator to solve this problem (\textit{hint: $3/5=0.6$}).
\end{problem}


\vfill ~

\begin{problem} (From the Kinematics Lab)
Here is a data table of times, positions, and velocities in SI units:\\
\rule{1.0in}{0pt}\begin{tabular}{c|c|c}
time $t$ ($\s$) & position $x$ ($\m$) & velocity $v$ ($\m\,\s^{-1}$) \\
\hline
1 & 1.15 & 1.3 \\
2 & 2.60 & 1.6 \\
3 & 4.35 & 1.9 \\
\hline
\end{tabular}\\
What is the average acceleration in the time interval from $2\,\s$ to $3\,\s$?
\end{problem}


\vfill ~

\begin{problem} (From Problem Set 1)
What is the approximate thickness of a stack of 1000 20-dollar bills?
No need to be precise, and use any units you like.
\end{problem}


\vfill ~

\begin{problem} (From Problem Set 1)
What is the relationship between the energy $E$ and wavelength
$\lambda$ of a photon? Give a formula that involves energy $E$,
Planck's Constant $h$, the speed of light $c$, and wavelength
$\lambda$ (or whatever you need).
\end{problem}

\vfill ~


\cleardoublepage



\noindent
Name: \rule[-1ex]{0.60\textwidth}{0.1pt}
NetID: \rule[-1ex]{0.20\textwidth}{0.1pt}

\section*{\textsl{Einstein's Universe} Term Exam 1}
\setcounter{problem}{1}


\begin{problem} (From the Math Review Lab)
What is this number? Give your answer in scientific notation.
$$
\frac{(7\times10^{-34})\times(3\times10^8)}{5\times10^{-7}}
$$
You don't need a calculator to solve this problem (\textit{hint: $3/5=0.6$}).
\end{problem}


\vfill ~

\begin{problem} (From the reading)
What musical instrument did Einstein most enjoy playing?
\end{problem}


\vfill ~

\begin{problem} (From Problem Set 1)
What is the approximate thickness of a stack of 1000 20-dollar bills?
No need to be precise, and use any units you like.
\end{problem}


\vfill ~

\begin{problem} (From Lecture on 2019-09-05)
The molar weight of water is $18\,\g$. How many molecules would there
be, therefore, in $18\,\g$ of water? You don't need a calculator for
this.
\end{problem}


\vfill ~


\clearpage


\begin{problem} (From the reading)
Classical mechanics, or Newtonian mechanics, is only valid in certain
circumstances. When do the laws of classical mechanics, like $F =
m\,a$ for example, become wrong or break down? There are many answers
to this problem; I will take anything correct.
\end{problem}


\vfill ~

\begin{problem} (From Problem Set 1)
What is the relationship between the energy $E$ and wavelength
$\lambda$ of a photon? Give a formula that involves energy $E$,
Planck's Constant $h$, the speed of light $c$, and wavelength
$\lambda$ (or whatever you need).
\end{problem}

\vfill ~

\begin{problem} (From the Kinematics Lab)
Here is a data table of times, positions, and velocities in SI units:\\
\rule{1.0in}{0pt}\begin{tabular}{c|c|c}
time $t$ ($\s$) & position $x$ ($\m$) & velocity $v$ ($\m\,\s^{-1}$) \\
\hline
1 & 1.15 & 1.3 \\
2 & 2.60 & 1.6 \\
3 & 4.35 & 1.9 \\
\hline
\end{tabular}\\
What is the average acceleration in the time interval from $2\,\s$ to $3\,\s$?
\end{problem}


\vfill ~

\begin{problem} (From Lecture on 2019-09-17)
If you are traveling at 60 miles per hour, how long does
it take you to go 300 miles?
\end{problem}


\vfill ~


\cleardoublepage



\noindent
Name: \rule[-1ex]{0.60\textwidth}{0.1pt}
NetID: \rule[-1ex]{0.20\textwidth}{0.1pt}

\section*{\textsl{Einstein's Universe} Term Exam 1}
\setcounter{problem}{1}


\begin{problem} (From the Math Review Lab)
What is this number? Give your answer in scientific notation.
$$
\frac{(7\times10^{-34})\times(3\times10^8)}{5\times10^{-7}}
$$
You don't need a calculator to solve this problem (\textit{hint: $3/5=0.6$}).
\end{problem}


\vfill ~

\begin{problem} (From the reading)
Classical mechanics, or Newtonian mechanics, is only valid in certain
circumstances. When do the laws of classical mechanics, like $F =
m\,a$ for example, become wrong or break down? There are many answers
to this problem; I will take anything correct.
\end{problem}


\vfill ~

\begin{problem} (From Problem Set 1)
What is the approximate thickness of a stack of 1000 20-dollar bills?
No need to be precise, and use any units you like.
\end{problem}


\vfill ~

\begin{problem} (From the reading)
What musical instrument did Einstein most enjoy playing?
\end{problem}


\vfill ~


\clearpage


\begin{problem} (From Problem Set 1)
What is the relationship between the energy $E$ and wavelength
$\lambda$ of a photon? Give a formula that involves energy $E$,
Planck's Constant $h$, the speed of light $c$, and wavelength
$\lambda$ (or whatever you need).
\end{problem}

\vfill ~

\begin{problem} (From the Kinematics Lab)
Here is a data table of times, positions, and velocities in SI units:\\
\rule{1.0in}{0pt}\begin{tabular}{c|c|c}
time $t$ ($\s$) & position $x$ ($\m$) & velocity $v$ ($\m\,\s^{-1}$) \\
\hline
1 & 1.15 & 1.3 \\
2 & 2.60 & 1.6 \\
3 & 4.35 & 1.9 \\
\hline
\end{tabular}\\
What is the average acceleration in the time interval from $2\,\s$ to $3\,\s$?
\end{problem}


\vfill ~

\begin{problem} (From Lecture on 2019-09-05)
The molar weight of water is $18\,\g$. How many molecules would there
be, therefore, in $18\,\g$ of water? You don't need a calculator for
this.
\end{problem}


\vfill ~

\begin{problem} (From Lecture on 2019-09-17)
If you are traveling at 60 miles per hour, how long does
it take you to go 300 miles?
\end{problem}


\vfill ~


\cleardoublepage



\noindent
Name: \rule[-1ex]{0.60\textwidth}{0.1pt}
NetID: \rule[-1ex]{0.20\textwidth}{0.1pt}

\section*{\textsl{Einstein's Universe} Term Exam 1}
\setcounter{problem}{1}


\begin{problem} (From Lecture on 2019-09-05)
The molar weight of water is $18\,\g$. How many molecules would there
be, therefore, in $18\,\g$ of water? You don't need a calculator for
this.
\end{problem}


\vfill ~

\begin{problem} (From the reading)
What musical instrument did Einstein most enjoy playing?
\end{problem}


\vfill ~

\begin{problem} (From Lecture on 2019-09-17)
If you are traveling at 60 miles per hour, how long does
it take you to go 300 miles?
\end{problem}


\vfill ~

\begin{problem} (From the Kinematics Lab)
Here is a data table of times, positions, and velocities in SI units:\\
\rule{1.0in}{0pt}\begin{tabular}{c|c|c}
time $t$ ($\s$) & position $x$ ($\m$) & velocity $v$ ($\m\,\s^{-1}$) \\
\hline
1 & 1.15 & 1.3 \\
2 & 2.60 & 1.6 \\
3 & 4.35 & 1.9 \\
\hline
\end{tabular}\\
What is the average acceleration in the time interval from $2\,\s$ to $3\,\s$?
\end{problem}


\vfill ~


\clearpage


\begin{problem} (From Problem Set 1)
What is the approximate thickness of a stack of 1000 20-dollar bills?
No need to be precise, and use any units you like.
\end{problem}


\vfill ~

\begin{problem} (From Problem Set 1)
What is the relationship between the energy $E$ and wavelength
$\lambda$ of a photon? Give a formula that involves energy $E$,
Planck's Constant $h$, the speed of light $c$, and wavelength
$\lambda$ (or whatever you need).
\end{problem}

\vfill ~

\begin{problem} (From the Math Review Lab)
What is this number? Give your answer in scientific notation.
$$
\frac{(7\times10^{-34})\times(3\times10^8)}{5\times10^{-7}}
$$
You don't need a calculator to solve this problem (\textit{hint: $3/5=0.6$}).
\end{problem}


\vfill ~

\begin{problem} (From the reading)
Classical mechanics, or Newtonian mechanics, is only valid in certain
circumstances. When do the laws of classical mechanics, like $F =
m\,a$ for example, become wrong or break down? There are many answers
to this problem; I will take anything correct.
\end{problem}


\vfill ~


\cleardoublepage



\noindent
Name: \rule[-1ex]{0.60\textwidth}{0.1pt}
NetID: \rule[-1ex]{0.20\textwidth}{0.1pt}

\section*{\textsl{Einstein's Universe} Term Exam 1}
\setcounter{problem}{1}


\begin{problem} (From the Math Review Lab)
What is this number? Give your answer in scientific notation.
$$
\frac{(7\times10^{-34})\times(3\times10^8)}{5\times10^{-7}}
$$
You don't need a calculator to solve this problem (\textit{hint: $3/5=0.6$}).
\end{problem}


\vfill ~

\begin{problem} (From the reading)
Classical mechanics, or Newtonian mechanics, is only valid in certain
circumstances. When do the laws of classical mechanics, like $F =
m\,a$ for example, become wrong or break down? There are many answers
to this problem; I will take anything correct.
\end{problem}


\vfill ~

\begin{problem} (From the Kinematics Lab)
Here is a data table of times, positions, and velocities in SI units:\\
\rule{1.0in}{0pt}\begin{tabular}{c|c|c}
time $t$ ($\s$) & position $x$ ($\m$) & velocity $v$ ($\m\,\s^{-1}$) \\
\hline
1 & 1.15 & 1.3 \\
2 & 2.60 & 1.6 \\
3 & 4.35 & 1.9 \\
\hline
\end{tabular}\\
What is the average acceleration in the time interval from $2\,\s$ to $3\,\s$?
\end{problem}


\vfill ~

\begin{problem} (From Problem Set 1)
What is the relationship between the energy $E$ and wavelength
$\lambda$ of a photon? Give a formula that involves energy $E$,
Planck's Constant $h$, the speed of light $c$, and wavelength
$\lambda$ (or whatever you need).
\end{problem}

\vfill ~


\clearpage


\begin{problem} (From Problem Set 1)
What is the approximate thickness of a stack of 1000 20-dollar bills?
No need to be precise, and use any units you like.
\end{problem}


\vfill ~

\begin{problem} (From Lecture on 2019-09-17)
If you are traveling at 60 miles per hour, how long does
it take you to go 300 miles?
\end{problem}


\vfill ~

\begin{problem} (From Lecture on 2019-09-05)
The molar weight of water is $18\,\g$. How many molecules would there
be, therefore, in $18\,\g$ of water? You don't need a calculator for
this.
\end{problem}


\vfill ~

\begin{problem} (From the reading)
What musical instrument did Einstein most enjoy playing?
\end{problem}


\vfill ~


\cleardoublepage



\noindent
Name: \rule[-1ex]{0.60\textwidth}{0.1pt}
NetID: \rule[-1ex]{0.20\textwidth}{0.1pt}

\section*{\textsl{Einstein's Universe} Term Exam 1}
\setcounter{problem}{1}


\begin{problem} (From the Math Review Lab)
What is this number? Give your answer in scientific notation.
$$
\frac{(7\times10^{-34})\times(3\times10^8)}{5\times10^{-7}}
$$
You don't need a calculator to solve this problem (\textit{hint: $3/5=0.6$}).
\end{problem}


\vfill ~

\begin{problem} (From the reading)
What musical instrument did Einstein most enjoy playing?
\end{problem}


\vfill ~

\begin{problem} (From Problem Set 1)
What is the approximate thickness of a stack of 1000 20-dollar bills?
No need to be precise, and use any units you like.
\end{problem}


\vfill ~

\begin{problem} (From Lecture on 2019-09-05)
The molar weight of water is $18\,\g$. How many molecules would there
be, therefore, in $18\,\g$ of water? You don't need a calculator for
this.
\end{problem}


\vfill ~


\clearpage


\begin{problem} (From Lecture on 2019-09-17)
If you are traveling at 60 miles per hour, how long does
it take you to go 300 miles?
\end{problem}


\vfill ~

\begin{problem} (From Problem Set 1)
What is the relationship between the energy $E$ and wavelength
$\lambda$ of a photon? Give a formula that involves energy $E$,
Planck's Constant $h$, the speed of light $c$, and wavelength
$\lambda$ (or whatever you need).
\end{problem}

\vfill ~

\begin{problem} (From the Kinematics Lab)
Here is a data table of times, positions, and velocities in SI units:\\
\rule{1.0in}{0pt}\begin{tabular}{c|c|c}
time $t$ ($\s$) & position $x$ ($\m$) & velocity $v$ ($\m\,\s^{-1}$) \\
\hline
1 & 1.15 & 1.3 \\
2 & 2.60 & 1.6 \\
3 & 4.35 & 1.9 \\
\hline
\end{tabular}\\
What is the average acceleration in the time interval from $2\,\s$ to $3\,\s$?
\end{problem}


\vfill ~

\begin{problem} (From the reading)
Classical mechanics, or Newtonian mechanics, is only valid in certain
circumstances. When do the laws of classical mechanics, like $F =
m\,a$ for example, become wrong or break down? There are many answers
to this problem; I will take anything correct.
\end{problem}


\vfill ~


\cleardoublepage



\noindent
Name: \rule[-1ex]{0.60\textwidth}{0.1pt}
NetID: \rule[-1ex]{0.20\textwidth}{0.1pt}

\section*{\textsl{Einstein's Universe} Term Exam 1}
\setcounter{problem}{1}


\begin{problem} (From Lecture on 2019-09-17)
If you are traveling at 60 miles per hour, how long does
it take you to go 300 miles?
\end{problem}


\vfill ~

\begin{problem} (From the Math Review Lab)
What is this number? Give your answer in scientific notation.
$$
\frac{(7\times10^{-34})\times(3\times10^8)}{5\times10^{-7}}
$$
You don't need a calculator to solve this problem (\textit{hint: $3/5=0.6$}).
\end{problem}


\vfill ~

\begin{problem} (From the reading)
What musical instrument did Einstein most enjoy playing?
\end{problem}


\vfill ~

\begin{problem} (From Lecture on 2019-09-05)
The molar weight of water is $18\,\g$. How many molecules would there
be, therefore, in $18\,\g$ of water? You don't need a calculator for
this.
\end{problem}


\vfill ~


\clearpage


\begin{problem} (From the reading)
Classical mechanics, or Newtonian mechanics, is only valid in certain
circumstances. When do the laws of classical mechanics, like $F =
m\,a$ for example, become wrong or break down? There are many answers
to this problem; I will take anything correct.
\end{problem}


\vfill ~

\begin{problem} (From Problem Set 1)
What is the relationship between the energy $E$ and wavelength
$\lambda$ of a photon? Give a formula that involves energy $E$,
Planck's Constant $h$, the speed of light $c$, and wavelength
$\lambda$ (or whatever you need).
\end{problem}

\vfill ~

\begin{problem} (From Problem Set 1)
What is the approximate thickness of a stack of 1000 20-dollar bills?
No need to be precise, and use any units you like.
\end{problem}


\vfill ~

\begin{problem} (From the Kinematics Lab)
Here is a data table of times, positions, and velocities in SI units:\\
\rule{1.0in}{0pt}\begin{tabular}{c|c|c}
time $t$ ($\s$) & position $x$ ($\m$) & velocity $v$ ($\m\,\s^{-1}$) \\
\hline
1 & 1.15 & 1.3 \\
2 & 2.60 & 1.6 \\
3 & 4.35 & 1.9 \\
\hline
\end{tabular}\\
What is the average acceleration in the time interval from $2\,\s$ to $3\,\s$?
\end{problem}


\vfill ~


\cleardoublepage



\noindent
Name: \rule[-1ex]{0.60\textwidth}{0.1pt}
NetID: \rule[-1ex]{0.20\textwidth}{0.1pt}

\section*{\textsl{Einstein's Universe} Term Exam 1}
\setcounter{problem}{1}


\begin{problem} (From Problem Set 1)
What is the approximate thickness of a stack of 1000 20-dollar bills?
No need to be precise, and use any units you like.
\end{problem}


\vfill ~

\begin{problem} (From the reading)
Classical mechanics, or Newtonian mechanics, is only valid in certain
circumstances. When do the laws of classical mechanics, like $F =
m\,a$ for example, become wrong or break down? There are many answers
to this problem; I will take anything correct.
\end{problem}


\vfill ~

\begin{problem} (From the Kinematics Lab)
Here is a data table of times, positions, and velocities in SI units:\\
\rule{1.0in}{0pt}\begin{tabular}{c|c|c}
time $t$ ($\s$) & position $x$ ($\m$) & velocity $v$ ($\m\,\s^{-1}$) \\
\hline
1 & 1.15 & 1.3 \\
2 & 2.60 & 1.6 \\
3 & 4.35 & 1.9 \\
\hline
\end{tabular}\\
What is the average acceleration in the time interval from $2\,\s$ to $3\,\s$?
\end{problem}


\vfill ~

\begin{problem} (From Lecture on 2019-09-05)
The molar weight of water is $18\,\g$. How many molecules would there
be, therefore, in $18\,\g$ of water? You don't need a calculator for
this.
\end{problem}


\vfill ~


\clearpage


\begin{problem} (From the reading)
What musical instrument did Einstein most enjoy playing?
\end{problem}


\vfill ~

\begin{problem} (From Lecture on 2019-09-17)
If you are traveling at 60 miles per hour, how long does
it take you to go 300 miles?
\end{problem}


\vfill ~

\begin{problem} (From Problem Set 1)
What is the relationship between the energy $E$ and wavelength
$\lambda$ of a photon? Give a formula that involves energy $E$,
Planck's Constant $h$, the speed of light $c$, and wavelength
$\lambda$ (or whatever you need).
\end{problem}

\vfill ~

\begin{problem} (From the Math Review Lab)
What is this number? Give your answer in scientific notation.
$$
\frac{(7\times10^{-34})\times(3\times10^8)}{5\times10^{-7}}
$$
You don't need a calculator to solve this problem (\textit{hint: $3/5=0.6$}).
\end{problem}


\vfill ~


\cleardoublepage



\noindent
Name: \rule[-1ex]{0.60\textwidth}{0.1pt}
NetID: \rule[-1ex]{0.20\textwidth}{0.1pt}

\section*{\textsl{Einstein's Universe} Term Exam 1}
\setcounter{problem}{1}


\begin{problem} (From Lecture on 2019-09-05)
The molar weight of water is $18\,\g$. How many molecules would there
be, therefore, in $18\,\g$ of water? You don't need a calculator for
this.
\end{problem}


\vfill ~

\begin{problem} (From Problem Set 1)
What is the approximate thickness of a stack of 1000 20-dollar bills?
No need to be precise, and use any units you like.
\end{problem}


\vfill ~

\begin{problem} (From the reading)
Classical mechanics, or Newtonian mechanics, is only valid in certain
circumstances. When do the laws of classical mechanics, like $F =
m\,a$ for example, become wrong or break down? There are many answers
to this problem; I will take anything correct.
\end{problem}


\vfill ~

\begin{problem} (From the reading)
What musical instrument did Einstein most enjoy playing?
\end{problem}


\vfill ~


\clearpage


\begin{problem} (From the Math Review Lab)
What is this number? Give your answer in scientific notation.
$$
\frac{(7\times10^{-34})\times(3\times10^8)}{5\times10^{-7}}
$$
You don't need a calculator to solve this problem (\textit{hint: $3/5=0.6$}).
\end{problem}


\vfill ~

\begin{problem} (From Lecture on 2019-09-17)
If you are traveling at 60 miles per hour, how long does
it take you to go 300 miles?
\end{problem}


\vfill ~

\begin{problem} (From Problem Set 1)
What is the relationship between the energy $E$ and wavelength
$\lambda$ of a photon? Give a formula that involves energy $E$,
Planck's Constant $h$, the speed of light $c$, and wavelength
$\lambda$ (or whatever you need).
\end{problem}

\vfill ~

\begin{problem} (From the Kinematics Lab)
Here is a data table of times, positions, and velocities in SI units:\\
\rule{1.0in}{0pt}\begin{tabular}{c|c|c}
time $t$ ($\s$) & position $x$ ($\m$) & velocity $v$ ($\m\,\s^{-1}$) \\
\hline
1 & 1.15 & 1.3 \\
2 & 2.60 & 1.6 \\
3 & 4.35 & 1.9 \\
\hline
\end{tabular}\\
What is the average acceleration in the time interval from $2\,\s$ to $3\,\s$?
\end{problem}


\vfill ~


\cleardoublepage



\noindent
Name: \rule[-1ex]{0.60\textwidth}{0.1pt}
NetID: \rule[-1ex]{0.20\textwidth}{0.1pt}

\section*{\textsl{Einstein's Universe} Term Exam 1}
\setcounter{problem}{1}


\begin{problem} (From the Math Review Lab)
What is this number? Give your answer in scientific notation.
$$
\frac{(7\times10^{-34})\times(3\times10^8)}{5\times10^{-7}}
$$
You don't need a calculator to solve this problem (\textit{hint: $3/5=0.6$}).
\end{problem}


\vfill ~

\begin{problem} (From Problem Set 1)
What is the relationship between the energy $E$ and wavelength
$\lambda$ of a photon? Give a formula that involves energy $E$,
Planck's Constant $h$, the speed of light $c$, and wavelength
$\lambda$ (or whatever you need).
\end{problem}

\vfill ~

\begin{problem} (From the reading)
What musical instrument did Einstein most enjoy playing?
\end{problem}


\vfill ~

\begin{problem} (From Problem Set 1)
What is the approximate thickness of a stack of 1000 20-dollar bills?
No need to be precise, and use any units you like.
\end{problem}


\vfill ~


\clearpage


\begin{problem} (From Lecture on 2019-09-05)
The molar weight of water is $18\,\g$. How many molecules would there
be, therefore, in $18\,\g$ of water? You don't need a calculator for
this.
\end{problem}


\vfill ~

\begin{problem} (From the Kinematics Lab)
Here is a data table of times, positions, and velocities in SI units:\\
\rule{1.0in}{0pt}\begin{tabular}{c|c|c}
time $t$ ($\s$) & position $x$ ($\m$) & velocity $v$ ($\m\,\s^{-1}$) \\
\hline
1 & 1.15 & 1.3 \\
2 & 2.60 & 1.6 \\
3 & 4.35 & 1.9 \\
\hline
\end{tabular}\\
What is the average acceleration in the time interval from $2\,\s$ to $3\,\s$?
\end{problem}


\vfill ~

\begin{problem} (From the reading)
Classical mechanics, or Newtonian mechanics, is only valid in certain
circumstances. When do the laws of classical mechanics, like $F =
m\,a$ for example, become wrong or break down? There are many answers
to this problem; I will take anything correct.
\end{problem}


\vfill ~

\begin{problem} (From Lecture on 2019-09-17)
If you are traveling at 60 miles per hour, how long does
it take you to go 300 miles?
\end{problem}


\vfill ~


\cleardoublepage



\noindent
Name: \rule[-1ex]{0.60\textwidth}{0.1pt}
NetID: \rule[-1ex]{0.20\textwidth}{0.1pt}

\section*{\textsl{Einstein's Universe} Term Exam 1}
\setcounter{problem}{1}


\begin{problem} (From the Math Review Lab)
What is this number? Give your answer in scientific notation.
$$
\frac{(7\times10^{-34})\times(3\times10^8)}{5\times10^{-7}}
$$
You don't need a calculator to solve this problem (\textit{hint: $3/5=0.6$}).
\end{problem}


\vfill ~

\begin{problem} (From the Kinematics Lab)
Here is a data table of times, positions, and velocities in SI units:\\
\rule{1.0in}{0pt}\begin{tabular}{c|c|c}
time $t$ ($\s$) & position $x$ ($\m$) & velocity $v$ ($\m\,\s^{-1}$) \\
\hline
1 & 1.15 & 1.3 \\
2 & 2.60 & 1.6 \\
3 & 4.35 & 1.9 \\
\hline
\end{tabular}\\
What is the average acceleration in the time interval from $2\,\s$ to $3\,\s$?
\end{problem}


\vfill ~

\begin{problem} (From Problem Set 1)
What is the approximate thickness of a stack of 1000 20-dollar bills?
No need to be precise, and use any units you like.
\end{problem}


\vfill ~

\begin{problem} (From Lecture on 2019-09-17)
If you are traveling at 60 miles per hour, how long does
it take you to go 300 miles?
\end{problem}


\vfill ~


\clearpage


\begin{problem} (From Problem Set 1)
What is the relationship between the energy $E$ and wavelength
$\lambda$ of a photon? Give a formula that involves energy $E$,
Planck's Constant $h$, the speed of light $c$, and wavelength
$\lambda$ (or whatever you need).
\end{problem}

\vfill ~

\begin{problem} (From the reading)
Classical mechanics, or Newtonian mechanics, is only valid in certain
circumstances. When do the laws of classical mechanics, like $F =
m\,a$ for example, become wrong or break down? There are many answers
to this problem; I will take anything correct.
\end{problem}


\vfill ~

\begin{problem} (From Lecture on 2019-09-05)
The molar weight of water is $18\,\g$. How many molecules would there
be, therefore, in $18\,\g$ of water? You don't need a calculator for
this.
\end{problem}


\vfill ~

\begin{problem} (From the reading)
What musical instrument did Einstein most enjoy playing?
\end{problem}


\vfill ~


\cleardoublepage



\noindent
Name: \rule[-1ex]{0.60\textwidth}{0.1pt}
NetID: \rule[-1ex]{0.20\textwidth}{0.1pt}

\section*{\textsl{Einstein's Universe} Term Exam 1}
\setcounter{problem}{1}


\begin{problem} (From Problem Set 1)
What is the relationship between the energy $E$ and wavelength
$\lambda$ of a photon? Give a formula that involves energy $E$,
Planck's Constant $h$, the speed of light $c$, and wavelength
$\lambda$ (or whatever you need).
\end{problem}

\vfill ~

\begin{problem} (From the Math Review Lab)
What is this number? Give your answer in scientific notation.
$$
\frac{(7\times10^{-34})\times(3\times10^8)}{5\times10^{-7}}
$$
You don't need a calculator to solve this problem (\textit{hint: $3/5=0.6$}).
\end{problem}


\vfill ~

\begin{problem} (From Lecture on 2019-09-05)
The molar weight of water is $18\,\g$. How many molecules would there
be, therefore, in $18\,\g$ of water? You don't need a calculator for
this.
\end{problem}


\vfill ~

\begin{problem} (From the reading)
What musical instrument did Einstein most enjoy playing?
\end{problem}


\vfill ~


\clearpage


\begin{problem} (From the reading)
Classical mechanics, or Newtonian mechanics, is only valid in certain
circumstances. When do the laws of classical mechanics, like $F =
m\,a$ for example, become wrong or break down? There are many answers
to this problem; I will take anything correct.
\end{problem}


\vfill ~

\begin{problem} (From Problem Set 1)
What is the approximate thickness of a stack of 1000 20-dollar bills?
No need to be precise, and use any units you like.
\end{problem}


\vfill ~

\begin{problem} (From the Kinematics Lab)
Here is a data table of times, positions, and velocities in SI units:\\
\rule{1.0in}{0pt}\begin{tabular}{c|c|c}
time $t$ ($\s$) & position $x$ ($\m$) & velocity $v$ ($\m\,\s^{-1}$) \\
\hline
1 & 1.15 & 1.3 \\
2 & 2.60 & 1.6 \\
3 & 4.35 & 1.9 \\
\hline
\end{tabular}\\
What is the average acceleration in the time interval from $2\,\s$ to $3\,\s$?
\end{problem}


\vfill ~

\begin{problem} (From Lecture on 2019-09-17)
If you are traveling at 60 miles per hour, how long does
it take you to go 300 miles?
\end{problem}


\vfill ~


\cleardoublepage



\noindent
Name: \rule[-1ex]{0.60\textwidth}{0.1pt}
NetID: \rule[-1ex]{0.20\textwidth}{0.1pt}

\section*{\textsl{Einstein's Universe} Term Exam 1}
\setcounter{problem}{1}


\begin{problem} (From Problem Set 1)
What is the approximate thickness of a stack of 1000 20-dollar bills?
No need to be precise, and use any units you like.
\end{problem}


\vfill ~

\begin{problem} (From Problem Set 1)
What is the relationship between the energy $E$ and wavelength
$\lambda$ of a photon? Give a formula that involves energy $E$,
Planck's Constant $h$, the speed of light $c$, and wavelength
$\lambda$ (or whatever you need).
\end{problem}

\vfill ~

\begin{problem} (From the reading)
What musical instrument did Einstein most enjoy playing?
\end{problem}


\vfill ~

\begin{problem} (From the Math Review Lab)
What is this number? Give your answer in scientific notation.
$$
\frac{(7\times10^{-34})\times(3\times10^8)}{5\times10^{-7}}
$$
You don't need a calculator to solve this problem (\textit{hint: $3/5=0.6$}).
\end{problem}


\vfill ~


\clearpage


\begin{problem} (From the Kinematics Lab)
Here is a data table of times, positions, and velocities in SI units:\\
\rule{1.0in}{0pt}\begin{tabular}{c|c|c}
time $t$ ($\s$) & position $x$ ($\m$) & velocity $v$ ($\m\,\s^{-1}$) \\
\hline
1 & 1.15 & 1.3 \\
2 & 2.60 & 1.6 \\
3 & 4.35 & 1.9 \\
\hline
\end{tabular}\\
What is the average acceleration in the time interval from $2\,\s$ to $3\,\s$?
\end{problem}


\vfill ~

\begin{problem} (From the reading)
Classical mechanics, or Newtonian mechanics, is only valid in certain
circumstances. When do the laws of classical mechanics, like $F =
m\,a$ for example, become wrong or break down? There are many answers
to this problem; I will take anything correct.
\end{problem}


\vfill ~

\begin{problem} (From Lecture on 2019-09-05)
The molar weight of water is $18\,\g$. How many molecules would there
be, therefore, in $18\,\g$ of water? You don't need a calculator for
this.
\end{problem}


\vfill ~

\begin{problem} (From Lecture on 2019-09-17)
If you are traveling at 60 miles per hour, how long does
it take you to go 300 miles?
\end{problem}


\vfill ~


\cleardoublepage



\noindent
Name: \rule[-1ex]{0.60\textwidth}{0.1pt}
NetID: \rule[-1ex]{0.20\textwidth}{0.1pt}

\section*{\textsl{Einstein's Universe} Term Exam 1}
\setcounter{problem}{1}


\begin{problem} (From Lecture on 2019-09-17)
If you are traveling at 60 miles per hour, how long does
it take you to go 300 miles?
\end{problem}


\vfill ~

\begin{problem} (From Problem Set 1)
What is the relationship between the energy $E$ and wavelength
$\lambda$ of a photon? Give a formula that involves energy $E$,
Planck's Constant $h$, the speed of light $c$, and wavelength
$\lambda$ (or whatever you need).
\end{problem}

\vfill ~

\begin{problem} (From Lecture on 2019-09-05)
The molar weight of water is $18\,\g$. How many molecules would there
be, therefore, in $18\,\g$ of water? You don't need a calculator for
this.
\end{problem}


\vfill ~

\begin{problem} (From the Kinematics Lab)
Here is a data table of times, positions, and velocities in SI units:\\
\rule{1.0in}{0pt}\begin{tabular}{c|c|c}
time $t$ ($\s$) & position $x$ ($\m$) & velocity $v$ ($\m\,\s^{-1}$) \\
\hline
1 & 1.15 & 1.3 \\
2 & 2.60 & 1.6 \\
3 & 4.35 & 1.9 \\
\hline
\end{tabular}\\
What is the average acceleration in the time interval from $2\,\s$ to $3\,\s$?
\end{problem}


\vfill ~


\clearpage


\begin{problem} (From Problem Set 1)
What is the approximate thickness of a stack of 1000 20-dollar bills?
No need to be precise, and use any units you like.
\end{problem}


\vfill ~

\begin{problem} (From the reading)
Classical mechanics, or Newtonian mechanics, is only valid in certain
circumstances. When do the laws of classical mechanics, like $F =
m\,a$ for example, become wrong or break down? There are many answers
to this problem; I will take anything correct.
\end{problem}


\vfill ~

\begin{problem} (From the Math Review Lab)
What is this number? Give your answer in scientific notation.
$$
\frac{(7\times10^{-34})\times(3\times10^8)}{5\times10^{-7}}
$$
You don't need a calculator to solve this problem (\textit{hint: $3/5=0.6$}).
\end{problem}


\vfill ~

\begin{problem} (From the reading)
What musical instrument did Einstein most enjoy playing?
\end{problem}


\vfill ~


\cleardoublepage



\noindent
Name: \rule[-1ex]{0.60\textwidth}{0.1pt}
NetID: \rule[-1ex]{0.20\textwidth}{0.1pt}

\section*{\textsl{Einstein's Universe} Term Exam 1}
\setcounter{problem}{1}


\begin{problem} (From Problem Set 1)
What is the approximate thickness of a stack of 1000 20-dollar bills?
No need to be precise, and use any units you like.
\end{problem}


\vfill ~

\begin{problem} (From the reading)
What musical instrument did Einstein most enjoy playing?
\end{problem}


\vfill ~

\begin{problem} (From the Math Review Lab)
What is this number? Give your answer in scientific notation.
$$
\frac{(7\times10^{-34})\times(3\times10^8)}{5\times10^{-7}}
$$
You don't need a calculator to solve this problem (\textit{hint: $3/5=0.6$}).
\end{problem}


\vfill ~

\begin{problem} (From Problem Set 1)
What is the relationship between the energy $E$ and wavelength
$\lambda$ of a photon? Give a formula that involves energy $E$,
Planck's Constant $h$, the speed of light $c$, and wavelength
$\lambda$ (or whatever you need).
\end{problem}

\vfill ~


\clearpage


\begin{problem} (From Lecture on 2019-09-17)
If you are traveling at 60 miles per hour, how long does
it take you to go 300 miles?
\end{problem}


\vfill ~

\begin{problem} (From the Kinematics Lab)
Here is a data table of times, positions, and velocities in SI units:\\
\rule{1.0in}{0pt}\begin{tabular}{c|c|c}
time $t$ ($\s$) & position $x$ ($\m$) & velocity $v$ ($\m\,\s^{-1}$) \\
\hline
1 & 1.15 & 1.3 \\
2 & 2.60 & 1.6 \\
3 & 4.35 & 1.9 \\
\hline
\end{tabular}\\
What is the average acceleration in the time interval from $2\,\s$ to $3\,\s$?
\end{problem}


\vfill ~

\begin{problem} (From the reading)
Classical mechanics, or Newtonian mechanics, is only valid in certain
circumstances. When do the laws of classical mechanics, like $F =
m\,a$ for example, become wrong or break down? There are many answers
to this problem; I will take anything correct.
\end{problem}


\vfill ~

\begin{problem} (From Lecture on 2019-09-05)
The molar weight of water is $18\,\g$. How many molecules would there
be, therefore, in $18\,\g$ of water? You don't need a calculator for
this.
\end{problem}


\vfill ~


\cleardoublepage



\noindent
Name: \rule[-1ex]{0.60\textwidth}{0.1pt}
NetID: \rule[-1ex]{0.20\textwidth}{0.1pt}

\section*{\textsl{Einstein's Universe} Term Exam 1}
\setcounter{problem}{1}


\begin{problem} (From Lecture on 2019-09-05)
The molar weight of water is $18\,\g$. How many molecules would there
be, therefore, in $18\,\g$ of water? You don't need a calculator for
this.
\end{problem}


\vfill ~

\begin{problem} (From Problem Set 1)
What is the approximate thickness of a stack of 1000 20-dollar bills?
No need to be precise, and use any units you like.
\end{problem}


\vfill ~

\begin{problem} (From the Math Review Lab)
What is this number? Give your answer in scientific notation.
$$
\frac{(7\times10^{-34})\times(3\times10^8)}{5\times10^{-7}}
$$
You don't need a calculator to solve this problem (\textit{hint: $3/5=0.6$}).
\end{problem}


\vfill ~

\begin{problem} (From Problem Set 1)
What is the relationship between the energy $E$ and wavelength
$\lambda$ of a photon? Give a formula that involves energy $E$,
Planck's Constant $h$, the speed of light $c$, and wavelength
$\lambda$ (or whatever you need).
\end{problem}

\vfill ~


\clearpage


\begin{problem} (From the reading)
Classical mechanics, or Newtonian mechanics, is only valid in certain
circumstances. When do the laws of classical mechanics, like $F =
m\,a$ for example, become wrong or break down? There are many answers
to this problem; I will take anything correct.
\end{problem}


\vfill ~

\begin{problem} (From the Kinematics Lab)
Here is a data table of times, positions, and velocities in SI units:\\
\rule{1.0in}{0pt}\begin{tabular}{c|c|c}
time $t$ ($\s$) & position $x$ ($\m$) & velocity $v$ ($\m\,\s^{-1}$) \\
\hline
1 & 1.15 & 1.3 \\
2 & 2.60 & 1.6 \\
3 & 4.35 & 1.9 \\
\hline
\end{tabular}\\
What is the average acceleration in the time interval from $2\,\s$ to $3\,\s$?
\end{problem}


\vfill ~

\begin{problem} (From Lecture on 2019-09-17)
If you are traveling at 60 miles per hour, how long does
it take you to go 300 miles?
\end{problem}


\vfill ~

\begin{problem} (From the reading)
What musical instrument did Einstein most enjoy playing?
\end{problem}


\vfill ~


\cleardoublepage



\noindent
Name: \rule[-1ex]{0.60\textwidth}{0.1pt}
NetID: \rule[-1ex]{0.20\textwidth}{0.1pt}

\section*{\textsl{Einstein's Universe} Term Exam 1}
\setcounter{problem}{1}


\begin{problem} (From the Kinematics Lab)
Here is a data table of times, positions, and velocities in SI units:\\
\rule{1.0in}{0pt}\begin{tabular}{c|c|c}
time $t$ ($\s$) & position $x$ ($\m$) & velocity $v$ ($\m\,\s^{-1}$) \\
\hline
1 & 1.15 & 1.3 \\
2 & 2.60 & 1.6 \\
3 & 4.35 & 1.9 \\
\hline
\end{tabular}\\
What is the average acceleration in the time interval from $2\,\s$ to $3\,\s$?
\end{problem}


\vfill ~

\begin{problem} (From Lecture on 2019-09-05)
The molar weight of water is $18\,\g$. How many molecules would there
be, therefore, in $18\,\g$ of water? You don't need a calculator for
this.
\end{problem}


\vfill ~

\begin{problem} (From Problem Set 1)
What is the relationship between the energy $E$ and wavelength
$\lambda$ of a photon? Give a formula that involves energy $E$,
Planck's Constant $h$, the speed of light $c$, and wavelength
$\lambda$ (or whatever you need).
\end{problem}

\vfill ~

\begin{problem} (From Problem Set 1)
What is the approximate thickness of a stack of 1000 20-dollar bills?
No need to be precise, and use any units you like.
\end{problem}


\vfill ~


\clearpage


\begin{problem} (From the Math Review Lab)
What is this number? Give your answer in scientific notation.
$$
\frac{(7\times10^{-34})\times(3\times10^8)}{5\times10^{-7}}
$$
You don't need a calculator to solve this problem (\textit{hint: $3/5=0.6$}).
\end{problem}


\vfill ~

\begin{problem} (From the reading)
Classical mechanics, or Newtonian mechanics, is only valid in certain
circumstances. When do the laws of classical mechanics, like $F =
m\,a$ for example, become wrong or break down? There are many answers
to this problem; I will take anything correct.
\end{problem}


\vfill ~

\begin{problem} (From Lecture on 2019-09-17)
If you are traveling at 60 miles per hour, how long does
it take you to go 300 miles?
\end{problem}


\vfill ~

\begin{problem} (From the reading)
What musical instrument did Einstein most enjoy playing?
\end{problem}


\vfill ~


\cleardoublepage



\noindent
Name: \rule[-1ex]{0.60\textwidth}{0.1pt}
NetID: \rule[-1ex]{0.20\textwidth}{0.1pt}

\section*{\textsl{Einstein's Universe} Term Exam 1}
\setcounter{problem}{1}


\begin{problem} (From the reading)
What musical instrument did Einstein most enjoy playing?
\end{problem}


\vfill ~

\begin{problem} (From Lecture on 2019-09-17)
If you are traveling at 60 miles per hour, how long does
it take you to go 300 miles?
\end{problem}


\vfill ~

\begin{problem} (From Problem Set 1)
What is the relationship between the energy $E$ and wavelength
$\lambda$ of a photon? Give a formula that involves energy $E$,
Planck's Constant $h$, the speed of light $c$, and wavelength
$\lambda$ (or whatever you need).
\end{problem}

\vfill ~

\begin{problem} (From the Kinematics Lab)
Here is a data table of times, positions, and velocities in SI units:\\
\rule{1.0in}{0pt}\begin{tabular}{c|c|c}
time $t$ ($\s$) & position $x$ ($\m$) & velocity $v$ ($\m\,\s^{-1}$) \\
\hline
1 & 1.15 & 1.3 \\
2 & 2.60 & 1.6 \\
3 & 4.35 & 1.9 \\
\hline
\end{tabular}\\
What is the average acceleration in the time interval from $2\,\s$ to $3\,\s$?
\end{problem}


\vfill ~


\clearpage


\begin{problem} (From the Math Review Lab)
What is this number? Give your answer in scientific notation.
$$
\frac{(7\times10^{-34})\times(3\times10^8)}{5\times10^{-7}}
$$
You don't need a calculator to solve this problem (\textit{hint: $3/5=0.6$}).
\end{problem}


\vfill ~

\begin{problem} (From the reading)
Classical mechanics, or Newtonian mechanics, is only valid in certain
circumstances. When do the laws of classical mechanics, like $F =
m\,a$ for example, become wrong or break down? There are many answers
to this problem; I will take anything correct.
\end{problem}


\vfill ~

\begin{problem} (From Lecture on 2019-09-05)
The molar weight of water is $18\,\g$. How many molecules would there
be, therefore, in $18\,\g$ of water? You don't need a calculator for
this.
\end{problem}


\vfill ~

\begin{problem} (From Problem Set 1)
What is the approximate thickness of a stack of 1000 20-dollar bills?
No need to be precise, and use any units you like.
\end{problem}


\vfill ~


\cleardoublepage



\noindent
Name: \rule[-1ex]{0.60\textwidth}{0.1pt}
NetID: \rule[-1ex]{0.20\textwidth}{0.1pt}

\section*{\textsl{Einstein's Universe} Term Exam 1}
\setcounter{problem}{1}


\begin{problem} (From Lecture on 2019-09-17)
If you are traveling at 60 miles per hour, how long does
it take you to go 300 miles?
\end{problem}


\vfill ~

\begin{problem} (From Problem Set 1)
What is the approximate thickness of a stack of 1000 20-dollar bills?
No need to be precise, and use any units you like.
\end{problem}


\vfill ~

\begin{problem} (From Lecture on 2019-09-05)
The molar weight of water is $18\,\g$. How many molecules would there
be, therefore, in $18\,\g$ of water? You don't need a calculator for
this.
\end{problem}


\vfill ~

\begin{problem} (From Problem Set 1)
What is the relationship between the energy $E$ and wavelength
$\lambda$ of a photon? Give a formula that involves energy $E$,
Planck's Constant $h$, the speed of light $c$, and wavelength
$\lambda$ (or whatever you need).
\end{problem}

\vfill ~


\clearpage


\begin{problem} (From the Math Review Lab)
What is this number? Give your answer in scientific notation.
$$
\frac{(7\times10^{-34})\times(3\times10^8)}{5\times10^{-7}}
$$
You don't need a calculator to solve this problem (\textit{hint: $3/5=0.6$}).
\end{problem}


\vfill ~

\begin{problem} (From the reading)
Classical mechanics, or Newtonian mechanics, is only valid in certain
circumstances. When do the laws of classical mechanics, like $F =
m\,a$ for example, become wrong or break down? There are many answers
to this problem; I will take anything correct.
\end{problem}


\vfill ~

\begin{problem} (From the Kinematics Lab)
Here is a data table of times, positions, and velocities in SI units:\\
\rule{1.0in}{0pt}\begin{tabular}{c|c|c}
time $t$ ($\s$) & position $x$ ($\m$) & velocity $v$ ($\m\,\s^{-1}$) \\
\hline
1 & 1.15 & 1.3 \\
2 & 2.60 & 1.6 \\
3 & 4.35 & 1.9 \\
\hline
\end{tabular}\\
What is the average acceleration in the time interval from $2\,\s$ to $3\,\s$?
\end{problem}


\vfill ~

\begin{problem} (From the reading)
What musical instrument did Einstein most enjoy playing?
\end{problem}


\vfill ~


\cleardoublepage



\noindent
Name: \rule[-1ex]{0.60\textwidth}{0.1pt}
NetID: \rule[-1ex]{0.20\textwidth}{0.1pt}

\section*{\textsl{Einstein's Universe} Term Exam 1}
\setcounter{problem}{1}


\begin{problem} (From Problem Set 1)
What is the approximate thickness of a stack of 1000 20-dollar bills?
No need to be precise, and use any units you like.
\end{problem}


\vfill ~

\begin{problem} (From the reading)
What musical instrument did Einstein most enjoy playing?
\end{problem}


\vfill ~

\begin{problem} (From the Math Review Lab)
What is this number? Give your answer in scientific notation.
$$
\frac{(7\times10^{-34})\times(3\times10^8)}{5\times10^{-7}}
$$
You don't need a calculator to solve this problem (\textit{hint: $3/5=0.6$}).
\end{problem}


\vfill ~

\begin{problem} (From Lecture on 2019-09-17)
If you are traveling at 60 miles per hour, how long does
it take you to go 300 miles?
\end{problem}


\vfill ~


\clearpage


\begin{problem} (From Lecture on 2019-09-05)
The molar weight of water is $18\,\g$. How many molecules would there
be, therefore, in $18\,\g$ of water? You don't need a calculator for
this.
\end{problem}


\vfill ~

\begin{problem} (From Problem Set 1)
What is the relationship between the energy $E$ and wavelength
$\lambda$ of a photon? Give a formula that involves energy $E$,
Planck's Constant $h$, the speed of light $c$, and wavelength
$\lambda$ (or whatever you need).
\end{problem}

\vfill ~

\begin{problem} (From the reading)
Classical mechanics, or Newtonian mechanics, is only valid in certain
circumstances. When do the laws of classical mechanics, like $F =
m\,a$ for example, become wrong or break down? There are many answers
to this problem; I will take anything correct.
\end{problem}


\vfill ~

\begin{problem} (From the Kinematics Lab)
Here is a data table of times, positions, and velocities in SI units:\\
\rule{1.0in}{0pt}\begin{tabular}{c|c|c}
time $t$ ($\s$) & position $x$ ($\m$) & velocity $v$ ($\m\,\s^{-1}$) \\
\hline
1 & 1.15 & 1.3 \\
2 & 2.60 & 1.6 \\
3 & 4.35 & 1.9 \\
\hline
\end{tabular}\\
What is the average acceleration in the time interval from $2\,\s$ to $3\,\s$?
\end{problem}


\vfill ~


\cleardoublepage



\noindent
Name: \rule[-1ex]{0.60\textwidth}{0.1pt}
NetID: \rule[-1ex]{0.20\textwidth}{0.1pt}

\section*{\textsl{Einstein's Universe} Term Exam 1}
\setcounter{problem}{1}


\begin{problem} (From Lecture on 2019-09-17)
If you are traveling at 60 miles per hour, how long does
it take you to go 300 miles?
\end{problem}


\vfill ~

\begin{problem} (From the Kinematics Lab)
Here is a data table of times, positions, and velocities in SI units:\\
\rule{1.0in}{0pt}\begin{tabular}{c|c|c}
time $t$ ($\s$) & position $x$ ($\m$) & velocity $v$ ($\m\,\s^{-1}$) \\
\hline
1 & 1.15 & 1.3 \\
2 & 2.60 & 1.6 \\
3 & 4.35 & 1.9 \\
\hline
\end{tabular}\\
What is the average acceleration in the time interval from $2\,\s$ to $3\,\s$?
\end{problem}


\vfill ~

\begin{problem} (From Lecture on 2019-09-05)
The molar weight of water is $18\,\g$. How many molecules would there
be, therefore, in $18\,\g$ of water? You don't need a calculator for
this.
\end{problem}


\vfill ~

\begin{problem} (From the Math Review Lab)
What is this number? Give your answer in scientific notation.
$$
\frac{(7\times10^{-34})\times(3\times10^8)}{5\times10^{-7}}
$$
You don't need a calculator to solve this problem (\textit{hint: $3/5=0.6$}).
\end{problem}


\vfill ~


\clearpage


\begin{problem} (From Problem Set 1)
What is the approximate thickness of a stack of 1000 20-dollar bills?
No need to be precise, and use any units you like.
\end{problem}


\vfill ~

\begin{problem} (From the reading)
Classical mechanics, or Newtonian mechanics, is only valid in certain
circumstances. When do the laws of classical mechanics, like $F =
m\,a$ for example, become wrong or break down? There are many answers
to this problem; I will take anything correct.
\end{problem}


\vfill ~

\begin{problem} (From the reading)
What musical instrument did Einstein most enjoy playing?
\end{problem}


\vfill ~

\begin{problem} (From Problem Set 1)
What is the relationship between the energy $E$ and wavelength
$\lambda$ of a photon? Give a formula that involves energy $E$,
Planck's Constant $h$, the speed of light $c$, and wavelength
$\lambda$ (or whatever you need).
\end{problem}

\vfill ~


\cleardoublepage



\noindent
Name: \rule[-1ex]{0.60\textwidth}{0.1pt}
NetID: \rule[-1ex]{0.20\textwidth}{0.1pt}

\section*{\textsl{Einstein's Universe} Term Exam 1}
\setcounter{problem}{1}


\begin{problem} (From Problem Set 1)
What is the approximate thickness of a stack of 1000 20-dollar bills?
No need to be precise, and use any units you like.
\end{problem}


\vfill ~

\begin{problem} (From the reading)
What musical instrument did Einstein most enjoy playing?
\end{problem}


\vfill ~

\begin{problem} (From the Math Review Lab)
What is this number? Give your answer in scientific notation.
$$
\frac{(7\times10^{-34})\times(3\times10^8)}{5\times10^{-7}}
$$
You don't need a calculator to solve this problem (\textit{hint: $3/5=0.6$}).
\end{problem}


\vfill ~

\begin{problem} (From Lecture on 2019-09-17)
If you are traveling at 60 miles per hour, how long does
it take you to go 300 miles?
\end{problem}


\vfill ~


\clearpage


\begin{problem} (From Lecture on 2019-09-05)
The molar weight of water is $18\,\g$. How many molecules would there
be, therefore, in $18\,\g$ of water? You don't need a calculator for
this.
\end{problem}


\vfill ~

\begin{problem} (From Problem Set 1)
What is the relationship between the energy $E$ and wavelength
$\lambda$ of a photon? Give a formula that involves energy $E$,
Planck's Constant $h$, the speed of light $c$, and wavelength
$\lambda$ (or whatever you need).
\end{problem}

\vfill ~

\begin{problem} (From the Kinematics Lab)
Here is a data table of times, positions, and velocities in SI units:\\
\rule{1.0in}{0pt}\begin{tabular}{c|c|c}
time $t$ ($\s$) & position $x$ ($\m$) & velocity $v$ ($\m\,\s^{-1}$) \\
\hline
1 & 1.15 & 1.3 \\
2 & 2.60 & 1.6 \\
3 & 4.35 & 1.9 \\
\hline
\end{tabular}\\
What is the average acceleration in the time interval from $2\,\s$ to $3\,\s$?
\end{problem}


\vfill ~

\begin{problem} (From the reading)
Classical mechanics, or Newtonian mechanics, is only valid in certain
circumstances. When do the laws of classical mechanics, like $F =
m\,a$ for example, become wrong or break down? There are many answers
to this problem; I will take anything correct.
\end{problem}


\vfill ~


\cleardoublepage



\noindent
Name: \rule[-1ex]{0.60\textwidth}{0.1pt}
NetID: \rule[-1ex]{0.20\textwidth}{0.1pt}

\section*{\textsl{Einstein's Universe} Term Exam 1}
\setcounter{problem}{1}


\begin{problem} (From Problem Set 1)
What is the relationship between the energy $E$ and wavelength
$\lambda$ of a photon? Give a formula that involves energy $E$,
Planck's Constant $h$, the speed of light $c$, and wavelength
$\lambda$ (or whatever you need).
\end{problem}

\vfill ~

\begin{problem} (From Lecture on 2019-09-05)
The molar weight of water is $18\,\g$. How many molecules would there
be, therefore, in $18\,\g$ of water? You don't need a calculator for
this.
\end{problem}


\vfill ~

\begin{problem} (From the reading)
Classical mechanics, or Newtonian mechanics, is only valid in certain
circumstances. When do the laws of classical mechanics, like $F =
m\,a$ for example, become wrong or break down? There are many answers
to this problem; I will take anything correct.
\end{problem}


\vfill ~

\begin{problem} (From the reading)
What musical instrument did Einstein most enjoy playing?
\end{problem}


\vfill ~


\clearpage


\begin{problem} (From the Math Review Lab)
What is this number? Give your answer in scientific notation.
$$
\frac{(7\times10^{-34})\times(3\times10^8)}{5\times10^{-7}}
$$
You don't need a calculator to solve this problem (\textit{hint: $3/5=0.6$}).
\end{problem}


\vfill ~

\begin{problem} (From the Kinematics Lab)
Here is a data table of times, positions, and velocities in SI units:\\
\rule{1.0in}{0pt}\begin{tabular}{c|c|c}
time $t$ ($\s$) & position $x$ ($\m$) & velocity $v$ ($\m\,\s^{-1}$) \\
\hline
1 & 1.15 & 1.3 \\
2 & 2.60 & 1.6 \\
3 & 4.35 & 1.9 \\
\hline
\end{tabular}\\
What is the average acceleration in the time interval from $2\,\s$ to $3\,\s$?
\end{problem}


\vfill ~

\begin{problem} (From Lecture on 2019-09-17)
If you are traveling at 60 miles per hour, how long does
it take you to go 300 miles?
\end{problem}


\vfill ~

\begin{problem} (From Problem Set 1)
What is the approximate thickness of a stack of 1000 20-dollar bills?
No need to be precise, and use any units you like.
\end{problem}


\vfill ~


\cleardoublepage



\noindent
Name: \rule[-1ex]{0.60\textwidth}{0.1pt}
NetID: \rule[-1ex]{0.20\textwidth}{0.1pt}

\section*{\textsl{Einstein's Universe} Term Exam 1}
\setcounter{problem}{1}


\begin{problem} (From the Math Review Lab)
What is this number? Give your answer in scientific notation.
$$
\frac{(7\times10^{-34})\times(3\times10^8)}{5\times10^{-7}}
$$
You don't need a calculator to solve this problem (\textit{hint: $3/5=0.6$}).
\end{problem}


\vfill ~

\begin{problem} (From Problem Set 1)
What is the relationship between the energy $E$ and wavelength
$\lambda$ of a photon? Give a formula that involves energy $E$,
Planck's Constant $h$, the speed of light $c$, and wavelength
$\lambda$ (or whatever you need).
\end{problem}

\vfill ~

\begin{problem} (From Lecture on 2019-09-05)
The molar weight of water is $18\,\g$. How many molecules would there
be, therefore, in $18\,\g$ of water? You don't need a calculator for
this.
\end{problem}


\vfill ~

\begin{problem} (From the reading)
Classical mechanics, or Newtonian mechanics, is only valid in certain
circumstances. When do the laws of classical mechanics, like $F =
m\,a$ for example, become wrong or break down? There are many answers
to this problem; I will take anything correct.
\end{problem}


\vfill ~


\clearpage


\begin{problem} (From the Kinematics Lab)
Here is a data table of times, positions, and velocities in SI units:\\
\rule{1.0in}{0pt}\begin{tabular}{c|c|c}
time $t$ ($\s$) & position $x$ ($\m$) & velocity $v$ ($\m\,\s^{-1}$) \\
\hline
1 & 1.15 & 1.3 \\
2 & 2.60 & 1.6 \\
3 & 4.35 & 1.9 \\
\hline
\end{tabular}\\
What is the average acceleration in the time interval from $2\,\s$ to $3\,\s$?
\end{problem}


\vfill ~

\begin{problem} (From Lecture on 2019-09-17)
If you are traveling at 60 miles per hour, how long does
it take you to go 300 miles?
\end{problem}


\vfill ~

\begin{problem} (From Problem Set 1)
What is the approximate thickness of a stack of 1000 20-dollar bills?
No need to be precise, and use any units you like.
\end{problem}


\vfill ~

\begin{problem} (From the reading)
What musical instrument did Einstein most enjoy playing?
\end{problem}


\vfill ~


\cleardoublepage



\noindent
Name: \rule[-1ex]{0.60\textwidth}{0.1pt}
NetID: \rule[-1ex]{0.20\textwidth}{0.1pt}

\section*{\textsl{Einstein's Universe} Term Exam 1}
\setcounter{problem}{1}


\begin{problem} (From the reading)
Classical mechanics, or Newtonian mechanics, is only valid in certain
circumstances. When do the laws of classical mechanics, like $F =
m\,a$ for example, become wrong or break down? There are many answers
to this problem; I will take anything correct.
\end{problem}


\vfill ~

\begin{problem} (From Lecture on 2019-09-17)
If you are traveling at 60 miles per hour, how long does
it take you to go 300 miles?
\end{problem}


\vfill ~

\begin{problem} (From the Kinematics Lab)
Here is a data table of times, positions, and velocities in SI units:\\
\rule{1.0in}{0pt}\begin{tabular}{c|c|c}
time $t$ ($\s$) & position $x$ ($\m$) & velocity $v$ ($\m\,\s^{-1}$) \\
\hline
1 & 1.15 & 1.3 \\
2 & 2.60 & 1.6 \\
3 & 4.35 & 1.9 \\
\hline
\end{tabular}\\
What is the average acceleration in the time interval from $2\,\s$ to $3\,\s$?
\end{problem}


\vfill ~

\begin{problem} (From the reading)
What musical instrument did Einstein most enjoy playing?
\end{problem}


\vfill ~


\clearpage


\begin{problem} (From Lecture on 2019-09-05)
The molar weight of water is $18\,\g$. How many molecules would there
be, therefore, in $18\,\g$ of water? You don't need a calculator for
this.
\end{problem}


\vfill ~

\begin{problem} (From the Math Review Lab)
What is this number? Give your answer in scientific notation.
$$
\frac{(7\times10^{-34})\times(3\times10^8)}{5\times10^{-7}}
$$
You don't need a calculator to solve this problem (\textit{hint: $3/5=0.6$}).
\end{problem}


\vfill ~

\begin{problem} (From Problem Set 1)
What is the approximate thickness of a stack of 1000 20-dollar bills?
No need to be precise, and use any units you like.
\end{problem}


\vfill ~

\begin{problem} (From Problem Set 1)
What is the relationship between the energy $E$ and wavelength
$\lambda$ of a photon? Give a formula that involves energy $E$,
Planck's Constant $h$, the speed of light $c$, and wavelength
$\lambda$ (or whatever you need).
\end{problem}

\vfill ~


\cleardoublepage



\noindent
Name: \rule[-1ex]{0.60\textwidth}{0.1pt}
NetID: \rule[-1ex]{0.20\textwidth}{0.1pt}

\section*{\textsl{Einstein's Universe} Term Exam 1}
\setcounter{problem}{1}


\begin{problem} (From Lecture on 2019-09-17)
If you are traveling at 60 miles per hour, how long does
it take you to go 300 miles?
\end{problem}


\vfill ~

\begin{problem} (From Lecture on 2019-09-05)
The molar weight of water is $18\,\g$. How many molecules would there
be, therefore, in $18\,\g$ of water? You don't need a calculator for
this.
\end{problem}


\vfill ~

\begin{problem} (From the reading)
Classical mechanics, or Newtonian mechanics, is only valid in certain
circumstances. When do the laws of classical mechanics, like $F =
m\,a$ for example, become wrong or break down? There are many answers
to this problem; I will take anything correct.
\end{problem}


\vfill ~

\begin{problem} (From Problem Set 1)
What is the approximate thickness of a stack of 1000 20-dollar bills?
No need to be precise, and use any units you like.
\end{problem}


\vfill ~


\clearpage


\begin{problem} (From the Math Review Lab)
What is this number? Give your answer in scientific notation.
$$
\frac{(7\times10^{-34})\times(3\times10^8)}{5\times10^{-7}}
$$
You don't need a calculator to solve this problem (\textit{hint: $3/5=0.6$}).
\end{problem}


\vfill ~

\begin{problem} (From Problem Set 1)
What is the relationship between the energy $E$ and wavelength
$\lambda$ of a photon? Give a formula that involves energy $E$,
Planck's Constant $h$, the speed of light $c$, and wavelength
$\lambda$ (or whatever you need).
\end{problem}

\vfill ~

\begin{problem} (From the Kinematics Lab)
Here is a data table of times, positions, and velocities in SI units:\\
\rule{1.0in}{0pt}\begin{tabular}{c|c|c}
time $t$ ($\s$) & position $x$ ($\m$) & velocity $v$ ($\m\,\s^{-1}$) \\
\hline
1 & 1.15 & 1.3 \\
2 & 2.60 & 1.6 \\
3 & 4.35 & 1.9 \\
\hline
\end{tabular}\\
What is the average acceleration in the time interval from $2\,\s$ to $3\,\s$?
\end{problem}


\vfill ~

\begin{problem} (From the reading)
What musical instrument did Einstein most enjoy playing?
\end{problem}


\vfill ~


\cleardoublepage



\noindent
Name: \rule[-1ex]{0.60\textwidth}{0.1pt}
NetID: \rule[-1ex]{0.20\textwidth}{0.1pt}

\section*{\textsl{Einstein's Universe} Term Exam 1}
\setcounter{problem}{1}


\begin{problem} (From Lecture on 2019-09-17)
If you are traveling at 60 miles per hour, how long does
it take you to go 300 miles?
\end{problem}


\vfill ~

\begin{problem} (From the reading)
Classical mechanics, or Newtonian mechanics, is only valid in certain
circumstances. When do the laws of classical mechanics, like $F =
m\,a$ for example, become wrong or break down? There are many answers
to this problem; I will take anything correct.
\end{problem}


\vfill ~

\begin{problem} (From the Math Review Lab)
What is this number? Give your answer in scientific notation.
$$
\frac{(7\times10^{-34})\times(3\times10^8)}{5\times10^{-7}}
$$
You don't need a calculator to solve this problem (\textit{hint: $3/5=0.6$}).
\end{problem}


\vfill ~

\begin{problem} (From Problem Set 1)
What is the relationship between the energy $E$ and wavelength
$\lambda$ of a photon? Give a formula that involves energy $E$,
Planck's Constant $h$, the speed of light $c$, and wavelength
$\lambda$ (or whatever you need).
\end{problem}

\vfill ~


\clearpage


\begin{problem} (From Lecture on 2019-09-05)
The molar weight of water is $18\,\g$. How many molecules would there
be, therefore, in $18\,\g$ of water? You don't need a calculator for
this.
\end{problem}


\vfill ~

\begin{problem} (From Problem Set 1)
What is the approximate thickness of a stack of 1000 20-dollar bills?
No need to be precise, and use any units you like.
\end{problem}


\vfill ~

\begin{problem} (From the Kinematics Lab)
Here is a data table of times, positions, and velocities in SI units:\\
\rule{1.0in}{0pt}\begin{tabular}{c|c|c}
time $t$ ($\s$) & position $x$ ($\m$) & velocity $v$ ($\m\,\s^{-1}$) \\
\hline
1 & 1.15 & 1.3 \\
2 & 2.60 & 1.6 \\
3 & 4.35 & 1.9 \\
\hline
\end{tabular}\\
What is the average acceleration in the time interval from $2\,\s$ to $3\,\s$?
\end{problem}


\vfill ~

\begin{problem} (From the reading)
What musical instrument did Einstein most enjoy playing?
\end{problem}


\vfill ~


\cleardoublepage



\noindent
Name: \rule[-1ex]{0.60\textwidth}{0.1pt}
NetID: \rule[-1ex]{0.20\textwidth}{0.1pt}

\section*{\textsl{Einstein's Universe} Term Exam 1}
\setcounter{problem}{1}


\begin{problem} (From the Kinematics Lab)
Here is a data table of times, positions, and velocities in SI units:\\
\rule{1.0in}{0pt}\begin{tabular}{c|c|c}
time $t$ ($\s$) & position $x$ ($\m$) & velocity $v$ ($\m\,\s^{-1}$) \\
\hline
1 & 1.15 & 1.3 \\
2 & 2.60 & 1.6 \\
3 & 4.35 & 1.9 \\
\hline
\end{tabular}\\
What is the average acceleration in the time interval from $2\,\s$ to $3\,\s$?
\end{problem}


\vfill ~

\begin{problem} (From Lecture on 2019-09-05)
The molar weight of water is $18\,\g$. How many molecules would there
be, therefore, in $18\,\g$ of water? You don't need a calculator for
this.
\end{problem}


\vfill ~

\begin{problem} (From Problem Set 1)
What is the relationship between the energy $E$ and wavelength
$\lambda$ of a photon? Give a formula that involves energy $E$,
Planck's Constant $h$, the speed of light $c$, and wavelength
$\lambda$ (or whatever you need).
\end{problem}

\vfill ~

\begin{problem} (From Problem Set 1)
What is the approximate thickness of a stack of 1000 20-dollar bills?
No need to be precise, and use any units you like.
\end{problem}


\vfill ~


\clearpage


\begin{problem} (From Lecture on 2019-09-17)
If you are traveling at 60 miles per hour, how long does
it take you to go 300 miles?
\end{problem}


\vfill ~

\begin{problem} (From the Math Review Lab)
What is this number? Give your answer in scientific notation.
$$
\frac{(7\times10^{-34})\times(3\times10^8)}{5\times10^{-7}}
$$
You don't need a calculator to solve this problem (\textit{hint: $3/5=0.6$}).
\end{problem}


\vfill ~

\begin{problem} (From the reading)
What musical instrument did Einstein most enjoy playing?
\end{problem}


\vfill ~

\begin{problem} (From the reading)
Classical mechanics, or Newtonian mechanics, is only valid in certain
circumstances. When do the laws of classical mechanics, like $F =
m\,a$ for example, become wrong or break down? There are many answers
to this problem; I will take anything correct.
\end{problem}


\vfill ~


\cleardoublepage



\noindent
Name: \rule[-1ex]{0.60\textwidth}{0.1pt}
NetID: \rule[-1ex]{0.20\textwidth}{0.1pt}

\section*{\textsl{Einstein's Universe} Term Exam 1}
\setcounter{problem}{1}


\begin{problem} (From Lecture on 2019-09-17)
If you are traveling at 60 miles per hour, how long does
it take you to go 300 miles?
\end{problem}


\vfill ~

\begin{problem} (From the Math Review Lab)
What is this number? Give your answer in scientific notation.
$$
\frac{(7\times10^{-34})\times(3\times10^8)}{5\times10^{-7}}
$$
You don't need a calculator to solve this problem (\textit{hint: $3/5=0.6$}).
\end{problem}


\vfill ~

\begin{problem} (From Lecture on 2019-09-05)
The molar weight of water is $18\,\g$. How many molecules would there
be, therefore, in $18\,\g$ of water? You don't need a calculator for
this.
\end{problem}


\vfill ~

\begin{problem} (From the reading)
What musical instrument did Einstein most enjoy playing?
\end{problem}


\vfill ~


\clearpage


\begin{problem} (From Problem Set 1)
What is the approximate thickness of a stack of 1000 20-dollar bills?
No need to be precise, and use any units you like.
\end{problem}


\vfill ~

\begin{problem} (From the reading)
Classical mechanics, or Newtonian mechanics, is only valid in certain
circumstances. When do the laws of classical mechanics, like $F =
m\,a$ for example, become wrong or break down? There are many answers
to this problem; I will take anything correct.
\end{problem}


\vfill ~

\begin{problem} (From the Kinematics Lab)
Here is a data table of times, positions, and velocities in SI units:\\
\rule{1.0in}{0pt}\begin{tabular}{c|c|c}
time $t$ ($\s$) & position $x$ ($\m$) & velocity $v$ ($\m\,\s^{-1}$) \\
\hline
1 & 1.15 & 1.3 \\
2 & 2.60 & 1.6 \\
3 & 4.35 & 1.9 \\
\hline
\end{tabular}\\
What is the average acceleration in the time interval from $2\,\s$ to $3\,\s$?
\end{problem}


\vfill ~

\begin{problem} (From Problem Set 1)
What is the relationship between the energy $E$ and wavelength
$\lambda$ of a photon? Give a formula that involves energy $E$,
Planck's Constant $h$, the speed of light $c$, and wavelength
$\lambda$ (or whatever you need).
\end{problem}

\vfill ~


\cleardoublepage



\noindent
Name: \rule[-1ex]{0.60\textwidth}{0.1pt}
NetID: \rule[-1ex]{0.20\textwidth}{0.1pt}

\section*{\textsl{Einstein's Universe} Term Exam 1}
\setcounter{problem}{1}


\begin{problem} (From the Kinematics Lab)
Here is a data table of times, positions, and velocities in SI units:\\
\rule{1.0in}{0pt}\begin{tabular}{c|c|c}
time $t$ ($\s$) & position $x$ ($\m$) & velocity $v$ ($\m\,\s^{-1}$) \\
\hline
1 & 1.15 & 1.3 \\
2 & 2.60 & 1.6 \\
3 & 4.35 & 1.9 \\
\hline
\end{tabular}\\
What is the average acceleration in the time interval from $2\,\s$ to $3\,\s$?
\end{problem}


\vfill ~

\begin{problem} (From Problem Set 1)
What is the approximate thickness of a stack of 1000 20-dollar bills?
No need to be precise, and use any units you like.
\end{problem}


\vfill ~

\begin{problem} (From the Math Review Lab)
What is this number? Give your answer in scientific notation.
$$
\frac{(7\times10^{-34})\times(3\times10^8)}{5\times10^{-7}}
$$
You don't need a calculator to solve this problem (\textit{hint: $3/5=0.6$}).
\end{problem}


\vfill ~

\begin{problem} (From the reading)
What musical instrument did Einstein most enjoy playing?
\end{problem}


\vfill ~


\clearpage


\begin{problem} (From Problem Set 1)
What is the relationship between the energy $E$ and wavelength
$\lambda$ of a photon? Give a formula that involves energy $E$,
Planck's Constant $h$, the speed of light $c$, and wavelength
$\lambda$ (or whatever you need).
\end{problem}

\vfill ~

\begin{problem} (From the reading)
Classical mechanics, or Newtonian mechanics, is only valid in certain
circumstances. When do the laws of classical mechanics, like $F =
m\,a$ for example, become wrong or break down? There are many answers
to this problem; I will take anything correct.
\end{problem}


\vfill ~

\begin{problem} (From Lecture on 2019-09-05)
The molar weight of water is $18\,\g$. How many molecules would there
be, therefore, in $18\,\g$ of water? You don't need a calculator for
this.
\end{problem}


\vfill ~

\begin{problem} (From Lecture on 2019-09-17)
If you are traveling at 60 miles per hour, how long does
it take you to go 300 miles?
\end{problem}


\vfill ~


\cleardoublepage



\noindent
Name: \rule[-1ex]{0.60\textwidth}{0.1pt}
NetID: \rule[-1ex]{0.20\textwidth}{0.1pt}

\section*{\textsl{Einstein's Universe} Term Exam 1}
\setcounter{problem}{1}


\begin{problem} (From Problem Set 1)
What is the relationship between the energy $E$ and wavelength
$\lambda$ of a photon? Give a formula that involves energy $E$,
Planck's Constant $h$, the speed of light $c$, and wavelength
$\lambda$ (or whatever you need).
\end{problem}

\vfill ~

\begin{problem} (From Problem Set 1)
What is the approximate thickness of a stack of 1000 20-dollar bills?
No need to be precise, and use any units you like.
\end{problem}


\vfill ~

\begin{problem} (From the Kinematics Lab)
Here is a data table of times, positions, and velocities in SI units:\\
\rule{1.0in}{0pt}\begin{tabular}{c|c|c}
time $t$ ($\s$) & position $x$ ($\m$) & velocity $v$ ($\m\,\s^{-1}$) \\
\hline
1 & 1.15 & 1.3 \\
2 & 2.60 & 1.6 \\
3 & 4.35 & 1.9 \\
\hline
\end{tabular}\\
What is the average acceleration in the time interval from $2\,\s$ to $3\,\s$?
\end{problem}


\vfill ~

\begin{problem} (From the Math Review Lab)
What is this number? Give your answer in scientific notation.
$$
\frac{(7\times10^{-34})\times(3\times10^8)}{5\times10^{-7}}
$$
You don't need a calculator to solve this problem (\textit{hint: $3/5=0.6$}).
\end{problem}


\vfill ~


\clearpage


\begin{problem} (From Lecture on 2019-09-05)
The molar weight of water is $18\,\g$. How many molecules would there
be, therefore, in $18\,\g$ of water? You don't need a calculator for
this.
\end{problem}


\vfill ~

\begin{problem} (From Lecture on 2019-09-17)
If you are traveling at 60 miles per hour, how long does
it take you to go 300 miles?
\end{problem}


\vfill ~

\begin{problem} (From the reading)
Classical mechanics, or Newtonian mechanics, is only valid in certain
circumstances. When do the laws of classical mechanics, like $F =
m\,a$ for example, become wrong or break down? There are many answers
to this problem; I will take anything correct.
\end{problem}


\vfill ~

\begin{problem} (From the reading)
What musical instrument did Einstein most enjoy playing?
\end{problem}


\vfill ~


\cleardoublepage



\noindent
Name: \rule[-1ex]{0.60\textwidth}{0.1pt}
NetID: \rule[-1ex]{0.20\textwidth}{0.1pt}

\section*{\textsl{Einstein's Universe} Term Exam 1}
\setcounter{problem}{1}


\begin{problem} (From Lecture on 2019-09-05)
The molar weight of water is $18\,\g$. How many molecules would there
be, therefore, in $18\,\g$ of water? You don't need a calculator for
this.
\end{problem}


\vfill ~

\begin{problem} (From the Math Review Lab)
What is this number? Give your answer in scientific notation.
$$
\frac{(7\times10^{-34})\times(3\times10^8)}{5\times10^{-7}}
$$
You don't need a calculator to solve this problem (\textit{hint: $3/5=0.6$}).
\end{problem}


\vfill ~

\begin{problem} (From the Kinematics Lab)
Here is a data table of times, positions, and velocities in SI units:\\
\rule{1.0in}{0pt}\begin{tabular}{c|c|c}
time $t$ ($\s$) & position $x$ ($\m$) & velocity $v$ ($\m\,\s^{-1}$) \\
\hline
1 & 1.15 & 1.3 \\
2 & 2.60 & 1.6 \\
3 & 4.35 & 1.9 \\
\hline
\end{tabular}\\
What is the average acceleration in the time interval from $2\,\s$ to $3\,\s$?
\end{problem}


\vfill ~

\begin{problem} (From the reading)
What musical instrument did Einstein most enjoy playing?
\end{problem}


\vfill ~


\clearpage


\begin{problem} (From Problem Set 1)
What is the approximate thickness of a stack of 1000 20-dollar bills?
No need to be precise, and use any units you like.
\end{problem}


\vfill ~

\begin{problem} (From Lecture on 2019-09-17)
If you are traveling at 60 miles per hour, how long does
it take you to go 300 miles?
\end{problem}


\vfill ~

\begin{problem} (From Problem Set 1)
What is the relationship between the energy $E$ and wavelength
$\lambda$ of a photon? Give a formula that involves energy $E$,
Planck's Constant $h$, the speed of light $c$, and wavelength
$\lambda$ (or whatever you need).
\end{problem}

\vfill ~

\begin{problem} (From the reading)
Classical mechanics, or Newtonian mechanics, is only valid in certain
circumstances. When do the laws of classical mechanics, like $F =
m\,a$ for example, become wrong or break down? There are many answers
to this problem; I will take anything correct.
\end{problem}


\vfill ~


\cleardoublepage



\noindent
Name: \rule[-1ex]{0.60\textwidth}{0.1pt}
NetID: \rule[-1ex]{0.20\textwidth}{0.1pt}

\section*{\textsl{Einstein's Universe} Term Exam 1}
\setcounter{problem}{1}


\begin{problem} (From Problem Set 1)
What is the approximate thickness of a stack of 1000 20-dollar bills?
No need to be precise, and use any units you like.
\end{problem}


\vfill ~

\begin{problem} (From the reading)
Classical mechanics, or Newtonian mechanics, is only valid in certain
circumstances. When do the laws of classical mechanics, like $F =
m\,a$ for example, become wrong or break down? There are many answers
to this problem; I will take anything correct.
\end{problem}


\vfill ~

\begin{problem} (From the Kinematics Lab)
Here is a data table of times, positions, and velocities in SI units:\\
\rule{1.0in}{0pt}\begin{tabular}{c|c|c}
time $t$ ($\s$) & position $x$ ($\m$) & velocity $v$ ($\m\,\s^{-1}$) \\
\hline
1 & 1.15 & 1.3 \\
2 & 2.60 & 1.6 \\
3 & 4.35 & 1.9 \\
\hline
\end{tabular}\\
What is the average acceleration in the time interval from $2\,\s$ to $3\,\s$?
\end{problem}


\vfill ~

\begin{problem} (From Problem Set 1)
What is the relationship between the energy $E$ and wavelength
$\lambda$ of a photon? Give a formula that involves energy $E$,
Planck's Constant $h$, the speed of light $c$, and wavelength
$\lambda$ (or whatever you need).
\end{problem}

\vfill ~


\clearpage


\begin{problem} (From the reading)
What musical instrument did Einstein most enjoy playing?
\end{problem}


\vfill ~

\begin{problem} (From Lecture on 2019-09-05)
The molar weight of water is $18\,\g$. How many molecules would there
be, therefore, in $18\,\g$ of water? You don't need a calculator for
this.
\end{problem}


\vfill ~

\begin{problem} (From Lecture on 2019-09-17)
If you are traveling at 60 miles per hour, how long does
it take you to go 300 miles?
\end{problem}


\vfill ~

\begin{problem} (From the Math Review Lab)
What is this number? Give your answer in scientific notation.
$$
\frac{(7\times10^{-34})\times(3\times10^8)}{5\times10^{-7}}
$$
You don't need a calculator to solve this problem (\textit{hint: $3/5=0.6$}).
\end{problem}


\vfill ~


\cleardoublepage



\noindent
Name: \rule[-1ex]{0.60\textwidth}{0.1pt}
NetID: \rule[-1ex]{0.20\textwidth}{0.1pt}

\section*{\textsl{Einstein's Universe} Term Exam 1}
\setcounter{problem}{1}


\begin{problem} (From Lecture on 2019-09-17)
If you are traveling at 60 miles per hour, how long does
it take you to go 300 miles?
\end{problem}


\vfill ~

\begin{problem} (From Problem Set 1)
What is the approximate thickness of a stack of 1000 20-dollar bills?
No need to be precise, and use any units you like.
\end{problem}


\vfill ~

\begin{problem} (From the reading)
Classical mechanics, or Newtonian mechanics, is only valid in certain
circumstances. When do the laws of classical mechanics, like $F =
m\,a$ for example, become wrong or break down? There are many answers
to this problem; I will take anything correct.
\end{problem}


\vfill ~

\begin{problem} (From the Math Review Lab)
What is this number? Give your answer in scientific notation.
$$
\frac{(7\times10^{-34})\times(3\times10^8)}{5\times10^{-7}}
$$
You don't need a calculator to solve this problem (\textit{hint: $3/5=0.6$}).
\end{problem}


\vfill ~


\clearpage


\begin{problem} (From the reading)
What musical instrument did Einstein most enjoy playing?
\end{problem}


\vfill ~

\begin{problem} (From Lecture on 2019-09-05)
The molar weight of water is $18\,\g$. How many molecules would there
be, therefore, in $18\,\g$ of water? You don't need a calculator for
this.
\end{problem}


\vfill ~

\begin{problem} (From the Kinematics Lab)
Here is a data table of times, positions, and velocities in SI units:\\
\rule{1.0in}{0pt}\begin{tabular}{c|c|c}
time $t$ ($\s$) & position $x$ ($\m$) & velocity $v$ ($\m\,\s^{-1}$) \\
\hline
1 & 1.15 & 1.3 \\
2 & 2.60 & 1.6 \\
3 & 4.35 & 1.9 \\
\hline
\end{tabular}\\
What is the average acceleration in the time interval from $2\,\s$ to $3\,\s$?
\end{problem}


\vfill ~

\begin{problem} (From Problem Set 1)
What is the relationship between the energy $E$ and wavelength
$\lambda$ of a photon? Give a formula that involves energy $E$,
Planck's Constant $h$, the speed of light $c$, and wavelength
$\lambda$ (or whatever you need).
\end{problem}

\vfill ~


\cleardoublepage



\noindent
Name: \rule[-1ex]{0.60\textwidth}{0.1pt}
NetID: \rule[-1ex]{0.20\textwidth}{0.1pt}

\section*{\textsl{Einstein's Universe} Term Exam 1}
\setcounter{problem}{1}


\begin{problem} (From Lecture on 2019-09-17)
If you are traveling at 60 miles per hour, how long does
it take you to go 300 miles?
\end{problem}


\vfill ~

\begin{problem} (From the Kinematics Lab)
Here is a data table of times, positions, and velocities in SI units:\\
\rule{1.0in}{0pt}\begin{tabular}{c|c|c}
time $t$ ($\s$) & position $x$ ($\m$) & velocity $v$ ($\m\,\s^{-1}$) \\
\hline
1 & 1.15 & 1.3 \\
2 & 2.60 & 1.6 \\
3 & 4.35 & 1.9 \\
\hline
\end{tabular}\\
What is the average acceleration in the time interval from $2\,\s$ to $3\,\s$?
\end{problem}


\vfill ~

\begin{problem} (From Problem Set 1)
What is the relationship between the energy $E$ and wavelength
$\lambda$ of a photon? Give a formula that involves energy $E$,
Planck's Constant $h$, the speed of light $c$, and wavelength
$\lambda$ (or whatever you need).
\end{problem}

\vfill ~

\begin{problem} (From Problem Set 1)
What is the approximate thickness of a stack of 1000 20-dollar bills?
No need to be precise, and use any units you like.
\end{problem}


\vfill ~


\clearpage


\begin{problem} (From the Math Review Lab)
What is this number? Give your answer in scientific notation.
$$
\frac{(7\times10^{-34})\times(3\times10^8)}{5\times10^{-7}}
$$
You don't need a calculator to solve this problem (\textit{hint: $3/5=0.6$}).
\end{problem}


\vfill ~

\begin{problem} (From the reading)
Classical mechanics, or Newtonian mechanics, is only valid in certain
circumstances. When do the laws of classical mechanics, like $F =
m\,a$ for example, become wrong or break down? There are many answers
to this problem; I will take anything correct.
\end{problem}


\vfill ~

\begin{problem} (From Lecture on 2019-09-05)
The molar weight of water is $18\,\g$. How many molecules would there
be, therefore, in $18\,\g$ of water? You don't need a calculator for
this.
\end{problem}


\vfill ~

\begin{problem} (From the reading)
What musical instrument did Einstein most enjoy playing?
\end{problem}


\vfill ~


\cleardoublepage



\noindent
Name: \rule[-1ex]{0.60\textwidth}{0.1pt}
NetID: \rule[-1ex]{0.20\textwidth}{0.1pt}

\section*{\textsl{Einstein's Universe} Term Exam 1}
\setcounter{problem}{1}


\begin{problem} (From the Kinematics Lab)
Here is a data table of times, positions, and velocities in SI units:\\
\rule{1.0in}{0pt}\begin{tabular}{c|c|c}
time $t$ ($\s$) & position $x$ ($\m$) & velocity $v$ ($\m\,\s^{-1}$) \\
\hline
1 & 1.15 & 1.3 \\
2 & 2.60 & 1.6 \\
3 & 4.35 & 1.9 \\
\hline
\end{tabular}\\
What is the average acceleration in the time interval from $2\,\s$ to $3\,\s$?
\end{problem}


\vfill ~

\begin{problem} (From the reading)
What musical instrument did Einstein most enjoy playing?
\end{problem}


\vfill ~

\begin{problem} (From Problem Set 1)
What is the relationship between the energy $E$ and wavelength
$\lambda$ of a photon? Give a formula that involves energy $E$,
Planck's Constant $h$, the speed of light $c$, and wavelength
$\lambda$ (or whatever you need).
\end{problem}

\vfill ~

\begin{problem} (From the Math Review Lab)
What is this number? Give your answer in scientific notation.
$$
\frac{(7\times10^{-34})\times(3\times10^8)}{5\times10^{-7}}
$$
You don't need a calculator to solve this problem (\textit{hint: $3/5=0.6$}).
\end{problem}


\vfill ~


\clearpage


\begin{problem} (From Lecture on 2019-09-05)
The molar weight of water is $18\,\g$. How many molecules would there
be, therefore, in $18\,\g$ of water? You don't need a calculator for
this.
\end{problem}


\vfill ~

\begin{problem} (From Lecture on 2019-09-17)
If you are traveling at 60 miles per hour, how long does
it take you to go 300 miles?
\end{problem}


\vfill ~

\begin{problem} (From Problem Set 1)
What is the approximate thickness of a stack of 1000 20-dollar bills?
No need to be precise, and use any units you like.
\end{problem}


\vfill ~

\begin{problem} (From the reading)
Classical mechanics, or Newtonian mechanics, is only valid in certain
circumstances. When do the laws of classical mechanics, like $F =
m\,a$ for example, become wrong or break down? There are many answers
to this problem; I will take anything correct.
\end{problem}


\vfill ~


\cleardoublepage



\noindent
Name: \rule[-1ex]{0.60\textwidth}{0.1pt}
NetID: \rule[-1ex]{0.20\textwidth}{0.1pt}

\section*{\textsl{Einstein's Universe} Term Exam 1}
\setcounter{problem}{1}


\begin{problem} (From the reading)
Classical mechanics, or Newtonian mechanics, is only valid in certain
circumstances. When do the laws of classical mechanics, like $F =
m\,a$ for example, become wrong or break down? There are many answers
to this problem; I will take anything correct.
\end{problem}


\vfill ~

\begin{problem} (From Lecture on 2019-09-05)
The molar weight of water is $18\,\g$. How many molecules would there
be, therefore, in $18\,\g$ of water? You don't need a calculator for
this.
\end{problem}


\vfill ~

\begin{problem} (From Problem Set 1)
What is the approximate thickness of a stack of 1000 20-dollar bills?
No need to be precise, and use any units you like.
\end{problem}


\vfill ~

\begin{problem} (From the Kinematics Lab)
Here is a data table of times, positions, and velocities in SI units:\\
\rule{1.0in}{0pt}\begin{tabular}{c|c|c}
time $t$ ($\s$) & position $x$ ($\m$) & velocity $v$ ($\m\,\s^{-1}$) \\
\hline
1 & 1.15 & 1.3 \\
2 & 2.60 & 1.6 \\
3 & 4.35 & 1.9 \\
\hline
\end{tabular}\\
What is the average acceleration in the time interval from $2\,\s$ to $3\,\s$?
\end{problem}


\vfill ~


\clearpage


\begin{problem} (From the reading)
What musical instrument did Einstein most enjoy playing?
\end{problem}


\vfill ~

\begin{problem} (From Lecture on 2019-09-17)
If you are traveling at 60 miles per hour, how long does
it take you to go 300 miles?
\end{problem}


\vfill ~

\begin{problem} (From Problem Set 1)
What is the relationship between the energy $E$ and wavelength
$\lambda$ of a photon? Give a formula that involves energy $E$,
Planck's Constant $h$, the speed of light $c$, and wavelength
$\lambda$ (or whatever you need).
\end{problem}

\vfill ~

\begin{problem} (From the Math Review Lab)
What is this number? Give your answer in scientific notation.
$$
\frac{(7\times10^{-34})\times(3\times10^8)}{5\times10^{-7}}
$$
You don't need a calculator to solve this problem (\textit{hint: $3/5=0.6$}).
\end{problem}


\vfill ~


\cleardoublepage



\noindent
Name: \rule[-1ex]{0.60\textwidth}{0.1pt}
NetID: \rule[-1ex]{0.20\textwidth}{0.1pt}

\section*{\textsl{Einstein's Universe} Term Exam 1}
\setcounter{problem}{1}


\begin{problem} (From the Math Review Lab)
What is this number? Give your answer in scientific notation.
$$
\frac{(7\times10^{-34})\times(3\times10^8)}{5\times10^{-7}}
$$
You don't need a calculator to solve this problem (\textit{hint: $3/5=0.6$}).
\end{problem}


\vfill ~

\begin{problem} (From Lecture on 2019-09-05)
The molar weight of water is $18\,\g$. How many molecules would there
be, therefore, in $18\,\g$ of water? You don't need a calculator for
this.
\end{problem}


\vfill ~

\begin{problem} (From the reading)
What musical instrument did Einstein most enjoy playing?
\end{problem}


\vfill ~

\begin{problem} (From the Kinematics Lab)
Here is a data table of times, positions, and velocities in SI units:\\
\rule{1.0in}{0pt}\begin{tabular}{c|c|c}
time $t$ ($\s$) & position $x$ ($\m$) & velocity $v$ ($\m\,\s^{-1}$) \\
\hline
1 & 1.15 & 1.3 \\
2 & 2.60 & 1.6 \\
3 & 4.35 & 1.9 \\
\hline
\end{tabular}\\
What is the average acceleration in the time interval from $2\,\s$ to $3\,\s$?
\end{problem}


\vfill ~


\clearpage


\begin{problem} (From Problem Set 1)
What is the approximate thickness of a stack of 1000 20-dollar bills?
No need to be precise, and use any units you like.
\end{problem}


\vfill ~

\begin{problem} (From Lecture on 2019-09-17)
If you are traveling at 60 miles per hour, how long does
it take you to go 300 miles?
\end{problem}


\vfill ~

\begin{problem} (From Problem Set 1)
What is the relationship between the energy $E$ and wavelength
$\lambda$ of a photon? Give a formula that involves energy $E$,
Planck's Constant $h$, the speed of light $c$, and wavelength
$\lambda$ (or whatever you need).
\end{problem}

\vfill ~

\begin{problem} (From the reading)
Classical mechanics, or Newtonian mechanics, is only valid in certain
circumstances. When do the laws of classical mechanics, like $F =
m\,a$ for example, become wrong or break down? There are many answers
to this problem; I will take anything correct.
\end{problem}


\vfill ~


\cleardoublepage



\noindent
Name: \rule[-1ex]{0.60\textwidth}{0.1pt}
NetID: \rule[-1ex]{0.20\textwidth}{0.1pt}

\section*{\textsl{Einstein's Universe} Term Exam 1}
\setcounter{problem}{1}


\begin{problem} (From the reading)
What musical instrument did Einstein most enjoy playing?
\end{problem}


\vfill ~

\begin{problem} (From Lecture on 2019-09-05)
The molar weight of water is $18\,\g$. How many molecules would there
be, therefore, in $18\,\g$ of water? You don't need a calculator for
this.
\end{problem}


\vfill ~

\begin{problem} (From Lecture on 2019-09-17)
If you are traveling at 60 miles per hour, how long does
it take you to go 300 miles?
\end{problem}


\vfill ~

\begin{problem} (From the Kinematics Lab)
Here is a data table of times, positions, and velocities in SI units:\\
\rule{1.0in}{0pt}\begin{tabular}{c|c|c}
time $t$ ($\s$) & position $x$ ($\m$) & velocity $v$ ($\m\,\s^{-1}$) \\
\hline
1 & 1.15 & 1.3 \\
2 & 2.60 & 1.6 \\
3 & 4.35 & 1.9 \\
\hline
\end{tabular}\\
What is the average acceleration in the time interval from $2\,\s$ to $3\,\s$?
\end{problem}


\vfill ~


\clearpage


\begin{problem} (From the Math Review Lab)
What is this number? Give your answer in scientific notation.
$$
\frac{(7\times10^{-34})\times(3\times10^8)}{5\times10^{-7}}
$$
You don't need a calculator to solve this problem (\textit{hint: $3/5=0.6$}).
\end{problem}


\vfill ~

\begin{problem} (From the reading)
Classical mechanics, or Newtonian mechanics, is only valid in certain
circumstances. When do the laws of classical mechanics, like $F =
m\,a$ for example, become wrong or break down? There are many answers
to this problem; I will take anything correct.
\end{problem}


\vfill ~

\begin{problem} (From Problem Set 1)
What is the relationship between the energy $E$ and wavelength
$\lambda$ of a photon? Give a formula that involves energy $E$,
Planck's Constant $h$, the speed of light $c$, and wavelength
$\lambda$ (or whatever you need).
\end{problem}

\vfill ~

\begin{problem} (From Problem Set 1)
What is the approximate thickness of a stack of 1000 20-dollar bills?
No need to be precise, and use any units you like.
\end{problem}


\vfill ~


\cleardoublepage



\noindent
Name: \rule[-1ex]{0.60\textwidth}{0.1pt}
NetID: \rule[-1ex]{0.20\textwidth}{0.1pt}

\section*{\textsl{Einstein's Universe} Term Exam 1}
\setcounter{problem}{1}


\begin{problem} (From the Kinematics Lab)
Here is a data table of times, positions, and velocities in SI units:\\
\rule{1.0in}{0pt}\begin{tabular}{c|c|c}
time $t$ ($\s$) & position $x$ ($\m$) & velocity $v$ ($\m\,\s^{-1}$) \\
\hline
1 & 1.15 & 1.3 \\
2 & 2.60 & 1.6 \\
3 & 4.35 & 1.9 \\
\hline
\end{tabular}\\
What is the average acceleration in the time interval from $2\,\s$ to $3\,\s$?
\end{problem}


\vfill ~

\begin{problem} (From the reading)
Classical mechanics, or Newtonian mechanics, is only valid in certain
circumstances. When do the laws of classical mechanics, like $F =
m\,a$ for example, become wrong or break down? There are many answers
to this problem; I will take anything correct.
\end{problem}


\vfill ~

\begin{problem} (From Problem Set 1)
What is the approximate thickness of a stack of 1000 20-dollar bills?
No need to be precise, and use any units you like.
\end{problem}


\vfill ~

\begin{problem} (From the reading)
What musical instrument did Einstein most enjoy playing?
\end{problem}


\vfill ~


\clearpage


\begin{problem} (From Lecture on 2019-09-05)
The molar weight of water is $18\,\g$. How many molecules would there
be, therefore, in $18\,\g$ of water? You don't need a calculator for
this.
\end{problem}


\vfill ~

\begin{problem} (From Problem Set 1)
What is the relationship between the energy $E$ and wavelength
$\lambda$ of a photon? Give a formula that involves energy $E$,
Planck's Constant $h$, the speed of light $c$, and wavelength
$\lambda$ (or whatever you need).
\end{problem}

\vfill ~

\begin{problem} (From the Math Review Lab)
What is this number? Give your answer in scientific notation.
$$
\frac{(7\times10^{-34})\times(3\times10^8)}{5\times10^{-7}}
$$
You don't need a calculator to solve this problem (\textit{hint: $3/5=0.6$}).
\end{problem}


\vfill ~

\begin{problem} (From Lecture on 2019-09-17)
If you are traveling at 60 miles per hour, how long does
it take you to go 300 miles?
\end{problem}


\vfill ~


\cleardoublepage



\noindent
Name: \rule[-1ex]{0.60\textwidth}{0.1pt}
NetID: \rule[-1ex]{0.20\textwidth}{0.1pt}

\section*{\textsl{Einstein's Universe} Term Exam 1}
\setcounter{problem}{1}


\begin{problem} (From Problem Set 1)
What is the approximate thickness of a stack of 1000 20-dollar bills?
No need to be precise, and use any units you like.
\end{problem}


\vfill ~

\begin{problem} (From Lecture on 2019-09-05)
The molar weight of water is $18\,\g$. How many molecules would there
be, therefore, in $18\,\g$ of water? You don't need a calculator for
this.
\end{problem}


\vfill ~

\begin{problem} (From the reading)
Classical mechanics, or Newtonian mechanics, is only valid in certain
circumstances. When do the laws of classical mechanics, like $F =
m\,a$ for example, become wrong or break down? There are many answers
to this problem; I will take anything correct.
\end{problem}


\vfill ~

\begin{problem} (From the reading)
What musical instrument did Einstein most enjoy playing?
\end{problem}


\vfill ~


\clearpage


\begin{problem} (From the Math Review Lab)
What is this number? Give your answer in scientific notation.
$$
\frac{(7\times10^{-34})\times(3\times10^8)}{5\times10^{-7}}
$$
You don't need a calculator to solve this problem (\textit{hint: $3/5=0.6$}).
\end{problem}


\vfill ~

\begin{problem} (From the Kinematics Lab)
Here is a data table of times, positions, and velocities in SI units:\\
\rule{1.0in}{0pt}\begin{tabular}{c|c|c}
time $t$ ($\s$) & position $x$ ($\m$) & velocity $v$ ($\m\,\s^{-1}$) \\
\hline
1 & 1.15 & 1.3 \\
2 & 2.60 & 1.6 \\
3 & 4.35 & 1.9 \\
\hline
\end{tabular}\\
What is the average acceleration in the time interval from $2\,\s$ to $3\,\s$?
\end{problem}


\vfill ~

\begin{problem} (From Problem Set 1)
What is the relationship between the energy $E$ and wavelength
$\lambda$ of a photon? Give a formula that involves energy $E$,
Planck's Constant $h$, the speed of light $c$, and wavelength
$\lambda$ (or whatever you need).
\end{problem}

\vfill ~

\begin{problem} (From Lecture on 2019-09-17)
If you are traveling at 60 miles per hour, how long does
it take you to go 300 miles?
\end{problem}


\vfill ~


\cleardoublepage



\noindent
Name: \rule[-1ex]{0.60\textwidth}{0.1pt}
NetID: \rule[-1ex]{0.20\textwidth}{0.1pt}

\section*{\textsl{Einstein's Universe} Term Exam 1}
\setcounter{problem}{1}


\begin{problem} (From the reading)
What musical instrument did Einstein most enjoy playing?
\end{problem}


\vfill ~

\begin{problem} (From Lecture on 2019-09-05)
The molar weight of water is $18\,\g$. How many molecules would there
be, therefore, in $18\,\g$ of water? You don't need a calculator for
this.
\end{problem}


\vfill ~

\begin{problem} (From Problem Set 1)
What is the relationship between the energy $E$ and wavelength
$\lambda$ of a photon? Give a formula that involves energy $E$,
Planck's Constant $h$, the speed of light $c$, and wavelength
$\lambda$ (or whatever you need).
\end{problem}

\vfill ~

\begin{problem} (From the Math Review Lab)
What is this number? Give your answer in scientific notation.
$$
\frac{(7\times10^{-34})\times(3\times10^8)}{5\times10^{-7}}
$$
You don't need a calculator to solve this problem (\textit{hint: $3/5=0.6$}).
\end{problem}


\vfill ~


\clearpage


\begin{problem} (From the Kinematics Lab)
Here is a data table of times, positions, and velocities in SI units:\\
\rule{1.0in}{0pt}\begin{tabular}{c|c|c}
time $t$ ($\s$) & position $x$ ($\m$) & velocity $v$ ($\m\,\s^{-1}$) \\
\hline
1 & 1.15 & 1.3 \\
2 & 2.60 & 1.6 \\
3 & 4.35 & 1.9 \\
\hline
\end{tabular}\\
What is the average acceleration in the time interval from $2\,\s$ to $3\,\s$?
\end{problem}


\vfill ~

\begin{problem} (From the reading)
Classical mechanics, or Newtonian mechanics, is only valid in certain
circumstances. When do the laws of classical mechanics, like $F =
m\,a$ for example, become wrong or break down? There are many answers
to this problem; I will take anything correct.
\end{problem}


\vfill ~

\begin{problem} (From Problem Set 1)
What is the approximate thickness of a stack of 1000 20-dollar bills?
No need to be precise, and use any units you like.
\end{problem}


\vfill ~

\begin{problem} (From Lecture on 2019-09-17)
If you are traveling at 60 miles per hour, how long does
it take you to go 300 miles?
\end{problem}


\vfill ~


\cleardoublepage



\noindent
Name: \rule[-1ex]{0.60\textwidth}{0.1pt}
NetID: \rule[-1ex]{0.20\textwidth}{0.1pt}

\section*{\textsl{Einstein's Universe} Term Exam 1}
\setcounter{problem}{1}


\begin{problem} (From Problem Set 1)
What is the approximate thickness of a stack of 1000 20-dollar bills?
No need to be precise, and use any units you like.
\end{problem}


\vfill ~

\begin{problem} (From the reading)
What musical instrument did Einstein most enjoy playing?
\end{problem}


\vfill ~

\begin{problem} (From the Kinematics Lab)
Here is a data table of times, positions, and velocities in SI units:\\
\rule{1.0in}{0pt}\begin{tabular}{c|c|c}
time $t$ ($\s$) & position $x$ ($\m$) & velocity $v$ ($\m\,\s^{-1}$) \\
\hline
1 & 1.15 & 1.3 \\
2 & 2.60 & 1.6 \\
3 & 4.35 & 1.9 \\
\hline
\end{tabular}\\
What is the average acceleration in the time interval from $2\,\s$ to $3\,\s$?
\end{problem}


\vfill ~

\begin{problem} (From Problem Set 1)
What is the relationship between the energy $E$ and wavelength
$\lambda$ of a photon? Give a formula that involves energy $E$,
Planck's Constant $h$, the speed of light $c$, and wavelength
$\lambda$ (or whatever you need).
\end{problem}

\vfill ~


\clearpage


\begin{problem} (From the reading)
Classical mechanics, or Newtonian mechanics, is only valid in certain
circumstances. When do the laws of classical mechanics, like $F =
m\,a$ for example, become wrong or break down? There are many answers
to this problem; I will take anything correct.
\end{problem}


\vfill ~

\begin{problem} (From the Math Review Lab)
What is this number? Give your answer in scientific notation.
$$
\frac{(7\times10^{-34})\times(3\times10^8)}{5\times10^{-7}}
$$
You don't need a calculator to solve this problem (\textit{hint: $3/5=0.6$}).
\end{problem}


\vfill ~

\begin{problem} (From Lecture on 2019-09-17)
If you are traveling at 60 miles per hour, how long does
it take you to go 300 miles?
\end{problem}


\vfill ~

\begin{problem} (From Lecture on 2019-09-05)
The molar weight of water is $18\,\g$. How many molecules would there
be, therefore, in $18\,\g$ of water? You don't need a calculator for
this.
\end{problem}


\vfill ~


\cleardoublepage



\noindent
Name: \rule[-1ex]{0.60\textwidth}{0.1pt}
NetID: \rule[-1ex]{0.20\textwidth}{0.1pt}

\section*{\textsl{Einstein's Universe} Term Exam 1}
\setcounter{problem}{1}


\begin{problem} (From the reading)
Classical mechanics, or Newtonian mechanics, is only valid in certain
circumstances. When do the laws of classical mechanics, like $F =
m\,a$ for example, become wrong or break down? There are many answers
to this problem; I will take anything correct.
\end{problem}


\vfill ~

\begin{problem} (From the reading)
What musical instrument did Einstein most enjoy playing?
\end{problem}


\vfill ~

\begin{problem} (From Problem Set 1)
What is the relationship between the energy $E$ and wavelength
$\lambda$ of a photon? Give a formula that involves energy $E$,
Planck's Constant $h$, the speed of light $c$, and wavelength
$\lambda$ (or whatever you need).
\end{problem}

\vfill ~

\begin{problem} (From Lecture on 2019-09-17)
If you are traveling at 60 miles per hour, how long does
it take you to go 300 miles?
\end{problem}


\vfill ~


\clearpage


\begin{problem} (From Problem Set 1)
What is the approximate thickness of a stack of 1000 20-dollar bills?
No need to be precise, and use any units you like.
\end{problem}


\vfill ~

\begin{problem} (From the Math Review Lab)
What is this number? Give your answer in scientific notation.
$$
\frac{(7\times10^{-34})\times(3\times10^8)}{5\times10^{-7}}
$$
You don't need a calculator to solve this problem (\textit{hint: $3/5=0.6$}).
\end{problem}


\vfill ~

\begin{problem} (From the Kinematics Lab)
Here is a data table of times, positions, and velocities in SI units:\\
\rule{1.0in}{0pt}\begin{tabular}{c|c|c}
time $t$ ($\s$) & position $x$ ($\m$) & velocity $v$ ($\m\,\s^{-1}$) \\
\hline
1 & 1.15 & 1.3 \\
2 & 2.60 & 1.6 \\
3 & 4.35 & 1.9 \\
\hline
\end{tabular}\\
What is the average acceleration in the time interval from $2\,\s$ to $3\,\s$?
\end{problem}


\vfill ~

\begin{problem} (From Lecture on 2019-09-05)
The molar weight of water is $18\,\g$. How many molecules would there
be, therefore, in $18\,\g$ of water? You don't need a calculator for
this.
\end{problem}


\vfill ~


\cleardoublepage



\noindent
Name: \rule[-1ex]{0.60\textwidth}{0.1pt}
NetID: \rule[-1ex]{0.20\textwidth}{0.1pt}

\section*{\textsl{Einstein's Universe} Term Exam 1}
\setcounter{problem}{1}


\begin{problem} (From Problem Set 1)
What is the relationship between the energy $E$ and wavelength
$\lambda$ of a photon? Give a formula that involves energy $E$,
Planck's Constant $h$, the speed of light $c$, and wavelength
$\lambda$ (or whatever you need).
\end{problem}

\vfill ~

\begin{problem} (From the reading)
What musical instrument did Einstein most enjoy playing?
\end{problem}


\vfill ~

\begin{problem} (From the Kinematics Lab)
Here is a data table of times, positions, and velocities in SI units:\\
\rule{1.0in}{0pt}\begin{tabular}{c|c|c}
time $t$ ($\s$) & position $x$ ($\m$) & velocity $v$ ($\m\,\s^{-1}$) \\
\hline
1 & 1.15 & 1.3 \\
2 & 2.60 & 1.6 \\
3 & 4.35 & 1.9 \\
\hline
\end{tabular}\\
What is the average acceleration in the time interval from $2\,\s$ to $3\,\s$?
\end{problem}


\vfill ~

\begin{problem} (From the Math Review Lab)
What is this number? Give your answer in scientific notation.
$$
\frac{(7\times10^{-34})\times(3\times10^8)}{5\times10^{-7}}
$$
You don't need a calculator to solve this problem (\textit{hint: $3/5=0.6$}).
\end{problem}


\vfill ~


\clearpage


\begin{problem} (From Lecture on 2019-09-05)
The molar weight of water is $18\,\g$. How many molecules would there
be, therefore, in $18\,\g$ of water? You don't need a calculator for
this.
\end{problem}


\vfill ~

\begin{problem} (From the reading)
Classical mechanics, or Newtonian mechanics, is only valid in certain
circumstances. When do the laws of classical mechanics, like $F =
m\,a$ for example, become wrong or break down? There are many answers
to this problem; I will take anything correct.
\end{problem}


\vfill ~

\begin{problem} (From Lecture on 2019-09-17)
If you are traveling at 60 miles per hour, how long does
it take you to go 300 miles?
\end{problem}


\vfill ~

\begin{problem} (From Problem Set 1)
What is the approximate thickness of a stack of 1000 20-dollar bills?
No need to be precise, and use any units you like.
\end{problem}


\vfill ~


\cleardoublepage



\noindent
Name: \rule[-1ex]{0.60\textwidth}{0.1pt}
NetID: \rule[-1ex]{0.20\textwidth}{0.1pt}

\section*{\textsl{Einstein's Universe} Term Exam 1}
\setcounter{problem}{1}


\begin{problem} (From the Kinematics Lab)
Here is a data table of times, positions, and velocities in SI units:\\
\rule{1.0in}{0pt}\begin{tabular}{c|c|c}
time $t$ ($\s$) & position $x$ ($\m$) & velocity $v$ ($\m\,\s^{-1}$) \\
\hline
1 & 1.15 & 1.3 \\
2 & 2.60 & 1.6 \\
3 & 4.35 & 1.9 \\
\hline
\end{tabular}\\
What is the average acceleration in the time interval from $2\,\s$ to $3\,\s$?
\end{problem}


\vfill ~

\begin{problem} (From the reading)
Classical mechanics, or Newtonian mechanics, is only valid in certain
circumstances. When do the laws of classical mechanics, like $F =
m\,a$ for example, become wrong or break down? There are many answers
to this problem; I will take anything correct.
\end{problem}


\vfill ~

\begin{problem} (From Problem Set 1)
What is the approximate thickness of a stack of 1000 20-dollar bills?
No need to be precise, and use any units you like.
\end{problem}


\vfill ~

\begin{problem} (From Lecture on 2019-09-17)
If you are traveling at 60 miles per hour, how long does
it take you to go 300 miles?
\end{problem}


\vfill ~


\clearpage


\begin{problem} (From the Math Review Lab)
What is this number? Give your answer in scientific notation.
$$
\frac{(7\times10^{-34})\times(3\times10^8)}{5\times10^{-7}}
$$
You don't need a calculator to solve this problem (\textit{hint: $3/5=0.6$}).
\end{problem}


\vfill ~

\begin{problem} (From Lecture on 2019-09-05)
The molar weight of water is $18\,\g$. How many molecules would there
be, therefore, in $18\,\g$ of water? You don't need a calculator for
this.
\end{problem}


\vfill ~

\begin{problem} (From the reading)
What musical instrument did Einstein most enjoy playing?
\end{problem}


\vfill ~

\begin{problem} (From Problem Set 1)
What is the relationship between the energy $E$ and wavelength
$\lambda$ of a photon? Give a formula that involves energy $E$,
Planck's Constant $h$, the speed of light $c$, and wavelength
$\lambda$ (or whatever you need).
\end{problem}

\vfill ~


\cleardoublepage



\noindent
Name: \rule[-1ex]{0.60\textwidth}{0.1pt}
NetID: \rule[-1ex]{0.20\textwidth}{0.1pt}

\section*{\textsl{Einstein's Universe} Term Exam 1}
\setcounter{problem}{1}


\begin{problem} (From Problem Set 1)
What is the relationship between the energy $E$ and wavelength
$\lambda$ of a photon? Give a formula that involves energy $E$,
Planck's Constant $h$, the speed of light $c$, and wavelength
$\lambda$ (or whatever you need).
\end{problem}

\vfill ~

\begin{problem} (From Problem Set 1)
What is the approximate thickness of a stack of 1000 20-dollar bills?
No need to be precise, and use any units you like.
\end{problem}


\vfill ~

\begin{problem} (From the Kinematics Lab)
Here is a data table of times, positions, and velocities in SI units:\\
\rule{1.0in}{0pt}\begin{tabular}{c|c|c}
time $t$ ($\s$) & position $x$ ($\m$) & velocity $v$ ($\m\,\s^{-1}$) \\
\hline
1 & 1.15 & 1.3 \\
2 & 2.60 & 1.6 \\
3 & 4.35 & 1.9 \\
\hline
\end{tabular}\\
What is the average acceleration in the time interval from $2\,\s$ to $3\,\s$?
\end{problem}


\vfill ~

\begin{problem} (From the reading)
What musical instrument did Einstein most enjoy playing?
\end{problem}


\vfill ~


\clearpage


\begin{problem} (From the reading)
Classical mechanics, or Newtonian mechanics, is only valid in certain
circumstances. When do the laws of classical mechanics, like $F =
m\,a$ for example, become wrong or break down? There are many answers
to this problem; I will take anything correct.
\end{problem}


\vfill ~

\begin{problem} (From Lecture on 2019-09-17)
If you are traveling at 60 miles per hour, how long does
it take you to go 300 miles?
\end{problem}


\vfill ~

\begin{problem} (From Lecture on 2019-09-05)
The molar weight of water is $18\,\g$. How many molecules would there
be, therefore, in $18\,\g$ of water? You don't need a calculator for
this.
\end{problem}


\vfill ~

\begin{problem} (From the Math Review Lab)
What is this number? Give your answer in scientific notation.
$$
\frac{(7\times10^{-34})\times(3\times10^8)}{5\times10^{-7}}
$$
You don't need a calculator to solve this problem (\textit{hint: $3/5=0.6$}).
\end{problem}


\vfill ~


\cleardoublepage



\noindent
Name: \rule[-1ex]{0.60\textwidth}{0.1pt}
NetID: \rule[-1ex]{0.20\textwidth}{0.1pt}

\section*{\textsl{Einstein's Universe} Term Exam 1}
\setcounter{problem}{1}


\begin{problem} (From Lecture on 2019-09-05)
The molar weight of water is $18\,\g$. How many molecules would there
be, therefore, in $18\,\g$ of water? You don't need a calculator for
this.
\end{problem}


\vfill ~

\begin{problem} (From Problem Set 1)
What is the approximate thickness of a stack of 1000 20-dollar bills?
No need to be precise, and use any units you like.
\end{problem}


\vfill ~

\begin{problem} (From Lecture on 2019-09-17)
If you are traveling at 60 miles per hour, how long does
it take you to go 300 miles?
\end{problem}


\vfill ~

\begin{problem} (From the reading)
What musical instrument did Einstein most enjoy playing?
\end{problem}


\vfill ~


\clearpage


\begin{problem} (From Problem Set 1)
What is the relationship between the energy $E$ and wavelength
$\lambda$ of a photon? Give a formula that involves energy $E$,
Planck's Constant $h$, the speed of light $c$, and wavelength
$\lambda$ (or whatever you need).
\end{problem}

\vfill ~

\begin{problem} (From the reading)
Classical mechanics, or Newtonian mechanics, is only valid in certain
circumstances. When do the laws of classical mechanics, like $F =
m\,a$ for example, become wrong or break down? There are many answers
to this problem; I will take anything correct.
\end{problem}


\vfill ~

\begin{problem} (From the Kinematics Lab)
Here is a data table of times, positions, and velocities in SI units:\\
\rule{1.0in}{0pt}\begin{tabular}{c|c|c}
time $t$ ($\s$) & position $x$ ($\m$) & velocity $v$ ($\m\,\s^{-1}$) \\
\hline
1 & 1.15 & 1.3 \\
2 & 2.60 & 1.6 \\
3 & 4.35 & 1.9 \\
\hline
\end{tabular}\\
What is the average acceleration in the time interval from $2\,\s$ to $3\,\s$?
\end{problem}


\vfill ~

\begin{problem} (From the Math Review Lab)
What is this number? Give your answer in scientific notation.
$$
\frac{(7\times10^{-34})\times(3\times10^8)}{5\times10^{-7}}
$$
You don't need a calculator to solve this problem (\textit{hint: $3/5=0.6$}).
\end{problem}


\vfill ~


\cleardoublepage



\noindent
Name: \rule[-1ex]{0.60\textwidth}{0.1pt}
NetID: \rule[-1ex]{0.20\textwidth}{0.1pt}

\section*{\textsl{Einstein's Universe} Term Exam 1}
\setcounter{problem}{1}


\begin{problem} (From the Kinematics Lab)
Here is a data table of times, positions, and velocities in SI units:\\
\rule{1.0in}{0pt}\begin{tabular}{c|c|c}
time $t$ ($\s$) & position $x$ ($\m$) & velocity $v$ ($\m\,\s^{-1}$) \\
\hline
1 & 1.15 & 1.3 \\
2 & 2.60 & 1.6 \\
3 & 4.35 & 1.9 \\
\hline
\end{tabular}\\
What is the average acceleration in the time interval from $2\,\s$ to $3\,\s$?
\end{problem}


\vfill ~

\begin{problem} (From Lecture on 2019-09-17)
If you are traveling at 60 miles per hour, how long does
it take you to go 300 miles?
\end{problem}


\vfill ~

\begin{problem} (From the reading)
Classical mechanics, or Newtonian mechanics, is only valid in certain
circumstances. When do the laws of classical mechanics, like $F =
m\,a$ for example, become wrong or break down? There are many answers
to this problem; I will take anything correct.
\end{problem}


\vfill ~

\begin{problem} (From the reading)
What musical instrument did Einstein most enjoy playing?
\end{problem}


\vfill ~


\clearpage


\begin{problem} (From Problem Set 1)
What is the approximate thickness of a stack of 1000 20-dollar bills?
No need to be precise, and use any units you like.
\end{problem}


\vfill ~

\begin{problem} (From Problem Set 1)
What is the relationship between the energy $E$ and wavelength
$\lambda$ of a photon? Give a formula that involves energy $E$,
Planck's Constant $h$, the speed of light $c$, and wavelength
$\lambda$ (or whatever you need).
\end{problem}

\vfill ~

\begin{problem} (From the Math Review Lab)
What is this number? Give your answer in scientific notation.
$$
\frac{(7\times10^{-34})\times(3\times10^8)}{5\times10^{-7}}
$$
You don't need a calculator to solve this problem (\textit{hint: $3/5=0.6$}).
\end{problem}


\vfill ~

\begin{problem} (From Lecture on 2019-09-05)
The molar weight of water is $18\,\g$. How many molecules would there
be, therefore, in $18\,\g$ of water? You don't need a calculator for
this.
\end{problem}


\vfill ~


\cleardoublepage



\noindent
Name: \rule[-1ex]{0.60\textwidth}{0.1pt}
NetID: \rule[-1ex]{0.20\textwidth}{0.1pt}

\section*{\textsl{Einstein's Universe} Term Exam 1}
\setcounter{problem}{1}


\begin{problem} (From Problem Set 1)
What is the approximate thickness of a stack of 1000 20-dollar bills?
No need to be precise, and use any units you like.
\end{problem}


\vfill ~

\begin{problem} (From the Kinematics Lab)
Here is a data table of times, positions, and velocities in SI units:\\
\rule{1.0in}{0pt}\begin{tabular}{c|c|c}
time $t$ ($\s$) & position $x$ ($\m$) & velocity $v$ ($\m\,\s^{-1}$) \\
\hline
1 & 1.15 & 1.3 \\
2 & 2.60 & 1.6 \\
3 & 4.35 & 1.9 \\
\hline
\end{tabular}\\
What is the average acceleration in the time interval from $2\,\s$ to $3\,\s$?
\end{problem}


\vfill ~

\begin{problem} (From Lecture on 2019-09-05)
The molar weight of water is $18\,\g$. How many molecules would there
be, therefore, in $18\,\g$ of water? You don't need a calculator for
this.
\end{problem}


\vfill ~

\begin{problem} (From the reading)
Classical mechanics, or Newtonian mechanics, is only valid in certain
circumstances. When do the laws of classical mechanics, like $F =
m\,a$ for example, become wrong or break down? There are many answers
to this problem; I will take anything correct.
\end{problem}


\vfill ~


\clearpage


\begin{problem} (From Lecture on 2019-09-17)
If you are traveling at 60 miles per hour, how long does
it take you to go 300 miles?
\end{problem}


\vfill ~

\begin{problem} (From the Math Review Lab)
What is this number? Give your answer in scientific notation.
$$
\frac{(7\times10^{-34})\times(3\times10^8)}{5\times10^{-7}}
$$
You don't need a calculator to solve this problem (\textit{hint: $3/5=0.6$}).
\end{problem}


\vfill ~

\begin{problem} (From Problem Set 1)
What is the relationship between the energy $E$ and wavelength
$\lambda$ of a photon? Give a formula that involves energy $E$,
Planck's Constant $h$, the speed of light $c$, and wavelength
$\lambda$ (or whatever you need).
\end{problem}

\vfill ~

\begin{problem} (From the reading)
What musical instrument did Einstein most enjoy playing?
\end{problem}


\vfill ~


\cleardoublepage



\noindent
Name: \rule[-1ex]{0.60\textwidth}{0.1pt}
NetID: \rule[-1ex]{0.20\textwidth}{0.1pt}

\section*{\textsl{Einstein's Universe} Term Exam 1}
\setcounter{problem}{1}


\begin{problem} (From Problem Set 1)
What is the approximate thickness of a stack of 1000 20-dollar bills?
No need to be precise, and use any units you like.
\end{problem}


\vfill ~

\begin{problem} (From the Kinematics Lab)
Here is a data table of times, positions, and velocities in SI units:\\
\rule{1.0in}{0pt}\begin{tabular}{c|c|c}
time $t$ ($\s$) & position $x$ ($\m$) & velocity $v$ ($\m\,\s^{-1}$) \\
\hline
1 & 1.15 & 1.3 \\
2 & 2.60 & 1.6 \\
3 & 4.35 & 1.9 \\
\hline
\end{tabular}\\
What is the average acceleration in the time interval from $2\,\s$ to $3\,\s$?
\end{problem}


\vfill ~

\begin{problem} (From Problem Set 1)
What is the relationship between the energy $E$ and wavelength
$\lambda$ of a photon? Give a formula that involves energy $E$,
Planck's Constant $h$, the speed of light $c$, and wavelength
$\lambda$ (or whatever you need).
\end{problem}

\vfill ~

\begin{problem} (From the reading)
Classical mechanics, or Newtonian mechanics, is only valid in certain
circumstances. When do the laws of classical mechanics, like $F =
m\,a$ for example, become wrong or break down? There are many answers
to this problem; I will take anything correct.
\end{problem}


\vfill ~


\clearpage


\begin{problem} (From Lecture on 2019-09-05)
The molar weight of water is $18\,\g$. How many molecules would there
be, therefore, in $18\,\g$ of water? You don't need a calculator for
this.
\end{problem}


\vfill ~

\begin{problem} (From Lecture on 2019-09-17)
If you are traveling at 60 miles per hour, how long does
it take you to go 300 miles?
\end{problem}


\vfill ~

\begin{problem} (From the Math Review Lab)
What is this number? Give your answer in scientific notation.
$$
\frac{(7\times10^{-34})\times(3\times10^8)}{5\times10^{-7}}
$$
You don't need a calculator to solve this problem (\textit{hint: $3/5=0.6$}).
\end{problem}


\vfill ~

\begin{problem} (From the reading)
What musical instrument did Einstein most enjoy playing?
\end{problem}


\vfill ~


\cleardoublepage



\noindent
Name: \rule[-1ex]{0.60\textwidth}{0.1pt}
NetID: \rule[-1ex]{0.20\textwidth}{0.1pt}

\section*{\textsl{Einstein's Universe} Term Exam 1}
\setcounter{problem}{1}


\begin{problem} (From Problem Set 1)
What is the approximate thickness of a stack of 1000 20-dollar bills?
No need to be precise, and use any units you like.
\end{problem}


\vfill ~

\begin{problem} (From the Math Review Lab)
What is this number? Give your answer in scientific notation.
$$
\frac{(7\times10^{-34})\times(3\times10^8)}{5\times10^{-7}}
$$
You don't need a calculator to solve this problem (\textit{hint: $3/5=0.6$}).
\end{problem}


\vfill ~

\begin{problem} (From Problem Set 1)
What is the relationship between the energy $E$ and wavelength
$\lambda$ of a photon? Give a formula that involves energy $E$,
Planck's Constant $h$, the speed of light $c$, and wavelength
$\lambda$ (or whatever you need).
\end{problem}

\vfill ~

\begin{problem} (From the reading)
What musical instrument did Einstein most enjoy playing?
\end{problem}


\vfill ~


\clearpage


\begin{problem} (From the reading)
Classical mechanics, or Newtonian mechanics, is only valid in certain
circumstances. When do the laws of classical mechanics, like $F =
m\,a$ for example, become wrong or break down? There are many answers
to this problem; I will take anything correct.
\end{problem}


\vfill ~

\begin{problem} (From the Kinematics Lab)
Here is a data table of times, positions, and velocities in SI units:\\
\rule{1.0in}{0pt}\begin{tabular}{c|c|c}
time $t$ ($\s$) & position $x$ ($\m$) & velocity $v$ ($\m\,\s^{-1}$) \\
\hline
1 & 1.15 & 1.3 \\
2 & 2.60 & 1.6 \\
3 & 4.35 & 1.9 \\
\hline
\end{tabular}\\
What is the average acceleration in the time interval from $2\,\s$ to $3\,\s$?
\end{problem}


\vfill ~

\begin{problem} (From Lecture on 2019-09-17)
If you are traveling at 60 miles per hour, how long does
it take you to go 300 miles?
\end{problem}


\vfill ~

\begin{problem} (From Lecture on 2019-09-05)
The molar weight of water is $18\,\g$. How many molecules would there
be, therefore, in $18\,\g$ of water? You don't need a calculator for
this.
\end{problem}


\vfill ~


\cleardoublepage



\noindent
Name: \rule[-1ex]{0.60\textwidth}{0.1pt}
NetID: \rule[-1ex]{0.20\textwidth}{0.1pt}

\section*{\textsl{Einstein's Universe} Term Exam 1}
\setcounter{problem}{1}


\begin{problem} (From the Math Review Lab)
What is this number? Give your answer in scientific notation.
$$
\frac{(7\times10^{-34})\times(3\times10^8)}{5\times10^{-7}}
$$
You don't need a calculator to solve this problem (\textit{hint: $3/5=0.6$}).
\end{problem}


\vfill ~

\begin{problem} (From Lecture on 2019-09-05)
The molar weight of water is $18\,\g$. How many molecules would there
be, therefore, in $18\,\g$ of water? You don't need a calculator for
this.
\end{problem}


\vfill ~

\begin{problem} (From Problem Set 1)
What is the relationship between the energy $E$ and wavelength
$\lambda$ of a photon? Give a formula that involves energy $E$,
Planck's Constant $h$, the speed of light $c$, and wavelength
$\lambda$ (or whatever you need).
\end{problem}

\vfill ~

\begin{problem} (From the reading)
Classical mechanics, or Newtonian mechanics, is only valid in certain
circumstances. When do the laws of classical mechanics, like $F =
m\,a$ for example, become wrong or break down? There are many answers
to this problem; I will take anything correct.
\end{problem}


\vfill ~


\clearpage


\begin{problem} (From Problem Set 1)
What is the approximate thickness of a stack of 1000 20-dollar bills?
No need to be precise, and use any units you like.
\end{problem}


\vfill ~

\begin{problem} (From Lecture on 2019-09-17)
If you are traveling at 60 miles per hour, how long does
it take you to go 300 miles?
\end{problem}


\vfill ~

\begin{problem} (From the reading)
What musical instrument did Einstein most enjoy playing?
\end{problem}


\vfill ~

\begin{problem} (From the Kinematics Lab)
Here is a data table of times, positions, and velocities in SI units:\\
\rule{1.0in}{0pt}\begin{tabular}{c|c|c}
time $t$ ($\s$) & position $x$ ($\m$) & velocity $v$ ($\m\,\s^{-1}$) \\
\hline
1 & 1.15 & 1.3 \\
2 & 2.60 & 1.6 \\
3 & 4.35 & 1.9 \\
\hline
\end{tabular}\\
What is the average acceleration in the time interval from $2\,\s$ to $3\,\s$?
\end{problem}


\vfill ~


\cleardoublepage



\noindent
Name: \rule[-1ex]{0.60\textwidth}{0.1pt}
NetID: \rule[-1ex]{0.20\textwidth}{0.1pt}

\section*{\textsl{Einstein's Universe} Term Exam 1}
\setcounter{problem}{1}


\begin{problem} (From the reading)
Classical mechanics, or Newtonian mechanics, is only valid in certain
circumstances. When do the laws of classical mechanics, like $F =
m\,a$ for example, become wrong or break down? There are many answers
to this problem; I will take anything correct.
\end{problem}


\vfill ~

\begin{problem} (From the Math Review Lab)
What is this number? Give your answer in scientific notation.
$$
\frac{(7\times10^{-34})\times(3\times10^8)}{5\times10^{-7}}
$$
You don't need a calculator to solve this problem (\textit{hint: $3/5=0.6$}).
\end{problem}


\vfill ~

\begin{problem} (From Problem Set 1)
What is the approximate thickness of a stack of 1000 20-dollar bills?
No need to be precise, and use any units you like.
\end{problem}


\vfill ~

\begin{problem} (From the Kinematics Lab)
Here is a data table of times, positions, and velocities in SI units:\\
\rule{1.0in}{0pt}\begin{tabular}{c|c|c}
time $t$ ($\s$) & position $x$ ($\m$) & velocity $v$ ($\m\,\s^{-1}$) \\
\hline
1 & 1.15 & 1.3 \\
2 & 2.60 & 1.6 \\
3 & 4.35 & 1.9 \\
\hline
\end{tabular}\\
What is the average acceleration in the time interval from $2\,\s$ to $3\,\s$?
\end{problem}


\vfill ~


\clearpage


\begin{problem} (From the reading)
What musical instrument did Einstein most enjoy playing?
\end{problem}


\vfill ~

\begin{problem} (From Lecture on 2019-09-17)
If you are traveling at 60 miles per hour, how long does
it take you to go 300 miles?
\end{problem}


\vfill ~

\begin{problem} (From Problem Set 1)
What is the relationship between the energy $E$ and wavelength
$\lambda$ of a photon? Give a formula that involves energy $E$,
Planck's Constant $h$, the speed of light $c$, and wavelength
$\lambda$ (or whatever you need).
\end{problem}

\vfill ~

\begin{problem} (From Lecture on 2019-09-05)
The molar weight of water is $18\,\g$. How many molecules would there
be, therefore, in $18\,\g$ of water? You don't need a calculator for
this.
\end{problem}


\vfill ~


\cleardoublepage



\noindent
Name: \rule[-1ex]{0.60\textwidth}{0.1pt}
NetID: \rule[-1ex]{0.20\textwidth}{0.1pt}

\section*{\textsl{Einstein's Universe} Term Exam 1}
\setcounter{problem}{1}


\begin{problem} (From the reading)
What musical instrument did Einstein most enjoy playing?
\end{problem}


\vfill ~

\begin{problem} (From Lecture on 2019-09-05)
The molar weight of water is $18\,\g$. How many molecules would there
be, therefore, in $18\,\g$ of water? You don't need a calculator for
this.
\end{problem}


\vfill ~

\begin{problem} (From Problem Set 1)
What is the approximate thickness of a stack of 1000 20-dollar bills?
No need to be precise, and use any units you like.
\end{problem}


\vfill ~

\begin{problem} (From Problem Set 1)
What is the relationship between the energy $E$ and wavelength
$\lambda$ of a photon? Give a formula that involves energy $E$,
Planck's Constant $h$, the speed of light $c$, and wavelength
$\lambda$ (or whatever you need).
\end{problem}

\vfill ~


\clearpage


\begin{problem} (From the reading)
Classical mechanics, or Newtonian mechanics, is only valid in certain
circumstances. When do the laws of classical mechanics, like $F =
m\,a$ for example, become wrong or break down? There are many answers
to this problem; I will take anything correct.
\end{problem}


\vfill ~

\begin{problem} (From Lecture on 2019-09-17)
If you are traveling at 60 miles per hour, how long does
it take you to go 300 miles?
\end{problem}


\vfill ~

\begin{problem} (From the Kinematics Lab)
Here is a data table of times, positions, and velocities in SI units:\\
\rule{1.0in}{0pt}\begin{tabular}{c|c|c}
time $t$ ($\s$) & position $x$ ($\m$) & velocity $v$ ($\m\,\s^{-1}$) \\
\hline
1 & 1.15 & 1.3 \\
2 & 2.60 & 1.6 \\
3 & 4.35 & 1.9 \\
\hline
\end{tabular}\\
What is the average acceleration in the time interval from $2\,\s$ to $3\,\s$?
\end{problem}


\vfill ~

\begin{problem} (From the Math Review Lab)
What is this number? Give your answer in scientific notation.
$$
\frac{(7\times10^{-34})\times(3\times10^8)}{5\times10^{-7}}
$$
You don't need a calculator to solve this problem (\textit{hint: $3/5=0.6$}).
\end{problem}


\vfill ~


\cleardoublepage



\noindent
Name: \rule[-1ex]{0.60\textwidth}{0.1pt}
NetID: \rule[-1ex]{0.20\textwidth}{0.1pt}

\section*{\textsl{Einstein's Universe} Term Exam 1}
\setcounter{problem}{1}


\begin{problem} (From Lecture on 2019-09-05)
The molar weight of water is $18\,\g$. How many molecules would there
be, therefore, in $18\,\g$ of water? You don't need a calculator for
this.
\end{problem}


\vfill ~

\begin{problem} (From the Math Review Lab)
What is this number? Give your answer in scientific notation.
$$
\frac{(7\times10^{-34})\times(3\times10^8)}{5\times10^{-7}}
$$
You don't need a calculator to solve this problem (\textit{hint: $3/5=0.6$}).
\end{problem}


\vfill ~

\begin{problem} (From the reading)
Classical mechanics, or Newtonian mechanics, is only valid in certain
circumstances. When do the laws of classical mechanics, like $F =
m\,a$ for example, become wrong or break down? There are many answers
to this problem; I will take anything correct.
\end{problem}


\vfill ~

\begin{problem} (From Problem Set 1)
What is the approximate thickness of a stack of 1000 20-dollar bills?
No need to be precise, and use any units you like.
\end{problem}


\vfill ~


\clearpage


\begin{problem} (From the Kinematics Lab)
Here is a data table of times, positions, and velocities in SI units:\\
\rule{1.0in}{0pt}\begin{tabular}{c|c|c}
time $t$ ($\s$) & position $x$ ($\m$) & velocity $v$ ($\m\,\s^{-1}$) \\
\hline
1 & 1.15 & 1.3 \\
2 & 2.60 & 1.6 \\
3 & 4.35 & 1.9 \\
\hline
\end{tabular}\\
What is the average acceleration in the time interval from $2\,\s$ to $3\,\s$?
\end{problem}


\vfill ~

\begin{problem} (From Lecture on 2019-09-17)
If you are traveling at 60 miles per hour, how long does
it take you to go 300 miles?
\end{problem}


\vfill ~

\begin{problem} (From Problem Set 1)
What is the relationship between the energy $E$ and wavelength
$\lambda$ of a photon? Give a formula that involves energy $E$,
Planck's Constant $h$, the speed of light $c$, and wavelength
$\lambda$ (or whatever you need).
\end{problem}

\vfill ~

\begin{problem} (From the reading)
What musical instrument did Einstein most enjoy playing?
\end{problem}


\vfill ~


\cleardoublepage



\noindent
Name: \rule[-1ex]{0.60\textwidth}{0.1pt}
NetID: \rule[-1ex]{0.20\textwidth}{0.1pt}

\section*{\textsl{Einstein's Universe} Term Exam 1}
\setcounter{problem}{1}


\begin{problem} (From the Math Review Lab)
What is this number? Give your answer in scientific notation.
$$
\frac{(7\times10^{-34})\times(3\times10^8)}{5\times10^{-7}}
$$
You don't need a calculator to solve this problem (\textit{hint: $3/5=0.6$}).
\end{problem}


\vfill ~

\begin{problem} (From Lecture on 2019-09-17)
If you are traveling at 60 miles per hour, how long does
it take you to go 300 miles?
\end{problem}


\vfill ~

\begin{problem} (From the reading)
Classical mechanics, or Newtonian mechanics, is only valid in certain
circumstances. When do the laws of classical mechanics, like $F =
m\,a$ for example, become wrong or break down? There are many answers
to this problem; I will take anything correct.
\end{problem}


\vfill ~

\begin{problem} (From Problem Set 1)
What is the approximate thickness of a stack of 1000 20-dollar bills?
No need to be precise, and use any units you like.
\end{problem}


\vfill ~


\clearpage


\begin{problem} (From the Kinematics Lab)
Here is a data table of times, positions, and velocities in SI units:\\
\rule{1.0in}{0pt}\begin{tabular}{c|c|c}
time $t$ ($\s$) & position $x$ ($\m$) & velocity $v$ ($\m\,\s^{-1}$) \\
\hline
1 & 1.15 & 1.3 \\
2 & 2.60 & 1.6 \\
3 & 4.35 & 1.9 \\
\hline
\end{tabular}\\
What is the average acceleration in the time interval from $2\,\s$ to $3\,\s$?
\end{problem}


\vfill ~

\begin{problem} (From the reading)
What musical instrument did Einstein most enjoy playing?
\end{problem}


\vfill ~

\begin{problem} (From Problem Set 1)
What is the relationship between the energy $E$ and wavelength
$\lambda$ of a photon? Give a formula that involves energy $E$,
Planck's Constant $h$, the speed of light $c$, and wavelength
$\lambda$ (or whatever you need).
\end{problem}

\vfill ~

\begin{problem} (From Lecture on 2019-09-05)
The molar weight of water is $18\,\g$. How many molecules would there
be, therefore, in $18\,\g$ of water? You don't need a calculator for
this.
\end{problem}


\vfill ~


\cleardoublepage



\noindent
Name: \rule[-1ex]{0.60\textwidth}{0.1pt}
NetID: \rule[-1ex]{0.20\textwidth}{0.1pt}

\section*{\textsl{Einstein's Universe} Term Exam 1}
\setcounter{problem}{1}


\begin{problem} (From Problem Set 1)
What is the approximate thickness of a stack of 1000 20-dollar bills?
No need to be precise, and use any units you like.
\end{problem}


\vfill ~

\begin{problem} (From the reading)
Classical mechanics, or Newtonian mechanics, is only valid in certain
circumstances. When do the laws of classical mechanics, like $F =
m\,a$ for example, become wrong or break down? There are many answers
to this problem; I will take anything correct.
\end{problem}


\vfill ~

\begin{problem} (From Lecture on 2019-09-17)
If you are traveling at 60 miles per hour, how long does
it take you to go 300 miles?
\end{problem}


\vfill ~

\begin{problem} (From Problem Set 1)
What is the relationship between the energy $E$ and wavelength
$\lambda$ of a photon? Give a formula that involves energy $E$,
Planck's Constant $h$, the speed of light $c$, and wavelength
$\lambda$ (or whatever you need).
\end{problem}

\vfill ~


\clearpage


\begin{problem} (From Lecture on 2019-09-05)
The molar weight of water is $18\,\g$. How many molecules would there
be, therefore, in $18\,\g$ of water? You don't need a calculator for
this.
\end{problem}


\vfill ~

\begin{problem} (From the Math Review Lab)
What is this number? Give your answer in scientific notation.
$$
\frac{(7\times10^{-34})\times(3\times10^8)}{5\times10^{-7}}
$$
You don't need a calculator to solve this problem (\textit{hint: $3/5=0.6$}).
\end{problem}


\vfill ~

\begin{problem} (From the reading)
What musical instrument did Einstein most enjoy playing?
\end{problem}


\vfill ~

\begin{problem} (From the Kinematics Lab)
Here is a data table of times, positions, and velocities in SI units:\\
\rule{1.0in}{0pt}\begin{tabular}{c|c|c}
time $t$ ($\s$) & position $x$ ($\m$) & velocity $v$ ($\m\,\s^{-1}$) \\
\hline
1 & 1.15 & 1.3 \\
2 & 2.60 & 1.6 \\
3 & 4.35 & 1.9 \\
\hline
\end{tabular}\\
What is the average acceleration in the time interval from $2\,\s$ to $3\,\s$?
\end{problem}


\vfill ~


\cleardoublepage



\noindent
Name: \rule[-1ex]{0.60\textwidth}{0.1pt}
NetID: \rule[-1ex]{0.20\textwidth}{0.1pt}

\section*{\textsl{Einstein's Universe} Term Exam 1}
\setcounter{problem}{1}


\begin{problem} (From Lecture on 2019-09-05)
The molar weight of water is $18\,\g$. How many molecules would there
be, therefore, in $18\,\g$ of water? You don't need a calculator for
this.
\end{problem}


\vfill ~

\begin{problem} (From Problem Set 1)
What is the relationship between the energy $E$ and wavelength
$\lambda$ of a photon? Give a formula that involves energy $E$,
Planck's Constant $h$, the speed of light $c$, and wavelength
$\lambda$ (or whatever you need).
\end{problem}

\vfill ~

\begin{problem} (From Problem Set 1)
What is the approximate thickness of a stack of 1000 20-dollar bills?
No need to be precise, and use any units you like.
\end{problem}


\vfill ~

\begin{problem} (From the reading)
Classical mechanics, or Newtonian mechanics, is only valid in certain
circumstances. When do the laws of classical mechanics, like $F =
m\,a$ for example, become wrong or break down? There are many answers
to this problem; I will take anything correct.
\end{problem}


\vfill ~


\clearpage


\begin{problem} (From Lecture on 2019-09-17)
If you are traveling at 60 miles per hour, how long does
it take you to go 300 miles?
\end{problem}


\vfill ~

\begin{problem} (From the reading)
What musical instrument did Einstein most enjoy playing?
\end{problem}


\vfill ~

\begin{problem} (From the Kinematics Lab)
Here is a data table of times, positions, and velocities in SI units:\\
\rule{1.0in}{0pt}\begin{tabular}{c|c|c}
time $t$ ($\s$) & position $x$ ($\m$) & velocity $v$ ($\m\,\s^{-1}$) \\
\hline
1 & 1.15 & 1.3 \\
2 & 2.60 & 1.6 \\
3 & 4.35 & 1.9 \\
\hline
\end{tabular}\\
What is the average acceleration in the time interval from $2\,\s$ to $3\,\s$?
\end{problem}


\vfill ~

\begin{problem} (From the Math Review Lab)
What is this number? Give your answer in scientific notation.
$$
\frac{(7\times10^{-34})\times(3\times10^8)}{5\times10^{-7}}
$$
You don't need a calculator to solve this problem (\textit{hint: $3/5=0.6$}).
\end{problem}


\vfill ~


\cleardoublepage



\noindent
Name: \rule[-1ex]{0.60\textwidth}{0.1pt}
NetID: \rule[-1ex]{0.20\textwidth}{0.1pt}

\section*{\textsl{Einstein's Universe} Term Exam 1}
\setcounter{problem}{1}


\begin{problem} (From Problem Set 1)
What is the relationship between the energy $E$ and wavelength
$\lambda$ of a photon? Give a formula that involves energy $E$,
Planck's Constant $h$, the speed of light $c$, and wavelength
$\lambda$ (or whatever you need).
\end{problem}

\vfill ~

\begin{problem} (From Lecture on 2019-09-05)
The molar weight of water is $18\,\g$. How many molecules would there
be, therefore, in $18\,\g$ of water? You don't need a calculator for
this.
\end{problem}


\vfill ~

\begin{problem} (From the reading)
What musical instrument did Einstein most enjoy playing?
\end{problem}


\vfill ~

\begin{problem} (From Lecture on 2019-09-17)
If you are traveling at 60 miles per hour, how long does
it take you to go 300 miles?
\end{problem}


\vfill ~


\clearpage


\begin{problem} (From the Math Review Lab)
What is this number? Give your answer in scientific notation.
$$
\frac{(7\times10^{-34})\times(3\times10^8)}{5\times10^{-7}}
$$
You don't need a calculator to solve this problem (\textit{hint: $3/5=0.6$}).
\end{problem}


\vfill ~

\begin{problem} (From the reading)
Classical mechanics, or Newtonian mechanics, is only valid in certain
circumstances. When do the laws of classical mechanics, like $F =
m\,a$ for example, become wrong or break down? There are many answers
to this problem; I will take anything correct.
\end{problem}


\vfill ~

\begin{problem} (From the Kinematics Lab)
Here is a data table of times, positions, and velocities in SI units:\\
\rule{1.0in}{0pt}\begin{tabular}{c|c|c}
time $t$ ($\s$) & position $x$ ($\m$) & velocity $v$ ($\m\,\s^{-1}$) \\
\hline
1 & 1.15 & 1.3 \\
2 & 2.60 & 1.6 \\
3 & 4.35 & 1.9 \\
\hline
\end{tabular}\\
What is the average acceleration in the time interval from $2\,\s$ to $3\,\s$?
\end{problem}


\vfill ~

\begin{problem} (From Problem Set 1)
What is the approximate thickness of a stack of 1000 20-dollar bills?
No need to be precise, and use any units you like.
\end{problem}


\vfill ~


\cleardoublepage



\noindent
Name: \rule[-1ex]{0.60\textwidth}{0.1pt}
NetID: \rule[-1ex]{0.20\textwidth}{0.1pt}

\section*{\textsl{Einstein's Universe} Term Exam 1}
\setcounter{problem}{1}


\begin{problem} (From the reading)
Classical mechanics, or Newtonian mechanics, is only valid in certain
circumstances. When do the laws of classical mechanics, like $F =
m\,a$ for example, become wrong or break down? There are many answers
to this problem; I will take anything correct.
\end{problem}


\vfill ~

\begin{problem} (From Problem Set 1)
What is the relationship between the energy $E$ and wavelength
$\lambda$ of a photon? Give a formula that involves energy $E$,
Planck's Constant $h$, the speed of light $c$, and wavelength
$\lambda$ (or whatever you need).
\end{problem}

\vfill ~

\begin{problem} (From the Kinematics Lab)
Here is a data table of times, positions, and velocities in SI units:\\
\rule{1.0in}{0pt}\begin{tabular}{c|c|c}
time $t$ ($\s$) & position $x$ ($\m$) & velocity $v$ ($\m\,\s^{-1}$) \\
\hline
1 & 1.15 & 1.3 \\
2 & 2.60 & 1.6 \\
3 & 4.35 & 1.9 \\
\hline
\end{tabular}\\
What is the average acceleration in the time interval from $2\,\s$ to $3\,\s$?
\end{problem}


\vfill ~

\begin{problem} (From the reading)
What musical instrument did Einstein most enjoy playing?
\end{problem}


\vfill ~


\clearpage


\begin{problem} (From the Math Review Lab)
What is this number? Give your answer in scientific notation.
$$
\frac{(7\times10^{-34})\times(3\times10^8)}{5\times10^{-7}}
$$
You don't need a calculator to solve this problem (\textit{hint: $3/5=0.6$}).
\end{problem}


\vfill ~

\begin{problem} (From Problem Set 1)
What is the approximate thickness of a stack of 1000 20-dollar bills?
No need to be precise, and use any units you like.
\end{problem}


\vfill ~

\begin{problem} (From Lecture on 2019-09-05)
The molar weight of water is $18\,\g$. How many molecules would there
be, therefore, in $18\,\g$ of water? You don't need a calculator for
this.
\end{problem}


\vfill ~

\begin{problem} (From Lecture on 2019-09-17)
If you are traveling at 60 miles per hour, how long does
it take you to go 300 miles?
\end{problem}


\vfill ~


\cleardoublepage



\noindent
Name: \rule[-1ex]{0.60\textwidth}{0.1pt}
NetID: \rule[-1ex]{0.20\textwidth}{0.1pt}

\section*{\textsl{Einstein's Universe} Term Exam 1}
\setcounter{problem}{1}


\begin{problem} (From the Math Review Lab)
What is this number? Give your answer in scientific notation.
$$
\frac{(7\times10^{-34})\times(3\times10^8)}{5\times10^{-7}}
$$
You don't need a calculator to solve this problem (\textit{hint: $3/5=0.6$}).
\end{problem}


\vfill ~

\begin{problem} (From Lecture on 2019-09-05)
The molar weight of water is $18\,\g$. How many molecules would there
be, therefore, in $18\,\g$ of water? You don't need a calculator for
this.
\end{problem}


\vfill ~

\begin{problem} (From Lecture on 2019-09-17)
If you are traveling at 60 miles per hour, how long does
it take you to go 300 miles?
\end{problem}


\vfill ~

\begin{problem} (From the Kinematics Lab)
Here is a data table of times, positions, and velocities in SI units:\\
\rule{1.0in}{0pt}\begin{tabular}{c|c|c}
time $t$ ($\s$) & position $x$ ($\m$) & velocity $v$ ($\m\,\s^{-1}$) \\
\hline
1 & 1.15 & 1.3 \\
2 & 2.60 & 1.6 \\
3 & 4.35 & 1.9 \\
\hline
\end{tabular}\\
What is the average acceleration in the time interval from $2\,\s$ to $3\,\s$?
\end{problem}


\vfill ~


\clearpage


\begin{problem} (From the reading)
Classical mechanics, or Newtonian mechanics, is only valid in certain
circumstances. When do the laws of classical mechanics, like $F =
m\,a$ for example, become wrong or break down? There are many answers
to this problem; I will take anything correct.
\end{problem}


\vfill ~

\begin{problem} (From Problem Set 1)
What is the relationship between the energy $E$ and wavelength
$\lambda$ of a photon? Give a formula that involves energy $E$,
Planck's Constant $h$, the speed of light $c$, and wavelength
$\lambda$ (or whatever you need).
\end{problem}

\vfill ~

\begin{problem} (From the reading)
What musical instrument did Einstein most enjoy playing?
\end{problem}


\vfill ~

\begin{problem} (From Problem Set 1)
What is the approximate thickness of a stack of 1000 20-dollar bills?
No need to be precise, and use any units you like.
\end{problem}


\vfill ~


\cleardoublepage



\noindent
Name: \rule[-1ex]{0.60\textwidth}{0.1pt}
NetID: \rule[-1ex]{0.20\textwidth}{0.1pt}

\section*{\textsl{Einstein's Universe} Term Exam 1}
\setcounter{problem}{1}


\begin{problem} (From Lecture on 2019-09-05)
The molar weight of water is $18\,\g$. How many molecules would there
be, therefore, in $18\,\g$ of water? You don't need a calculator for
this.
\end{problem}


\vfill ~

\begin{problem} (From Lecture on 2019-09-17)
If you are traveling at 60 miles per hour, how long does
it take you to go 300 miles?
\end{problem}


\vfill ~

\begin{problem} (From the Math Review Lab)
What is this number? Give your answer in scientific notation.
$$
\frac{(7\times10^{-34})\times(3\times10^8)}{5\times10^{-7}}
$$
You don't need a calculator to solve this problem (\textit{hint: $3/5=0.6$}).
\end{problem}


\vfill ~

\begin{problem} (From Problem Set 1)
What is the relationship between the energy $E$ and wavelength
$\lambda$ of a photon? Give a formula that involves energy $E$,
Planck's Constant $h$, the speed of light $c$, and wavelength
$\lambda$ (or whatever you need).
\end{problem}

\vfill ~


\clearpage


\begin{problem} (From the reading)
What musical instrument did Einstein most enjoy playing?
\end{problem}


\vfill ~

\begin{problem} (From the reading)
Classical mechanics, or Newtonian mechanics, is only valid in certain
circumstances. When do the laws of classical mechanics, like $F =
m\,a$ for example, become wrong or break down? There are many answers
to this problem; I will take anything correct.
\end{problem}


\vfill ~

\begin{problem} (From Problem Set 1)
What is the approximate thickness of a stack of 1000 20-dollar bills?
No need to be precise, and use any units you like.
\end{problem}


\vfill ~

\begin{problem} (From the Kinematics Lab)
Here is a data table of times, positions, and velocities in SI units:\\
\rule{1.0in}{0pt}\begin{tabular}{c|c|c}
time $t$ ($\s$) & position $x$ ($\m$) & velocity $v$ ($\m\,\s^{-1}$) \\
\hline
1 & 1.15 & 1.3 \\
2 & 2.60 & 1.6 \\
3 & 4.35 & 1.9 \\
\hline
\end{tabular}\\
What is the average acceleration in the time interval from $2\,\s$ to $3\,\s$?
\end{problem}


\vfill ~


\cleardoublepage



\noindent
Name: \rule[-1ex]{0.60\textwidth}{0.1pt}
NetID: \rule[-1ex]{0.20\textwidth}{0.1pt}

\section*{\textsl{Einstein's Universe} Term Exam 1}
\setcounter{problem}{1}


\begin{problem} (From Lecture on 2019-09-05)
The molar weight of water is $18\,\g$. How many molecules would there
be, therefore, in $18\,\g$ of water? You don't need a calculator for
this.
\end{problem}


\vfill ~

\begin{problem} (From Lecture on 2019-09-17)
If you are traveling at 60 miles per hour, how long does
it take you to go 300 miles?
\end{problem}


\vfill ~

\begin{problem} (From the reading)
Classical mechanics, or Newtonian mechanics, is only valid in certain
circumstances. When do the laws of classical mechanics, like $F =
m\,a$ for example, become wrong or break down? There are many answers
to this problem; I will take anything correct.
\end{problem}


\vfill ~

\begin{problem} (From Problem Set 1)
What is the relationship between the energy $E$ and wavelength
$\lambda$ of a photon? Give a formula that involves energy $E$,
Planck's Constant $h$, the speed of light $c$, and wavelength
$\lambda$ (or whatever you need).
\end{problem}

\vfill ~


\clearpage


\begin{problem} (From the Math Review Lab)
What is this number? Give your answer in scientific notation.
$$
\frac{(7\times10^{-34})\times(3\times10^8)}{5\times10^{-7}}
$$
You don't need a calculator to solve this problem (\textit{hint: $3/5=0.6$}).
\end{problem}


\vfill ~

\begin{problem} (From the Kinematics Lab)
Here is a data table of times, positions, and velocities in SI units:\\
\rule{1.0in}{0pt}\begin{tabular}{c|c|c}
time $t$ ($\s$) & position $x$ ($\m$) & velocity $v$ ($\m\,\s^{-1}$) \\
\hline
1 & 1.15 & 1.3 \\
2 & 2.60 & 1.6 \\
3 & 4.35 & 1.9 \\
\hline
\end{tabular}\\
What is the average acceleration in the time interval from $2\,\s$ to $3\,\s$?
\end{problem}


\vfill ~

\begin{problem} (From the reading)
What musical instrument did Einstein most enjoy playing?
\end{problem}


\vfill ~

\begin{problem} (From Problem Set 1)
What is the approximate thickness of a stack of 1000 20-dollar bills?
No need to be precise, and use any units you like.
\end{problem}


\vfill ~


\cleardoublepage



\noindent
Name: \rule[-1ex]{0.60\textwidth}{0.1pt}
NetID: \rule[-1ex]{0.20\textwidth}{0.1pt}

\section*{\textsl{Einstein's Universe} Term Exam 1}
\setcounter{problem}{1}


\begin{problem} (From Problem Set 1)
What is the approximate thickness of a stack of 1000 20-dollar bills?
No need to be precise, and use any units you like.
\end{problem}


\vfill ~

\begin{problem} (From the reading)
Classical mechanics, or Newtonian mechanics, is only valid in certain
circumstances. When do the laws of classical mechanics, like $F =
m\,a$ for example, become wrong or break down? There are many answers
to this problem; I will take anything correct.
\end{problem}


\vfill ~

\begin{problem} (From the reading)
What musical instrument did Einstein most enjoy playing?
\end{problem}


\vfill ~

\begin{problem} (From the Kinematics Lab)
Here is a data table of times, positions, and velocities in SI units:\\
\rule{1.0in}{0pt}\begin{tabular}{c|c|c}
time $t$ ($\s$) & position $x$ ($\m$) & velocity $v$ ($\m\,\s^{-1}$) \\
\hline
1 & 1.15 & 1.3 \\
2 & 2.60 & 1.6 \\
3 & 4.35 & 1.9 \\
\hline
\end{tabular}\\
What is the average acceleration in the time interval from $2\,\s$ to $3\,\s$?
\end{problem}


\vfill ~


\clearpage


\begin{problem} (From the Math Review Lab)
What is this number? Give your answer in scientific notation.
$$
\frac{(7\times10^{-34})\times(3\times10^8)}{5\times10^{-7}}
$$
You don't need a calculator to solve this problem (\textit{hint: $3/5=0.6$}).
\end{problem}


\vfill ~

\begin{problem} (From Problem Set 1)
What is the relationship between the energy $E$ and wavelength
$\lambda$ of a photon? Give a formula that involves energy $E$,
Planck's Constant $h$, the speed of light $c$, and wavelength
$\lambda$ (or whatever you need).
\end{problem}

\vfill ~

\begin{problem} (From Lecture on 2019-09-05)
The molar weight of water is $18\,\g$. How many molecules would there
be, therefore, in $18\,\g$ of water? You don't need a calculator for
this.
\end{problem}


\vfill ~

\begin{problem} (From Lecture on 2019-09-17)
If you are traveling at 60 miles per hour, how long does
it take you to go 300 miles?
\end{problem}


\vfill ~


\cleardoublepage



\noindent
Name: \rule[-1ex]{0.60\textwidth}{0.1pt}
NetID: \rule[-1ex]{0.20\textwidth}{0.1pt}

\section*{\textsl{Einstein's Universe} Term Exam 1}
\setcounter{problem}{1}


\begin{problem} (From the reading)
Classical mechanics, or Newtonian mechanics, is only valid in certain
circumstances. When do the laws of classical mechanics, like $F =
m\,a$ for example, become wrong or break down? There are many answers
to this problem; I will take anything correct.
\end{problem}


\vfill ~

\begin{problem} (From Lecture on 2019-09-17)
If you are traveling at 60 miles per hour, how long does
it take you to go 300 miles?
\end{problem}


\vfill ~

\begin{problem} (From the Kinematics Lab)
Here is a data table of times, positions, and velocities in SI units:\\
\rule{1.0in}{0pt}\begin{tabular}{c|c|c}
time $t$ ($\s$) & position $x$ ($\m$) & velocity $v$ ($\m\,\s^{-1}$) \\
\hline
1 & 1.15 & 1.3 \\
2 & 2.60 & 1.6 \\
3 & 4.35 & 1.9 \\
\hline
\end{tabular}\\
What is the average acceleration in the time interval from $2\,\s$ to $3\,\s$?
\end{problem}


\vfill ~

\begin{problem} (From Lecture on 2019-09-05)
The molar weight of water is $18\,\g$. How many molecules would there
be, therefore, in $18\,\g$ of water? You don't need a calculator for
this.
\end{problem}


\vfill ~


\clearpage


\begin{problem} (From Problem Set 1)
What is the relationship between the energy $E$ and wavelength
$\lambda$ of a photon? Give a formula that involves energy $E$,
Planck's Constant $h$, the speed of light $c$, and wavelength
$\lambda$ (or whatever you need).
\end{problem}

\vfill ~

\begin{problem} (From Problem Set 1)
What is the approximate thickness of a stack of 1000 20-dollar bills?
No need to be precise, and use any units you like.
\end{problem}


\vfill ~

\begin{problem} (From the Math Review Lab)
What is this number? Give your answer in scientific notation.
$$
\frac{(7\times10^{-34})\times(3\times10^8)}{5\times10^{-7}}
$$
You don't need a calculator to solve this problem (\textit{hint: $3/5=0.6$}).
\end{problem}


\vfill ~

\begin{problem} (From the reading)
What musical instrument did Einstein most enjoy playing?
\end{problem}


\vfill ~


\cleardoublepage



\noindent
Name: \rule[-1ex]{0.60\textwidth}{0.1pt}
NetID: \rule[-1ex]{0.20\textwidth}{0.1pt}

\section*{\textsl{Einstein's Universe} Term Exam 1}
\setcounter{problem}{1}


\begin{problem} (From the reading)
What musical instrument did Einstein most enjoy playing?
\end{problem}


\vfill ~

\begin{problem} (From Problem Set 1)
What is the relationship between the energy $E$ and wavelength
$\lambda$ of a photon? Give a formula that involves energy $E$,
Planck's Constant $h$, the speed of light $c$, and wavelength
$\lambda$ (or whatever you need).
\end{problem}

\vfill ~

\begin{problem} (From Problem Set 1)
What is the approximate thickness of a stack of 1000 20-dollar bills?
No need to be precise, and use any units you like.
\end{problem}


\vfill ~

\begin{problem} (From Lecture on 2019-09-05)
The molar weight of water is $18\,\g$. How many molecules would there
be, therefore, in $18\,\g$ of water? You don't need a calculator for
this.
\end{problem}


\vfill ~


\clearpage


\begin{problem} (From Lecture on 2019-09-17)
If you are traveling at 60 miles per hour, how long does
it take you to go 300 miles?
\end{problem}


\vfill ~

\begin{problem} (From the reading)
Classical mechanics, or Newtonian mechanics, is only valid in certain
circumstances. When do the laws of classical mechanics, like $F =
m\,a$ for example, become wrong or break down? There are many answers
to this problem; I will take anything correct.
\end{problem}


\vfill ~

\begin{problem} (From the Math Review Lab)
What is this number? Give your answer in scientific notation.
$$
\frac{(7\times10^{-34})\times(3\times10^8)}{5\times10^{-7}}
$$
You don't need a calculator to solve this problem (\textit{hint: $3/5=0.6$}).
\end{problem}


\vfill ~

\begin{problem} (From the Kinematics Lab)
Here is a data table of times, positions, and velocities in SI units:\\
\rule{1.0in}{0pt}\begin{tabular}{c|c|c}
time $t$ ($\s$) & position $x$ ($\m$) & velocity $v$ ($\m\,\s^{-1}$) \\
\hline
1 & 1.15 & 1.3 \\
2 & 2.60 & 1.6 \\
3 & 4.35 & 1.9 \\
\hline
\end{tabular}\\
What is the average acceleration in the time interval from $2\,\s$ to $3\,\s$?
\end{problem}


\vfill ~


\cleardoublepage



\noindent
Name: \rule[-1ex]{0.60\textwidth}{0.1pt}
NetID: \rule[-1ex]{0.20\textwidth}{0.1pt}

\section*{\textsl{Einstein's Universe} Term Exam 1}
\setcounter{problem}{1}


\begin{problem} (From Problem Set 1)
What is the approximate thickness of a stack of 1000 20-dollar bills?
No need to be precise, and use any units you like.
\end{problem}


\vfill ~

\begin{problem} (From the reading)
What musical instrument did Einstein most enjoy playing?
\end{problem}


\vfill ~

\begin{problem} (From the Kinematics Lab)
Here is a data table of times, positions, and velocities in SI units:\\
\rule{1.0in}{0pt}\begin{tabular}{c|c|c}
time $t$ ($\s$) & position $x$ ($\m$) & velocity $v$ ($\m\,\s^{-1}$) \\
\hline
1 & 1.15 & 1.3 \\
2 & 2.60 & 1.6 \\
3 & 4.35 & 1.9 \\
\hline
\end{tabular}\\
What is the average acceleration in the time interval from $2\,\s$ to $3\,\s$?
\end{problem}


\vfill ~

\begin{problem} (From Lecture on 2019-09-05)
The molar weight of water is $18\,\g$. How many molecules would there
be, therefore, in $18\,\g$ of water? You don't need a calculator for
this.
\end{problem}


\vfill ~


\clearpage


\begin{problem} (From the Math Review Lab)
What is this number? Give your answer in scientific notation.
$$
\frac{(7\times10^{-34})\times(3\times10^8)}{5\times10^{-7}}
$$
You don't need a calculator to solve this problem (\textit{hint: $3/5=0.6$}).
\end{problem}


\vfill ~

\begin{problem} (From Lecture on 2019-09-17)
If you are traveling at 60 miles per hour, how long does
it take you to go 300 miles?
\end{problem}


\vfill ~

\begin{problem} (From Problem Set 1)
What is the relationship between the energy $E$ and wavelength
$\lambda$ of a photon? Give a formula that involves energy $E$,
Planck's Constant $h$, the speed of light $c$, and wavelength
$\lambda$ (or whatever you need).
\end{problem}

\vfill ~

\begin{problem} (From the reading)
Classical mechanics, or Newtonian mechanics, is only valid in certain
circumstances. When do the laws of classical mechanics, like $F =
m\,a$ for example, become wrong or break down? There are many answers
to this problem; I will take anything correct.
\end{problem}


\vfill ~


\cleardoublepage



\noindent
Name: \rule[-1ex]{0.60\textwidth}{0.1pt}
NetID: \rule[-1ex]{0.20\textwidth}{0.1pt}

\section*{\textsl{Einstein's Universe} Term Exam 1}
\setcounter{problem}{1}


\begin{problem} (From Lecture on 2019-09-17)
If you are traveling at 60 miles per hour, how long does
it take you to go 300 miles?
\end{problem}


\vfill ~

\begin{problem} (From Problem Set 1)
What is the approximate thickness of a stack of 1000 20-dollar bills?
No need to be precise, and use any units you like.
\end{problem}


\vfill ~

\begin{problem} (From the reading)
What musical instrument did Einstein most enjoy playing?
\end{problem}


\vfill ~

\begin{problem} (From the Math Review Lab)
What is this number? Give your answer in scientific notation.
$$
\frac{(7\times10^{-34})\times(3\times10^8)}{5\times10^{-7}}
$$
You don't need a calculator to solve this problem (\textit{hint: $3/5=0.6$}).
\end{problem}


\vfill ~


\clearpage


\begin{problem} (From the Kinematics Lab)
Here is a data table of times, positions, and velocities in SI units:\\
\rule{1.0in}{0pt}\begin{tabular}{c|c|c}
time $t$ ($\s$) & position $x$ ($\m$) & velocity $v$ ($\m\,\s^{-1}$) \\
\hline
1 & 1.15 & 1.3 \\
2 & 2.60 & 1.6 \\
3 & 4.35 & 1.9 \\
\hline
\end{tabular}\\
What is the average acceleration in the time interval from $2\,\s$ to $3\,\s$?
\end{problem}


\vfill ~

\begin{problem} (From Lecture on 2019-09-05)
The molar weight of water is $18\,\g$. How many molecules would there
be, therefore, in $18\,\g$ of water? You don't need a calculator for
this.
\end{problem}


\vfill ~

\begin{problem} (From Problem Set 1)
What is the relationship between the energy $E$ and wavelength
$\lambda$ of a photon? Give a formula that involves energy $E$,
Planck's Constant $h$, the speed of light $c$, and wavelength
$\lambda$ (or whatever you need).
\end{problem}

\vfill ~

\begin{problem} (From the reading)
Classical mechanics, or Newtonian mechanics, is only valid in certain
circumstances. When do the laws of classical mechanics, like $F =
m\,a$ for example, become wrong or break down? There are many answers
to this problem; I will take anything correct.
\end{problem}


\vfill ~


\cleardoublepage



\noindent
Name: \rule[-1ex]{0.60\textwidth}{0.1pt}
NetID: \rule[-1ex]{0.20\textwidth}{0.1pt}

\section*{\textsl{Einstein's Universe} Term Exam 1}
\setcounter{problem}{1}


\begin{problem} (From Problem Set 1)
What is the approximate thickness of a stack of 1000 20-dollar bills?
No need to be precise, and use any units you like.
\end{problem}


\vfill ~

\begin{problem} (From Problem Set 1)
What is the relationship between the energy $E$ and wavelength
$\lambda$ of a photon? Give a formula that involves energy $E$,
Planck's Constant $h$, the speed of light $c$, and wavelength
$\lambda$ (or whatever you need).
\end{problem}

\vfill ~

\begin{problem} (From the reading)
What musical instrument did Einstein most enjoy playing?
\end{problem}


\vfill ~

\begin{problem} (From Lecture on 2019-09-17)
If you are traveling at 60 miles per hour, how long does
it take you to go 300 miles?
\end{problem}


\vfill ~


\clearpage


\begin{problem} (From the reading)
Classical mechanics, or Newtonian mechanics, is only valid in certain
circumstances. When do the laws of classical mechanics, like $F =
m\,a$ for example, become wrong or break down? There are many answers
to this problem; I will take anything correct.
\end{problem}


\vfill ~

\begin{problem} (From Lecture on 2019-09-05)
The molar weight of water is $18\,\g$. How many molecules would there
be, therefore, in $18\,\g$ of water? You don't need a calculator for
this.
\end{problem}


\vfill ~

\begin{problem} (From the Kinematics Lab)
Here is a data table of times, positions, and velocities in SI units:\\
\rule{1.0in}{0pt}\begin{tabular}{c|c|c}
time $t$ ($\s$) & position $x$ ($\m$) & velocity $v$ ($\m\,\s^{-1}$) \\
\hline
1 & 1.15 & 1.3 \\
2 & 2.60 & 1.6 \\
3 & 4.35 & 1.9 \\
\hline
\end{tabular}\\
What is the average acceleration in the time interval from $2\,\s$ to $3\,\s$?
\end{problem}


\vfill ~

\begin{problem} (From the Math Review Lab)
What is this number? Give your answer in scientific notation.
$$
\frac{(7\times10^{-34})\times(3\times10^8)}{5\times10^{-7}}
$$
You don't need a calculator to solve this problem (\textit{hint: $3/5=0.6$}).
\end{problem}


\vfill ~


\cleardoublepage



\noindent
Name: \rule[-1ex]{0.60\textwidth}{0.1pt}
NetID: \rule[-1ex]{0.20\textwidth}{0.1pt}

\section*{\textsl{Einstein's Universe} Term Exam 1}
\setcounter{problem}{1}


\begin{problem} (From the Math Review Lab)
What is this number? Give your answer in scientific notation.
$$
\frac{(7\times10^{-34})\times(3\times10^8)}{5\times10^{-7}}
$$
You don't need a calculator to solve this problem (\textit{hint: $3/5=0.6$}).
\end{problem}


\vfill ~

\begin{problem} (From Lecture on 2019-09-17)
If you are traveling at 60 miles per hour, how long does
it take you to go 300 miles?
\end{problem}


\vfill ~

\begin{problem} (From Problem Set 1)
What is the approximate thickness of a stack of 1000 20-dollar bills?
No need to be precise, and use any units you like.
\end{problem}


\vfill ~

\begin{problem} (From the Kinematics Lab)
Here is a data table of times, positions, and velocities in SI units:\\
\rule{1.0in}{0pt}\begin{tabular}{c|c|c}
time $t$ ($\s$) & position $x$ ($\m$) & velocity $v$ ($\m\,\s^{-1}$) \\
\hline
1 & 1.15 & 1.3 \\
2 & 2.60 & 1.6 \\
3 & 4.35 & 1.9 \\
\hline
\end{tabular}\\
What is the average acceleration in the time interval from $2\,\s$ to $3\,\s$?
\end{problem}


\vfill ~


\clearpage


\begin{problem} (From the reading)
What musical instrument did Einstein most enjoy playing?
\end{problem}


\vfill ~

\begin{problem} (From Problem Set 1)
What is the relationship between the energy $E$ and wavelength
$\lambda$ of a photon? Give a formula that involves energy $E$,
Planck's Constant $h$, the speed of light $c$, and wavelength
$\lambda$ (or whatever you need).
\end{problem}

\vfill ~

\begin{problem} (From Lecture on 2019-09-05)
The molar weight of water is $18\,\g$. How many molecules would there
be, therefore, in $18\,\g$ of water? You don't need a calculator for
this.
\end{problem}


\vfill ~

\begin{problem} (From the reading)
Classical mechanics, or Newtonian mechanics, is only valid in certain
circumstances. When do the laws of classical mechanics, like $F =
m\,a$ for example, become wrong or break down? There are many answers
to this problem; I will take anything correct.
\end{problem}


\vfill ~


\cleardoublepage



\noindent
Name: \rule[-1ex]{0.60\textwidth}{0.1pt}
NetID: \rule[-1ex]{0.20\textwidth}{0.1pt}

\section*{\textsl{Einstein's Universe} Term Exam 1}
\setcounter{problem}{1}


\begin{problem} (From Problem Set 1)
What is the approximate thickness of a stack of 1000 20-dollar bills?
No need to be precise, and use any units you like.
\end{problem}


\vfill ~

\begin{problem} (From the reading)
What musical instrument did Einstein most enjoy playing?
\end{problem}


\vfill ~

\begin{problem} (From Lecture on 2019-09-05)
The molar weight of water is $18\,\g$. How many molecules would there
be, therefore, in $18\,\g$ of water? You don't need a calculator for
this.
\end{problem}


\vfill ~

\begin{problem} (From the Kinematics Lab)
Here is a data table of times, positions, and velocities in SI units:\\
\rule{1.0in}{0pt}\begin{tabular}{c|c|c}
time $t$ ($\s$) & position $x$ ($\m$) & velocity $v$ ($\m\,\s^{-1}$) \\
\hline
1 & 1.15 & 1.3 \\
2 & 2.60 & 1.6 \\
3 & 4.35 & 1.9 \\
\hline
\end{tabular}\\
What is the average acceleration in the time interval from $2\,\s$ to $3\,\s$?
\end{problem}


\vfill ~


\clearpage


\begin{problem} (From Lecture on 2019-09-17)
If you are traveling at 60 miles per hour, how long does
it take you to go 300 miles?
\end{problem}


\vfill ~

\begin{problem} (From Problem Set 1)
What is the relationship between the energy $E$ and wavelength
$\lambda$ of a photon? Give a formula that involves energy $E$,
Planck's Constant $h$, the speed of light $c$, and wavelength
$\lambda$ (or whatever you need).
\end{problem}

\vfill ~

\begin{problem} (From the Math Review Lab)
What is this number? Give your answer in scientific notation.
$$
\frac{(7\times10^{-34})\times(3\times10^8)}{5\times10^{-7}}
$$
You don't need a calculator to solve this problem (\textit{hint: $3/5=0.6$}).
\end{problem}


\vfill ~

\begin{problem} (From the reading)
Classical mechanics, or Newtonian mechanics, is only valid in certain
circumstances. When do the laws of classical mechanics, like $F =
m\,a$ for example, become wrong or break down? There are many answers
to this problem; I will take anything correct.
\end{problem}


\vfill ~


\cleardoublepage



\noindent
Name: \rule[-1ex]{0.60\textwidth}{0.1pt}
NetID: \rule[-1ex]{0.20\textwidth}{0.1pt}

\section*{\textsl{Einstein's Universe} Term Exam 1}
\setcounter{problem}{1}


\begin{problem} (From Problem Set 1)
What is the approximate thickness of a stack of 1000 20-dollar bills?
No need to be precise, and use any units you like.
\end{problem}


\vfill ~

\begin{problem} (From Lecture on 2019-09-05)
The molar weight of water is $18\,\g$. How many molecules would there
be, therefore, in $18\,\g$ of water? You don't need a calculator for
this.
\end{problem}


\vfill ~

\begin{problem} (From the reading)
Classical mechanics, or Newtonian mechanics, is only valid in certain
circumstances. When do the laws of classical mechanics, like $F =
m\,a$ for example, become wrong or break down? There are many answers
to this problem; I will take anything correct.
\end{problem}


\vfill ~

\begin{problem} (From the Math Review Lab)
What is this number? Give your answer in scientific notation.
$$
\frac{(7\times10^{-34})\times(3\times10^8)}{5\times10^{-7}}
$$
You don't need a calculator to solve this problem (\textit{hint: $3/5=0.6$}).
\end{problem}


\vfill ~


\clearpage


\begin{problem} (From the Kinematics Lab)
Here is a data table of times, positions, and velocities in SI units:\\
\rule{1.0in}{0pt}\begin{tabular}{c|c|c}
time $t$ ($\s$) & position $x$ ($\m$) & velocity $v$ ($\m\,\s^{-1}$) \\
\hline
1 & 1.15 & 1.3 \\
2 & 2.60 & 1.6 \\
3 & 4.35 & 1.9 \\
\hline
\end{tabular}\\
What is the average acceleration in the time interval from $2\,\s$ to $3\,\s$?
\end{problem}


\vfill ~

\begin{problem} (From Problem Set 1)
What is the relationship between the energy $E$ and wavelength
$\lambda$ of a photon? Give a formula that involves energy $E$,
Planck's Constant $h$, the speed of light $c$, and wavelength
$\lambda$ (or whatever you need).
\end{problem}

\vfill ~

\begin{problem} (From Lecture on 2019-09-17)
If you are traveling at 60 miles per hour, how long does
it take you to go 300 miles?
\end{problem}


\vfill ~

\begin{problem} (From the reading)
What musical instrument did Einstein most enjoy playing?
\end{problem}


\vfill ~


\cleardoublepage



\noindent
Name: \rule[-1ex]{0.60\textwidth}{0.1pt}
NetID: \rule[-1ex]{0.20\textwidth}{0.1pt}

\section*{\textsl{Einstein's Universe} Term Exam 1}
\setcounter{problem}{1}


\begin{problem} (From the Math Review Lab)
What is this number? Give your answer in scientific notation.
$$
\frac{(7\times10^{-34})\times(3\times10^8)}{5\times10^{-7}}
$$
You don't need a calculator to solve this problem (\textit{hint: $3/5=0.6$}).
\end{problem}


\vfill ~

\begin{problem} (From the reading)
What musical instrument did Einstein most enjoy playing?
\end{problem}


\vfill ~

\begin{problem} (From Problem Set 1)
What is the approximate thickness of a stack of 1000 20-dollar bills?
No need to be precise, and use any units you like.
\end{problem}


\vfill ~

\begin{problem} (From Problem Set 1)
What is the relationship between the energy $E$ and wavelength
$\lambda$ of a photon? Give a formula that involves energy $E$,
Planck's Constant $h$, the speed of light $c$, and wavelength
$\lambda$ (or whatever you need).
\end{problem}

\vfill ~


\clearpage


\begin{problem} (From the reading)
Classical mechanics, or Newtonian mechanics, is only valid in certain
circumstances. When do the laws of classical mechanics, like $F =
m\,a$ for example, become wrong or break down? There are many answers
to this problem; I will take anything correct.
\end{problem}


\vfill ~

\begin{problem} (From the Kinematics Lab)
Here is a data table of times, positions, and velocities in SI units:\\
\rule{1.0in}{0pt}\begin{tabular}{c|c|c}
time $t$ ($\s$) & position $x$ ($\m$) & velocity $v$ ($\m\,\s^{-1}$) \\
\hline
1 & 1.15 & 1.3 \\
2 & 2.60 & 1.6 \\
3 & 4.35 & 1.9 \\
\hline
\end{tabular}\\
What is the average acceleration in the time interval from $2\,\s$ to $3\,\s$?
\end{problem}


\vfill ~

\begin{problem} (From Lecture on 2019-09-17)
If you are traveling at 60 miles per hour, how long does
it take you to go 300 miles?
\end{problem}


\vfill ~

\begin{problem} (From Lecture on 2019-09-05)
The molar weight of water is $18\,\g$. How many molecules would there
be, therefore, in $18\,\g$ of water? You don't need a calculator for
this.
\end{problem}


\vfill ~


\cleardoublepage



\noindent
Name: \rule[-1ex]{0.60\textwidth}{0.1pt}
NetID: \rule[-1ex]{0.20\textwidth}{0.1pt}

\section*{\textsl{Einstein's Universe} Term Exam 1}
\setcounter{problem}{1}


\begin{problem} (From Lecture on 2019-09-05)
The molar weight of water is $18\,\g$. How many molecules would there
be, therefore, in $18\,\g$ of water? You don't need a calculator for
this.
\end{problem}


\vfill ~

\begin{problem} (From Lecture on 2019-09-17)
If you are traveling at 60 miles per hour, how long does
it take you to go 300 miles?
\end{problem}


\vfill ~

\begin{problem} (From the reading)
What musical instrument did Einstein most enjoy playing?
\end{problem}


\vfill ~

\begin{problem} (From Problem Set 1)
What is the relationship between the energy $E$ and wavelength
$\lambda$ of a photon? Give a formula that involves energy $E$,
Planck's Constant $h$, the speed of light $c$, and wavelength
$\lambda$ (or whatever you need).
\end{problem}

\vfill ~


\clearpage


\begin{problem} (From the reading)
Classical mechanics, or Newtonian mechanics, is only valid in certain
circumstances. When do the laws of classical mechanics, like $F =
m\,a$ for example, become wrong or break down? There are many answers
to this problem; I will take anything correct.
\end{problem}


\vfill ~

\begin{problem} (From the Kinematics Lab)
Here is a data table of times, positions, and velocities in SI units:\\
\rule{1.0in}{0pt}\begin{tabular}{c|c|c}
time $t$ ($\s$) & position $x$ ($\m$) & velocity $v$ ($\m\,\s^{-1}$) \\
\hline
1 & 1.15 & 1.3 \\
2 & 2.60 & 1.6 \\
3 & 4.35 & 1.9 \\
\hline
\end{tabular}\\
What is the average acceleration in the time interval from $2\,\s$ to $3\,\s$?
\end{problem}


\vfill ~

\begin{problem} (From Problem Set 1)
What is the approximate thickness of a stack of 1000 20-dollar bills?
No need to be precise, and use any units you like.
\end{problem}


\vfill ~

\begin{problem} (From the Math Review Lab)
What is this number? Give your answer in scientific notation.
$$
\frac{(7\times10^{-34})\times(3\times10^8)}{5\times10^{-7}}
$$
You don't need a calculator to solve this problem (\textit{hint: $3/5=0.6$}).
\end{problem}


\vfill ~


\cleardoublepage



\noindent
Name: \rule[-1ex]{0.60\textwidth}{0.1pt}
NetID: \rule[-1ex]{0.20\textwidth}{0.1pt}

\section*{\textsl{Einstein's Universe} Term Exam 1}
\setcounter{problem}{1}


\begin{problem} (From the reading)
Classical mechanics, or Newtonian mechanics, is only valid in certain
circumstances. When do the laws of classical mechanics, like $F =
m\,a$ for example, become wrong or break down? There are many answers
to this problem; I will take anything correct.
\end{problem}


\vfill ~

\begin{problem} (From Problem Set 1)
What is the approximate thickness of a stack of 1000 20-dollar bills?
No need to be precise, and use any units you like.
\end{problem}


\vfill ~

\begin{problem} (From Lecture on 2019-09-17)
If you are traveling at 60 miles per hour, how long does
it take you to go 300 miles?
\end{problem}


\vfill ~

\begin{problem} (From the Math Review Lab)
What is this number? Give your answer in scientific notation.
$$
\frac{(7\times10^{-34})\times(3\times10^8)}{5\times10^{-7}}
$$
You don't need a calculator to solve this problem (\textit{hint: $3/5=0.6$}).
\end{problem}


\vfill ~


\clearpage


\begin{problem} (From Lecture on 2019-09-05)
The molar weight of water is $18\,\g$. How many molecules would there
be, therefore, in $18\,\g$ of water? You don't need a calculator for
this.
\end{problem}


\vfill ~

\begin{problem} (From Problem Set 1)
What is the relationship between the energy $E$ and wavelength
$\lambda$ of a photon? Give a formula that involves energy $E$,
Planck's Constant $h$, the speed of light $c$, and wavelength
$\lambda$ (or whatever you need).
\end{problem}

\vfill ~

\begin{problem} (From the Kinematics Lab)
Here is a data table of times, positions, and velocities in SI units:\\
\rule{1.0in}{0pt}\begin{tabular}{c|c|c}
time $t$ ($\s$) & position $x$ ($\m$) & velocity $v$ ($\m\,\s^{-1}$) \\
\hline
1 & 1.15 & 1.3 \\
2 & 2.60 & 1.6 \\
3 & 4.35 & 1.9 \\
\hline
\end{tabular}\\
What is the average acceleration in the time interval from $2\,\s$ to $3\,\s$?
\end{problem}


\vfill ~

\begin{problem} (From the reading)
What musical instrument did Einstein most enjoy playing?
\end{problem}


\vfill ~


\cleardoublepage



\noindent
Name: \rule[-1ex]{0.60\textwidth}{0.1pt}
NetID: \rule[-1ex]{0.20\textwidth}{0.1pt}

\section*{\textsl{Einstein's Universe} Term Exam 1}
\setcounter{problem}{1}


\begin{problem} (From Problem Set 1)
What is the approximate thickness of a stack of 1000 20-dollar bills?
No need to be precise, and use any units you like.
\end{problem}


\vfill ~

\begin{problem} (From Lecture on 2019-09-17)
If you are traveling at 60 miles per hour, how long does
it take you to go 300 miles?
\end{problem}


\vfill ~

\begin{problem} (From the reading)
What musical instrument did Einstein most enjoy playing?
\end{problem}


\vfill ~

\begin{problem} (From the Math Review Lab)
What is this number? Give your answer in scientific notation.
$$
\frac{(7\times10^{-34})\times(3\times10^8)}{5\times10^{-7}}
$$
You don't need a calculator to solve this problem (\textit{hint: $3/5=0.6$}).
\end{problem}


\vfill ~


\clearpage


\begin{problem} (From Problem Set 1)
What is the relationship between the energy $E$ and wavelength
$\lambda$ of a photon? Give a formula that involves energy $E$,
Planck's Constant $h$, the speed of light $c$, and wavelength
$\lambda$ (or whatever you need).
\end{problem}

\vfill ~

\begin{problem} (From Lecture on 2019-09-05)
The molar weight of water is $18\,\g$. How many molecules would there
be, therefore, in $18\,\g$ of water? You don't need a calculator for
this.
\end{problem}


\vfill ~

\begin{problem} (From the reading)
Classical mechanics, or Newtonian mechanics, is only valid in certain
circumstances. When do the laws of classical mechanics, like $F =
m\,a$ for example, become wrong or break down? There are many answers
to this problem; I will take anything correct.
\end{problem}


\vfill ~

\begin{problem} (From the Kinematics Lab)
Here is a data table of times, positions, and velocities in SI units:\\
\rule{1.0in}{0pt}\begin{tabular}{c|c|c}
time $t$ ($\s$) & position $x$ ($\m$) & velocity $v$ ($\m\,\s^{-1}$) \\
\hline
1 & 1.15 & 1.3 \\
2 & 2.60 & 1.6 \\
3 & 4.35 & 1.9 \\
\hline
\end{tabular}\\
What is the average acceleration in the time interval from $2\,\s$ to $3\,\s$?
\end{problem}


\vfill ~


\cleardoublepage



\noindent
Name: \rule[-1ex]{0.60\textwidth}{0.1pt}
NetID: \rule[-1ex]{0.20\textwidth}{0.1pt}

\section*{\textsl{Einstein's Universe} Term Exam 1}
\setcounter{problem}{1}


\begin{problem} (From the Kinematics Lab)
Here is a data table of times, positions, and velocities in SI units:\\
\rule{1.0in}{0pt}\begin{tabular}{c|c|c}
time $t$ ($\s$) & position $x$ ($\m$) & velocity $v$ ($\m\,\s^{-1}$) \\
\hline
1 & 1.15 & 1.3 \\
2 & 2.60 & 1.6 \\
3 & 4.35 & 1.9 \\
\hline
\end{tabular}\\
What is the average acceleration in the time interval from $2\,\s$ to $3\,\s$?
\end{problem}


\vfill ~

\begin{problem} (From Problem Set 1)
What is the relationship between the energy $E$ and wavelength
$\lambda$ of a photon? Give a formula that involves energy $E$,
Planck's Constant $h$, the speed of light $c$, and wavelength
$\lambda$ (or whatever you need).
\end{problem}

\vfill ~

\begin{problem} (From the reading)
Classical mechanics, or Newtonian mechanics, is only valid in certain
circumstances. When do the laws of classical mechanics, like $F =
m\,a$ for example, become wrong or break down? There are many answers
to this problem; I will take anything correct.
\end{problem}


\vfill ~

\begin{problem} (From Problem Set 1)
What is the approximate thickness of a stack of 1000 20-dollar bills?
No need to be precise, and use any units you like.
\end{problem}


\vfill ~


\clearpage


\begin{problem} (From the reading)
What musical instrument did Einstein most enjoy playing?
\end{problem}


\vfill ~

\begin{problem} (From Lecture on 2019-09-17)
If you are traveling at 60 miles per hour, how long does
it take you to go 300 miles?
\end{problem}


\vfill ~

\begin{problem} (From the Math Review Lab)
What is this number? Give your answer in scientific notation.
$$
\frac{(7\times10^{-34})\times(3\times10^8)}{5\times10^{-7}}
$$
You don't need a calculator to solve this problem (\textit{hint: $3/5=0.6$}).
\end{problem}


\vfill ~

\begin{problem} (From Lecture on 2019-09-05)
The molar weight of water is $18\,\g$. How many molecules would there
be, therefore, in $18\,\g$ of water? You don't need a calculator for
this.
\end{problem}


\vfill ~


\cleardoublepage



\noindent
Name: \rule[-1ex]{0.60\textwidth}{0.1pt}
NetID: \rule[-1ex]{0.20\textwidth}{0.1pt}

\section*{\textsl{Einstein's Universe} Term Exam 1}
\setcounter{problem}{1}


\begin{problem} (From Lecture on 2019-09-05)
The molar weight of water is $18\,\g$. How many molecules would there
be, therefore, in $18\,\g$ of water? You don't need a calculator for
this.
\end{problem}


\vfill ~

\begin{problem} (From the Math Review Lab)
What is this number? Give your answer in scientific notation.
$$
\frac{(7\times10^{-34})\times(3\times10^8)}{5\times10^{-7}}
$$
You don't need a calculator to solve this problem (\textit{hint: $3/5=0.6$}).
\end{problem}


\vfill ~

\begin{problem} (From Lecture on 2019-09-17)
If you are traveling at 60 miles per hour, how long does
it take you to go 300 miles?
\end{problem}


\vfill ~

\begin{problem} (From the reading)
What musical instrument did Einstein most enjoy playing?
\end{problem}


\vfill ~


\clearpage


\begin{problem} (From Problem Set 1)
What is the approximate thickness of a stack of 1000 20-dollar bills?
No need to be precise, and use any units you like.
\end{problem}


\vfill ~

\begin{problem} (From the Kinematics Lab)
Here is a data table of times, positions, and velocities in SI units:\\
\rule{1.0in}{0pt}\begin{tabular}{c|c|c}
time $t$ ($\s$) & position $x$ ($\m$) & velocity $v$ ($\m\,\s^{-1}$) \\
\hline
1 & 1.15 & 1.3 \\
2 & 2.60 & 1.6 \\
3 & 4.35 & 1.9 \\
\hline
\end{tabular}\\
What is the average acceleration in the time interval from $2\,\s$ to $3\,\s$?
\end{problem}


\vfill ~

\begin{problem} (From the reading)
Classical mechanics, or Newtonian mechanics, is only valid in certain
circumstances. When do the laws of classical mechanics, like $F =
m\,a$ for example, become wrong or break down? There are many answers
to this problem; I will take anything correct.
\end{problem}


\vfill ~

\begin{problem} (From Problem Set 1)
What is the relationship between the energy $E$ and wavelength
$\lambda$ of a photon? Give a formula that involves energy $E$,
Planck's Constant $h$, the speed of light $c$, and wavelength
$\lambda$ (or whatever you need).
\end{problem}

\vfill ~


\cleardoublepage



\noindent
Name: \rule[-1ex]{0.60\textwidth}{0.1pt}
NetID: \rule[-1ex]{0.20\textwidth}{0.1pt}

\section*{\textsl{Einstein's Universe} Term Exam 1}
\setcounter{problem}{1}


\begin{problem} (From the reading)
What musical instrument did Einstein most enjoy playing?
\end{problem}


\vfill ~

\begin{problem} (From Problem Set 1)
What is the relationship between the energy $E$ and wavelength
$\lambda$ of a photon? Give a formula that involves energy $E$,
Planck's Constant $h$, the speed of light $c$, and wavelength
$\lambda$ (or whatever you need).
\end{problem}

\vfill ~

\begin{problem} (From the Math Review Lab)
What is this number? Give your answer in scientific notation.
$$
\frac{(7\times10^{-34})\times(3\times10^8)}{5\times10^{-7}}
$$
You don't need a calculator to solve this problem (\textit{hint: $3/5=0.6$}).
\end{problem}


\vfill ~

\begin{problem} (From Problem Set 1)
What is the approximate thickness of a stack of 1000 20-dollar bills?
No need to be precise, and use any units you like.
\end{problem}


\vfill ~


\clearpage


\begin{problem} (From Lecture on 2019-09-17)
If you are traveling at 60 miles per hour, how long does
it take you to go 300 miles?
\end{problem}


\vfill ~

\begin{problem} (From Lecture on 2019-09-05)
The molar weight of water is $18\,\g$. How many molecules would there
be, therefore, in $18\,\g$ of water? You don't need a calculator for
this.
\end{problem}


\vfill ~

\begin{problem} (From the Kinematics Lab)
Here is a data table of times, positions, and velocities in SI units:\\
\rule{1.0in}{0pt}\begin{tabular}{c|c|c}
time $t$ ($\s$) & position $x$ ($\m$) & velocity $v$ ($\m\,\s^{-1}$) \\
\hline
1 & 1.15 & 1.3 \\
2 & 2.60 & 1.6 \\
3 & 4.35 & 1.9 \\
\hline
\end{tabular}\\
What is the average acceleration in the time interval from $2\,\s$ to $3\,\s$?
\end{problem}


\vfill ~

\begin{problem} (From the reading)
Classical mechanics, or Newtonian mechanics, is only valid in certain
circumstances. When do the laws of classical mechanics, like $F =
m\,a$ for example, become wrong or break down? There are many answers
to this problem; I will take anything correct.
\end{problem}


\vfill ~


\cleardoublepage



\noindent
Name: \rule[-1ex]{0.60\textwidth}{0.1pt}
NetID: \rule[-1ex]{0.20\textwidth}{0.1pt}

\section*{\textsl{Einstein's Universe} Term Exam 1}
\setcounter{problem}{1}


\begin{problem} (From Lecture on 2019-09-05)
The molar weight of water is $18\,\g$. How many molecules would there
be, therefore, in $18\,\g$ of water? You don't need a calculator for
this.
\end{problem}


\vfill ~

\begin{problem} (From the reading)
What musical instrument did Einstein most enjoy playing?
\end{problem}


\vfill ~

\begin{problem} (From Problem Set 1)
What is the approximate thickness of a stack of 1000 20-dollar bills?
No need to be precise, and use any units you like.
\end{problem}


\vfill ~

\begin{problem} (From the Kinematics Lab)
Here is a data table of times, positions, and velocities in SI units:\\
\rule{1.0in}{0pt}\begin{tabular}{c|c|c}
time $t$ ($\s$) & position $x$ ($\m$) & velocity $v$ ($\m\,\s^{-1}$) \\
\hline
1 & 1.15 & 1.3 \\
2 & 2.60 & 1.6 \\
3 & 4.35 & 1.9 \\
\hline
\end{tabular}\\
What is the average acceleration in the time interval from $2\,\s$ to $3\,\s$?
\end{problem}


\vfill ~


\clearpage


\begin{problem} (From the reading)
Classical mechanics, or Newtonian mechanics, is only valid in certain
circumstances. When do the laws of classical mechanics, like $F =
m\,a$ for example, become wrong or break down? There are many answers
to this problem; I will take anything correct.
\end{problem}


\vfill ~

\begin{problem} (From Lecture on 2019-09-17)
If you are traveling at 60 miles per hour, how long does
it take you to go 300 miles?
\end{problem}


\vfill ~

\begin{problem} (From Problem Set 1)
What is the relationship between the energy $E$ and wavelength
$\lambda$ of a photon? Give a formula that involves energy $E$,
Planck's Constant $h$, the speed of light $c$, and wavelength
$\lambda$ (or whatever you need).
\end{problem}

\vfill ~

\begin{problem} (From the Math Review Lab)
What is this number? Give your answer in scientific notation.
$$
\frac{(7\times10^{-34})\times(3\times10^8)}{5\times10^{-7}}
$$
You don't need a calculator to solve this problem (\textit{hint: $3/5=0.6$}).
\end{problem}


\vfill ~


\cleardoublepage



\noindent
Name: \rule[-1ex]{0.60\textwidth}{0.1pt}
NetID: \rule[-1ex]{0.20\textwidth}{0.1pt}

\section*{\textsl{Einstein's Universe} Term Exam 1}
\setcounter{problem}{1}


\begin{problem} (From the reading)
Classical mechanics, or Newtonian mechanics, is only valid in certain
circumstances. When do the laws of classical mechanics, like $F =
m\,a$ for example, become wrong or break down? There are many answers
to this problem; I will take anything correct.
\end{problem}


\vfill ~

\begin{problem} (From the Kinematics Lab)
Here is a data table of times, positions, and velocities in SI units:\\
\rule{1.0in}{0pt}\begin{tabular}{c|c|c}
time $t$ ($\s$) & position $x$ ($\m$) & velocity $v$ ($\m\,\s^{-1}$) \\
\hline
1 & 1.15 & 1.3 \\
2 & 2.60 & 1.6 \\
3 & 4.35 & 1.9 \\
\hline
\end{tabular}\\
What is the average acceleration in the time interval from $2\,\s$ to $3\,\s$?
\end{problem}


\vfill ~

\begin{problem} (From Problem Set 1)
What is the relationship between the energy $E$ and wavelength
$\lambda$ of a photon? Give a formula that involves energy $E$,
Planck's Constant $h$, the speed of light $c$, and wavelength
$\lambda$ (or whatever you need).
\end{problem}

\vfill ~

\begin{problem} (From the Math Review Lab)
What is this number? Give your answer in scientific notation.
$$
\frac{(7\times10^{-34})\times(3\times10^8)}{5\times10^{-7}}
$$
You don't need a calculator to solve this problem (\textit{hint: $3/5=0.6$}).
\end{problem}


\vfill ~


\clearpage


\begin{problem} (From the reading)
What musical instrument did Einstein most enjoy playing?
\end{problem}


\vfill ~

\begin{problem} (From Lecture on 2019-09-05)
The molar weight of water is $18\,\g$. How many molecules would there
be, therefore, in $18\,\g$ of water? You don't need a calculator for
this.
\end{problem}


\vfill ~

\begin{problem} (From Lecture on 2019-09-17)
If you are traveling at 60 miles per hour, how long does
it take you to go 300 miles?
\end{problem}


\vfill ~

\begin{problem} (From Problem Set 1)
What is the approximate thickness of a stack of 1000 20-dollar bills?
No need to be precise, and use any units you like.
\end{problem}


\vfill ~


\cleardoublepage



\noindent
Name: \rule[-1ex]{0.60\textwidth}{0.1pt}
NetID: \rule[-1ex]{0.20\textwidth}{0.1pt}

\section*{\textsl{Einstein's Universe} Term Exam 1}
\setcounter{problem}{1}


\begin{problem} (From the Kinematics Lab)
Here is a data table of times, positions, and velocities in SI units:\\
\rule{1.0in}{0pt}\begin{tabular}{c|c|c}
time $t$ ($\s$) & position $x$ ($\m$) & velocity $v$ ($\m\,\s^{-1}$) \\
\hline
1 & 1.15 & 1.3 \\
2 & 2.60 & 1.6 \\
3 & 4.35 & 1.9 \\
\hline
\end{tabular}\\
What is the average acceleration in the time interval from $2\,\s$ to $3\,\s$?
\end{problem}


\vfill ~

\begin{problem} (From Problem Set 1)
What is the relationship between the energy $E$ and wavelength
$\lambda$ of a photon? Give a formula that involves energy $E$,
Planck's Constant $h$, the speed of light $c$, and wavelength
$\lambda$ (or whatever you need).
\end{problem}

\vfill ~

\begin{problem} (From the reading)
Classical mechanics, or Newtonian mechanics, is only valid in certain
circumstances. When do the laws of classical mechanics, like $F =
m\,a$ for example, become wrong or break down? There are many answers
to this problem; I will take anything correct.
\end{problem}


\vfill ~

\begin{problem} (From Problem Set 1)
What is the approximate thickness of a stack of 1000 20-dollar bills?
No need to be precise, and use any units you like.
\end{problem}


\vfill ~


\clearpage


\begin{problem} (From the Math Review Lab)
What is this number? Give your answer in scientific notation.
$$
\frac{(7\times10^{-34})\times(3\times10^8)}{5\times10^{-7}}
$$
You don't need a calculator to solve this problem (\textit{hint: $3/5=0.6$}).
\end{problem}


\vfill ~

\begin{problem} (From Lecture on 2019-09-17)
If you are traveling at 60 miles per hour, how long does
it take you to go 300 miles?
\end{problem}


\vfill ~

\begin{problem} (From Lecture on 2019-09-05)
The molar weight of water is $18\,\g$. How many molecules would there
be, therefore, in $18\,\g$ of water? You don't need a calculator for
this.
\end{problem}


\vfill ~

\begin{problem} (From the reading)
What musical instrument did Einstein most enjoy playing?
\end{problem}


\vfill ~


\cleardoublepage



\end{document}

