\documentclass[12pt, letterpaper]{article}
\usepackage{url, graphicx, epstopdf, amsmath}

% page layout
\setlength{\topmargin}{-0.25in}
\setlength{\textheight}{9.5in}
\setlength{\headheight}{0in}
\setlength{\headsep}{0in}
\setlength{\parindent}{1.1\baselineskip}

\newcommand{\examheader}[1]{
\noindent
Name:\rule[-1ex]{0.40\textwidth}{0.1pt}
NetID:\rule[-1ex]{0.13\textwidth}{0.1pt}
Lab time:\rule[-1ex]{0.13\textwidth}{0.1pt}

\section*{\textsl{Einstein's Universe} {#1}}
\setcounter{problem}{1}}


% problem formatting
\newcommand{\problemname}{Problem}
\newcounter{problem}\setcounter{problem}{1}
\newenvironment{problem}{%
  \addvspace{\baselineskip}\noindent\textbf{Problem~\theproblem:}\refstepcounter{problem}
}{%
  \par\addvspace{\baselineskip}
}

% words
\newcommand{\foreign}[1]{\textsl{#1}}
\newcommand{\vs}{\foreign{vs}}

% math
\newcommand{\dd}{\mathrm{d}}
\newcommand{\e}{\mathrm{e}}

% primary or base units
\newcommand{\rad}{\mathrm{rad}}
\newcommand{\kg}{\mathrm{kg}}
\newcommand{\m}{\mathrm{m}}
\newcommand{\s}{\mathrm{s}}

% secondary units
\renewcommand{\deg}{\mathrm{deg}}
\newcommand{\g}{\mathrm{g}}
\newcommand{\km}{\mathrm{km}}
\newcommand{\cm}{\mathrm{cm}}
\newcommand{\mm}{\mathrm{mm}}
\newcommand{\mum}{\mathrm{\mu m}}
\newcommand{\ft}{\mathrm{ft}}
\newcommand{\mi}{\mathrm{mi}}
\newcommand{\AU}{\mathrm{AU}}
\newcommand{\Mpc}{\mathrm{Mpc}}
\newcommand{\ns}{\mathrm{ns}}
\newcommand{\h}{\mathrm{h}}
\newcommand{\yr}{\mathrm{yr}}
\newcommand{\N}{\mathrm{N}}
\newcommand{\J}{\mathrm{J}}
\newcommand{\eV}{\mathrm{eV}}
\newcommand{\W}{\mathrm{W}}
\newcommand{\Pa}{\mathrm{Pa}}

% derived units
\newcommand{\mps}{\m\,\s^{-1}}
\newcommand{\mph}{\mi\,\h^{-1}}
\newcommand{\mpss}{\m\,\s^{-2}}

% random stuff
\sloppy\sloppypar\raggedbottom\frenchspacing


\newcommand{\escape}{\mathrm{escape}}

\begin{document}

\section*{\textsl{Einstein's Universe} Problem Set 6}

This problem set is not to be handed in for credit. But it is due
before \textbf{Thursday December 12}, when some of these problems
will appear in part, near-verbatim, on Term Exam 6.

\begin{problem}
The formula for escape velocity $v_{\escape}$ from the surface of the Earth
is
\begin{equation}
\frac{1}{2}\,v_{\escape}^2 - \frac{G\,M}{R} = 0
\quad,
\end{equation}
where $G$ is Newton's constant, $M$ is the mass of the Earth, and $R$
is the radius of the Earth.
Now imagine that the Earth was compressed into a sphere so small that
this escape velocity became the speed of light! That's a proposal for what
makes a black hole. What would be the radius of that small sphere?
Now look up the true formula for the radius of a black hole and compare
your answer to the true answer.

If you make $M$ the mass of the Sun, and $R$ the radius (1\,AU) of
Earth's orbit around the Sun, then the velocity you compute is the
escape velocity from the Earth's orbit to outside the Solar System!
This is the velocity at which the {\small NASA} \textsl{Voyager} probes had to
be launched.
How massive would the Sun have to be for the escape velocity at 1\,AU
to be the speed of light? You can use either the equation above, or
else the true formula for a black hole.
\end{problem}

\begin{problem}
The \textsl{\small LIGO} discovery of the neutron-star-merger system
{\small GW170817} gave an opportunity to measure the speed of
gravitational waves (GWs), because the discovered event had both a
light signal (a flash) and a GW signal (a chirp).
Look up the distance to this event.
Now imagine that GWs travel more slowly than the speed of light.
How much slower than the speed of light would GWs have to travel for
the two signals to arrive at Earth separated by 10 seconds?
\end{problem}

\begin{problem}
At the center of the Milky Way, there is a star called {\small S2}
orbiting a supermassive black hole (called Sag A$\ast$).
The star orbits the black hole with a period of 16\,yr, on an orbit
that has a mean radius (semi-major axis) of 1000\,AU.
Plug these numbers into the classic Kepler--Newton formula for orbital
period $T$ given semi-major axis $a$ to estimate the mass of the black
hole.
\begin{equation}
T^2 = \frac{4\,\pi^2}{G\,M}\,a^3
\end{equation}
You might have to do some unit conversions!

Also compute the mean orbital velocity $v$ of the star using
\begin{equation}
v \approx \frac{2\,\pi\,a}{T}
\end{equation}
Do you think the non-relativistic Kepler formula is very wrong in this
case? Why or why not?
\end{problem}

\begin{problem}
When a black hole of mass $M_1$ merges with a black hole of mass $M_2$,
it creates a new black hole of mass $M_3$. Do you expect $M_3$ to be
equal to, less than, or more than, the sum $M_1 + M_2$? Recall mass--energy
equivalence and the fact that there is gravitational radiation emitted as
they merge.

Now look up the first \textsl{\small LIGO} discovery {\small GW150914} and
look up $M_1, M_2, M_3$ for this discovery. Are you right?
\end{problem}

\end{document}
