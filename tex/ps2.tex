\documentclass[12pt, letterpaper]{article}
\include{eu}

\begin{document}

\section*{\textsl{Einstein's Universe} Problem Set 2}

This problem set is not to be handed in for credit. But it is due
before \textbf{Thursday October 3}, when some of these problems
will appear in part, near-verbatim, on Term Exam 2.

\begin{problem}
In class on 2019-09-05 we started to compute the volume and size of a
water molecule by considering the volume of a block of ice. What,
roughly, is the volume of $18\,\g$ of ice, and therefore what,
roughly, is the volume of an ice molecule? If you think of the
molecule as a cube (yes this is a bad approximation), how big is that
cube (how long are the sides of the cube)? Now do the same for an iron
atom by considering solid iron, and for a uranium atom by considering
solid uranium.
\end{problem}

\begin{problem}
compute the size of a molecule a different way: imagine that when you
drive a car around, the tire leaves a mono-molecular layer of rubber
everywhere it goes. sounds crazy, right? but it's close to true:

A typical tire lasts 50,000 miles. go look at a tire and estimate the
depth of the tread, and therefore the thickness of the rubber left on
the road, on average, over those 50,000 miles. is that thickness
reasonable to be mono-molecular?
\end{problem}

\begin{problem}
Piano strings (it turns out) are characterized by a mass $m$, a length
$L$, and a tension $T$ (force). Find an expression that combines $m$,
$L$, and $T$ into a quantity with the units of frequency.

Now imagine a piano string has a diameter of $D=1\,\mm$, a length of
$L=2\,\m$, and is made of steel. What, roughly, is its mass $m$? And
roughly what is the tension $T$ it must have to play middle C? You
might have to look up the frequency of middle C on the internets.
\end{problem}

\begin{problem}
Explain in words why you get a ``diffraction pattern'' when light of
one single wavelength (like laser light) goes through a pair of slits
separated in space by a many wavelengths. That is, explain the
double-slit experiment.

Draw a diagram that shows the slits and the screen and the position on
the screen of the central bright (constructive interference) point
relative to the two slits. Add to your diagram the positions of the
next dark and bright points going away from that central bright
point. Label these bright and dark points with the path differences
from the two slits.
\end{problem}

\begin{problem}
Imagine that electrons are bound to a metal with a voltage of $2\,V$
(that is, it takes $2\,\eV$ of energy to liberate the electron, or
the work function is $2\,\eV$). This metal is illuminated with light
of different wavelengths: $1\,\mum$, $0.5\,\mum$, and $0.25\,\mum$.

For which wavelengths is it possible for the light to liberate
electrons by the photo-electric effect?  And, for the wavelengths that
can liberate electrons, what are the kinetic energies and speeds of
the liberated electrons? Use the classical $(1/2)\,m\,v^2$ formula for
kinetic energy. Is that formula okay, or should we be using something
more relativistically correct?
\end{problem}

\end{document}
