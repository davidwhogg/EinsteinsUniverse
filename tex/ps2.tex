\documentclass[12pt, letterpaper]{article}
\include{eu}

\begin{document}

\section*{\textsl{Einstein's Universe} Problem Set 2}

This problem set is not to be handed in for credit. But it is due
before \textbf{Thursday October 3}, when some of these problems
will appear in part, near-verbatim, on Term Exam 2.

\begin{problem}
In class on 2019-09-05 we started to compute the volume and size of a
water molecule by considering the volume of a block of ice. What,
roughly, is the volume of $18\,\g$ of ice, and therefore what,
roughly, is the volume of an ice molecule? If you think of the
molecule as a cube (yes this is a bad approximation), how big is that
cube (how long are the sides of the cube)? Now do the same for an iron
atom by considering solid iron, and for a lead atom by considering
solid lead.
\end{problem}

\begin{problem}
compute the size of a molecule a different way: imagine that when you
drive a car around, the tire leaves a mono-molecular layer of rubber
everywhere it goes. sounds crazy, right? but it's close to true:

a typical tire lasts 50,000 miles. go look at a tire and estimate the
depth of the tread, and therefore the thickness of the rubber left on
the road, on average, over those 50,000 miles. is that thickness
reasonable to be mono-molecular?
\end{problem}

\end{document}
